\documentclass[parskip=half, fontsize=12pt]{scrartcl}

\usepackage[ngerman]{babel}
% \usepackage[english]{babel}         % Deutsches Sprachpaket

\usepackage{iftex}
\ifPDFTeX
   \usepackage[utf8]{inputenc}% Eingaben codieren
   \usepackage{fourier}
   % \usepackage[T1]{fontenc}   % Umlaute codieren, Silbentrennung
\else
   \usepackage{lmodern}
   % \usepackage[no-math]{fontspec}
   % \setmainfont{Utopia Std}
\fi

\usepackage{amsmath, amssymb}       % Mathe, \mathbb{R}
\usepackage{amsthm,amstext}         % Theoreme, \text im Mathe-Modus
\usepackage{mathtools}              % \Aboxed für Boxen in Align Umgebungen
\usepackage[arrowdel]{physics}      % Ableitungen \dv{B}{t} \pdv \dd{t}
\usepackage[left=2.5cm, right=2.5cm, top=3cm, bottom=3cm]{geometry}
\usepackage{graphicx}               % \includegraphics
\usepackage[extendedchars]{grffile} % extends file name processing of graphics
\usepackage[section]{placeins}      % \Floatbarrier
\usepackage{wrapfig}                % Bilder umfließen
\usepackage{enumerate}              % Aufzählungen
\usepackage{enumitem}               % \begin{enumerate}[label=\alph*]
\usepackage{footnote}               % Fußzeilen
\usepackage{booktabs}               % publication quality tables
\usepackage{tikz, pgfplots}         % TIKZ ist kein Zeichenprogramm
\usepackage[europeanvoltages,europeanresistors]{circuitikz}
\usepackage{bm}                     % bold symbols \bm{r}
\usepackage{dsfont}                 % identity matrix \mathds{1}
\usepackage{esint}                  % Doppelintegrale
\usepackage{mathrsfs}               % \mathscr{} statt \mathcal{}
\usepackage{placeins}               % FloatBarrier
\usepackage{subcaption}
\usepackage{multirow}
\usepackage{mdframed}               % Deckblatt
\usepackage{xcolor}                 % Deckblatt
\usepackage{fancyhdr}               % Kopfzeile
\usepackage{aligned-overset}        % Ausrichtungen mit stackrel oder overset
\usepackage{cancel}
\usepackage{float}
\usepackage{cite}
\usepackage[version=4]{mhchem}                 % Chemistry Package
\usepackage{multicol}
% \usepackage{pdfpages}             % insert whole pdf files

\definecolor{FSUblau}{cmyk}{1,0.7,0.1,0.5}
\definecolor{PAForange}{cmyk}{0.1,0.7,1,0}
\definecolor{Gruen}{cmyk}{1,0.1,0.7,0.5}

\usepackage[final,
    pdfauthor={Martin Beyer},
    pdffitwindow=false,     % resize document window to fit document size
    pdftoolbar=false,        % Adobe Toolbar
    bookmarks=true,         % Anzeigen der Kapitel
    bookmarksopen=true,
    bookmarksopenlevel=0,
    bookmarksnumbered=true,
    colorlinks=true,        % fuer Druckversion auf "false"
    linkcolor=FSUblau,         % Table of Contents, Footnotes
    urlcolor=FSUblau,          % fuer eingebunden URLs
    citecolor=FSUblau,         % Equations, References
    filecolor=FSUblau,
    pdfborder={0 0 0},      % keine Rahmen um Verlinkungen: {0 0 0}
    pagebackref=false
]{hyperref}

\pgfplotsset{compat=1.18}
\newcommand\mydots{\makebox[1em][c]{.\hfil.\hfil.}}
\newcommand{\minus}{\scalebox{0.75}[1.0]{$-$}}
\newcommand{\e}{\mathrm{e}}
\renewcommand{\i}{\mathrm{i}}

\usepackage[detect-all,
            locale=DE,
            exponent-product = \cdot,
            per-mode=fraction]{siunitx}
\usepackage[position=below,
            format=hang,
            figurename=Fig.,
            labelfont={bf},
            font=small]{caption}

\sisetup{range-phrase = {\mydots}}
% Commands
\usetikzlibrary{positioning,intersections,calc,external}
\usepgfplotslibrary{fillbetween, groupplots}
\pgfplotsset{
tick label style={font=\small},
label style={font=\small},
legend style={font=\footnotesize},
every axis post/.style={legend cell align={left}}}
\tikzstyle{every node}=[font=\small]


\setlength{\parindent}{0px}         % keine Absätze durch Leerzeilen im Code
\numberwithin{equation}{section}

% Remove page number from \thispagestyle{empty}
\makeatletter\let\ps@plain\ps@fancy\makeatother

% Deckblatt
\newcommand{\HRule}{\rule{\linewidth}{0.5mm}}
\newcommand{\Deckblatt}[5][\LaTeX-Satz und Design von Martin Beyer]{
  \begin{titlepage}
    \center
    \textsc{\LARGE Friedrich-Schiller-Universität Jena\\[1ex]
    \Large Physikalisch-Astronomische-Fakultät}
    \begin{figure}[h!]
       \centering
       \includegraphics[scale=0.75]{uni-Logo_neu.pdf}
    \end{figure}\\
    \vspace{2em}
    \textsc{\Large #2}\\[0.35cm]
    \HRule \\[0.4cm]
    { \Huge \bfseries #3}\\[0.15cm]
    \HRule \\[0.5cm]
    \textsc{\Large #4}\\[0.35cm]
    \vfill
    \begin{mdframed}[backgroundcolor=gray!20]
      \begin{center}
        #1
      \end{center}
    \end{mdframed}
  \end{titlepage}

  \pagestyle{fancy}
  \fancyhead[R]{\textbf{#5}}
  \fancyfoot[C]{\bfseries\thepage}
  \fancyhead[L]{\rightmark}

  \fancypagestyle{plain}{
    \fancyfoot[C]{\bfseries\thepage}
    \fancyhead[R]{}
    \fancyhead[L]{}
    \renewcommand{\headrulewidth}{0pt}
  }
}
\renewcommand{\sectionmark}[1]{\markright{#1}}
\renewcommand{\headrulewidth}{0.5pt}
\renewcommand{\footrulewidth}{0.5pt}

% \input{../Optikz.tex}

\pgfplotsset{table/search path={Plots}}
\newtheorem*{satz}{Satz}
%%%%
% \includeonly{05_Summenzeichen}
% \includeonly{06_Exp_Logarithmen}
% \includeonly{07_Trigonometrische_Funktionen}
% \includeonly{Skript/08_Differentialrechnung} 
% \includeonly{Skript/09_Vollständige_Induktion} 
% \includeonly{Skript/10_Reihen}
% \includeonly{Skript/11_binomischer_Satz}
% \includeonly{Skript/12_Vektoren_Matrizen}
% \includeonly{Skript/13_Integralrechnung}
%%%% 
 
\usetikzlibrary{tikzmark}  

\begin{document}
\graphicspath{{./Bilder/}}
\Deckblatt{Wintersemester 2023/24}
          {Mathematik - Ein Vorkurs für Studienanfänger}
          {Martin Beyer}
          {Mathematik} 
 
\thispagestyle{plain}  
\tableofcontents
\newpage  
 
\thispagestyle{plain}
\section{Grundrechnungsarten}

Wir beginnen bei den elementaren Regeln des Rechnes. Dazu gehören Addition, Subtraktion, sowie Multiplikation und Division reeller Zahlen. Wir führen die Bruchschreibweise, als auch Potenzen und Wurzeln ein. 

\subsection{Addition und Subtraktion}
algebraische Eigenschaften der Addition reeller Zahlen ($a,b,c \in \mathbb{R}$):
\begin{itemize}
    \item Kommutativität: $a+b = b+a$
    \item Assoziativität: $a+(b+c) = (a+b)+c = a+b+c$
\end{itemize}

Auf dem (reellen) Zahlenstrahl unterscheiden wir zwischen positiven und negativen Zahlen. Der \emph{Betrag} (einer Zahl ungleich Null) liefert immer eine positive Zahl. 

\begin{figure}[htp]
    \centering
    \begin{tikzpicture}
        \draw[thick,-{latex}] (-5.5,0) -- (5.5,0)node[right]{$\mathbb{R}$};
        \foreach \x in {-5,-4,...,5}{
            \node (A) at (\x,-.5){\x};
            \draw[thick] (\x,.15) -- (\x,-.15);
        }
        \draw[thick, dashed] (0,-.8) -- (0,-3);
        \node (A) at (-2.5,-1.5){``negative'' Zahlen};
        \node (A) at (2.5,-1.5){``positive'' Zahlen};
        \node (A) at (-2.5,-2.2){$a<0,\quad |a| = -a$};
        \node (A) at (2.5,-2.2){$a>0,\quad |a| = a$};
    \end{tikzpicture}
\end{figure}
Offenbar gilt für den Betrag bei Addition und Subtraktion 
\begin{align}
    |a|-|b| \le |a\pm b| \le |a|+|b|.
\end{align}
Die Addition von negativen Zahlen entspricht einer Subtraktion 
\begin{align}
    a+ (-b) = a-b, \quad (-a) + b = b-a \quad (-a)+(-b) = -(a+b).
\end{align}
Die Zahl \emph{Null} hat eine Sonderrolle. Sie ergibt sich bei der Addition einer Zahl $a$ mit ihrem additiven Inversen $-a$ und ist das neutrale Element der Addition:
\begin{align}
    a + (-a) = 0, \quad a + 0 = a.
\end{align}

\clearpage
\subsection{Multiplikation und Division}

algebraische Eigenschaften der Multiplikation reeller Zahlen ($a,b,c \in \mathbb{R}$):
\begin{itemize}
    \item Kommutativität: $ab = ba$ 
    \item Assoziativität: $a(bc) = (ab)c = abc$
    \item Distributivität: $a(b+c) = ab+ac$
\end{itemize}
Das Distributivitätsgesetz liefert die Rechenregeln zum \emph{Ausmultiplizieren} und \emph{Ausklammern} 
\begin{align}
    (a+b)\underbrace{(c+d)}_{f} = af+bf &= a(c+d) + b(c+d)\notag \\[-1em] 
    &= ac + ad + bc+ bd.
\end{align}

Für die Multiplikation mit negativen Zahlen gilt $``+''$ mal $``-'' = ``-''$ und $``-''$ mal $``-'' = ``+''$
\begin{align}
    (+a)(-b) = -(ab), \quad (-a)(+b)=-(ab), \quad (-a)(-b) = +(ab).
\end{align}
Alternativ lässt sich auch schreiben $-a = (-1)a$ und $(-1)(-1) = 1$. Die Zahl Null besitzt in der Multiplikation ebenfalls eine Sonderrolle: $a \cdot 0 = 0$.

Eine wichtige Reihe an Rechenregeln stellen die \emph{binomischen} Formeln dar.

\begin{mymathbox}[ams align, title={Binomische Formeln}, colframe={FSUblau}]
    \begin{split}\label{eqn:1_binomische_Formeln}
      (a+b)^2 &= a^2 + b^2 +2ab \\
      (a-b)^2 &= a^2 + b^2 -2ab \\
      (a+b)(a-b) &= a^2 - b^2.
    \end{split}
\end{mymathbox}. 
  
Wir können nun die Division als Umkehrung der Multiplikation einführen:
\begin{align}
    a = bc \Rightarrow b = \frac{a}{c}, \qq{es sei denn} c=0.
\end{align}
Dies führt dazu, dass nicht jede Multiplikation in eine Division überführt werden kann. Ebenso ist die Division mit Null nicht möglich: 

\begin{itemize}
    \item $\dfrac{a}{0}$ nicht definiert, da keine Zahl mit Null multipliziert $a$ ergibt 
    \item $\dfrac{0}{0}$ unbestimmt, da \emph{jede} Zahl mit Null multipliziert Null ergibt.
\end{itemize}

\newpage
\subsection{Bruchrechnung}

Wir haben gerade schon von der Notation eines Bruches Gebrauch gemacht, um die Division zweier Zahlen zu beschreiben. Ein Bruch besteht aus \emph{Zähler} (oben) und \emph{Nenner} (unten).
\begin{itemize}
    \item Multiplikation immer im Zähler: $k \dfrac{a}{b} = \dfrac{k}{1}\dfrac{a}{b} = \dfrac{ka}{b}$. 
    \item Divison immer im Nenner: $\dfrac{1}{k}\cdot \dfrac{a}{b} = \dfrac{a}{kb}$
    \item Kürzen: $\dfrac{k a}{kb} = \dfrac{k}{k} \dfrac{a}{b} = \dfrac{a}{b} \qq{da} \dfrac{k}{k}=1$
    \item Erweitern: $\dfrac{a}{b} = 1\cdot \dfrac{a}{b} = \dfrac{k}{k}\dfrac{a}{b} = \dfrac{ka}{kb}$.
\end{itemize}
Den \emph{Kehrwert} bzw. das Reziproke eines Bruches zu bilden, heißt:
\begin{align}
    \frac{a}{b} \longrightarrow \frac{b}{a}, \qq{sodass} \frac{a}{b}\cdot\frac{b}{a} = 1.
\end{align}
Wir können damit auch den Begriff der \emph{Mehrfachbrüche} einführen, d.\,h. 
\begin{align}
    \frac{a}{b} = \frac{1}{\frac{b}{a}} \qq{oder} \frac{\frac{a}{b}}{\frac{c}{d}} = \frac{a}{b} \frac{1}{\frac{c}{d}} = \frac{a}{b}\frac{d}{c} = \frac{ad}{bc}.
\end{align}
\vspace{-11mm}

\paragraph{Addition von Brüchen}$~$

Sofern die Nenner zweier Brüche gleich sind, können die Zähler addiert werden 
\begin{align}
    \frac{a}{n}+\frac{b}{n} = \frac{a+b}{n}.
\end{align}
Andernfalls wird zunächst der Hauptnenner gebildet 
\begin{align}
    \frac{a}{n}+\frac{b}{m} = \frac{a}{n} \cdot \frac{m}{m} + \frac{b}{m}\cdot \frac{n}{n} = \frac{am}{nm} + \frac{bn}{nm} = \frac{am + bn}{nm}.
\end{align}
Der Hauptnenner kann durch Multiplikation der beiden Nenner gebildet werden. Enthalten beide Nenner jeweils den gleichen Faktor, so kann der Hauptnenner einfacher gebildet werden, beispielsweise: 
\begin{align}
    \frac{2c-5b}{6ab - 10b^2} - \frac{5(2c-3a)}{18a^2 -30ab} &= \frac{2c-5b}{2b(3a-5b)} - \frac{5(2c-3a)}{6a(3a-5b)} \notag \\
    &= \frac{1}{2(3a-5b)} \qty[\frac{2c-5b}{b}-\frac{5(2c-3a)}{3a}] \notag \\
    &= \frac{1}{2(3a-5b)} \qty[\frac{3a(2c-5b) -5b(2c-3a)}{3ab}] \notag \\
    &= \frac{1}{6ab(3a-5b)}\qty(6ac -\cancel{15ab} -10bc +\cancel{15ab}) \notag \\
    &= \frac{1}{6ab\cancel{(3a-5b)}} 2c\cancel{(3a-5b)} = \uuline{\frac{c}{3ab}}.
\end{align}

\subsection{Potenzen und Wurzeln}

\subsubsection{Potenzen}
Potenzen drücken die mehrfache Ausführung eine Multiplikation in einer kompakten Schreibweise aus
\begin{align}
    \underbrace{a \cdot a \cdot a \cdot a \cdot \hdots \cdot a}_{n \text{ gleiche Faktoren}} = \tikzmarknode{eq1}{}a^n = b\tikzmarknode{eq2}{}.
\end{align}
\tikz[overlay,remember picture]{
\draw[shorten >=2pt,shorten <=2pt, thick, -{latex}] ($(eq1)+(0.5,-0.6)$)node[right]{Basis} -- ($(eq1)+(0.1,0)$);
\draw[shorten >=2pt,shorten <=2pt, thick, -{latex}] ($(eq1)+(0.7,0.8)$)node[right]{Exponent} -- ($(eq1)+(0.35,0.35)$);
\draw[shorten >=2pt,shorten <=2pt, thick, -{latex}] ($(eq2)+(0.9,0)$)node[right]{Potenzwert} -- ($(eq2)+(0.1,0.2)$);
}
Spezielle Werte des Potenzierens sind im Folgenden aufgelistet: 
\begin{align}
    a^0 &= 1, \quad 0^n = 0, \quad 1^n = 1. \\
    (-1)^n &= \begin{cases}
        1, & n \qq{gerade} \\
        -1, & n \qq{ungerade}
    \end{cases} \qq{oder:} \begin{cases}
        (-1)^{2n} = 1 \\
        (-1)^{2n+1} = -1.
    \end{cases}
\end{align}
Der Term $0^0$ ist hingegen nicht definiert, da der Wert durch Grenzwertbildung von verschiedenen Zahlenfolgen $\lim_{x\to 0} x^0 = 1$ bzw. $\lim_{x\to 0} 0^x = 0$ verschiedene Werte liefert. 

\paragraph{Potenzgesetze}$~$

Wir wollen im Folgenden die wichtigsten Potenzgesetze anschaulich herleiten: 
\begin{align}
    a^m \cdot a^n &= a^{m+n}, \qq{denn:} a^m \cdot a^n = \underbrace{\underbrace{(a\cdot a \cdot \hdots \cdot a)}_{m \text{ Faktoren}} \cdot \underbrace{(a\cdot a \cdot \hdots \cdot a)}_{n \text{ Faktoren}}}_{m+n \text{ Faktoren}}\label{eqn:01_Potenz1}\\
    a^n \cdot b^n &= (ab)^n, \qq{denn:} a^n \cdot b^n = (a \cdot a \cdot \hdots \cdot a)(b \cdot b \cdot \hdots b) = \underbrace{(ab)\cdot (ab) \cdot \hdots \cdot (ab)}_{n \text{ Paare}}\label{eqn:01_Potenz2} \\
    (a^m)^n &= (a^n)^m = a^{mn}, \qq{denn:} (a^m)^n = \underbrace{a^m \cdot a^m \cdot \hdots \cdot a^m}_{n \text{ Faktoren}} = \underbrace{a \cdot a \cdot \hdots \cdot a}_{m\cdot n \text{ Faktoren}}.\label{eqn:01_Potenz3}
\end{align}
Wir können zudem negative Exponenten einführen 
\begin{align}
    a^{-n} = \frac{1}{a^n} \qq{bzw.} a^n = \frac{1}{a^{-n}} = (a^{-n})^{-1} = (a^{-1})^{-n}.\label{eqn:01_Potenz_negativ}
\end{align}
Daraus können wir weitere Potenzgesetze ableiten 
\begin{align}
    \frac{a^m}{a^n} &= a^m (a^n)^{-1} \overset{\eqref{eqn:01_Potenz3}}{=} a^m \cdot a^{-n}  \overset{\eqref{eqn:01_Potenz1}}{=} a^{m-n} \\ 
    \frac{a^n}{b^n} &= a^n b^{-n} \overset{\eqref{eqn:01_Potenz3}}{=} a^n (b^{-1})^n  \overset{\eqref{eqn:01_Potenz2}}{=} (a\cdot b^{-1})^n  \overset{\eqref{eqn:01_Potenz_negativ}}{=} \qty(\frac{a}{b})^n \\
    \qty(\frac{a}{b})^{-n} &= (a \cdot b^{-1})^{-n} = a^{-n} \cdot b^n = (b \cdot a^{-1})^n = \qty(\frac{b}{a})^n.
\end{align}

\subsubsection{Wurzeln}
Möchte man die Gleichung $b^n = a$ nach $b$ lösen, so hat man die $n$-te Wurzel zu ziehen
\begin{align}
     \tikzmarknode{eq1}{}\sqrt[n]{a}= b\tikzmarknode{eq2}{}.
\end{align}
\tikz[overlay,remember picture]{
\draw[shorten >=2pt,shorten <=2pt, thick, -{latex}] ($(eq1)+(1.3,-0.5)$)node[right]{Radikand} -- ($(eq1)+(0.5,0)$);
\draw[shorten >=2pt,shorten <=2pt, thick, -{latex}] ($(eq1)+(-0.7,0.25)$)node[left]{Wurzelexponent} -- ($(eq1)+(0,0.25)$);
\draw[shorten >=2pt,shorten <=2pt, thick, -{latex}] ($(eq2)+(0.9,0.2)$)node[right]{Potenzwert} -- ($(eq2)+(0.1,0.2)$);
}

Der Ausdruck $\sqrt[n]{a}$ hat als Ergebnis die Zahl $b$, die in die $n$-te Potenz erhoben $a$ ergibt, $b^n = a$. Für den Fall $n=2$ lässt man den Wurzelexponenten weg: $\sqrt[2]{a} \equiv \sqrt{a}$.

Spezielle Werte des Wurzelziehens (für $n> 0$) sind im Folgenden aufgelistet: 
\begin{align}
    \sqrt[n]{0} = 0, \sqrt[n]{1} = 1, \sqrt[1]{a} = a.
\end{align}
Beachte, dass die Gleichung $b^0 = a$ keine Lösung für $b$ hat, wenn $a \neq=1$, da $b^0 = 1$, aber beliebig viele hat, wenn $a=1$.

\paragraph{Quadratwurzel}$~$

Die Quadratwurzel $\sqrt{a}$ ist im Reellen nicht definiert für $a<0$. Ist also $x^2 = a$ ($a\ge0$), dann folgt $\sqrt{x^2}= \sqrt{a}$ und daraus $|x| = \sqrt{a}$. Wir müssen also eine Fallunterscheidung treffen: 
\begin{align}
    \begin{split}
        x > 0: &\quad x = \sqrt{a} \\
        x < 0: &\quad x = -\sqrt{a}
    \end{split}
\end{align}
Es wäre hingegen falsch zu schreiben $\sqrt{9} = \pm 3$. Die Wurzel selbst ist positiv definiert.

\paragraph{Wurzelgesetze}$~$

Wir können die Wurzelgesetze über die Potenzschreibweise der Wurzeln auf die Potenzgesetze zurückführen. Es gilt 
\begin{align}
    \sqrt[n]{a} = a^{\frac{1}{n}}, \qq{denn:} \qty(\sqrt[n]{a})^n = \qty(a^{\frac{1}{n}})^n = a.
\end{align}

% Daraus folgen die Rechenregeln:
Es gilt weiterhin zu beachten, dass Summen von Wurzeln in der Regel nicht vereinfacht werden können 
\begin{align}
    \sqrt{a} + \sqrt{b} \neq \sqrt{a+b}.
\end{align}
Abschließend wollen wir diskutieren, wie wir Brüche mit Wurzeltermen vereinfachen können. Steht im Nenner des Bruchs ein Wurzelausdruck, so lässt sich dies durch ``Rationalmachen'' des Nenners vereinfachen: 
\begin{align}
    \frac{6}{2+\sqrt{2}} = \frac{6}{2+\sqrt{2}}\cdot\frac{2-\sqrt{2}}{2-\sqrt{2}} = \frac{6(2-\sqrt{2})}{4-2} = 3(2-\sqrt{2}).
\end{align}
Wir haben hierbei den Bruch geschickt erweitert, sodass wir die dritte binomische Formel $(a+b)(a-b) = a^2-b^2$ nutzen konnten.
\begin{mymathbox}[ams align, title={Potenzgesetze}, colframe={FSUblau}]
    \begin{split}
      a^m \cdot a^n = a^{m+n}, \quad a^n \cdot b^n &= (ab)^n, \quad (a^m)^n = (a^n)^m = a^{mn}\\
      \frac{a^m}{a^n} = a^{m-n}, \quad &\frac{a^n}{b^n} = \qty(\frac{a}{b})^n.
    \end{split}
\end{mymathbox}

\begin{mymathbox}[ams align, title={Wurzelgesetze}, colframe={FSUblau}]
    \begin{split}
      &\sqrt[n]{a}\sqrt[n]{b} = \sqrt[n]{ab}, \quad \sqrt[n]{a^n b} =  a \sqrt[n]{b}, \quad \qty(\sqrt[n]{a})^m = \sqrt[n]{a^m}, \quad \sqrt[m]{\sqrt[n]{a}} = \sqrt[mn]{a} \\
      &\sqrt[p]{a^m} \sqrt[q]{a^n} = \sqrt[pq]{a^{mq+np}}, \quad \frac{\sqrt[n]{a}}{\sqrt[n]{b}} = \sqrt[n]{\frac{a}{b}}, \quad \frac{\sqrt[p]{a^m}}{\sqrt[q]{a^n}} = \sqrt[pq]{a^{mq-np}}
    \end{split}
\end{mymathbox}

\subsection{Gleichungen}
Wir wollen nun noch allgemeine Eigenschaften von Gleichungen diskutieren, die wir für nachfolgende Kapitel als Grundlage benötigen. Eigenschaften des Gleichheitszeichens sind:
\begin{itemize}
    \item Reflexivität, d.\,h. es gilt $a = a$;
    \item Symmetrie, d.\,h. gilt $a= b$, dann gilt auch $b=a$;
    \item Transitivität, d.\,h. gilt $a=b$ und $b=c$, dann gilt auch $a=c$.
\end{itemize}
Gleichungen sind Darstellungen logischer Aussagen, deren Form mit Hilfe von \emph{Äquivalenzumformungen} manipuliert werden kann. Eine Äquivalenzumformung kann durch eine Umkehrung der Rechenoperation rückgängig gemacht werden. Außerdem bleibt die Lösungsmenge der Gleichung unverändert. Wir können beispielsweise auf beiden Seiten einer Gleichung Terme addieren oder subtrahieren
\begin{align}
    a+3 &= x^2 - 5 \notag \\
    \tikzmarknode{eq1}{}\Longleftrightarrow\qquad a&= x^2 - 8.
\end{align}
\tikz[overlay,remember picture]{
\draw[shorten >=2pt,shorten <=2pt, thick, -{latex}] ($(eq1)+(-1.2,0.6)$)node[left]{Äquivalenzpfeil} -- ($(eq1)+(-0.1,0.2)$);
}
Bei Multiplikation/Division einer Zahl muss jedoch sichergestellt werden, dass diese nicht Null ist \vspace{-3mm}
\begin{align}
    \begin{split}
        \frac{x}{a-b} &= z\hphantom{, a\neq b} \,\cancel{\Longleftrightarrow}\;\; x = z(a-b)\\
        \frac{x}{a-b} &= z, a\neq b \Longleftrightarrow\;  x = z(a-b).
    \end{split}
\end{align}
Wir müssen ebenfalls beachten, dass die Information über das Vorzeichen einer Zahl beim Quadrieren verloren geht. Es handelt sich deshalb nicht um eine Äquivalenzrelation 
\begin{align}
    \begin{split}
        x-1 &= a\hphantom{^2} \qq{hat eine Lösung,}\hphantom{en} x\hphantom{_1}=1+a \\
        (x-1)^2 &= a^2 \qq{hat zwei Lösungen,} x_1=1+a, x_2 =1-a.
    \end{split}
\end{align}
Weiterhin ist explizites Auflösen einer Gleichung nach einer Größe nicht immer möglich,
\begin{align}
    \qq{beispielsweise} x + \sin(x) = 0.
\end{align}  
\thispagestyle{plain}
\section{Lineare Gleichungssysteme}

In diesem Kapitel beschäftigen wir uns mit der Verknüpfung von zwei oder mehreren Gleichungen zu einem Gleichungssystem und diskutieren den Mengenbegriff.

Grundsätzlich können wir Gleichungen in vier Arten unterscheiden:
\begin{itemize}
    \item \emph{Identische Gleichungen} sind Darstellungen wahrer mathematischer Aussagen, z.\,B. $5+2=7$ oder $a\cdot b = b\cdot a$. 
    \item \emph{Funktionsgleichungen} stellen Zusammenhänge zwischen verschieden variablen Größen her; beispielsweise gehorcht der Flächeninhalt $A$ eines Kreises in Abhängigkeit des Radius $r$ der Gleichung $A(r) = \pi r^2$. 
    \item \emph{Definitionsgleichungen} ordnen mathematischen Ausdrücken eine Bezeichnung durch ein Symbol zu; man schreibt z.\,B. $z := 2x^2 +1$.
    \item \emph{Bestimmungsgleichungen} enthalten eine Variable, deren Wert grundsätzlich jede (reelle) Zahl sein kann. Die Menge aller Werte der Variable, für die eine Gleichung den Wahrheitswert ``wahr'' hat, heißt \emph{Lösungsmenge} $\mathbb{L}$ dieser Gleichung.
\end{itemize}
In diesem Abschnitt geht es um lineare Gleichungen, also Bestimmungsgleichungen, deren Variablen höchstens in erster Potenz auftreten.


\subsection{Mengen und Intervalle}

Eine \emph{Menge} ist eine Zusammenfassung von Objekten, die Elemente der Menge genannt werden. Ist ein Element $e$ in einer Menge $M$ enthalten, so schreibt man $e \in M$.

Mengen können mit Hilfe der Mengenklammer $\{\hdots\}$ definiert werden, überlicherweise geschieht dies über eine explizite Auflistung, bspw. $M=\{a,b,k,s\}$, oder durch Angabe einer definierenden Eigenschaft ihrer Elemente beispielsweise ist $M$
\begin{align}
    M = \{n |\, n \text{ ist eine gerade Zahl}\},
\end{align}
die Menge der geraden Zahlen. Mengen können ebenfalls Mengen als Elemente enthalten. Zum Beispiel enthält die Menge $N$ als Element die Menge $\{a,b\}$
\begin{align}
    N = \{1,\{a,b\}, 3\}.
\end{align}
Wichtige Mengen sind die Zahlenbereiche, insbesondere 
\begin{table}[htp]
    \centering
    \begin{tabular}{l l}
        die natürlichen Zahlen & $\mathbb{N} = \{1,2,3,\hdots\}$;\\
        die ganzen Zahlen & $\mathbb{Z} = \{\hdots,-3,-2,-1,0,1,2,3,\hdots\}$;\\
        die rationalen Zahlen & $\mathbb{Q} = \{\frac{p}{q} | p,q \in \mathbb{Z}, q\neq 0\}$;\\
        die reellen Zahlen & $\mathbb{R}$.
    \end{tabular}
\end{table}
Diese speziellen Mengen haben die besondere Eigenschaft, einer Ordnungsrelation zu unterliegen, d.\,h. man kann angeben, ob ein bestimmtes Element kleiner, größer oder gleich einem anderen Element ist. Dadurch ist es möglich, \emph{Intervalle} zu definieren.

\begin{figure}[htp]
    \centering
    \begin{tikzpicture}
        \node(A) at (-3,1){$a<b :$};
        \draw[very thick, -{latex}] (-2,1) -- +(4,0)node[right]{$x$};
        \draw[thick] (-1,1+0.1) -- +(0,-0.2)node[below]{$\vphantom{b}a$};
        \draw[thick] (1,1+0.1) -- +(0,-0.2)node[below]{$b$};
        \node(A) at (-3,0){$a=b :$};
        \draw[very thick, -{latex}] (-2,0) -- +(4,0)node[right]{$x$};
        \draw[thick] (0,0.1) -- +(0,-0.2)node[below]{$a=b$};
        \node(A) at (-3,-1){$a>b :$};
        \draw[very thick, -{latex}] (-2,-1) -- +(4,0)node[right]{$x$};
        \draw[thick] (1,-1+0.1) -- +(0,-0.2)node[below]{$\vphantom{b}a$};
        \draw[thick] (-1,-1+0.1) -- +(0,-0.2)node[below]{$b$};
    \end{tikzpicture}
\end{figure}

Im Folgenden wollen wir uns speziell auf die reellen Zahlen. Ein Intervall ist eine Teilmenge von $\mathbb{R}$, die durch die Angabe begrenzender Elemente gebildet wird. Es gibt zwei Möglichkeiten, eine Intervallgrenze $a\in \mathbb{R}$ aufzufassen:

\begin{itemize}
    \item Die Grenze $a$ ist Teil des Intervalls, $x \ge a$ oder $x \le a.$
    \begin{figure}[htp]
        \centering
        \begin{tikzpicture}
            \node(A) at (-3,1){$x\ge a $};
            \draw[very thick, -{latex}] (-2,1) -- +(4,0)node[right]{$x$};
            \draw[very thick, PAForange] (-1,1) -- +(2.74,0);
            \draw[very thick] (-0.9,1+0.3) -- (-1,1+0.3) -- ++(0,-0.6) -- ++(0.1,0) node[below]{$a\;$};
            \node(A) at (-3,0){$x \le a$};
            \draw[very thick, -{latex}] (-2,0) -- +(4,0)node[right]{$x$};
            \draw[very thick, PAForange] (-2,0) -- +(3,0);
            \draw[very thick] (0.9,0.3) -- (1,0.3) -- ++(0,-0.6) -- ++(-0.1,0) node[below]{$\;a$};
        \end{tikzpicture}
    \end{figure}
    \item Die Grenze $a$ ist nicht Teil des Intervalls, $x>a$ oder $x<a$. 
    \begin{figure}[htp]
        \centering
        \begin{tikzpicture}
            \node(A) at (-3,1){$x\ge a $};
            \draw[very thick, -{latex}] (-2,1) -- +(4,0)node[right]{$x$};
            \draw[very thick, PAForange] (-1,1) -- +(2.74,0);
            \draw[very thick] ($(0,1)+(160:1)$) arc (160:200:1) node[below]{$a\;$};
            \node(A) at (-3,0){$x \le a$};
            \draw[very thick, -{latex}] (-2,0) -- +(4,0)node[right]{$x$};
            \draw[very thick, PAForange] (-2,0) -- +(3,0);
            \draw[very thick] ($(0,0)+(-20:1)$)node[below]{$a\;$} arc (-20:20:1);
        \end{tikzpicture}
    \end{figure}
\end{itemize}

Damit lassen sich endliche Intervalle definieren als 
\begin{align}
    \begin{split}
        &\;[a,b] = \{x \in\mathbb{R} | a \le x \le b\} \qquad \text{abgeschlossenes Intervall} \\
        &\begin{rcases}
            [a,b) = \{x \in\mathbb{R} | a \le x < b\} \\
            (a,b] = \{x \in\mathbb{R} | a < x \le b\}   
        \end{rcases} \quad\;\text{halboffene Intervalle} \\
        &\;(a,b) = \{x \in\mathbb{R} | a < x < b\} \qquad \text{offenes Intervall} \\
    \end{split}
\end{align}
und unendliche Intervalle als 
\begin{align}
    \begin{split}
        [a,\infty) &= \{x\in\mathbb{R}| a \le x\} \hspace{6.2cm}\\
        (a,\infty) &= \{x\in\mathbb{R}| a < x\} \\
        (\minus\infty,b] &= \{x\in\mathbb{R}| x \le b\} \\
        (\minus\infty,b) &= \{x\in\mathbb{R}| x < b\}
    \end{split}
\end{align}
Weiterhin können wir bestimmte Teilmengen der reellen Zahlen definieren, beispielsweise 
\begin{align}
    \mathbb{R}^+ = (0,\infty), \quad \mathbb{R}^- = (\minus\infty,0) \quad \Rightarrow \quad \mathbb{R} = \mathbb{R}^- \cup \{0\} \cup \mathbb{R}^+ = (\minus\infty,\infty).
\end{align}
Wir müssen beachten, dass das $\infty$-Zeichen nur ein Symbol ist und kein Element der reellen Zahlen. Es gibt demnach keine unendlichen abgeschlossenen Intervalle.

Im Umgang mit Ungleichungen sind folgende Rechenregeln zu beachten: 
\begin{align}
    \begin{split}
        a < b &\Longleftrightarrow b>a, \\
        a < b &\Longleftrightarrow a+c < b+c,\\
        a < b \wedge c > 0 &\Longleftrightarrow ca < cb, 
    \end{split}
\end{align}
%% Potentielle Aufgabe: Goldener Schnitt

\thispagestyle{plain}
\section{Quadratische Gleichungssysteme}

In diesem Abschnitt geht es um quadratische Gleichungen, also Bestimmungsgleichungen, deren Variablen höchsten in zweiter Potenz auftreten, sowie um quadratische Funktionen und Systeme aus quadratischen Gleichungen.

\subsection{Die quadratische Gleichung}

Wir schauen uns zunächst die allgemeine Form einer quadratischen Gleichung an 
\begin{align}
    \tikzmarknode{eq1}{A} x^2 + \tikzmarknode{eq2}{B} x + \tikzmarknode{eq3}{C} =0.
\end{align}
\tikz[overlay,remember picture]{
\draw[shorten >=2pt,shorten <=2pt, thick, -{latex}] ($(eq1)+(-.4,-1)$)node[left]{quadratisches Glied} -- ($(eq1)+(0,-.2)$);
\draw[shorten >=2pt,shorten <=2pt, thick, -{latex}] ($(eq2)+(0,-1)$)node[below]{lineares Glied} -- ($(eq2)+(0,-.2)$);
\draw[shorten >=2pt,shorten <=2pt, thick, -{latex}] ($(eq3)+(.4,-1)$)node[right]{Absolutglied} -- ($(eq3)+(0,-.2)$);
}

\vspace{3mm}
Wir nehmen im Folgenden o.\,B.\,d.\,A. an, es sei $A >0$. Zudem sind $A,B$ und $C$ reelle Konstanten. Wir betrachten zunächst zwei Spezialfälle
\begin{alignat}{2}
    B &= 0 \qquad &&\qquad\; A x^2 + C = 0 \\[-3mm]
      &  &&\Rightarrow \qq{für} C \le 0: \qq{Lösungen} x_1 = \sqrt{-\frac{C}{A}}, x_2 = -\sqrt{-\frac{C}{A}} \notag \\
      & &&\Rightarrow \qq{für} C >0: \qq{keine (reellen) Lösungen} \notag\\
    C &= 0 \qquad &&\qquad\; A x^2 + Bx = 0 \\
      & && \qquad\; x(Ax + B) = 0 \notag \\
      & &&\Rightarrow \qq{Lösungen} x_1 = 0, x_2 = -\frac{B}{A}. \notag %\hspace{3.85cm}
\end{alignat}

Zur Bestimmung einer allgemeinen Lösung verwenden wir die \emph{Methode der quadratischen Ergänzung}:
\begin{enumerate}
    \item Überführung in die \emph{Normalform} mit Umbenennung $p \equiv \frac{B}{A}, q \equiv \frac{C}{A}$
    \begin{align}
        x^2 + \frac{B}{A} x + \frac{C}{A} = 0\qq{,} x^2 + px + q = 0;
    \end{align}
    \item quadratische Ergänzung: \vspace{-1.1cm}
    \begin{align}
        x^2 + px &= -q \hphantom{+\qty(\frac{p}{2})^2} \bigg| +\qty(\frac{p}{2})^2 \\
        \Rightarrow x^2 + px +\qty(\frac{p}{2})^2 &= -q +\qty(\frac{p}{2})^2 \notag \\
        \qty(x+\frac{p}{2})^2 &= - q +\qty(\frac{p}{2})^2.
    \end{align}
    \item Auflösen nach $x$ 
    \begin{align}
        x + \frac{p}{2} = \pm \sqrt{\qty(\frac{p}{2})^2 - q}.
    \end{align}
\end{enumerate}
\begin{mymathbox}[ams align, title={$p$-$q$-Lösungsformel}, colframe={FSUblau}]
    x^2 + px + q = 0 \quad \Rightarrow \quad x_{1/2} = -\frac{p}{2} \pm \sqrt{\qty(\frac{p}{2})^2 - q}.
\end{mymathbox}
Man bezeichnet $D \equiv \qty(\frac{p}{2})^2 - q$ als \emph{Diskriminante}. Anhand ihres Vorzeichens können folgende Fälle auftreten:
\begin{itemize}
    \item $D > 0$, es existieren zwei reelle Lösungen 
    \item $D = 0$, beide Lösungen fallen zusammen $x_1 = x_2 = -\frac{p}{2}$ 
    \item $D < 0$, es existiert keine (reelle) Lösung.
\end{itemize}
Betrachten wir für den Fall $D \ge 0$ die Summe und das Produkt der beiden allgemeinen Lösungen,
\begin{align}
        x_1 + x_2 &= -\frac{p}{2} + \cancel{\sqrt{\qty(\frac{p}{2})^2 - q}} - \frac{p}{2} - \cancel{\sqrt{\qty(\frac{p}{2})^2 - q}} = -p \\ 
        x_1 \cdot x_2 &= \qty(-\frac{p}{2}+ \sqrt{\qty(\frac{p}{2})^2 - q})\qty(-\frac{p}{2}- \sqrt{\qty(\frac{p}{2})^2 - q}) \notag \\
        \overset{\eqref{eqn:1_binomische_Formeln}}&{=} \frac{p^2}{4} - \qty(\sqrt{\qty(\frac{p}{2})^2 - q})^2 = \cancel{\frac{p^2}{4}} - \cancel{\frac{p^2}{4}} + q = q, 
\end{align}
dann folgt damit der \emph{Vieta'sche Wurzelsatz}\footnote{Wurzeln sind eine Bezeichnung für die Nullstellen eines Polynoms.} für die Lösungen quadratischer Gleichungen, 
\begin{align}
    x_1 + x_2 = -p \qq{,} x_1 x_2 = q, \qq{Vieta'scher Wurzelsatz.}
\end{align}
Setzen wir dieses Resultat in die Normalform ein, so ergibt sich 
\begin{align}
        x^2 + px + q &= x^2 - (x_1 + x_2) x + x_1 x_2 \notag \\
        &= (x-x_1) (x-x_2), 
\end{align}
also die Zerlegung in Linearfaktoren.

\clearpage
\subsection{Quadratische Funktionen}

Für quadratische Funktionen lautet die allgemeine Form 
\begin{align}
    f(x) = Ax^2 + Bx + C \qq{,} A,B,C \in \mathbb{R}.
\end{align}
Wir nehmen wieder o.\,B.\,d.\,A. $A > 0$ an. Offenbar enspricht das Lösen einer quadratischen Gleichung der Suche nach den Nullstellen einer quadratischen Funktion (und umgekehrt). Betrachten wir zunächst den Fall $B = C = 0$: 

\begin{minipage}{0.45\textwidth}
    \begin{itemize}
        \item Definitionsbereich $x \in \mathbb{R}$ \\
        Wertebereich $y \ge 0$;
        \item $A = 1$: \emph{Normalparabel} 
        \item $A > 1$: gestauchte Normalparabel ;
        \item $A < 1$: gestreckte Normalparabel;
        \item doppelte Nullstele bei $x_{1/2} = 0.$
    \end{itemize}
\end{minipage}
\begin{minipage}{0.55\textwidth}
        \centering
        \begin{tikzpicture}
            \begin{axis}[disabledatascaling, axis lines=middle, xlabel={$x$}, ylabel={$y$},xtick={-3,-2,-1,1,2,3}, ytick={1,2,3,4,5,6,7,8,9}, height=8cm, width=\textwidth, xmin=-3.5, xmax=4, ymin=-1.5, ymax = 9.9]
                \addplot[no marks, FSUblau, thick, domain=-3.5:3]{x^2}node[right]{$A=1$};
                \addplot[no marks, FSUblau, dashed, thick, domain=-3.5:0]{1.4*x^2};
                \addplot[no marks, FSUblau, dashed, thick, domain=-3.5:0]{0.7*x^2};
                \draw[dashed] (1,0) -- (1,1) -- (0,1);
                \draw[dashed] (2,0) -- (2,4) -- (0,4);
                \draw[dashed] (3,0) -- (3,9) -- (0,9);
                \node(A) at (-1.5,8){$A>1$};
                \node(A) at (-2.5,2){$A<1$};
            \end{axis}
        \end{tikzpicture}
\end{minipage}

Durch Addition der Konstante $y_0$ verschiebt sich die Parabel entlang der $y$-Achse. Für $y_0 > 0$ bzw. $y_0 <0$ existieren keine bzw. zwei Nullstellen.

Durch die Substitution $x \mapsto x+x_0$ verschiebt sich die $y$-Achse um den Wert $x_0$, bzw. die Parabel um $\minus x_0$ entlang der $x$-Achse. Wir erhalten damit die quadratische Funktion
\begin{align}
    f(x) = A(x+x_0)^2 + y_0 \qq{Scheitelpunktform,}
\end{align}
wobei\quad$x_0 > 0$: Verschiebung der $y$-Achse (Parabel) nach rechts (links) \\
\hphantom{wobei}\quad$x_0 < 0$: Verschiebung der $y$-Achse (Parabel) nach links (rechts).

\begin{wrapfigure}{r}{6cm}
    \centering
    \vspace{-5mm}
        \begin{tikzpicture}
            \begin{axis}[disabledatascaling, axis lines=middle, xtick={1}, xticklabels={$-x_0$}, ytick={-2}, yticklabels={$y_0$}, xlabel={$x$}, ylabel={$y$}, height=6cm, width=6cm, ymin = -2.3]
                \addplot[no marks, FSUblau, thick, domain=-1:3] {(x-1)^2-2};
                \draw[thick, dashed] (0,-2) -- (1,-2) -- (1,0);
            \end{axis}
        \end{tikzpicture}
    \vspace{-5mm}
\end{wrapfigure}

Vergleichen wir mittels quadratischer Ergänzung die Scheitelpunktform mit der Normalform, so ergibt sich 
\begin{align}
    x_0 = \frac{B}{2A} \qq{,} y_0 = C -A x_0^2.
\end{align}
Das heißt, dass jede quadratische Funktion durch geeignete Verschiebung des Koordinatensystems auf eine Parabel der Form $f(x) = A x^2$ zurückgeführt werden kann.

\clearpage
\subsection{Quadratische Gleichungssystem mit zwei Unbekannten}

Die allgemeine Form eines quadratischen Gleichungssystems mit zwei Unbekannten Größen $x,y$ lautet 
\begin{align}
    a_i x^2 + b_i y^2 + c_i xy + d_i x + e_i y + f_i = 0.
\end{align}
Die allgemeine ist jedoch nur umständlich zu diskutieren und es existieren viele Lösungsmöglichkeiten. Daher beschränken wir uns hier auf zwei Beispiele. 

\paragraph{Beispiel 1}
\begin{subequations}
    \begin{align}
        x^2 +y^2 - 2x &= \frac{11}{2}, \label{eqn:3_QGS_Bsp1a}\\
        2xy - 2y &= \frac{5}{2}\label{eqn:3_QGS_Bsp1b}
    \end{align}
\end{subequations}
Wir addieren nun $\eqref{eqn:3_QGS_Bsp1a} \pm \eqref{eqn:3_QGS_Bsp1b}$ und erhalten 
\begin{align}
    2\underbrace{(x^2 + y^2 \pm 2xy)}_{(x\pm y)^2} - 4(x\pm y) = 11 \pm 5.
\end{align}
Wir substituieren nun $u \equiv x+y$ und $v \equiv x-y$, damit haben wir zwei voneinander unabhängige quadratische Gleichungen 
\begin{align}
    \begin{split}
        u^2 - 2u -8 &= 0\\
        v^2 - 2v -3 &= 0
    \end{split} \qq{mit den Lösungen} \begin{split}
        u_{1/2} &= 1 \pm \sqrt{1+8} \quad \Rightarrow \quad u_1 = 4, u_2 = -2 \\
        v_{1/2} &= 1 \pm \sqrt{1+3} \quad \Rightarrow \quad v_1 = 3, v_2 = -1.
    \end{split}
\end{align}

Resubstituieren wir jetzt nun $x = \frac{1}{2}(u+v), y = \frac{1}{2}(u-v)$, dann existieren vier Lösungspaare für $(x,y)$, da vier Paare $(u_i, v_j)$ beildet werden können 
\begin{align}
    \begin{split}
        x_1 &= \frac{1}{2}(u_1+v_1) = \hphantom{\minus}\frac{7}{2} \qq{,} y_1 = \frac{1}{2}(u_1-v_1) = \frac{1}{2};\\
        x_2 &= \frac{1}{2}(u_1+v_2) = \hphantom{\minus}\frac{3}{2} \qq{,} y_2 = \frac{1}{2}(u_1-v_2) = \frac{5}{2};\\
        x_3 &= \frac{1}{2}(u_2+v_1) = \hphantom{\minus}\frac{1}{2} \qq{,} y_3 = \frac{1}{2}(u_2-v_1) = \minus\frac{5}{2};\\
        x_4 &= \frac{1}{2}(u_2+v_2) = \minus\frac{3}{2}             \qq{,} y_4 = \frac{1}{2}(u_2-v_2) = \minus\frac{1}{2};
    \end{split}
\end{align}
\begin{align}
    \Rightarrow \mathbb{L} = \qty{\qty(\frac{7}{2};\frac{1}{2}),\qty(\frac{3}{2};\frac{5}{2}),\qty(\frac{1}{2};\minus\frac{5}{2}),\qty(\minus\frac{3}{2};\minus\frac{1}{2})}.
\end{align}
Im Zweifelsfall (``Sind wirklich alle Kombinationen erlaubt?'') sollte die Probe durchgeführt werden. 

\emph{Bemerkung:} Eine geometrische Interpretation ist auch hier möglich: Gleichung~\eqref{eqn:3_QGS_Bsp1a} beschreibt einen Kreis des Radius $\sqrt{\frac{13}{2}}$, dessen Mittelpunkt um 1 nach rechts verschoben ist, da die Gleichung in der Form $\textstyle (x-1)^2 + y^2 = \frac{13}{2}$. Dahingegen lässt sich \eqref{eqn:3_QGS_Bsp1b} als Funktion $y = \frac{5}{4}\frac{1}{x-1}$ schreiben. Die Schnittpunktkoordinaten der Hyperbeln mit dem Kreis sind die Lösungspaare. 
\begin{figure}[htp]
    \centering
    \begin{tikzpicture}
        \begin{axis}[axis equal, disabledatascaling, axis lines=middle, xlabel={$x$}, ylabel={$y$}, width=0.7\textwidth, ymin=-3.5, ymax=3.5]
            \addplot[no marks, FSUblau, samples=100, thick, domain=1.1:4]{5/4*1/(x-1)};
            \addplot[no marks, FSUblau, samples=100, thick, domain=-2:0.9]{5/4*1/(x-1)};
            \draw[thick] (1,0) circle (2.55);
            \draw[dashed] (1,4) -- (1,-4);
            \fill (7/2,1/2) circle (2pt) node[above right]{$\qty(\frac{7}{2};\frac{1}{2})$};
            \fill (3/2,5/2) circle (2pt) node[above right]{$\qty(\frac{3}{2};\frac{5}{2})$};
            \fill (-3/2,-1/2) circle (2pt) node[below left]{$\qty(\minus\frac{3}{2};\minus\frac{1}{2})$};
            \fill (1/2,-5/2) circle (2pt) node[above right]{$\qty(\frac{1}{2};\minus\frac{5}{2})$};
        \end{axis}
    \end{tikzpicture}
    \caption{Grafische Lösung des quadratischen Gleichungssystems.}
\end{figure}

\paragraph{Beispiel 2}$~$
\begin{align}
    \begin{split}     
        x + y^2 &= 2, \\
        x y^2 &= -8 
    \end{split} 
    \qquad\Rightarrow\qq{Substitution} y^2 \equiv z 
    \begin{split}
        x+ z &= 2, \\
        xz &= -8
    \end{split}\label{eqn:3_QGS_Bsp2}
\end{align}
Der Satz von Vieta angewandt auf das Gleichungssystem~\eqref{eqn:3_QGS_Bsp2} zeigt, dass $x$ und $z$ Lösungen einer qudratischen Gleichung mit $p = -2$ und $q = -8$ sind, also der Gleichung 
\begin{align}
    u^2 + pu +q = u^2 -2u - 8 = 0 \quad \Rightarrow \quad u_{1/2} = 1 \pm \sqrt{9} = \begin{cases}
        4 \\ -2.
    \end{cases}
\end{align}
Wir könnten nun $u_1 = x, u_2 =z$ oder $u_1 = z, u_2 = x$ identifizieren, allerdings muss $z = y^2$ positiv sein. wir erhalten somit die Lösungen 
\begin{align}
    x= -2, y = \pm 2 \quad \Rightarrow \quad \uuline{\mathbb{L} = \{(-2;2),(-2,-2)\}}.
\end{align}
\section{Umgang mit beliebigen Potenzen}

Bisher haben wir uns auf Gleichungen/Funktionen beschränkt, deren Variablen höchstens in erster oder zweiter Potenz aufgetreten sind. Nun sollen Methoden zum Umgang mit beliebigen (ganzzahligen) Potenzen behandelt werden. 

\subsection{Polynome und Polynomdivision}

Ein \emph{Polynom} $n$\emph{-ten Grades} ist eine Funktion der Form 
\begin{align}
    f_n(x) = a_n x^n + a_{n-1} x^{n-1} + \hdots + a_2 x^2 + a_1 x + a_0
\end{align}
mit den reellen Konstanten $a_i,\; i = 0,1,\hdots,n$. Die Nullstellen der Funktion $f_n(x)$ werden auch \emph{Wurzeln} des Polynoms genannt; ein Polynom $n$-ten Grades besitzt höchstens $n$ reelle Wurzeln. 

Besitzt ein Polynom $f_n(x)$ genau $n$ reelle Wurzeln, dann kann es als Produkt von Linearfaktoren geschrieben werden (vergleiche den Fall $n=2$ mit dem Satz von Vieta), 
\begin{align}
    f_n(x) = a_n (x-x_1)(x-x_2)\hdots (x-x_{n-1})(x-x_n),
\end{align}
mit Wurzeln $x_i, \; i=1,2,\hdots,n$. Demnach kann eine Wurzel gemäß $f_n(x) = (x-x_n)f_{n-1}(x)$ aus einem Polynom abgespalten werden, wobei $f_{n-1}(x)$, bei bekanntem $x_n$, mit Hilfe der \emph{Methode der Polynomdivision} zu bestimmen ist,
\begin{align}
    f_{n-1}(x) = f_n(x) : (x-x_n).
\end{align}
Möchte man einen unbekannten Linearfaktor abspalten, so ist ein $x_i$ zu erraten.

Wir wollen jetzt das Verfahren der Polynomdivision am Beispiel folgender Funktion diskutieren: 
\begin{align}
    f_3(x) = x^3 - 5x^2 + 8x -4, \qquad x_3 = 2.
\end{align}
Der Algorithmus für die schriftliche Polynomdivision besteht aus drei Schritten:
\begin{enumerate}
    \item Division: Man dividiere das Glied der höchsten Potenz des Zählerpolynoms durch das Glied der höchsten Potenz des Nennerpolynoms und schreibt das Ergebnis neben dem Gleichheitszeichen auf. 
    \item Multiplikation: Man multipliziert das Ergebnis von Schritt 1 mit dem Nennerpolynom und schreibt das Ergebnis unter das Zählerpolynom. 
    \item Subtraktion: Man subtrahiert das Ergebnis von Schritt 2 vom Zählerpolynom und beginnt wieder von vorne.
\end{enumerate}
\begin{align}
    \qq{Ergebnis: }\begin{array}{r@{} r@{} r@{} r r}
        x^3 &{}-5x^2\hphantom{)}&{}+8x&-4\hphantom{)} &:(x-2) = x^2 -3x +2 \\
      -(x^3 &{}-2x^2) &\\
      \cmidrule{1-2}
            & -3x^2\hphantom{)} &{}+8x&-4\hphantom{)}\\
            -&(-3x^2\hphantom{)} &{}+6x&\hphantom{-4}) \\
      \cmidrule{2-4}
            & &{}2x&{}-4\hphantom{)}\\
            & &-(2x&{}-4)\\
      \cmidrule{3-4} 
            & & & 0
    \end{array}
\end{align}

Die Nullstellen des Restpolynoms $x^2 -3x +2$ erhalten wir mittels $p$-$q$-Fromel: 
\begin{align}
    x_{1/2} = \frac{3}{2} \pm \sqrt{\frac{9}{4}-2} = \frac{3}{2} \pm \frac{1}{2} \quad \Rightarrow \quad x_1 = 2, x_2 =1.
\end{align}
Damit lautet die Linearfaktorzelegung des Polynoms  
\begin{align}
    \uuline{f_3(x) = (x-1) (x-2)^2} \textcolor{gray}{=\underbrace{ (x-1)(x^2-4x+4) = x^3-5x^2 + 8x-4}_{\text{Probe}}}.
\end{align}

\subsection{Partialbruchzerlegung}

Oft möchte man in einem Nenner auftretende Polynome zugunsten von Linearfaktoren zerlegen. Dies ist insbesondere bei der Berechnung von Integralen hilfreich. Wir betrachten das Verfahren anhand des Quotienten einer linearen und einer quadratischen Funktion: 
\begin{align}
    f(x) = \frac{mx + n}{x^2 +px + q} = \frac{mx +n}{(x-x_1)(x-x_2)} \tikzmarknode{eq1}{\overset{!}{=}} \frac{\alpha}{x-x_1} + \frac{\beta}{x-x_2}.
\end{align}
\tikz[overlay,remember picture]{
\draw[shorten >=2pt,shorten <=2pt, thick, -{latex}] ($(eq1)+(0,-.8)$)node[below]{Ansatz} -- ($(eq1)+(0,-.2)$);
}

Die Koeffizienten $\alpha$ und $\beta$ sind nun so zu bestimmen, dass die Forderung erfüllt ist. Wir multiplizieren das in Linearfaktoren zerlegte Polynom auf die rechte Seite 
\begin{align}
    mx + n &= \qty(\frac{\alpha}{x-x_1} + \frac{\beta}{x-x_2}) \cdot (x-x_1)(x-x_2) = \alpha (x-x_2) + \beta(x-x_1) \notag \\
    &= (\alpha + \beta) x - (\alpha x_2 + \beta x_1).
\end{align}
Im zweiten Schritt führen wir einen Koeffizientenvergleich durch nach den Potenzen von $x$ um ein lineares Gleichungssystem für $\alpha$ und $\beta$ zu erhalten
\begin{subequations}
    \begin{align}
        x^1: \quad m &= \alpha + \beta \\
        x^0: \quad \;\,n &= -(\alpha x_2 + \beta x_1).
    \end{align}
\end{subequations}
Dieses lösen wir nun mit der Additionsmethode durch Elimination von $\beta$ 
\begin{align}
    m x_1 + n = \alpha (x_1 - x_2) \quad \Rightarrow \quad \alpha &= \frac{m x_1 + n}{x_1 -x_2} \qq{,} x_1 \neq x_2. \\
    \Rightarrow \quad \beta &= -\frac{m x_2 + n}{x_1 -x_2}.
\end{align}
Offenbar versagt dieser Ansatz für $x_1 = x_2$, wenn also der Nenner eine doppelte Nullstelle hat. In diesem Fall jedoch hat $f(x)$ bereits die gewünschte Form, 
\begin{align}
    f(x) = \frac{mx + n}{(x-x_0)^2} \qq{,} x_0 \equiv x_1 = x_2.
\end{align}
Im allgemeinen Fall haben wir den Quotienten aus einem Polynom $p$-ten Grades und einem Polynom $q$-ten Grades $(p < q)$ zu zerlegen\footnote{Für $p\ge q$ kann eine Polynomdivision durchgeführt werden.}. Dann sind zunächst die Nullstellen des Nenners $x_i$ zu bestimmen; der Ansatz enthält für jede einfache Nullstelle einen Summanden 
\begin{align}
    \frac{\alpha_i}{x-x_i} \qq{,} \alpha_i \qq{const.,}
\end{align}
und für jede $k$-fache Nullstelle $k$ Summanden, einen für jede mögliche Potenz zwischen $1$ und $k$, 
\begin{align}
    \frac{\alpha_i^{(1)}}{x-x_i} + \frac{\alpha_i^{(2)}}{(x-x_i)^2} + \hdots + \frac{\alpha_i^{(k)}}{(x-x_i)^k} \qq{,} \alpha_i^{(j)} \qq{const.}
\end{align} 

\paragraph{Beispiel} Betrachten wir nun folgenden Bruch mit Polynomen 
\begin{align}
    Q(x) = \frac{3x^2 + 5}{(x+1)(x-1)^2} \overset{!}&{=} \frac{\alpha}{x+1} + \frac{\beta}{x-1} + \frac{\gamma}{(x-1)^2} \\
    \Rightarrow \quad 3x^2 +5 &= \alpha (x-1)^2 + \beta (x^2-1) + \gamma (x+1) \notag \\
                              &= (\alpha + \beta) x^2 + (\gamma - 2\alpha) x + \alpha -\beta +\gamma.
\end{align}
Wir führen nun den Koeffizientenvergleich durch 
\begin{align}
    \begin{split}
        x^2: \quad 3 &= \alpha + \beta  \hphantom{+\gamma}\;\,\quad(1)\\
        x^1: \quad 0 &= \gamma - 2\alpha \hphantom{+}\;\;\quad(2)\\
        x^0: \quad 5 &= \alpha -\beta + \gamma \quad (3)
    \end{split}
    \qquad\Rightarrow \quad (1)+(2)+(3): \quad 2 \gamma = 8 \quad \Rightarrow 
    \begin{split}
        \gamma &= 4 \\
        \alpha &= 2 \\
        \beta &= 1.
    \end{split} 
\end{align}
Damit folgt als Ergebnis 
\begin{align}
    \uuline{Q(x) = \frac{2}{x+1} + \frac{1}{x-1} + \frac{4}{(x-1)^2}}.
\end{align}

\paragraph{Bemerkung:} Jede rationale Funktion, d.\,h. eine Funktion, die als Quotient zweier Polynome geschrieben werden kann, lässt sich als Summe von Brüchen der Form $\frac{\alpha_i}{(x-x_i)^j}$ sowie gegebenenfalls einer ``reinen'' Polynomfunktion darstellen. 

\paragraph{Bemerkung} Sind Zähler und Nenner vom gleichen Grade, ist dem Ansatz ein konstanter Summand hinzuzufügen. Alternativ kann zunächst auch eine Polynomdivision mit Rest durchgeführt werden; der Rest ist dann ein Bruch mit kleinerem Zähler- als Nennergrad, sodass ``normal'' weitergerechnet werden kann.


\subsection{Potenzfunktionen}

Potenzfunktionen sind funktionen der Form $f(x) = a x^n$, wobei hier nur die Fälle $a \in \mathbb{R}, a > 0, n \in \mathbb{Z}$ betrachtet werden sollen. 

\paragraph{1.) Parabeln $n$-ter Ordnung: $n > 0$}$~$

\begin{minipage}{0.5\textwidth}
        \begin{tikzpicture}
            \begin{axis}[disabledatascaling, axis lines=middle, xtick={-1,1}, ytick={1}, yticklabels={$a$}, xlabel={$x$}, ylabel={$y$}, height=7cm, width=\textwidth, ymin=-1, ymax = 3, samples=100, legend pos = north west]
                \addplot[no marks, FSUblau, thick, domain=-1.5:1.5]{x^2};
                \addplot[no marks, PAForange, thick, domain=-1.5:1.5]{x^4};
                \draw[dashed, opacity=0.6] (1,0) -- (1,1) -- (-1,1) -- (-1,0);
                \legend{$a x^2$, $a x^4$};
            \end{axis}
        \end{tikzpicture}
        \begin{itemize}
            \item Definitionsbereich: $x \in \mathbb{R}$ 
            \item Wertebereich: $y \in [0, \infty)$
            \item gerade Funktion: $f(-x) = f(x)$
            \item Axialsymmetrie zur $y$-Achse
        \end{itemize}
\end{minipage}
\begin{minipage}{0.5\textwidth}
    \begin{tikzpicture}
        \begin{axis}[disabledatascaling, axis lines=middle, xtick={-1,1}, ytick={-1,1}, yticklabels={$-a$,$a$}, xlabel={$x$}, ylabel={$y$}, height=7cm, width=\textwidth, ymin=-2, ymax = 2, samples=100, legend pos = north west]
            \addplot[no marks, FSUblau, thick, domain=-1.5:1.5]{x^3};
            \addplot[no marks, PAForange, thick, domain=-1.5:1.5]{x^5};
            \draw[dashed, opacity=0.6] (1,0) -- (1,1) -- (0,1);
            \draw[dashed, opacity=0.6](-1,0) -- (-1,-1) -- (0,-1);
            \legend{$a x^3$, $a x^5$};
        \end{axis}
    \end{tikzpicture}
    \begin{itemize}
        \item Definitionsbereich: $x \in \mathbb{R}$ 
        \item Wertebereich: $y \in \mathbb{R}$
        \item ungerade Funktion: $f(-x) = -f(x)$
        \item Punktsymmetrie zum Ursprung
    \end{itemize}
\end{minipage}
\paragraph{2.) Hyperbeln $n$-ter Ordnung: $n < 0$}$~$

\begin{minipage}{0.5\textwidth}
    \begin{tikzpicture}
        \begin{axis}[disabledatascaling, axis lines=middle, xtick={-1,1}, ytick={1}, yticklabels={$a$}, xlabel={$x$}, ylabel={$y$}, height=7cm, width=\textwidth, ymin=-1, ymax = 3, samples=100, legend pos = north west]
            \addplot[no marks, FSUblau, thick, domain=-3:-0.4]{1/x^2};
            \addplot[no marks, PAForange, thick, domain=-3:-0.4]{1/x^4};
            \addplot[no marks, FSUblau, thick, domain=0.4:3]{1/x^2};
            \addplot[no marks, PAForange, thick, domain=0.4:3]{1/x^4};
            \draw[dashed, opacity=0.6] (1,0) -- (1,1) -- (-1,1) -- (-1,0);
            \legend{$a x^{-2}$, $a x^{-4}$};
        \end{axis}
    \end{tikzpicture}
    \begin{itemize}
        \item Definitionsbereich: $x \in \mathbb{R}\backslash \{0\}$ 
        \item Wertebereich: $y \in (0, \infty)$
        \item gerade Funktion: $f(-x) = f(x)$
        \item Axialsymmetrie zur $y$-Achse
    \end{itemize}
\end{minipage}
\begin{minipage}{0.5\textwidth}
\begin{tikzpicture}
    \begin{axis}[disabledatascaling, axis lines=middle, xtick={-1,1}, ytick={-1,1}, yticklabels={$-a$,$a$}, xlabel={$x$}, ylabel={$y$}, height=7cm, width=\textwidth, ymin=-2, ymax = 2, samples=100, legend pos = north west]
        \addplot[no marks, FSUblau, thick, domain=-3:-0.4]{1/x};
            \addplot[no marks, PAForange, thick, domain=-3:-0.4]{1/x^3};
            \addplot[no marks, FSUblau, thick, domain=0.4:3]{1/x};
            \addplot[no marks, PAForange, thick, domain=0.4:3]{1/x^3};
        \draw[dashed, opacity=0.6] (1,0) -- (1,1) -- (0,1);
        \draw[dashed, opacity=0.6](-1,0) -- (-1,-1) -- (0,-1);
        \legend{$a x^{-1}$, $a x^{-3}$};
    \end{axis}
\end{tikzpicture}
\begin{itemize}
    \item Definitionsbereich: $x \in \mathbb{R}\backslash\{0\}$ 
    \item Wertebereich: $y \in \mathbb{R}\backslash\{0\}$
    \item ungerade Funktion: $f(-x) = -f(x)$
    \item Punktsymmetrie zum Ursprung
\end{itemize}
\end{minipage}
\section{Das Summenzeichen}

Für viele Anwendungen erweist es sich als praktisch, sehr lange (oder auch unendliche) Summen kompakt aufzuschreiben. Betrachten wir eine Summe $S$ aus $n$ Summanden, 
\begin{align}
    S = s_1 + s_2 + s_3 + \hdots + s_n,
\end{align}
wobei jeder Summand mit einem \emph{Index} gekennzeichnet ist, 
\begin{align}
    s_i \qq{,} i=1,2,\hdots,n.
\end{align}
Die Indizes dienen zunächst nur dazu, die Summanden zu unterscheiden und sind im Allgemeinen willkürlich gesetzt - schließlich hängt der Wert der Summe nicht von der Summationsreihenfolge ab. Als Kurzschreibweise für Summen verwendet man den großen griechischen Buchstaben Sigma:
\begin{align}
    S = \sum_{i=1}^n s_i \qq{, mit Startwert $i=1$ und Endwert $i=n$.} 
\end{align}

\paragraph{Beispiel: Polynome}$~$

Hier sind die Summanden die jeweiligen Potenzen von $x$ mit ihren Vorfaktoren und es bietet sich an, die Potenzen als Indizes zu benutzen: 
\begin{align}
    P_n(x) &= a_0 + a_1 x^1 + a_2 x^2 + \hdots + a_{n-1} x^{n-1} + a_n x^n  = \sum_{i=0}^n a_i x^i.
\end{align}

\paragraph{Beispiel: Summe der ersten $n$ Zahlen}$~$

Hier bietet sich an, die Zahlen selbst als Indizes zu benutzen: 
\begin{align}
    \sum_{k=1}^n k = 1 + 2 + 3 + \hdots + n-1 +n.
\end{align}

\paragraph{Eigenschaften:}$~$

\begin{itemize}
    \item Die Benennung des Summationsindex ist irrelevant, 
    \begin{align}
        \sum_{i=1}^n s_i = \sum_{k=1}^n s_k.
    \end{align}
    \item Summen von Summen/Differenzen sind Summen/Differenzen von Summen, 
    \begin{align}
        \sum_{i=1}^n (a_i \pm b_i) = \sum_{i=1}^n a_i \pm \sum_{i=1}^n b_i.
    \end{align}
    \item Gleiche (vom Index unabhängige) Faktoren können ausgeklammert werden, 
    \begin{align}
        \sum_{i=1}^n (a s_i) &= a \sum_{i=1}^n s_i. \notag \\
        \Rightarrow \qq{speziell:} \sum_{i=1}^n a &= a \sum_{i=1}^n 1 = a\underbrace{(1+1+1+\hdots+1)}_{n\text{-mal}} = n \cdot a. \notag 
    \end{align}
    \item Summen können aufgeteilt werden,
    \begin{align}
        \sum_{i=1}^n s_i = \sum_{i=1}^m s_i + \sum_{i=m+1}^n s_i.
    \end{align}
    \item Startwerte können verändert werden, 
    \begin{align}
        \sum_{k=m}^n s_k &= \sum_{k=1}^n s_k - \sum_{k=1}^{m-1} s_k \\ 
        \Rightarrow \qq{speziell:} \sum_{k=m}^n 1 &= \sum_{k=1}^n 1 - \sum_{k=1}^{m-1} 1 = n -(m-1) = 1 + n -m. \notag 
    \end{align}
\end{itemize}

\paragraph{Bemerkung} Es gibt einige weitere Möglichkeiten, Summen zu schreiben, bspw. kann der Index Werte einer (abzählbaren) Menge $M$ annehmen, 
\begin{align}
    \sum_{i \in M} s_i \qq{,}
\end{align}
oder man lässt die Summationsgrenzen weg, wenn in irgendeiner Weise ``klar'' ist, wie summiert wird, 
\begin{align}
    \sum_i s_i. 
\end{align}
Mehrfache Summen können mit einem Summenzeichen geschrieben werden, sofern sie unabhängig voneinander laufen:
\begin{align}
    \sum_{i=1}^n \qty(\sum_{j=1}^n s_{ij}) = \sum_{i,j=1}^n s_{ij} = s_{11} + s_{12} + \hdots + s_{21} + s_{22} + \hdots + s_{nn}.
\end{align}
Dabei ist es egal, welche Summe zuerst ausgeführt wird. Letzteres gilt nicht in Fällen wie beispielsweise 
\begin{align}
    \sum_{i=1}^n \sum_{j=1}^i s_{ij},
\end{align}
da hier der Endwert der zweiten Summe vom Index der ersten Summe abhängt.

\clearpage
\paragraph{Beispiel: Differenzen benachbarter Quadratzahlen}$~$

Wenn wir untersuchen wollen, was die Summe aus den Differenzen benachbarter Quadratzahlen ist, so können wir das zunächst für die ersten paar Glieder aufschreiben: 
\begin{figure}[htp]
    \centering
    \begin{tikzpicture}
        \draw[thick, -{latex}] (0,0) -- (10,0)node[right]{$n$};
        \foreach \x in {0,1,4,9}{
            \draw (\x,.1) -- (\x,-.1)node[below]{\x};
        }
        \draw [decorate, decoration = {calligraphic brace}, thick] (1,-.7) --node[below]{1} (0,-.7);
        \draw [decorate, decoration = {calligraphic brace}, thick] (4,-.7) --node[below]{3} (1,-.7);
        \draw [decorate, decoration = {calligraphic brace}, thick] (9,-.7) --node[below]{5} (4,-.7);
    \end{tikzpicture}
\end{figure}
Dies legt die Vermutung nahe, dass es sich um die Summe aller ungeraden Zahlen handelt. Rechnen wir dies nach, ergibt sich 
\begin{align}
    \sum_{k=0}^n \qty[(k+1)^2 - k^2] &= \sum_{k=0}^n (k+1)^2 - \sum_{k=0}^n k^2 \notag \\   
    &= \sum_{k=0}^n (\cancel{k^2} +2k +1) - \cancel{\sum_{k=0}^n k^2} = \sum_{k=0}^n (2k+1).
\end{align}
Das ist in der Tat die Summe aller ungeraden Zahlen von 1 bis $2n+1$.

\paragraph{Indexverschiebung}$~$

Da die Indizierung einzelner Summanden willkürlich ist, ist es möglich, eine \emph{Indexverschiebung} vorzunehmen:
\begin{align}
    S &\tikzmarknode{eq1}{=} \sum_{i=1}^n s_i  \qq{, Substitution} i = j + a  \notag \\
                        & \hspace{3cm} \Rightarrow \qq{Startwert: aus} i=1 \qq{folgt} j = 1-a \notag \\
                        & \hspace{3cm} \Rightarrow \qq{Endwert: aus} i=n \qq{folgt} j = n-a \notag \\[-1em]
      &\tikzmarknode{eq2}{=} \sum_{j=1-a}^{n-a} s_{j+a}.
\end{align}
\tikz[overlay,remember picture]{
\draw[thick] ($(eq1)+(0,-.2)$)-- ($(eq2)+(0,.2)$);
}
Schauen wir uns dazu das vorherige Beispiel nochmal an:
\begin{align}
    \sum_{k=0}^n (k+1)^2 - \sum_{k=0}^n k^2 \overset{k = i+1}&{=} \sum_{k=0}^n (k+1)^2 - \sum_{i=-1}^{n-1} (i+1)^2 \overset{i\to k}{=} \underbrace{\sum_{k=0}^n (k+1)^2}_{(1)} - \underbrace{\sum_{k=-1}^{n-1} (k+1)^2}_{(2)} \notag\\
     &= \underbrace{(n+1)^2}_{\text{letztes Glied (1)}} + \cancel{\sum_{k=0}^{n-1}(k+1)^2} + \underbrace{(-1+1)^2}_{\text{erstes Glied (2)}} - \cancel{\sum_{k=0}^{n-1}(k+1)} \notag \\
     \Rightarrow \uuline{\sum_{k=0}^n (2k+1) = (n+1)^2}.
\end{align}
Die Summe der ersten $n$ ungeraden Zahlen ist gleich der $(n+1)$-ten Quadratzahl.
\section{Exponentialfunktionen und Logarithmen}

Wir wollen uns nun mit Funktionen beschäftigen, die exponentielles Wachstum beschreiben. \emph{Exponentialfunktionen} sind Funktionen, deren Variable im Exponenten steht $f(x) = a^x$. Hierbei muss die Basis $a > 0$ sein. Unabhängig von $a$ gilt dann $a^0 = 1$, d.\,h. alle Exponentialfunktionen schneiden die $y$-Achse im Punkt $(0,1)$. 

Wir können, abhängig von $a$, drei Fälle unterscheiden 
\begin{itemize}
    \item $a > 1: \quad \lim_{x\to -\infty} a^x = 0$, asymptotische Annäherung an $x$-Achse von rechts
    \item $a < 1: \quad \lim_{x\to \infty} a^x = 0$, asymptotische Annäherung an $x$-Achse von links 
    \item $a = 1: \quad y=1$ für alle $x$
\end{itemize}

\begin{figure}[htp]
    \centering
    \begin{tikzpicture}
        \begin{axis}[disabledatascaling, axis lines=middle, xlabel={$x$}, ylabel={$y$}, height=7cm, width=0.9\textwidth, ymax=2.99, legend pos = outer north east, xmin=-1.1, xmax=1.1, ytick={2}]
            \addplot[no marks, FSUblau, thick, domain=-1:1]{2^x};
            \addplot[no marks, FSUblau, thick, dashed, domain=-1:1]{5^x};
            \addplot[no marks, PAForange, thick, domain=-1:1]{(1/2)^x};
            \addplot[no marks, PAForange, thick, dashed, domain=-1:1]{(1/5)^x};
            \addplot[no marks, Gruen, thick, domain=-1:1]{1};
            \node (A) at (-0.04,0.8){1};
            \legend{$a = 2 >1$, $a = 5 > 1$,$a = 1/2 <1$, $a=1/5 < 1$, $a=1$};
        \end{axis}
    \end{tikzpicture}
    \caption{Darstellung von Exponentialfunktionen für verschiedene Werte von $a$.}
\end{figure} 
Die Funktionen sind spiegelbildlich zur $x$-Achse, denn 
\begin{align}
    \begin{rcases}
        f_1(x) = a^x \\
        f_2(x) = \qty(\frac{1}{a})^x = a^{-x} = f_1(-x) 
    \end{rcases} \qq{wenn} a >1, \qq{dann} \frac{1}{a} < 1.
\end{align}
Das heißt, zu jeder \textcolor{FSUblau}{blauen} Funktion mit Basis $a$, findet man eine spiegelsymmetrische \textcolor{PAForange}{orangene} Funktion mit Basis $\frac{1}{a}$.

\subsection{Logarithmen}

Wir wollen uns nun die Frage stellen, welchen Wert $n$ ein Exponent zu einer gegebenen Basis $b$ haben muss, damit der Potenzwert $a$ herauskommt. Also es gelte $a = b^n$ für ein bekanntes $a$ und $b$, was ist dann $n$?

Die Antwort auf diese Frage liefert die Logarithmusfunktion: $(a,b > 0; b \neq 1)$
\begin{align}
    \tikzmarknode{eq1}{}n = \log_b\tikzmarknode{eq2}{}(\tikzmarknode{eq3}{}a).
\end{align}
\tikz[overlay,remember picture]{
\draw[shorten >=2pt,shorten <=2pt, thick, -{latex}] ($(eq2)+(0.8,-0.7)$)node[right]{Radikand} -- ($(eq2)+(-0.2,-0.1)$);
\draw[shorten >=2pt,shorten <=2pt, thick, -{latex}] ($(eq1)+(-0.7,0.1)$)node[left]{Exponent} -- ($(eq1)+(0,0.1)$);
\draw[shorten >=2pt,shorten <=2pt, thick, -{latex}] ($(eq3)+(1.3,0.15)$)node[right]{Numerus (Potenzwert)} -- ($(eq3)+(0.5,0.15)$);
}

Beispielsweise $n = \log_5(625)$ heißt, dass gilt: $5^n = 625$. Wir finden damit als Ergebnis $n=4$. Spezielle Werte des Logarithmus sind 
\begin{align}
    \begin{split}
        \log_b (1) &= 0 \qq{da} b^n = 1 \quad \Rightarrow \quad n=0\\
        \log_b (b) &= 1 \qq{da} b^n = b \quad \Rightarrow \quad n=1.
    \end{split}
\end{align}
Der Logarithmus ist die Umkehrfunktion der Exponentialfunktion. Es gilt demzufolge 
\begin{align}
    b^{\log_b(a)} = b^n = a \qq{und} \log_b(b^n) = \log_b(a) = n.
\end{align}
Beachte: Der Logarithmus von Null ist nicht definiert, da $10^b = 0$ keine reelle Lösung für $b$ hat. Ebenso ist der Logarithmus negativer Zahlen (hier) nicht definiert, da $10^b = x$ für $x <0$ keine reelle Lösung für $b$ hat.

\paragraph{Logarithmengesetze}$~$

Wir wollen im Folgenden die Logarithmengesetze aus den Potenzgesetzen herleiten: 

\begin{minipage}{0.5\textwidth}
    \begin{tikzpicture}
        \node (A) at (-2,1){$\displaystyle u = b^p, v = b^q$};
        \node (A) at (2,1){$\displaystyle  uv = b^{p+q}$};
        \node (A) at (-2,-1){$\displaystyle p = \log_b(u), q=\log_b(v)$};
        \node (A) at (2,-1){$\displaystyle \log_b(uv) = p+q$};
        \draw[thick, -{latex}] (-.5,1) -- (.5,1);
        \draw[thick, -{latex}] (-2,0.5) -- +(0,-1);
        \draw[thick, -{latex}] (2,0.5) -- +(0,-1);
    \end{tikzpicture}
\end{minipage}
\begin{minipage}{0.49\textwidth}
    \vspace{1.3cm}
    \begin{align}
        \Rightarrow \quad \log_b(uv) = \log_b(u) + \log_b(v).
    \end{align}
\end{minipage}

\begin{minipage}{0.5\textwidth}
    \begin{tikzpicture}
        \node (A) at (-2,1){$\displaystyle u = b^p, v = b^q$};
        \node (A) at (2,1){$\displaystyle  \frac{u}{v} = b^{p-q}$};
        \node (A) at (-2,-1){$\displaystyle p = \log_b(u), q=\log_b(v)$};
        \node (A) at (2,-1){$\displaystyle \log_b\qty(\frac{u}{v}) = p-q$};
        \draw[thick, -{latex}] (-.5,1) -- (.5,1);
        \draw[thick, -{latex}] (-2,0.5) -- +(0,-1);
        \draw[thick, -{latex}] (2,0.5) -- +(0,-1);
    \end{tikzpicture}
\end{minipage}
\begin{minipage}{0.49\textwidth}
    \vspace{1.4cm}
    \begin{align}
        \Rightarrow \quad \log_b\qty(\frac{u}{v}) = \log_b(u) - \log_b(v).
    \end{align}
\end{minipage}

\begin{minipage}{0.5\textwidth}
    \begin{tikzpicture}
        \node (A) at (-3.82,1){};
        \node (A) at (-2,1){$\displaystyle b^x = u^m$};
        \node (A) at (2,1){$\displaystyle  u = b^{\frac{x}{m}}$};
        \node (A) at (-2,-1){$\displaystyle x = \log_b(u^m)$};
        \node (A) at (2,-1){$\displaystyle\log_b(u) = \frac{x}{m}$};
        \draw[thick, -{latex}] (-.5,1) -- (.5,1);
        \draw[thick, -{latex}] (-2,0.5) -- +(0,-1);
        \draw[thick, -{latex}] (2,0.5) -- +(0,-1);
    \end{tikzpicture}
\end{minipage}
\begin{minipage}{0.49\textwidth}
    \vspace{1.3cm}
    \begin{align}
        \Rightarrow \quad \log_b(u^m) = m \log_b(u).
    \end{align}
\end{minipage}

Die Logarithmus-Funktion wandelt Potenzen und Wurzeln in Produkte und Brüche und diese wiederum in Summen und Differenzen um.

\paragraph{Wechsel der Basis}$~$

Alle Logarithmen können bezüglich der gleichen Basis ausgedrückt werden. 

\begin{minipage}{0.5\textwidth}
    \begin{tikzpicture}
        \node (A) at (-3.82,1){};
        \node (A) at (-2,1){$\displaystyle x = a^u$};
        \node (A) at (1.5,1){$\displaystyle  u = \log_a(x)$};
        \node (A) at (0.3,-1){$\displaystyle \log_b(x) = \log_b(a^u) = u \log_b(a) = \log_a(x) \log_b(a)$};
        % \node (A) at (2,-1){$\displaystyle\log_b(u) = \frac{x}{m}$};
        \draw[thick, -{latex}] (-.7,1) --(0,1);
        \draw[thick, -{latex}] (-2,0.5) --node[right]{$\log_b$}  +(0,-1);
        \draw[thick, -{latex}] (1.5,0.5) -- +(0,-1);
    \end{tikzpicture}
\end{minipage}
\begin{minipage}{0.49\textwidth}
    \vspace{1.4cm}
    \begin{align}
        \Rightarrow \quad \log_a(x) = \frac{1}{\log_b(a)} \log_b(x).
    \end{align}
\end{minipage}

Dabei heißt $\frac{1}{\log_b(a)}$ der \emph{Modul} von $a$ bezüglich der Basis $b$. Für $x = b$ gilt speziell 
\begin{align}
    \log_a(b) = \frac{1}{\log_b(a)}\underbrace{\log_b(b)}_{1}.
\end{align}
Gebräuchliche Logarithmensysteme sind 
\begin{itemize}
    \item der \emph{dekadische} Logarithmus: $\log_{10}(x) \equiv \text{lg}(x)$, 
    \item der \emph{natürliche} Logarithmus: $\log_\text{e}(x) \equiv \ln(x)$
\end{itemize}
Hierbei ist e die \emph{Eulersche Zahl} (später mehr dazu)
\begin{align}
    e = \sum_{k=0}^\infty \frac{1}{k!} = 2.71828182845945\hdots
\end{align}
Häufig wird ein Basiswechsel zum natürlichen Logarithmus durchgeführt: 
\begin{align}
    \log_a(x) = \frac{\ln(x)}{\ln(a)}.
\end{align}
Fassen wir abschließend die Logarithmengesetze und den Basiswechsel nochmal zusammen:
\begin{mymathbox}[ams align, title={Logarithmengesetze, Basiswechsel}, colframe={FSUblau}]
    \begin{split}
        \log_b(uv) = \log_b(u) + \log_b(v), \quad \log_b\qty(\frac{u}{v}) = \log_b(u)-\log_b(v) \\
        \log_b(u^m) = m \log_b(u), \quad \log_a(x) = \frac{1}{\log_b(a)} \log_b(x).       
    \end{split}
\end{mymathbox}

\paragraph{Der dekadische Logarithmus $\log_{10}(x) = \lg(x)$}$~$

Der dekadische Logarithmus ist zur Basis 10 definiert und wird beispielsweise in der Chemie verwendet, um den pH-Wert zu definieren: 
\begin{align}
    \text{pH} = - \log_{10} [\text{H}^+], \qquad [\text{H}^+] \qq{- molare Konzentration von H$^+$ in der Lösung.}
\end{align}
Das Minuszeichen ist begründet dadurch, dass der Logarithmus für Zahlen zwischen 0 und 1 negative Werte annimmt, z.\,B. 
\begin{align}
    \lg(0,00213) = \lg (2,13\cdot{10^{-3}}) =\lg(2,13) + \underbrace{\lg (10^{-3})}_{\lg(10^n) = n} = \underbrace{\lg(2,13)}_{\lg{1} = 0 < 1 = \lg{10}} - 3 < 0.
\end{align}
Dabei wird der erste Term $\lg(2,13)$ \emph{Mantisse} und der zweite Term $(-3)$ \emph{Kernzahl} genannt.

\newpage
\paragraph{Grafische Darstellung}$~$

Die der Logarithmus die Umkehrfunktio der Exponentialfunktion ist, erhält man sie durch Spiegellung an der Geraden $y=x$. Für reelle $y$ ist der Definitionsbereich $0 < x < \infty$. Wir betrachten den Fall $a > 1$.
\begin{figure}[htp]
    \centering
    \begin{tikzpicture}[every text node part/.style={align=center}]
        \begin{axis}[disabledatascaling, axis equal, axis lines=middle, xtick=1, ytick=1, xlabel={$x$}, ylabel={$y$}, width=0.7\textwidth, ymax=3.5, ymin = -3.5, samples=100, legend pos = outer north east]
            \addplot[no marks, FSUblau, thick, domain=-2.5:4]{exp(x)};
            \addplot[no marks, PAForange, thick, domain=0.03:4]{ln(x)};
            \addplot[no marks, Gruen, thick, domain=-2.5:4]{x};
            \legend{$y=\e^x$, $y = \log_{\e}(x)$, $y = x$};
            \draw[gray] (1,0) -- (0,1);
            \coordinate (A1) at ($(2,{ln(2)})$);
            \coordinate (A2) at ($({ln(2)},2)$);
            \draw[gray] (A1) -- (A2);
            \coordinate (A1) at ($(1/3,{ln(1/3)})$);
            \coordinate (A2) at ($({ln(1/3)},1/3)$);
            \draw[gray] (A1) -- (A2);
            \draw[{latex}-, thick] (1.2,-.2) -- +(1.5,-1)node[below]{Alle logarithmischen Kurven\\ durchlaufen den Punkt (1,0).};
        \end{axis}
    \end{tikzpicture}
    \caption{Darstellung von Exponentialfunktion und Logarithmusfunktion für $a = \e$.}
    \label{}
\end{figure}


\subsection{Die Exponentialfunktion}
Bisher haben wir Exponentialfunktionen im Allgemeinen behandelt. Als \emph{die Exponentialfunktion} wird gemeinhin eine Exponentialfunktion zur Basis e (Eulersche Zahl) bezeichnet - kurz: e-Funktion.

Sie dient der Beschreibung von Wachstum und Zerfall (z.\,B. Radioaktivität) und als Lösung von Differentialgleichungen (siehe Mathematische Methoden der Physik I).

Die allgemeine Form der Exponentialfunktion ist 
\begin{align}
    f(x) = A \e^{c x} \equiv A \exp(cx)\qq{,} A = \text{const, } c = \text{const.}
\end{align}
Abhängig vom Vorzeichen von $c$ wird entweder exponentielles Wachstum ($c>0$) oder exponentieller Zerfall ($c <0$) beschrieben. 

Das besondere der e-Funktion ist, dass sie proportional zu ihrem eigenen Anstieg ist. Dieses Verhalten ist typisch für viele Phänomene in der Physik, weshalb die e-Funktion dort sehr häufig auftritt. 

\paragraph{Beispiel: Halbwertszeit}$~$

Wir betrachten einen exponentiellen Zerfall mit der Zeit $t$. Wir stellen uns nun die Frage wann ein Anfangswert auf seine Hälfte zerfallen ist und nennen diesen Zeitraum $\tau$. Mathematisch lässt sich dieser Zerfallsprozess formulieren als 
\begin{align}
    f(x) = A \e^{-ct}, \quad c > 0.
\end{align}
Es soll nun also $f(t_0 + \tau)$ halb so groß sein wie $f(t_0)$, wobei $t_0$ ein willkürlicher Zeitpunkt ist: 
\begin{align}
    f(t_0 + \tau) \overset{!}&{=} \frac{1}{2} f(t_0) \notag \\
    \cancel{A} \e^{-c(t_0 + \tau)} &= \frac{\cancel{A}}{2} \e^{-ct_0}  \quad (A \neq 0) \notag \\
    \e^{-c t_0} \e^{-c\tau} &= \frac{1}{2} \e^{-c t_0} \quad (\e^{x} \neq 0 \qq{für alle}x) \notag \\ 
    \Rightarrow -c\tau &= \ln(\frac{1}{2}) = - \ln(2) \quad \Rightarrow \quad \tau = \frac{\ln(2)}{c}.
\end{align}

\paragraph{Definitionen der Eulerschen Zahl:}$~$

Es gibt verschiedene (zueinander äquivalente) Möglichkeiten die Eulersche Zahl zu definieren: 

\begin{enumerate}
    \item Die e-Funktion lässt sich über ihre Reihendarstellung beschreiben:
    \begin{align}
        \e^{x} &= 1 + \frac{x}{1!} + \frac{x^2}{2!} + \frac{x^3}{3!} + \hdots = \sum_{k=0}^\infty \frac{x^k}{k!} \notag \\
        \intertext{Man kann zeigen, dass die Reihenglieder für jeden Wert von $x$ ab einer hinreichend hohen Ordnung immer kleiner werden. Man sagt, die Reihe \emph{konvergiert} (siehe Analysis I).}
        \Rightarrow \e &= 1 + 1 + \frac{1}{2} + \frac{1}{6} + \frac{1}{24} + \hdots = \sum_{k=0}^\infty \frac{1}{k!}.    
    \end{align}
    Hierbei ist die \emph{Fakultät} $n!$ einer natürlichen Zahl $n \in \mathbb{N}$ definiert als $n! := 1\cdot 2\cdot 3 \cdot \hdots \cdot (n-1)\cdot n$, wobei zusätzlich noch $0! := 1$ gilt.
    \item Definition als Grenzwert: 
    \begin{align}
        \e^x = \lim_{n\to \infty} \qty(1+\frac{x}{n})^n  \quad \Rightarrow \quad \e = \lim_{n\to \infty} \qty(1+\frac{1}{n})^n.
    \end{align}
    Diese Darstellung geht auf das Problem der stetigen Verzinsung nach Jakob Bernoulli (1654-1705) zurück: 
    \begin{quote}
        ``Eine Summe Geldes sei auf Zinsen angelegt, dass in den einzelnen Augenblicken ein proportionaler Teil der Jahreszinsen zum Kapital geschlagen wird.''
    \end{quote}
    Betrachten wir also ein Anfangsguthaben $A$ und einen Zinssatz von \SI{100}{\percent}: 
    \begin{align}
        \qq{Verzinsung nach 1 Jahr:} \text{Guth.} &= A + A = (1+1) A \notag \\
        \qq{Verzinsung nach $\frac{1}{2}$ Jahr:} \text{Guth.} &= \underbrace{A + \frac{1}{2} A}_{\text{1. Halbjahr}}  + \underbrace{\frac{1}{2}\qty( A + \frac{1}{2} A)}_{\text{2. Halbjahr}} = \qty(1+\frac{1}{2})^2 A \notag \\
        \qq{Verzinsung nach $\frac{1}{3}$ Jahr:} \text{Guth.} &= \qty(1+\frac{1}{3})^3 A \notag \\
        \qq{Verzinsung nach $\frac{1}{n}$ Jahr:} \text{Guth.} &= \qty(1+\frac{1}{n})^n A.
    \end{align}
    Für eine Verzinsung in einzelnen Augenblicken gilt nun: 
    \begin{align}
        \text{Guth.} = \lim_{n\to\infty} \qty(1+\frac{1}{n})^n A = e\cdot A.
    \end{align}
\end{enumerate}
\thispagestyle{plain}
\section{Trigonometrische Funktionen} 

Dieser Abschnitt widmet sich der Definition von Sinus- und Kosinusfunktionen sowie deren Eigenschaften. 

\paragraph{Bogenmaß und Gradmaß}$~$

\begin{wrapfigure}{r}{7cm}
    \centering
    \vspace{-1cm}
        \begin{tikzpicture}[scale=2.5]
            \draw[thick, -{latex}] ( -1.1,0) -- (1.2,0)node[above]{$x$};
            \draw[thick, -{latex}] (0,-1.1) -- (0,1.2)node[right]{$y$};
            \draw (0,0) circle (1cm);
            \fill (1,0) circle (1pt)node[below left]{$1$};
            \fill (0,1) circle (1pt)node[below right]{$1$};
            \fill (-1,0) circle (1pt)node[above right]{$\minus1$};
            \fill (0,-1) circle (1pt)node[above left]{$\minus1$};
            \draw[thick, -{latex}] (0,0) --node[above left, rotate=40, pos=0.8]{$r=1$} (40:1);
            \draw[-{latex}] (0.4,0) arc (0:40:.4);
            \node (A) at (20:.3){$\alpha$};
            \draw[thick, PAForange] (1,0) arc (0:40:1); 
        \end{tikzpicture}
    \vspace{-3mm}
\end{wrapfigure}

Wir betrachten die (reelle) Zahlenebene, in der alle Abstände in Einheit 1 gemessen werden, also \emph{dimensionslos} sind. Der Umfang des Einheitskreises beträgt definitionsgemäß 
\begin{align}
    u = 2\pi r = 2\pi.
\end{align}
Wir können nun zwischen den beiden Maßsystemen \emph{Gradmaß} und \emph{Bogenmaß} unterscheiden. Für das Gradmaß ist der Vollkreis in 360 Abschnitte eingeteilt, während im Bogenmaß der Winkel in Bruchteilen des Kreisumfangs gemessen wird:
\begin{align}
    \begin{split}
        \text{Gradmaß: } [\alpha] &= \si{\degree} \quad(\text{deg}) \qq{,} \text{Vollkreis: } \SI{360}{\degree}\\
        \text{Borgenmaß: } [\alpha] &= 1 \quad(\text{rad}) \qq{,} \text{Vollkreis: } 2\pi.
    \end{split}
\end{align}
Wir können beide Maßsysteme mittels einer Verhältnisgleichung ineinander überführen: 
\begin{align}
    \frac{\alpha [\text{rad}]}{2\pi} = \frac{\alpha [\text{deg}]}{\SI{360}{\degree}}.
\end{align}
Einige Werte zur Umrechnung sind in folgender Tabelle aufgelistet:

\begin{table}[htp]
    \centering
    \caption{Umrechnungstabelle für Bogen- und Gradmaß}
    \begin{tabular}{c c c c c c c c c c c}
        \toprule 
        $\alpha$ [deg] & 30 & 45 & 60 & 90 & 120 & 150 & 180 & 270 & 360 \\
        \midrule
        $\alpha$ [rad] & $\frac{\pi}{6}$ & $\frac{\pi}{4}$ & $\frac{\pi}{3}$ & $\frac{\pi}{2}$ & $\frac{2\pi}{3}$ & $\frac{5\pi}{6}$ & $\pi$ & $\frac{3\pi}{2}$ & $2\pi$ 
    \end{tabular}
\end{table}

\subsection{Winkelfunktionen}

\begin{minipage}{0.5\textwidth}
    Alle Winkelfunktionen sind dimensionslos definiert. Die $x$-Koordinate eines Punktes des Einheitskreises ist der Kosinus des Winkels zwischen seinem Ortsvektor und der Abszisse, während die $y$-Koordinate der Sinus dieses Winkels ist. Weitere trigonometrische Funktionen sind wie folgt defniert: 
    \begin{itemize}
        \item Tangens:\hphantom{Ko} $\displaystyle \tan(\alpha) := \frac{\sin(\alpha)}{\cos(\alpha)}$
        \item Kotangens: \,$\displaystyle \cot(\alpha) := \frac{\cos(\alpha)}{\sin(\alpha)} = \frac{1}{\tan(\alpha)}$
        \item Sekans:\;\;\hphantom{Ko} $\displaystyle \sec(\alpha) := \frac{1}{\cos(\alpha)}$
        \item Kosekans: \;\;$\displaystyle \csc(\alpha) := \frac{1}{\sin(\alpha)}$
    \end{itemize}
\end{minipage}
\begin{minipage}{0.5\textwidth}
    \centering
    \begin{tikzpicture}[scale=5]
        \draw[thick, -{latex}] ( -.3,0) -- (1.2,0)node[above]{$x$};
        \draw[thick, -{latex}] (0,-.3) -- (0,1.2)node[right]{$y$};
        \draw (-10:1) arc (-10:100:1);
        \fill (1,0) circle (0.5pt)node[below left]{$1$};
        \fill (0,1) circle (0.5pt)node[below right]{$1$};
        % \fill (-1,0) circle (1pt)node[above right]{$\minus1$};
        % \fill (0,-1) circle (1pt)node[above left]{$\minus1$};
        \draw[thick, -{latex}] (0,0) --node[above left, rotate=40, pos=0.6]{$r=1$} (40:1);
        \draw[-{latex}] (0.4,0) arc (0:40:.4);
        \node (A) at (20:.3){$\alpha$};

        \draw[ultra thick, FSUblau] ($(0,sin{40})$) --node[above]{$\cos(\alpha)$} ($({cos{40}},sin{40})$);
        \draw[ultra thick, PAForange, pos=0.4] (40:1) --node[rotate=-90,yshift=0.3cm,, xshift=0.3cm]{$\sin(\alpha)$} ($({cos{40}},0)$);
        \draw[ultra thick, Gruen] (1,0) --node[rotate=-90, yshift=0.3cm]{$\tan(\alpha)$} ($(1,tan{40})$);
        \draw[ultra thick, red!70!black] (0,1) --node[above]{$\cot(\alpha)$} ($(cot{40},1)$);
        \draw[ultra thick, PAForange!60!white] (0,0) --node[above, rotate=40]{$\csc(\alpha)$} ($(cot{40},1)$);
        \draw[thick, FSUblau!60!white] (0,0) --node[below, rotate=40]{$\sec(\alpha)$} ($(1, tan{40})$);
        % \draw[thick, PAForange] (1,0) arc (0:40:1); 
    \end{tikzpicture}
\end{minipage}

Aus der Abbildung können wir mithilfe des Satzes des Pythagoras folgende Relation ablesen 
\begin{mymathbox}[ams align, title={Trigonometrischer Pythagoras}, colframe={FSUblau}]
    \sin^2(\alpha) + \cos^2(\alpha) = 1.
\end{mymathbox}

Durch weitere rechtwinklige Dreiecke können wir ebenfalls folgende Relationen ablesen: 
\begin{align}
    \begin{split}
        \sec^2(\alpha) &= \frac{1}{\cos^2(\alpha)} = 1 + \tan^2(\alpha) \\
        \csc^2(\alpha) &= \frac{1}{\sin^2(\alpha)} = 1 + \cot^2(\alpha).
    \end{split}
\end{align}
Beide Relationen lassen sich auch direkt mithilfe der Definitioen und des trigonometrischen Pythagoras herleiten.
Anhand der Konstruktion können wir auch folgende Eigenschaften der Funktionen ablesen, wie z.\,B. die Wertebereiche 
\begin{align}
    \begin{split}
        &\sin(\alpha) \in [-1,1] \qquad \tan(\alpha) \in (-\infty,\infty) \\
        &\cos(\alpha) \in [-1,1] \qquad \cot(\alpha) \in (-\infty,\infty).
    \end{split}
\end{align}
Die Vorzeichen der Funktionen in den einzelnen Quadranten lauten:
\begin{table}[htp]
    \centering
    \caption{Vorzeichen/Signum (sgn) der Funktionen}
    \begin{tabular}{c c c c c c c c c c c}
        \toprule 
        Quadrant & sgn($\sin\alpha$) & sgn($\cos\alpha$) & sgn($\tan\alpha$) & sgn($\cot\alpha$)  \\
        \midrule
        I & + & + & + & + \\
        II & + & - & - & - \\
        III & - & - & + & + \\
        IV & - & + & - & -
    \end{tabular}
    \vspace{-1cm}
\end{table}

Spezielle Werte der Funktionen sind in folgender Tabelle zusammengefasst: 
\begin{table}[htp]
    \centering
    \caption{Vorzeichen/Signum (sgn) der Funktionen}
    \begin{tabular}{c c c c c c c c c c c }
        \toprule 
         $\alpha$ [deg] & 0 & 30 & 45 & 60 & 90 & 120 & 150 & 180 & 270 & 360 \\
         $\alpha$ [rad] & 0 & $\dfrac{\pi}{6}$ & $\dfrac{\pi}{4}$ & $\dfrac{\pi}{3}$ & $\dfrac{\pi}{2}$ & $\dfrac{2\pi}{3}$ & $\dfrac{5\pi}{6}$ & $\pi$ & $\dfrac{3\pi}{2}$ & $2\pi$\\[.4em]
         \midrule 
         $\sin\alpha$   & 0 & $\dfrac{1}{2}$ & $\dfrac{\sqrt{2}}{2}$ & $\dfrac{\sqrt{3}}{2}$ & 1 & $\dfrac{\sqrt{3}}{2}$ & $\dfrac{1}{2}$ & 0 & -1 & 0 \\[.7em]
         $\cos\alpha$   & 1 & $\dfrac{\sqrt{3}}{2}$ & $\dfrac{\sqrt{2}}{2}$ & $\dfrac{1}{2}$ & 0 & -$\dfrac{1}{2}$ & -$\dfrac{\sqrt{3}}{2}$ & -1 & 0 & 1 \\[.7em]
         $\tan\alpha$   & 0 & $\dfrac{1}{\sqrt{3}}$ & $1$ & $\sqrt{3}$ & $\pm \infty$ & -$\sqrt{3}$ & -$\dfrac{1}{\sqrt{3}}$ & 0 & $\pm \infty$ & 0 \\[.7em]
         $\cot\alpha$   & $\pm\infty$ & $\sqrt{3}$ & $1$ & $\dfrac{1}{\sqrt{3}}$ & 0 & -$\dfrac{1}{\sqrt{3}}$ & -$\sqrt{3}$ & $\pm \infty$ & 0 & $\pm \infty$ \\

    \end{tabular}
    \vspace{-1cm}
\end{table}

\subsection{Graphische Darstellung der Winkelfunktionen}

Wir erlauben alle reellen Zahlen $x$ als Argumente in den Winkelfunktionen, d.\,h. der Definitionsbereich ist ganz $\mathbb{R}$ (für $\sin(x)$ und $\cos(x)$). Dabei sind die Funktionen periodisch 
\begin{align}
    \begin{split}
        \sin(x) &= \sin(x+2\pi n), \qquad \tan(x) = \tan(x+\pi n) \\
        \cos(x) &= \cos(x+2\pi n), \qquad \cot(x+\pi n)
    \end{split} \qq{mit} n \in \mathbb{Z}.
\end{align}

\begin{figure}[htp]
    \centering
    \begin{tikzpicture}
        \begin{axis}[disabledatascaling, axis lines=middle, xtick ={-pi,pi,2*pi,3*pi}, xticklabels={-$\pi$,$\pi$,$2\pi$, $3\pi$},ytick={-1,1}, xlabel={$x$}, ylabel={$y$}, height=5.5cm, width=0.9\textwidth, xmin =-pi-0.1, xmax=3*pi+0.2, samples =100, ymin=-1.3, ymax=1.3, legend pos = outer north east]
            \addplot[no marks, FSUblau, thick, domain=-pi:3*pi]{sin{deg(x)}};
            \addplot[no marks, PAForange, thick, domain=-pi:3*pi]{cos{deg(x)}};
            \legend{$\sin(x)$, $\cos(x)$}
        \end{axis}
    \end{tikzpicture}
    \begin{tikzpicture}
        \begin{axis}[disabledatascaling, axis lines=middle, xtick ={-pi,-pi/2,pi,2*pi,3*pi}, xticklabels={-$\pi$,,$\pi$,$2\pi$, $3\pi$},ytick={-1,1}, xlabel={$x$}, ylabel={$y$}, height=5.5cm, width=0.9\textwidth, xmin =-pi-0.1, xmax=3*pi+0.2, samples =100, ymin=-2.5, ymax=2.5, legend pos=outer north east]
            \addplot[no marks, FSUblau, thick, domain=-pi:-pi/2-0.1]{tan{deg(x)}};
            \addplot[no marks, PAForange, thick, domain=-pi:0-0.1]{cot{deg(x)}};
            \legend{$\tan(x)$, $\cot(x)$}
            \addplot[no marks, FSUblau, thick, domain=-pi/2+0.1:pi/2-0.1]{tan{deg(x)}};
            \addplot[no marks, FSUblau, thick, domain=-pi/2+0.1:pi/2-0.1]{tan{deg(x)}};
            \addplot[no marks, FSUblau, thick, domain=pi/2+0.1:3*pi/2-0.1]{tan{deg(x)}};
            \addplot[no marks, FSUblau, thick, domain=3*pi/2+0.1:5*pi/2-0.1]{tan{deg(x)}};
            \addplot[no marks, PAForange, thick, domain=0.1:pi-0.1]{cot{deg(x)}};
            \addplot[no marks, PAForange, thick, domain=pi+0.1:2*pi-0.1]{cot{deg(x)}};
            \addplot[no marks, PAForange, thick, domain=2*pi+0.1:3*pi-0.1]{cot{deg(x)}};
            \draw[thick, dashed] (2*pi,-2.5) -- (2*pi,2.5);
            \draw[thick, dashed] (5*pi/2,-2.5) -- (5*pi/2,2.5);
        \end{axis}
    \end{tikzpicture}
    \caption{Grafische Darstellung von Sinus bzw. Kosinus (oben) und Tangens bzw. Kotangens (unten).}
\end{figure}
Symmetrie-Eigenschaften der Funktionen lauten 
\begin{itemize}
    \item $\sin(-x) = -\sin(x), $ ungerade
    \item $\cos(-x) = +\cos(x), $ gerade
    \item $\tan(-x) = -\tan(x), $ ungerade
    \item $\cot(-x) = -\cot(x), $ ungerade
\end{itemize}

\paragraph{Umkehrfunktionen}$~$

Wir erhalten die Umkehrfunktionen durch Auflösen nach dem Argument $x$. Dabei definieren wir die folgendermaßen
\begin{itemize}
    \item Arkussinus: \hphantom{ko}\;\qquad$\arcsin(\sin x) = x$,
    \item Arkuskosinus: \qquad\;$\arccos(\cos x) = x$,
    \item Arkustangens: \quad\hphantom{ko}$\arctan(\tan x) = x$,
    \item Arkuskotangens: \quad$\text{acot}(\cot x) = x.$
\end{itemize}

Wir müssen beachten, dass der Definitionsbereich der periodischen Winkelfunktionen eingeschränkt werden muss, damit die Umkehrfunktionen eindeutig sind. Wir beschränken die Argumente (bzw. die Funktionswerte der Umkehrfunktionen) wie folgt: 
\begin{itemize}
    \item $\sin(x), \quad x\in\qty[-\frac{\pi}{2},\frac{\pi}{2}]$,
    \item $\cos(x), \quad x\in[0,\pi]$,
    \item $\tan(x), \quad x \in \qty(-\frac{\pi}{2},\frac{\pi}{2})$,
    \item $\text{acot}(x) \quad x \in (0,\pi).$
\end{itemize}

\begin{figure}[htp]
    \centering
    \begin{tikzpicture}
        \begin{axis}[disabledatascaling, axis lines=middle, xtick={-1,1}, ytick={-90,90, 180}, yticklabels={$-\frac{\pi}{2}$,$\frac{\pi}{2}$, $\pi$}, xlabel={$x$}, ylabel={$y$}, height=7cm, width=0.57\textwidth, xmin=-1.8, xmax=1.8, samples=100, ymax=220, ymin=-130]
            \addplot[no marks, FSUblau, thick, domain=-1:1]{asin(x)};
            \addplot[no marks, PAForange, thick, domain=-1:1]{acos(x)};
            \legend{$\arcsin(x)$, $\arccos(x)$}
            \draw[thin, dashed] (-1,180) -- (-.2,180);
            \draw[thin, dashed] (-1,-90) -- (-.4,-90);
        \end{axis}
    \end{tikzpicture}
    \hfill
    \begin{tikzpicture}
        \begin{axis}[disabledatascaling, axis lines=middle, ytick={-180,90,180}, yticklabels={$-\pi$,$\frac{\pi}{2}$,$\pi$}, xlabel={$x$}, ylabel={$y$}, height=7cm, width=0.57\textwidth, xmin=-5, xmax=5, samples=100, ymax=220, ymin=-130]
            \addplot[no marks, FSUblau, thick, domain=-5:5]{atan(x)};
            \addplot[no marks, PAForange, thick, domain=-5:5]{90 - atan(x)};
            \legend{$\arctan(x)$, $\text{acot}(x)$}
            \draw[thin,dashed] (-5,180) -- (-.7,180);
            \draw[thin,dashed] (-5,-90) -- (0,-90);\draw[thin,dashed] (0,90) -- (5,90);
        \end{axis}
    \end{tikzpicture}
    \caption{Grafische Darstellung der Umkehrfunktionen von Sinus bzw. Kosinus (links) und Tangens bzw. Kotangens (rechts).}
\end{figure}
Wir können aus der graphischen Darstellung die folgenden analytischen Eigenschaften ablesen: 
\begin{align}
    \arccos(x) &= \frac{\pi}{2} - \arcsin(x), \quad \text{atan}(x) = \frac{\pi}{2} - \arctan(x) \\
    \lim_{x\to \pm \infty} (\arctan(x)) &= \pm \frac{\pi}{2} \qq{und} \lim_{x\to\infty}(\text{acot}(x)) = 0, \quad \lim_{x\to-\infty}(\text{acot}(x)) = \pi.
\end{align}

\subsection{Definition durch Reihen}

Ebenso wie die Exponentialfunktion lassen sich auch die trigonometrischen Funktionen als Reihen darstellen bzw. definieren. Der Zusammenhang der Wineklfunktionen mit der e-Funktion wird im Vorkurs (mit Einführung der komplexen Zahlen) klar werden. Es gilt: 
\begin{align}
    \begin{split}
        \sin(x) &= \frac{x}{1!} - \frac{x^3}{3!} + \frac{x^5}{5!} - \frac{x^7}{7!} \pm \hdots = \sum_{n=0}^\infty (-1)^n \frac{(x)^{2n+1}}{(2n+1)!} \\
        \cos(x) &= 1 - \frac{x^2}{2!} + \frac{x^4}{4!} - \frac{x^6}{6!} \pm \hdots = \sum_{n=0}^\infty (-1)^n \frac{(x)^{2n}}{(2n)!}.
    \end{split}
\end{align}
Wir können weiterhin auch die Symmetrieeigenschaften von Sinus und Kosinus in den Reihenentwicklungen erkennen. Da die Sinusfunktion eine ungerade Funktion ist, tauchen hier nur ungerade Potenzen von $x$ auf. Weiterhin gilt 
\begin{align}
    \arctan(x) = x - \frac{x^3}{3} + \frac{x^5}{5} - \frac{x^7}{7} \pm \hdots = \sum_{n=0}^\infty (-1)^{n} \frac{(x)^{2n+1}}{2n+1}.
\end{align}
Die übrigen Reihen sehen etwas komplizierter aus und werden hier nicht aufgeführt.

\subsection{Additionstheoreme}

Für die Winkelfunktionen gelten die folgenden Additionstheoreme: 
\begin{mymathbox}[ams align, title={Additionstheoreme}, colframe={FSUblau}]
    \begin{split}
        \sin(x\pm y) &= \sin(x) \cos(y) \pm \cos(x) \sin(y),\\
    \cos(x\pm y) &= \cos(x) \cos(y) \mp \sin(x)\sin(y).
    \end{split}
\end{mymathbox}
Das zweite Theorem folgt aus dem Ersten (siehe Übung). Ebenfalls folgen Regeln für Tangens und Cotangens
\begin{align}
    \tan(x\pm y) = \frac{\tan(x) \pm \tan(y)}{1 \mp \tan(x) \tan(y)}.
\end{align}
Auf einen Beweis wird an dieser Stelle verzichtet.

Die Additionstheoreme können genutzt werden, um die Relationen für die Doppelwinkelfunktionen herzuleiten: 
\begin{align}
    \sin(2x) &= 2 \sin(x) \cos(x) \label{eqn:07_Doppelwinkel_sin}\\
    \cos(2x) &= 2 \cos^2(x) -1. \label{eqn:07_Doppelwinkel_cos}
\end{align}

Des weiteren lässt sich die Summe zweier Kosinusfunktionen auch als Produkt schreiben: 
\begin{align}
    \cos(x) + \cos(y) = 2 \cos(\frac{x+y}{2}) \cos(\frac{x-y}{2}).
\end{align}
% Diese Identität wird beispielsweise in der Mechanik genutzt, um die Schwebung zwischen zwei Tonsignalen der ähnlicher Frequenzen $f_1$ und $f_2$ zu veranschaulichen. Die (langsame) Frequenz des Schwebungssignals ist dann $\tilde{f} = \frac{|f_1 - f_2|}{2}$. 

\subsection{Ebene Trigonometrie}

\paragraph{Das rechtwinklige Dreieck}$~$

Sinus, Kosinus und Tangens können auch über Winkel und Längenverhältnisse im rechtwinkligen Dreieck definiert werden. 
\begin{figure}[htp]
    \centering
    \begin{tikzpicture}
        \coordinate (A) at (0,3){};
        \coordinate (B) at (-4,0){};
        \coordinate (C) at (0,0){};
        \draw[thick] (A) --node[above]{$c$} (B) --node[below]{$a$} (C) --node[right]{$b$} (A);
        \draw ($(A)+(0,-0.9)$) arc (-90:-90-53:.9);
        \draw ($(B)+(0.9,0)$) arc (0:37:.9);
        \draw ($(C) + (0,.7)$) arc (90:180:.7);
        \node (D) at ($(A) + (-110:.6)$){$\beta$};
        \node (D) at ($(B) + (20:.6)$){$\alpha$};
        \fill ($(C) + (135:.35)$) circle (1pt);
        \node[anchor=west] (text) at (2,3){$a:$ Ankathete (von $\alpha$)};
        \node[anchor=west] (text) at (2,2.2){$b:$ Gegenkathete (von $\alpha$)};
        \node[anchor=west] (text) at (2,1.4){$c:$ Hypotenuse};
        \node[anchor=west] (text) at (2,.4){Pythagoras: $a^2 + b^2 = c^2$};
        \node[anchor=west] (text) at (2,-.6){Innenwinkel: $\alpha + \beta = \frac{\pi}{2}$};
    \end{tikzpicture}
\end{figure}

Wir können nun die Winkelfunktionen definieren als 
\begin{align}
    \sin(\alpha) = \frac{b}{c}, \quad \cos(\alpha) = \frac{a}{c}, \quad \tan(\alpha) = \frac{b}{a}.
\end{align}
Setzen wir $c=1$ so erhalten wir wieder die Definition der Winkelfunktionen am Einheitskreis. Der Flächeninhalt des rechtwinkligen Dreiecks ergibt sich zu $A = \frac{1}{2}ab$.

\paragraph{Schiefwinkliges Dreieck}$~$

Für ein allgemeines, schiefwinkliges Dreieck können wir o.\,B.\,d.\,A. für $\alpha < \frac{\pi}{2}, \gamma < \frac{\pi}{2}$ zwei Fälle unterscheiden: 

\begin{figure}[htp]
    \centering
    \begin{tikzpicture}
        \coordinate (A) at (-4,0){};
        \coordinate (B) at (2,0){};
        \coordinate (C) at ($(A)+(45:5.5)$){};
        \draw[thick] (A) --node[below]{$c$} (B) --node[right]{$a$} (C) --node[above left]{$b$} (A);
        \draw ($(A)+(0.9,0)$) arc (0:45:.9);
        \draw ($(B)+(-0.9,0)$) arc (180:115:.9);
        \draw ($(C) + (225:.9)$) arc (225:300:.9);
        \node (D) at ($(B) + (150:.6)$){$\beta$};
        \node (D) at ($(A) + (22.5:.6)$){$\alpha$};
        \node (D) at ($(C) + (250+12.5:.6)$){$\gamma$};
        \draw[gray, dashed] (A) --node[above]{$h_a$} +(25:6);
        \draw[gray, dashed] (B) --node[above, pos=0.3]{$h_b$} +(180-42:5);
        \draw[gray, dashed] (C) --node[right,pos=0.7]{$h_c$} (0,0);

        \node[anchor=west] (text) at (-4,-1.5){Innenwinkel: $\alpha + \beta +\gamma = \pi$};
        \node[anchor=west] (text) at (-4,-2.3){Flächeninhalt: $A = \frac{1}{2}ab \sin\gamma$};
        \node[anchor=west] (text) at (-4,4){$\beta < \frac{\pi}{2}$};
        \draw[thick, {latex}-{latex}] (-4,-.5) --node[below]{$p$} +(5.5/1.41,0); 
        \draw[thick, {latex}-{latex}] (-4+5.5/1.41,-.5) --node[below]{$q$} (2,-.5); 

        \begin{scope}[shift={(6,0)}]
            \coordinate (A) at (-3,0){};
            \coordinate (B) at (2,0){};
            \coordinate (C) at ($(A)+(30:7.5)$){};
            \draw[thick] (A) --node[below]{$c$} (B) --node[right]{$a$} (C) --node[above left]{$b$} (A);
            \draw ($(A)+(1.2,0)$) arc (0:30:1.2);
            \draw ($(B)+(-0.9,0)$) arc (180:70:.9);
            \draw ($(C) + (180+30:.9)$) arc (180+30:180+30+40:.9);
            \node (D) at ($(B) + (125:.6)$){$\beta$};
            \node (D) at ($(A) + (15:.9)$){$\alpha$};
            \node (D) at ($(C) + (180+30+20:.6)$){$\gamma$}; 
            \draw[gray, dashed] (A) --node[above]{$h_a$} +(-22:5.5);
            \draw[gray, dashed] (B) --node[above]{$h_b$} +(120:3);
            \draw[gray, dashed] (C) --node[right]{$h_c$} +(0,-4);
            \draw[gray] (B) -- +(2,0);
            \draw[gray] (B) -- +(180+68:3.5);
            
        \node[anchor=west] (text) at (-3,4){$\beta > \frac{\pi}{2}$};
        \end{scope}
    \end{tikzpicture}
    \caption{Geometrie eines allgemeinen schiefwinkligen Dreiecks für die beiden Fälle $\beta < \frac{\pi}{2}$ und $\beta > \frac{\pi}{2}$. Die Höhen der einzelnen Seiten sind jeweils auch grau gestrichelt eingezeichnet.}
    \label{fig:07_Dreieck}
\end{figure}

In beiden Varianten lässt sich folgendes ablesen: 
\begin{align}
    \begin{rcases}
        h_a = c \sin(\beta) = b \sin(\gamma) \\
        h_b = c \sin(\alpha) = a \sin(\gamma) \\
        h_c = a \sin(\beta) = b \sin(\alpha)
    \end{rcases} \quad \Rightarrow \quad \frac{a}{\sin(\alpha)} = \frac{b}{\sin(\beta)} = \frac{c}{\sin(\gamma)}.
\end{align}
Dieses Ergebnis wird \emph{Sinussatz} genannt. Die Seiten eines Dreiecks verhalten sich also zueinander wie die Sinus der gegenüberliegenden Winkel.

\paragraph{Kosinussatz}$~$

Wir wollen uns nun noch einmal o.\,B.\,d.\,A. mit dem Fall $\beta < \frac{\pi}{2}$ beschäftigen und das Dreieck (siehe Abb.~\ref{fig:07_Dreieck} links) entlang der Höhe $h_c$ in zwei Teildreiecke aufteilen. Dann gilt nach Pythagoras 
\begin{align}
    \begin{rcases}
        a^2 = h_c^2 + q^2 \\
        b^2 = h_c^2 + p^2
    \end{rcases} \quad a^2 = (b^2-p^2) + q^2 \overset{q = c-p}{=} b^2 + c^2 -2cp.
\end{align}
Nutzen wir außerdem noch $p = b \cos(\alpha)$, so folgt daraus 
\begin{mymathbox}[ams align, title={Kosinussatz}, colframe={FSUblau}]
    \begin{split}
        a^2 &= b^2 + c^2 - 2bc \cos(\alpha)\\
        b^2 &= c^2 + a^2 - 2ac \cos(\beta)\\
        c^2 &= a^2 + b^2 - 2ab \cos(\gamma).
    \end{split}
\end{mymathbox}
Die anderen Gleichungen erhalten wir durch zyklische Vertauschung der Variablen $a,b,c$ und $\alpha,\beta,\gamma$.

Formulieren wir den Kosinussatz noch einmal in Worten: Das Quadrat einer Seite ist gleich der Summe der Quadrate der beiden anderen Seiten, verringert um das doppelte Produkt diesr beiden Seiten mit dem Kosinus des von ihnen eingeschlossenen Winkels.

Wir wollen abschließend noch anmerken, dass der Kosinussatz ebenfalls auch für den Fall $\beta > \frac{\pi}{2}$ gilt. Er stellt eine Verallgemeinerung des Satzes des Pythagoras für allgemeine Dreiecke dar.

\clearpage
\paragraph{Beispiel: Heron'sche Inhaltsformel (Heron von Alexandria)}$~$

Wir wollen in diesem Abschnitt eine Formel für den Flächeninhalt eines Dreiecks angeben, welches nur durch seine Seitenlängen bestimmt ist. Beginnen wir beim Kosinussatz und formen dies etwas um, so folgt 
\begin{align}
    \cos(\alpha) &= \frac{b^2 + c^2 - a^2}{2bc} \notag \\
    \Leftrightarrow 1+\cos(\alpha) &= \frac{b^2 + c^2 -a^2 +2bc}{2bc} = \frac{(b+c)^2 -a^2}{2bc} \overset{\eqref{eqn:1_binomische_Formeln}}{=} \frac{(b+c+a)(b+c-a)}{2bc} \notag \\
     2 \cos^2 \qty(\frac{\alpha}{2}) &= \frac{(b+c+a)(b+c-a)}{2bc}. 
\end{align}
Die letzte Gleichheit folgt dabei aus den Formeln für die Doppelwinkelfunktionen~\eqref{eqn:07_Doppelwinkel_cos}. Führen wir nun die Variable 
\begin{align}
    s = \frac{1}{2}(a+b+c)
\end{align}
als halben Umfang des Dreiecks ein, so erhalten wir mit 
\begin{align}
    s-a = \frac{b+c-a}{2}, \quad s-b = \frac{a+c-b}{2}, \quad s-c = \frac{a+b-c}{2}
\end{align}
durch zyklisches Durchtauschen folgendes Ergebnis: 
\begin{align}
    \Rightarrow 2\cos^2\qty(\frac{\alpha}{2}) = \frac{2s \cancel{2}(s-a)}{\cancel{2}bc} \quad \Rightarrow \quad \cos^2\qty(\frac{\alpha}{2}) &= \frac{s(s-a)}{bc} \notag \\
    \Rightarrow \quad \cos^2\qty(\frac{\beta}{2}) &= \frac{s(s-b)}{ac} \\
    \Rightarrow \quad \cos^2\qty(\frac{\gamma}{2}) &= \frac{s(s-c)}{ab}. \notag
\end{align}
Auf dem selben Weg folgen aus der Relation 
\begin{align}
    1- \cos(\alpha) &= \frac{a^2 -b^2 -c^2 +2bc}{2bc} \notag \\
    \Rightarrow \sin^2\qty(\frac{\alpha}{2}) &= \frac{(s-b)(s-c)}{bc}, \quad \sin^2\qty(\frac{\beta}{2}) = \frac{(s-a)(s-c)}{ac}, \quad \sin^2\qty(\frac{\gamma}{2}) = \frac{(s-a)(s-b)}{ab}.
\end{align}
Damit ergibt sich insgesamt für den Flächeninhalt des Dreiecks 
\begin{align}
    A = \frac{1}{2} ab \sin(\gamma) \overset{\eqref{eqn:07_Doppelwinkel_sin}}&{=} ab \sin(\frac{\gamma}{2})\cos(\frac{\gamma}{2}) = \cancel{ab} \sqrt{\frac{(s-a)(s-b)}{\cancel{ab}}} \sqrt{\frac{s(s-c)}{\cancel{ab}}} \notag \\
    &= \sqrt{s(s-a)(s-b)(s-c)}.
\end{align}

\thispagestyle{plain}
\section{Grundlagen der Differentialrechnung (Ableiten)}

\begin{wrapfigure}{r}{8cm}
    \centering
    \vspace{-5mm}
        \begin{tikzpicture}
            \begin{axis}[disabledatascaling, axis lines=middle, xtick={1}, xticklabels={$x_0$}, ytick={0.83}, yticklabels={$f(x_0)$}, xlabel={$x$}, ylabel={$y$}, height=6cm, width=9cm, ymax= 5, ymin=-2, samples=100]
                \addplot[no marks, FSUblau, thick, domain=-2:3]{0.5*x^2 + 0.33*x^3 - x +1};
                \addplot[no marks, PAForange, thick, domain=-1:3]{x+0.83-1};
                \legend{Funktion $f(x)$, Tangente}
            \end{axis}
        \end{tikzpicture}
    \vspace{-5mm}
\end{wrapfigure}

Wir möchten in diesem Kapitel die Frage stellen, wie man den Anstieg einer beliebigen Funktion $f(x)$ an einem Punkt $x=x_0$ bestimmen kann. Dabei meinen wir den Anstieg der Geraden, die am Punkt $x_0$ als Tangente angelegt wird. Da das in jedem beliebigen Punkt $x=x_0$ möglich ist, ist der Anstieg selbst wieder eine Funktion von $x$ die wir Ableitung $f'(x)$ nennen wollen.

\paragraph{Konstruktion der Ableitung}$~$

\begin{figure}[htp]
    \centering
    \begin{tikzpicture}
        \begin{axis}[disabledatascaling, axis lines=middle, xtick={0.576, 1.951}, xticklabels={$x_0$,$x_0 + \varepsilon$}, ytick={0.652,3.402}, yticklabels={$f(x_0)$, $f(x_0 +\varepsilon)$}, xlabel={$x$}, ylabel={$y$}, height=6cm, width=0.8\textwidth, ymax= 5, ymin=-2, legend pos=outer north east, samples=100]
            \addplot[no marks, FSUblau, thick, domain=-2:3]{0.5*x^2 + 0.33*x^3 - x +1};
            \addplot[no marks, PAForange, thick, domain=-1:3]{2*x-0.5};
            \legend{Funktion $f(x)$, Sekante}
            \draw[dashed, gray] (0.576,0) -- (0.576,0.652) -- (0,0.652);
            \draw[dashed, gray] (1.951,0) -- (1.951,3.402) -- (0,3.402);
            \draw [decorate, decoration = {calligraphic brace}, thick] (0.2,3.402)--node[right]{\scriptsize$f(x_0+\varepsilon)-f(x_0)$}(0.2,0.652);
            \draw [decorate, decoration = {calligraphic brace}, thick] (1.951,-1)--node[below]{$\varepsilon$}(0.576,-1);
        \end{axis}
    \end{tikzpicture}
    \caption{Für die Konstruktion der Ableitung am Punkt $x_0$ legen wir zunächst eine Sekante des Graphen durch den Punkt $(x_0, f(x_0))$ und einen Punkt $(x_0 + \varepsilon, f(x_0+\varepsilon))$ und bilden anschließend den Grenzwert $\varepsilon \to 0$.} 
\end{figure}

Wir definieren die Ableitung als Grenzwert des Differenzenquotienten: 
\begin{align}
    f'(x) := \lim_{\varepsilon \to 0} \frac{f(x_0 + \varepsilon)-f(x_0)}{\varepsilon}.
\end{align}
Wir können die erhaltene Ableitungsfunktion erneut ableiten und erhalten damit die zweite Ableitung. Für Ableitungen $n$-ter Ordnung schreiben wir schließlich 
\begin{align}
    &\text{1. Ableitung:} \qquad f'(x) = \dv{f(x)}{x} \equiv \qty(\dv{f}{x})(x) \notag \\
    &\text{2. Ableitung:} \qquad f''(x) = \dv[2]{f(x)}{x} \notag \\
    &\text{n. Ableitung:} \qquad f^{(n)}(x) = \dv[n]{f(x)}{x}.
\end{align}
Ableitungen spezieller Funktionen sind zum Beispiel 
\begin{align}
    \begin{split}
        f(x) &= a =\text{const.} \quad \Rightarrow \quad f'(x) = \dv{x}(a) = 0, \\
        f(x) &= x \hphantom{= const.} \quad \Rightarrow \quad f'(x) = \dv{x}(x) = \lim_{\varepsilon \to 0} \frac{(x+\varepsilon) - x }{\varepsilon} = 1.
    \end{split}
\end{align}

\subsection{Allgemeine Eigenschaften}

Der Operator $\dv{x}$ weißt bestimmte Eigenschaften auf, die wir als Ableitungsregeln bezeichnen: 
\begin{itemize}
    \item Linearität: ($a,b \in \mathbb{R}$)
    \begin{align}
        \dv{x}(a f(x) + b g(x)) = a \dv{f}{x} + b \dv{g}{x}
    \end{align}
    \item Produktregel (Leibniz-Regel):
    \begin{align}
        \dv{x}(f(x)\cdot g(x)) = \dv{f}{x} \cdot g(x) + f(x) \cdot \dv{g}{x}.
    \end{align}
    \item Kettenregel: \vspace{-0.5cm}
    \begin{align}
        \dv{x} f(g(x)) = \underbrace{\qty(\dv{f}{g})(g(x))}_{\mathclap{\text{äußere Ableitung}}}\cdot \overbrace{\dv{g}{x}}^{\mathclap{\text{innere Ableitung}}}.
    \end{align}
    \item Quotientenregel: 
    \begin{align}
        \dv{x}\qty(\frac{f(x)}{g(x)}) = \frac{1}{g(x)^2} \qty(\dv{f}{x}\cdot g(x) - f(x) \cdot \dv{g}{x}).
    \end{align}
    \item Potenzregel: 
    \begin{align}
        \dv{x}\qty(x^n) = n \, x^{n-1} \quad \Rightarrow \quad \text{insbesondere: } \dv{x}\qty(\frac{1}{x^n}) = - \frac{n}{x^{n+1}.}
    \end{align}
\end{itemize}
Wenden wir einige dieser Regeln mal an einem praktischen Beispiel an: 
\begin{align}
    f(x) &=x \cdot g(x) + 7 h(y(x)) + \frac{x^2}{j(x)} \qq{mit} y(x) = ax + b \notag \\
    \dv{f}{x} &= \underbrace{\dv{x}{x} \cdot g(x) + x \cdot \dv{g}{x}}_{\text{Produktregel}} + \underbrace{7 \qty(\dv{h}{y})(y) \cdot \dv{y}{x}}_{\text{Kettenregel}} + \underbrace{\frac{2 x j(x) - x^2 \dv{j}{x}}{j(x)^2}}_{\text{Quotientenregel}}\notag \\
    &= g(x) +  x \cdot \dv{g}{x} + 7a \qty(\dv{h}{y})(y)  + \frac{2x j(x) - x^2 \dv{j}{x}}{j(x)^2}.
\end{align}
Wichtig ist es, bei der Kettenregel nach der richtigen Variable abzuleiten, d.\,h. wir müssen $h(y)$ nach dem Argument $y = ax+ b$ ableiten und nicht nach $x$. 


\subsection{Ableitungen spezieller Funktionen}

Wir wollen im Folgenden eine Liste von häufig verwendeten Funktionen und deren Ableitungen angeben. 

\begin{table}[htp]
    \centering
    \caption{Ableitungen spezieller Funktionen}
    \begin{tabular}[t]{l l}
        \toprule
        $f(x)$ & $f'(x)$ \\
        \midrule 
        $x^n$ & $n x^n$ \\
        $\sqrt{x}$ & $\dfrac{1}{2\sqrt{x}}$ \\
        $\ln(x)$ & $\dfrac{1}{x}$ \\
        $\sin(x)$ & $\cos(x)$ \\
        $\cos(x)$ & $-\sin(x)$ \\ 
        $\tan(x)$ & $1 + \tan^2(x)$ \\
        $\exp(x)$ & $\exp(x)$ \\
        $a^x$ & $\ln(a) a^x$ 
    \end{tabular}
    \hspace{1cm}
    \begin{tabular}[t]{l l}
        \toprule
        $f(x)$ & $f'(x)$ \\
        \midrule 
        $\arcsin(x)$ & $\dfrac{1}{\sqrt{1-x^2}}$ \\
        $\arccos(x)$ & $\dfrac{-1}{\sqrt{1-x^2}}$ \\
        $\arctan(x)$ & $\dfrac{1}{1+x^2}$ \\
        $\sinh(x)$ & $\cosh(x)$ \\
        $\cosh(x)$ & $\sinh(x)$ \\
    \end{tabular}
\end{table}

Schauen wir uns mal ein Beispiel an: 
\begin{align}
    f(x) &= \sin(\sqrt[n]{x^3}) \cdot \ln(\cos(x)-3\e^{x\cdot \ln(x)}) \notag \\
    \dv{f}{x} &= \dv{x}\qty[\sin(\sqrt[n]{x^3})]\cdot \ln(\cos(x)- 3\e^{x\ln(x)}) + \sin(\sqrt[n]{x^3}) \dv{x} \ln\underbrace{\big(\cos(x)-3\e^{x\cdot \ln(x)}\big)}_{\equiv A(x)} \notag \\
    &= \ln(A(x)) \cdot \cos(\sqrt[n]{x^3}) \cdot \dv{x}\underbrace{(\sqrt[n]{x^3})}_{x^{3/n}} + \sin(\sqrt[n]{x^3}) + \sin(\sqrt[n]{x^3}) \frac{1}{A(x)} \dv{A}{x} \notag \\
    &= \ln(A(x)) \cdot \cos(\sqrt[n]{x^3}) \cdot \frac{3}{n} \cdot \underbrace{x^{\frac{3}{n}-1}}_{\mathclap{x^{\frac{3-n}{n}} =\sqrt[n]{x^{3-n} = \sqrt[n]{x^3}\cdot x^{-1}}}} + \sin(\sqrt[n]{x^3}) \frac{1}{A(x)}\qty[-\sin(x)-3 \dv{x}\e^{x \ln(x)}] \notag\\
    &= \frac{3 \ln(A(x)) \sqrt[n]{x^3}}{nx}\cos(\sqrt[n]{x^3}) - \frac{\sin(\sqrt[n]{x^3})}{A(x)} \qty[\sin(x)+3 \e^{x \ln(x)}(\ln(x)+1)]
\end{align}

\newpage
\subsection{Kurvendiskussion} 
Verschwindet die Ableitung an einem Punkt, $f'(x) = 0$, dann bedeutet dies, dass dort ein lokales Extremum (bzw. ein Sattelpunkt) vorliegt. Bei einem Maximum ist die zweite Ableitung $f''(x_0) < 0$, während sie bei einem lokalen Minimum $f''(x_0) >0$ ist. Für den Fall $f''(x_0) = 0$ liegt ein Sattelpunkt vor, falls $f^{(3)}(x_0) \neq 0$ ist. Ansonsten muss durch das Bilden von höheren Ableitungen weiter entschieden werden\footnote{Ein Beispiel dafür ist $f(x) = x^4$. Im Nullpunkt liegt ein Minimum vor, jedoch verschwinden dort die ersten drei Ableitungen.}. 

\begin{figure}[htp]
    \centering
    \begin{tikzpicture}
        \begin{axis}[disabledatascaling, axis lines=middle, xtick={2.5}, xticklabels={$x_0$}, ytick, xlabel={$x$}, ylabel={$y$}, height=6cm, width=0.4\textwidth, xmin = 0, xmax=5, ymin=0, ymax=5]
            \addplot[no marks, FSUblau, thick, domain=0:5]{-0.5*(x-2.5)^2+4};
            \draw[thick] (1,4) -- (4,4);
            \draw[thick, dashed] (2.5,0) -- (2.5,4);
        \end{axis}
        \node (A) at (2.3,-1){$f''(x_0) <0$ Maximum};
    \end{tikzpicture}
    \hfill
    \begin{tikzpicture}
        \begin{axis}[disabledatascaling, axis lines=middle, xtick={2.5}, xticklabels={$x_0$}, ytick, xlabel={$x$}, ylabel={$y$}, height=6cm, width=0.4\textwidth, xmin = 0, xmax=5, ymin=0, ymax=5]
            \addplot[no marks, FSUblau, thick, domain=0:5]{0.5*(x-2.5)^2+1};
            \draw[thick] (1,1) -- (4,1);
            \draw[thick, dashed] (2.5,0) -- (2.5,1);
            
        \end{axis}
        \node (A) at (2.3,-1){$f''(x_0) < 0$ Minimum};
    \end{tikzpicture}
    \hfill
    \begin{tikzpicture}
        \begin{axis}[disabledatascaling, axis lines=middle, xtick={2.5}, xticklabels={$x_0$}, ytick, xlabel={$x$}, ylabel={$y$}, height=6cm, width=0.4\textwidth, xmin = 0, xmax=5, ymin=0, ymax=5]
            \addplot[no marks, FSUblau, thick, domain=0:5]{0.2*(x-2.5)^3+2.5};
            \draw[thick] (1,2.5) -- (4,2.5);
            \draw[thick, dashed] (2.5,0) -- (2.5,2.5);
            
        \end{axis}
        \node (A) at (2.3,-1){$f''(x_0) = 0$ Sattelpunkt};
    \end{tikzpicture}
    \caption{lokale Extremstellen einer Funktion abhängig vom Vorzeichen der zweiten Ableitung $f''(x_0)$. Für den Sattelpunkt muss außerdem noch gelten: $f^{(3)}(x_0) \neq 0$.}
\end{figure}

Das Vorzeichen der zweiten Ableitung gibt Auskunft über die Krümmung der Kurve:
\begin{figure}[htp]
    \centering
    \begin{tikzpicture}
        \begin{axis}[disabledatascaling, axis lines=middle, xtick={2,2.5,3}, xticklabels={$x_1$,$x_2$,$x_3$}, ytick, xlabel={$x$}, ylabel={$y$}, height=6cm, width=0.55\textwidth, xmin=1,xmax=4.5,ymin=0,ymax=10]
            \addplot[no marks, FSUblau, very thick, domain=0:3.7]{(x-1)^2+1} node[above]{$f(x)$};
            \fill (2,2) circle (2pt);
            \fill (3,5) circle (2pt);
            \fill (2.5,3.25) circle (2pt);
            \addplot[no marks, PAForange, domain=0:3.7]{2*x-2}node[right]{$f'(x_1)$};
            \addplot[no marks, PAForange, domain=0:3.7]{3*x-4.25}node[right]{$f'(x_2)$};
            \addplot[no marks, PAForange, domain=0:3.7]{4*x-7}node[right]{$f'(x_3)$};

            \draw[gray, dashed] (2,0) -- (2,2);
            \draw[gray, dashed] (2.5,0) -- (2.5,3.25);
            \draw[gray, dashed] (3,0) -- (3,5);
            \node (A) at (2,7){konvex};
        \end{axis}
    \end{tikzpicture}
    \hfill 
    \begin{tikzpicture}
        \begin{axis}[disabledatascaling, axis lines=middle, xtick={2,2.5,3}, xticklabels={$x_1$,$x_2$,$x_3$}, ytick, xlabel={$x$}, ylabel={$y$}, height=6cm, width=0.55\textwidth, xmin=1,xmax=4.5,ymin=0,ymax=10]
            \addplot[no marks, FSUblau, very thick, domain=0:3.7]{-(x-1)^2+8} node[left]{$f(x)$};
            \fill (2,8-1) circle (2pt);
            \fill (3,8-4) circle (2pt);
            \fill (2.5,8-2.25) circle (2pt);
            \addplot[no marks, PAForange, domain=0:3.7]{-2*x+2+9}node[right]{$f'(x_1)$};
            \addplot[no marks, PAForange, domain=0:3.7]{-3*x+4.25+9}node[right]{$f'(x_2)$};
            \addplot[no marks, PAForange, domain=0:3.7]{-4*x+7+9}node[right]{$f'(x_3)$};

            \draw[gray, dashed] (2,0) -- (2,8-1);
            \draw[gray, dashed] (2.5,0) -- (2.5,8-2.25);
            \draw[gray, dashed] (3,0) -- (3,8-4);
            \node (A) at (3.5,7){konkav};
        \end{axis}
    \end{tikzpicture}
    \caption{Links: Beispiel für eine konvexe Kurve. Der Anstieg $f'(x)$ nimmt zu, es gilt also $f''(x) > 0$ \\
    Rechts: Beispiel für eine konkave Kurve. Der Anstieg $f'(x)$ nimmt ab, es gilt also $f''(x) < 0$}
\end{figure}
\thispagestyle{plain}
\section{Die Methode der vollständigen Induktion}

Wir wollen in diesem Kapitel ein wichtiges Beweisverfahren der Mathematik einführen, die \emph{vollständige Induktion}. Dabei wollen wir zunächst das Beweisverfahren selbst beweisen:

\begin{satz}[Vollständige Induktion]
    Eine Aussage ist für jede natürliche Zahl $n \in \mathbb{N}$ richtig, wenn 
    \begin{enumerate}
        \item sie für $n=1$ richtig ist \emph{und}
        \item aus der Richtigkeit der Aussage für eine willkürliche natürliche Zahl $n=k$ die Richtigkeit für $n=k+1$ folgt.
    \end{enumerate}
\end{satz}

\begin{proof}
    Annahme: Eine Aussage sei \emph{nicht} für jede natürliche Zahl gültig, \emph{obwohl} (1) und (2) gelten. $\Rightarrow$ Dann existiert ein $m$, sodass die Aussage für $n=m$ falsch ist und für $n < m$ richtig. 

    Aber: Die Aussage gilt für $n=1$. Also muss $m > 1$ sein. Dann ist $m-1$ eine natürliche Zahl, für die die Aussage richtig ist, ohne es für die darauffolgende zu sein. Das stellt allerdings einen Widerspruch zu Annahme (2) dar. \\
    $\Longrightarrow$ Die Annahme ist falsch (Beweis durch Widerspruch).
\end{proof}

Für eine bessere Strukturierung des Induktionsbeweises wollen wir nochmal die Schritte notieren, die uns zum erfolgreichen Beweis führen:
\begin{enumerate}
    \item Induktionsanfang (IA): Zeige, dass die Aussage für einen bestimmten Startwert z.\,B. $n=0,1$ gültig ist.
    \item Induktionsvoraussetzung (IV): Wir nehmen an, die Aussage sei für $n=k$ gültig und notieren sie. 
    \item Induktionsbehauptung (IB): Wir notieren wie die Aussage für $n=k+1$ lautet. 
    \item Beweis: Für führen den Induktionsschritt aus und benutzen die Induktionsvoraussetzung, um die Aussage für $n=k+1$ zu zeigen.
\end{enumerate}

Wir wollen im Folgenden ein paar Beispiele angeben:

\paragraph{Summe: $\displaystyle S_n = \frac{1}{1\cdot 2} + \frac{1}{2\cdot 3}+ \frac{1}{3\cdot 4} + \hdots + \frac{1}{n\cdot (n+1)} \overset{?}{=} \frac{n}{n+1}$}$~$

\begin{enumerate}
    \item[(IA)] $\displaystyle n=1: \quad S_1 = \frac{1}{1\cdot 2} = \frac{1}{1+1} = \frac{1}{2} .\quad\checkmark$ 
    \item[(IV)] $\displaystyle n=k: \quad S_k = \frac{k}{k+1}$
    \item[(IB)] $\displaystyle n=k+1:\quad S_{k+1} = \frac{k+1}{k+2}$
    \begin{proof}$~$\\[-1.5cm]
        \begin{align}
            S_{k+1} &= S_k + \frac{1}{(k+1)(k+2)} \overset{(\text{IV})}{=} \frac{k}{k+1} + \frac{1}{(k+1)(k+2)} = \frac{1}{k+1} \qty(k+\frac{1}{k+2}) \notag \\
            &= \frac{1}{k+1} \frac{k^2+2k+1}{k+2} = \frac{(k+1)^{\cancel{2}}}{\cancel{(k+1)}(k+2)} = \frac{k+1}{k+2}
        \end{align}
    \end{proof}
\end{enumerate}

\paragraph{Summe der ersten $n$ ungeraden Zahlen} $S_n = 1 +3 + \hdots + (2n-1) \overset{?}{=} n^2$

\begin{enumerate}
    \item[(IA)] $\displaystyle n=1: \quad S_1 = 1 = 1^2 .\quad\checkmark$ 
    \item[(IV)] $\displaystyle n=k: \quad S_k = k^2$
    \item[(IB)] $\displaystyle n=k+1:\quad S_{k+1} = (k+1)^2$
    \begin{proof}$~$\\[-1.5cm]
        \begin{align}
            S_{k+1} &= S_k + (2(k+1)-1) = k^2 + 2k +1 = (k+1)^2
        \end{align}
    \end{proof}
\end{enumerate}

\paragraph{Gauß'sche Summenformel} $S_n = 1 +2+3+\hdots + n = \sum_{m=1}^n m \overset{?}{=} \frac{n(n+1)}{2}$

\begin{enumerate}
    \item[(IA)] $\displaystyle n=1: \quad S_1 = \frac{1(1+1)}{2} = 1 .\quad\checkmark$ 
    \item[(IV)] $\displaystyle n=k: \quad S_k = \frac{k(k+1)}{2}$
    \item[(IB)] $\displaystyle n=k+1:\quad S_{k+1} = \frac{(k+1)(k+2)}{2}$
    \begin{proof}$~$\\[-1.5cm]
        \begin{align}
            S_{k+1} &= S_k + (k+1) = (k+1)\qty(\frac{k}{2}+1) = \frac{(k+1)(k+2)}{2}.
        \end{align}
    \end{proof}
\end{enumerate}

\paragraph{Summe der Quadratzahlen} $S_n = 1^2 + 2^2 + 3^2+\hdots + n^2 = \sum_{m=1}^n m^2 \overset{?}{=} \frac{n(n+1)(2n+1)}{6}$

\begin{enumerate}
    \item[(IA)] $\displaystyle n=1: \quad S_1 = \frac{1(1+1)(2+1)}{6} = 1 .\quad\checkmark$ 
    \item[(IV)] $\displaystyle n=k: \quad S_k = \frac{k(k+1)(2k+1)}{6}$
    \item[(IB)] $\displaystyle n=k+1:\quad S_{k+1} = \frac{(k+1)(k+2)(2k+3)}{6}$
    \begin{proof}$~$\\[-1.5cm]
        \begin{align}
            S_{k+1} &= S_k + (k+1)^2 = \frac{1}{6}\qty[k(k+1)(2k+1)+6(k+1)^2] \notag \\
            &= \frac{k+1}{6}\qty[k(2k+1)+6(k+1)] = \frac{k+1}{6}\qty(2k^2+7k+6).
        \end{align}
        \begin{align}
            \begin{array}{r r@{} r@{}  r@{} r}
                \text{Polynomdivision}&(2k^2 &{}+7k\hp{)}&{}+6) &\;:(k+2) = 2k+3 \\
              &-(2k^2 &{}+4k) \\ 
              \cmidrule{2-3}
                    & & 3k \hp{)} &{}+6\hp{)}\\
                    & &-(3k\hp{)}&{}+6) \\
              \cmidrule{3-4}
                    & & & 0
            \end{array}
        \end{align}
    \end{proof}
    Alternativ lässt sich die Behauptung zeigen, indem man das Ergebnis und den Anfang jeweils so weit wie möglich ausmultipliziert und damit die Gleichheit beider Terme zeigt.
\end{enumerate}

\paragraph{Endliche geometrische Reihe} $S_n = 1+ x+x^2+ x^3+ \hdots + x^n =\sum_{m=0}^n x^m = \frac{x^{n+1}-1}{x-1}$ für $x\neq 1$ 

\begin{enumerate}
    \item[(IA)] $\displaystyle n=0: \quad S_0 = 1, \quad n=1: \quad S_1 = \frac{x^2 -1}{x-1} = \frac{\cancel{(x-1)}(x+1)}{\cancel{x-1}} = x+1 .\quad\checkmark$ 
    \item[(IV)] $\displaystyle n=k: \quad S_k = \frac{x^{k+1}-1}{x-1}$
    \item[(IB)] $\displaystyle n=k+1:\quad S_{k+1} = \frac{x^{k+2}-1}{x-1}$
    \begin{proof}$~$\\[-1.5cm]
        \begin{align}
            S_{k+1} &= S_k + x^{k+1} = \frac{x^{k+1}-1}{x-1} + x^{k+1} = \frac{x^{k+1}-1 + (x-1)x^{k+1}}{x-1} \notag \\
            &= \frac{\cancel{x^{k+1}}-1 + x^{k+2}-\cancel{x^{k+1}}}{x-1}.
        \end{align}
    \end{proof}
\end{enumerate}

\paragraph{Bernoulli'sche Ungleichung} $(1+\alpha)^n > 1 + n\alpha \qquad (\alpha > -1, \alpha \neq 0, n>1)$ 

\begin{enumerate}
    \item[(IA)] $\displaystyle n=2: \quad (1+\alpha)^2 = 1 + 2\alpha + \alpha^2 > 1+2\alpha \Rightarrow \alpha^2 > 0. \quad\checkmark$ 
    \item[(IV)] $\displaystyle n=k: \quad (1+\alpha)^k > 1 + k\alpha \quad$
    \item[(IB)] $\displaystyle n=k+1:\quad (1+\alpha)^{k+1} > 1+(k+1)\alpha$
    \begin{proof}$~$\\[-1.5cm]
        \begin{align}
            (1+\alpha)^{k+1} &= (1+\alpha)^k (1+\alpha) \notag \\
            \overset{(\text{IV})}&{>} (1+k\alpha)(1+\alpha) \qq{da $1+\alpha >0$} \notag \\
            &= 1 + \alpha + k\alpha + k\alpha^2 = 1 + (k+1)\alpha + \underbrace{k \alpha^2}_{>0} \notag \\[-2mm]
            &> 1+(k+1)\alpha. 
        \end{align}
    \end{proof}
\end{enumerate}

\section{Arithmetische und geometrische Reihen}

Eine \emph{Folge} ist eine Liste nummerierter Objekte (endlich oder unendlich viele).
\begin{align}
    \qq{Schreibweise:} (a_k)_{k=1,\hdots,n} \qq{oder} (a_k)_{k\in\mathbb{N}}.
\end{align}
Folgen können definiert werden durch explizite Angabe der Folgenglieder, z.\,B. 
\begin{align}
    (a_k)_{k=1,\hdots,4} = (2,3,5,7), \qq{oder} a_1 = 2, a_2=3, a_3=5, a_4 = 7,
\end{align}
oder durch eine (explizite oder rekursive) Bildungsvorschrift wie z.\,B. 
\begin{align}
    a_k = 2^k \qq{,} k\in\mathbb{N}.
\end{align}
Eine \emph{Reihe} ist eine Liste von Summen aus Folgengliedern, also 
\begin{align}
    (s_n)_{n\in\mathbb{N}} \qq{, wobei} s_n = \sum_{k=1}^n a_k  \quad \leftarrow \qq{``Partialsummen''.}
\end{align}
Eine Reihe ist selbst wieder eine Folge. 

\paragraph{Beispiel:}$~$ Sei $(a_k)_{k=1}^4$ die Folg der ersten vier Primzahlen, also $a_1 = 2, a_2 = 3, a_3 = 5, a_4 = 7$. Dann sind die Partialsummen gegeben als 
\begin{alignat}{3}
    s_1 &= \sum_{k=1}^1 a_k = a_1 = 2, \quad & s_2&= \sum_{k=1}^2 a_k = a_1+a_2 = 5, \notag \\
    s_3 &= \sum_{k=1}^3 a_k = a_1 + a_2 + a_3 = 10, \quad & s_4&= \sum_{k=1}^4 a_k = a_1 + a_2 + a_3 + a_4 = 17.
\end{alignat}
Also ist die Folge der Reihe: 
\begin{align}
    (s_n)_{n=1}^4 = (2,5,10,17).
\end{align}

\subsection{Arithmetische Reihen}
Bei einer arithmetischen Folge ist die Differenz $d$ zwei aufeinander folgender Glieder konstant. Wir können die Folge entweder \emph{rekursiv} oder \emph{explizit} definieren 
\begin{subequations}
    \begin{alignat}{3}
            &\qq{rekursiv:}\quad &a_{k+1} &= a_k + d \\
            &\qq{explizit:}\quad &a_k &= a_0 + k\cdot d,
    \end{alignat}
\end{subequations}
wobei $a_0$ das Anfangsglied der Folge ist. Die Glieder einer arithmetischen Reihe sind nun die Partialsummen einer arithmetischen Folge: 
\begin{align}
    s_0 &= a_0 \notag \\
    s_1 &= a_0 + (a_0 + d) \notag \\
    s_2 &= a_0 + (a_0 + d) + (a_0 + 2d) \notag \\
    s_n &= a_0 + (a_0 + d) + \hdots + (a_0 + nd) = \sum_{k=0}^n (a_0 + k d). 
\end{align}
Wir können durch vollständige Induktion zeigen, dass für das $n$-te Glied der Reihe gilt: 
\begin{mymathbox}[ams align, title={Arithmetische Reihe}, colframe={FSUblau}]
    s_n = (n+1) \qty(a_0 + n \frac{d}{2}).
\end{mymathbox}
\begin{enumerate}
    \item[(IA)] $\displaystyle n=0: \quad s_n = a_0$ 
    \item[(IV)] $\displaystyle n=k: \quad s_k = (k+1)\qty(a_0 + k \frac{d}{2})$
    \item[(IB)] $\displaystyle n=k+1:\quad s_{k+1} = (k+2)\qty(a_0 + (k+1)\frac{d}{2})$
    \begin{proof}$~$\\[-1.5cm]
        \begin{align}
           s_{k+1} &= s_k + a_0 + (k+1) d = (k+1)a_0 + k(k+1)\frac{d}{2} + a_0 + (k+1)d \notag \\
           &= (k+2) a_0 + (k+1)(k+2)\frac{d}{2} = (k+2)\qty[a_0 + (k+1)\frac{d}{2}]. 
        \end{align}
    \end{proof}
\end{enumerate}
Es lässt sich $s_n$ auch durch $a_n$ ausdrücken. Dadurch eliminieren wir $d$: 
\begin{align}
    a_n &= a_0 + n d \quad \Rightarrow \quad nd = a_n - a_0 \notag \\
    \Rightarrow s_n &= (n+1)\qty(a_0 + \frac{a_n -a_0}{2}) = \uuline{\frac{n+1}{2}(a_0 + a_n)}.
\end{align}
Wir können an dieser Darstellung durch das Anfangs- und Endglied bereits die Namensgebung der arithmetischen Folge/Reihe erahnen: 
\begin{align}
    a_{k+1} &= a_0 + (k+1)d \notag \\
    a_{k-1} &= a_0 + (k-1)d \notag \\
    \Rightarrow a_{k+1} + a_{k-1} &= 2 a_0 +2kd = 2\underbrace{(a_0 + kd)}_{a_k} \notag \\[-2mm]
    \Rightarrow a_k &= \uuline{\frac{1}{2}(a_{k+1}+a_{k-1})}.
\end{align}
Die Glieder der arithmetischen Folge sind gleich dem arithmetischen Mittel (s.u.) ihrer Nachbarglieder.

\subsection{Geometrische Reihen}
Bei einer arithmetischen Folge ist der Quotient $q$ zwei aufeinander folgender Glieder konstant. Wir können die Folge entweder \emph{rekursiv} oder \emph{explizit} definieren 
\begin{subequations}
    \begin{alignat}{3}
            &\qq{rekursiv:}\quad &a_{k+1} &= q \cdot a_k \\
            &\qq{explizit:}\quad &a_k &= q^k \cdot a_0,
    \end{alignat}
\end{subequations}
wobei $a_0$ das Anfangsglied der geometrischen Folge darstellt. Die Gleider einer geometrischen Reihe sind die Partialsummen einer geometrischen Folge: 
\begin{align}
    s_0 &= a_0 \notag \\
    s_1 &= a_0 + q a_0 = (1+q) a_0 \notag \\
    s_2 &= a_0 + q a_0 + q^2 a_0 = (1+q+q^2) a_0 \notag \\
    s_n &= (1+q+q^2 + \hdots + q^n) a_0 = a_0 \sum_{k=0}^n q^k. 
\end{align}
Wir können, abhängig von $q$, unterschiedliche Fälle unterscheiden: 
\begin{align}
    q &> 0 \quad \begin{cases}
        q > 1: & \text{Die Glieder der Folge werden größer.} \\
        q = 1: & \text{Alle Glieder der Folge sind gleich.} \\
        q < 1: & \text{Die Glieder der Folge werden kleiner.}
    \end{cases} \\
    q &< 0: \quad \text{Die Folge alterniert.}
\end{align}
Wir zeigen durch vollständige Induktion, dass für das $n$-te Glied der Reihe gilt: 
\begin{mymathbox}[ams align, title={Arithmetische Reihe}, colframe={FSUblau}]
    s_n = a_0 \frac{1-q^{n+1}}{1-q} \qq{wobei} q \neq 1.
\end{mymathbox}
\begin{enumerate}
    \item[(IA)] $\displaystyle n=0: \quad s_n = a_0$ 
    \item[(IV)] $\displaystyle n=k: \quad s_k = a_0 \frac{1-q^{k+1}}{1-q}$
    \item[(IB)] $\displaystyle n=k+1:\quad s_{k+1} = a_0 \frac{1-q^{k+2}}{1-q}$
    \begin{proof}$~$\\[-1.5cm]
        \begin{align}
           s_{k+1} &= s_k + a_{k+1} = a_0 \frac{1-q^{k+1}}{1-q} + a_0 q^{k+1} \notag \\
           &= \frac{a_0}{1-q} \qty[1-q^{k+1} + (1-q)q^{k+1}] = \frac{a_0}{1-q}\qty(1-\cancel{q^{k+1}} + \cancel{q^{k+1}} - q^{k+2}) \notag \\
           &= a_0 \frac{1-q^{k+2}}{1-q}. 
        \end{align}
    \end{proof}
\end{enumerate}
Es lässt sich $s_k$ auch durch das Endglied $a_n$ ausdrücken (Eliminierung der Potenz): 
\begin{align}
    a_n &= a_0 q^k \quad \Rightarrow \quad q^n = \frac{a_n}{a_0} \notag\\
    \Rightarrow s_n = a_0 \frac{1-q \frac{a_n}{a_0}}{1-q} = \uuline{\frac{a_0 - q a_n}{1-q}}.
\end{align}
Wir können wieder anhand dieser Darstellung durch das Anfangs- und Endglied bereits die Namensgebung der geometrischen Folge/Reihe erahnen: 
\begin{align}
    a_{k+1} &= a_0 q^{k+1} \notag \\
    a_{k-1} &= a_0 q^{k-1} \notag \\
    \Rightarrow a_{k+1} \cdot a_{k-1} &= a_0^2 q^{2k} = (a_0 q^k)^2 = a_k^2 \notag \\
    \Rightarrow a_k &= \uuline{\sqrt{a_{k+1}\cdot a_{k+1}}}.
\end{align}
Die Glieder der geometrischen Folge sind gleich dem geometrischen Mittel (s.u.) ihrer Nachbarglieder.

\paragraph{Bemerkung}$~$

Für $|q| < 1$ gilt $\lim_{n\to\infty}(q^{n+1}) = 0$. Demnach nimmt dann die unendliche geometrische Reihe einen Wert an, 
\begin{mymathbox}[ams align, title={Unendliche geometrische Reihe}, colframe={FSUblau}]
    \sum_{n=0}^\infty q^n = \frac{1}{1-q} \qq{für} |q| < 1.
\end{mymathbox}
Anders herum betrachtet haben wir damit eine Reihendarstellung der Funktion 
\begin{align}
    f(x) = \frac{1}{1-x}, \qq{für} 0 < x < 1
\end{align}
gefunden.

\newpage
\subsection{Arithmetisches und geometrisches Mittel}
Es seien $a_1, a_2$ zwei reelle Zahlen. 
\begin{itemize}
    \item \emph{arithmetisches Mittel} von $a_1$ und $a_2$: $A = \dfrac{a_1+a_2}{2}$ 
    \item \emph{geometrisches Mittel} von $a_1$ und $a_2$: $G = \sqrt{a_1 a_2}$.
\end{itemize}
Es gilt allgemein 
\begin{align}
    G \le A.
\end{align}
\begin{proof}$~$\\[-1.5cm]
    \begin{alignat}{3}
        && \sqrt{a_1 a_2} &\le \frac{a_1 + a_2}{2} \notag \\
        &\Longleftrightarrow &4 a_1 a_2 &\le (a_1+a_2)^2 = a_1^2 + a_2^2 + 2a_1 a_2 \notag \\
        &\Longleftrightarrow & 0 &\le a_1^2 +a_2^2 - 2a_1 a_2 = (a_1-a_2)^2.  
    \end{alignat}
\end{proof}

\paragraph{geometrische Interpretation}$~$

\begin{wrapfigure}{r}{6cm}
    \centering
    \vspace{-5mm}
    \begin{tikzpicture}
        \fill (0,0) circle (2pt);
        \draw (0:2.5) arc (0:180:2.5);
        \draw[thick] ($(cos{70}*2.5,sin{70}*2.5)$) --node[left]{$G$} ($(cos{70}*2.5,0)$);
        \draw[thick, dashed] (180:2.5) --node[left]{$u$} ($(cos{70}*2.5,sin{70}*2.5)$);
        \draw[thick, dashed] (0:2.5) --node[right]{$v$} ($(cos{70}*2.5,sin{70}*2.5)$);
        \draw[thick, FSUblau] (180:2.5) --node[below]{$a_1$} ($(cos{70}*2.5,0)$);
        \draw[thick, PAForange] (0:2.5) --node[below]{$a_2$} ($(cos{70}*2.5,0)$);
        \draw[thick, {latex}-{latex}] (-2.5,-.5) --node[below]{$A$} (0,-.5);
    \end{tikzpicture}
    \vspace{-5mm}
\end{wrapfigure}

Für einen Punkt auf einem Halbkreis bildet das Dreieck, welches aus dem Punkt und den beiden Endpunkten gebildet wird, stets ein rechtwinkliges Dreieck (Thales-Kreis). Es gilt dann 
\begin{align}
    u^2 + v^2 &= (a_1 + a_2)^2 \notag \\
    (G^2 + a_1^2) + (G^2 + a_2^2) &= (a_1 + a_2)^2 \notag \\
    2G^2 + \cancel{a_1^2} + \cancel{a_2}^2 &= \cancel{a_1^2} + \cancel{a_2}^2 + 2 a_1 a_2 \notag \\
    \Rightarrow G = \sqrt{a_1 a_2}. 
\end{align}
Das arithmetische Mittel ist also der Radius des Kreis mit Durchmesser $(a_1+a_2)$. Das geometrische Mittel ist die halbe Länge der Sehne, senkrecht zum Durchmesser, in dem Punkt, an dem $a_1$ und $a_2$ aneinander stoßen. Mit der geometrischen Interpretation wird auch direkt offensichtlich, warum $G \le A$ gelten muss.

\section{Der binomische Satz}

Der binomische Satz (auch: Binomialtheorem) ermöglicht die Entwicklung von Binomen $(a+b)^n$ in Potenzen von $a$ und $b$, also das ``Ausmultiplizieren''. 

\subsection{Binomialkoeffizienten}
Für zwei natürliche Zahlen $n,k \in \mathbb{N}_0$ ist der Binomialkoeffizient $\binom{n}{k}$, sprich: ``$n$ über $k$'', definiert als 
\begin{align}
    \binom{n}{k} := \frac{n(n-1)(n-2)\hdots (n-k+1)}{k!}.
\end{align}
Beachte, dass der Binomialkoeffizient nur von Null verschieden ist, wenn $k \le n$. Betrachten wir einige Beispiele: 
\begin{align}
    \begin{split}
        \binom{7}{3} &= \frac{7\cdot 6\cdot 5}{3!} = 35, \quad \binom{10}{7} = \frac{10\cdot 9 \cdot 8 \cdot 7 \cdot 6 \cdot 5 \cdot 4}{7!} = 120, \\
        \binom{3}{5} &= \frac{3 \cdot 2 \cdot 1 \cdot 0 \cdot (-1)}{5!} = 0.
    \end{split}
\end{align}
Sofern $k \le n$ gilt, kann man schreiben: 
\begin{align}
    \binom{n}{k} = \frac{n(n-1)\hdots (n-k+1)}{k!} \cdot \underbrace{\frac{(n-k)(n-k-1)\hdots 2 \cdot 1}{(n-k)!}}_{1} = \frac{n!}{k!(n-k)!}.
\end{align}
Eine interessante Eigenschaft des Binomialkoeffizienten ist, dass sich der Wert nicht ändern, wenn man $k$ durch $n-k$ ersetzt:
\begin{mymathbox}[ams align, title={Binomialkoeffizent}, colframe={FSUblau}]
\binom{n}{k} = \frac{n!}{k!(n-k)!} \qq{,} \binom{n}{k} = \binom{n}{n-k} \qq{für} k \le n.
\end{mymathbox}

Wir betrachten im Folgenden einige Spezialfälle: 
\begin{align}
    \binom{n}{0} &= \frac{n!}{n! 0!} = 1, \notag \\
    \binom{n}{1} &= \frac{n!}{1!(n-1)!} = \frac{n(n-1)!}{(n-1)!} =n, \\
    \binom{n}{n} &= \frac{n!}{n! 0!} = 1. \notag
\end{align}
Außerdem gilt die Rekursionsrelation 
\begin{align}
    \binom{n}{k} + \binom{n}{k+1} &= \binom{n+1}{k+1}, \\
    \qq{denn:} \binom{n}{k} + \binom{n}{k+1} &= \frac{n(n-1)\hdots (n-k+1)}{k!} + \frac{n(n-1)\hdots (n-k)}{(k+1)!} \notag \\
    &= \frac{n(n-1)\hdots (n-k+1)}{(k+1)!} \qty[(\cancel{k}+1)+(n-\cancel{k})] \notag \\
    &= \frac{(n+1)n(n-1) \hdots (n-k+1)}{(k+1)!} = \binom{n+1}{k+1}.
\end{align}

Das Produkt $n(n-1)\hdots(n-k+1)$ wird auch als ``fallende Faktorielle'' oder ``absteigendes Pochhammer-Symbol'' bezeichnet und mit $[n]_k, (n)_k$ oder $n^{\underline{k}}$ abgekürzt.

Eine Verallgemeinerung der Binomialkoeffizienten für reelle $n$ (und natürliche $k$) ist mit oben gegebener Definition ohne Weiteres möglich. Beachte, dass $\binom{n}{k}$ dann auch für $k>n$ von Null verschiedene Werte annimmt, bspw. 
\begin{align}
    \binom{1/2}{4} = \frac{1/2\cdot (-1/2)\cdot (-3/2) \cdot (-5/2)}{4!} = -\frac{5}{128}. 
\end{align}
Die Binomialkoeffizienten können (sowohl in $n$ als auch in $k$) als Folge aufgefasst werden; sie ist weder arithmetisch noch geometrisch. 

\subsection{Der binomische Satz}

Mithilfe des Binomialkoeffizienten des letzten Abschnitts können wir nun den binomsichen Satz formulieren: 
\begin{mymathbox}[ams align, title={binomischer Lehrsatz}, colframe={FSUblau}]
    (a+b)^n = \sum_{k=0}^n \binom{n}{k} a^{n-k} b^k \qq{für} a,b \in \mathbb{R}, n\in\mathbb{N}_0
\end{mymathbox}
Wir wollen im Folgenden den Satz der Induktion beweisen. Dafür betrachten wir zunächst die Formel für $n=0,1,2$ 
\begin{enumerate}
    \item[(IA)] $\displaystyle n=0: \quad  \binom{0}{0} a^0 b^0 = 1\quad \checkmark$ \\
    $\displaystyle n=1: \quad  \binom{1}{0} a^1 b^0 + \binom{1}{1} a^0 b^1= a+b \quad\checkmark$ \\
    $\displaystyle n=1: \quad  \binom{2}{0} a^2 b^0 + \binom{2}{1} a^1 b^1 + \binom{2}{2} a^0 b^2= a^2 + 2ab +b^2 \quad\checkmark$
    \item[(IV)] $\displaystyle n=l: \quad (a+b)^l = \sum_{k=0}^l \binom{l}{k} a^{l-k} b^k$
    \item[(IB)] $\displaystyle n=l: \quad (a+b)^{l+1} = \sum_{k=0}^{l+1} \binom{l+1}{k} a^{l+1-k} b^k$\\
    \begin{proof}$~$\\[-1.65cm]
        \begin{align}
            \quad(a+b)^{l+1} &= (a+b)^l (a+b) \overset{\text{(IV)}}{=} \sum_{k=0}^l \binom{l}{k} a^{l-k} b^k (a+b) \notag \\
            &= \sum_{k=0}^l \binom{l}{k} a^{l+1-k} b^k + \sum_{k=0}^l \binom{l}{k} a^{l-k} b^{k+1} \notag \\
            &\qquad \text{Indextransformation } m=k+1 \Rightarrow k=m-1: \notag \\
            &\qquad \sum_{m=1}^{l+1} \binom{l}{m-1} a^{l+1-m} b^m \qq{,} \qq{Umbennennung} m \to k \notag \\
            &= \binom{l}{0} a^{l+1} + \sum_{k=1}^l \qty[\binom{l}{k} + \binom{l}{k-1}] a^{l+1-k} b^k + \binom{l}{l} b^{l+1} \notag \\
            &= a^{l+1} + \sum_{k=1}^l \binom{l+1}{k} a^{l+1-k} b^k + b^{l+1} \notag \\
            & \qquad \text{für } k=0: \quad \binom{l+1}{0} a^{l+1} = \binom{l}{0} a^{l+1} = a^{l+1}. \notag \\ 
            & \qquad \hp{für } k=l+1: \quad \binom{l+1}{l+1} b^{l+1} = \binom{l}{l} b^{l+1} = b^{l+1}. \notag \\
            &= \sum_{k=0}^{l+1} \binom{l+1}{k} a^{l+1-k} b^k.  
        \end{align}
    \end{proof}
\end{enumerate}
\section{Rechnen mit Vektoren und Matrizen}

Vektoren und Matrizen kommen in allen Teilbereichen der Physik fundamentale Rollen zu. Hier betrachten wir den dreidimensionalen euklidischen Raum $\mathbb{R}^3$, bestehend aus Punkten, die durch Angabe ihrer Koordinaten $x,y,z$ in einem kartesischen Koordinatensystem gekennzeichnet sind. 

\begin{wrapfigure}{r}{5cm}
    \centering
    \vspace{-5mm}
    \begin{tikzpicture}
        \draw[thick, -{latex}] (0,0) -- (2.6,0)node[right]{$y$};
        \draw[thick, -{latex}] (0,0) -- (-1.6,-1.6)node[below]{$x$};
        \draw[thick, -{latex}] (0,0) -- (0,2.6)node[right]{$z$};
        \draw[thick, PAForange, -{latex}] (0,0) -- (1.5,1); 
        \draw[dashed, gray] (0,0) -- (1.5,-1);
        \draw[dashed, gray] (-1,-1) -- (1.5,-1);
        \draw[dashed, gray] (2.5,0) -- (1.5,-1);
        \draw[dashed, gray] (1.5,1) -- (1.5,-1);
        \draw[dashed, gray] (0,2) -- (1.5,1);
        \fill (1.5,1) circle (1.5pt);
    \end{tikzpicture}
    \vspace{-5mm}
\end{wrapfigure}
Ziehen wir eine Verbindungslinie vom Koordinatenursprung zu einem Punkt mit den Koordinaten $(x;y;z)$, dann entspricht das dem \emph{Ortsvektor} dieses Punktes und wir schreiben\footnote{Wir verwenden die häufig in Büchern benutzte Notation, Vektoren \textbf{fett} zu schreiben, statt mit einem Vektorpfeil.}
\begin{align}
    \bm{r} = \mqty(x\\y\\z) \qq{,} \bm{r} \in \mathbb{R}^3.
\end{align}
Vorteil einer solchen vektoriellen Größe ist, dass sie neben ihrem Betrag auch Information über die Richtung (inkl. Orientierung) enthält.

\subsection{Grundlagen der Vektorrechnung}

Offenbar können Vektoren \emph{addiert} bzw. \emph{subtrahiert} werden, 
\begin{align}
    \bm{r}_1 + \bm{r}_2 = \bm{r}_3 = \bm{r}_2 + \bm{r}_1 \qq{bzw.} \bm{r}_1 - \bm{r}_2 = \bm{r}_3.
\end{align} 
\begin{figure}[htp]
    \centering
    \begin{tikzpicture}
        \draw[thick, FSUblau, -{latex}] (0,0) --node[below]{$\bm{r}_1$} (3,0);
        \draw[thick, FSUblau, -{latex}] (3,0) --node[right]{$\bm{r}_2$} (4.5,2);
        \draw[thick, FSUblau, -{latex}] (0,0) --node[above]{$\bm{r}_3$} (4.5,2); 

        \begin{scope}[shift={(6,0)}]
            \draw[thick, FSUblau, -{latex}] (0,0) --node[below]{$\bm{r}_1$} (3,0);
        \draw[thick, FSUblau, -{latex}] (0,0) --node[left]{$\bm{r}_2$} (1.5,2);
        \draw[thick, FSUblau, -{latex}] (1.5,2) --node[right]{$\bm{r}_3$} (3,0);
        \end{scope}
    \end{tikzpicture}
    \caption{Addition (links) und Subtraktion (rechts) von zwei Vektoren $\bm{r}_1$ und $\bm{r}_2$.}
\end{figure}

Jeder dreidimensionale Vektor kann als Linearkombination dreier \emph{Basisvektoren} geschrieben werden, 
\begin{align}
    \bm{r} = \mqty(x\\y\\z) = x \vu{e}_x + y \vu{e}_y + z \vu{e}_z;
\end{align}
damit gilt bei Addition 
\begin{align}
    \bm{r}_1 + \bm{r}_2 = \mqty(x_1\\y_1\\z_1) + \mqty(x_2\\y_2\\z_2) = \mqty(x_1+x_2\\y_1+y_2\\z_1+z_2) = (x_1+x_2)\vu{e}_x + (y_1+y_2)\vu{e}_y + (z_1+z_2)\vu{e}_z.
\end{align}
Der \emph{Betrag} (die Länge) eines Vektors ist definiert als 
\begin{align}
    |\bm{r}| = \sqrt{x^2+y^2+z^2} \ge 0 \qq{(dreidimensionaler Pythagoras).}
\end{align}
Speziell gilt für die Basisvektoren $|\vu{e}_x| = |\vu{e}_y| = |\vu{e}_z| = 1$ sowie für denn Nullvektor $|\bm{0}| = 0$. Es gilt zudem die \emph{Dreiecksungleichung}
\begin{align}
    |\bm{r}_1 + \bm{r}_2| \le |\bm{r}_1| + |\bm{r}_2|. 
\end{align} 
Vektoren können mit Skalaren (``Zahlen'' ohne Richtung, hier: Elemente des $\mathbb{R}$) multipliziert werden. 

Jedem Vektor kann ein \emph{Einheitsvektor} zugeordnet werden, 
\begin{align}
    \vu{e}_r = \frac{\bm{r}}{|\bm{r}|} \qq{,} \qq{sodass} |\vu{e}_r| = 1.
\end{align}

\subsection{Das Vektorprodukt}

Das Vektorprodukt (auch: Kreuzprodukt oder äußeres Produkt) bietet eine Möglichkeit, Vektoren miteinander zu multiplizieren, im Sinne eines äußeren Produktes (``Vektor mal Vektor gleich Vektor'')
\begin{align}
    &\text{Schreibweise: } & \bm{r}_1 \times \bm{r}_2 &= \bm{r}_3 \notag \\
    &\text{Konstruktion: } & \bm{r}_1 \times \bm{r}_2 &= \mqty(x_1\\y_1\\z_1) \times \mqty(x_2\\y_2\\z_2) = \mqty(y_1 z_2 - y_2 z_1 \\ z_1 x_2 - z_2 x_1 \\ x_1 y_2 - x_2 y_1). 
\end{align}
Das Vektorprodukt $ \bm{r}_1 \times \bm{r}_2$ steht senkrecht sowohl auf $\bm{r}_1$ als auch auf $\bm{r}_2$ und seine Richtung ist rechtsdrehend positiv (Rechte-Hand-Regel, Korkenzieher-Regel).

Für den Betrag des Vektorproduktes gilt:
\begin{align}
    | \bm{r}_1 \times \bm{r}_2| = |\bm{r}_1| \cdot |\bm{r}_2|\cdot \sin(\sphericalangle(\bm{r}_1,\bm{r}_2)).
\end{align}

Der Betrag des Vektorproduktes ist gleich dem Flächeninhalt des durch die Vektoren aufgespannten Parallelogramms. 
\begin{figure}[htp]
    \centering
    \begin{tikzpicture}
        \draw[thick, FSUblau, -{latex}] (0,0) -- (0,2)node[left]{$\bm{r}_1\times \bm{r}_2$};
        \fill[FSUblau, fill opacity = 0.3] (0,0) -- (1.4,1.2) --+(3,0.5) -- (3,0.5) --cycle;
        \draw[thick, FSUblau, -{latex}] (0,0) -- (1.4,1.2)node[above]{$\bm{r}_2$};
        \draw[thick, FSUblau, -{latex}] (0,0) -- (3,0.5)node[below]{$\bm{r}_1$};
        \draw (0.5,1/12) arc (0:90:0.52);
        \fill (0.25,0.3) circle (1pt); 
        \node[rotate=12] (A) at (2.2,0.85){$A = |\bm{r}_1 \times \bm{r}_2|$};
    \end{tikzpicture}
\end{figure}
Beachte: Das Vektorprodukt kann in dieser Form nur in drei Dimensionen existieren. 

\paragraph{Algebraische Eigenschaften des Vektorprodukts}$~$

\begin{itemize}
    \item nicht kommutativ, $\bm{a}\times \bm{b} \neq \bm{b} \times \bm{a}$, 
    \item dafür \emph{anti-kommutativ}: $\bm{a}\times\bm{b} = - \bm{b}\times \bm{a} \quad (\Rightarrow \bm{a}\times\bm{a} = \bm{0})$
    \item nicht assoziativ, $\bm{a}\times (\bm{b}\times \bm{c}) \neq (\bm{a}\times\bm{b})\times\bm{c}$; 
    \item distributiv, $\bm{a} \times (\bm{b}+\bm{c}) = \bm{a}\times\bm{b} + \bm{a}\times \bm{c}.$
\end{itemize}

Eine wichtige algebraische Eigenschaft ist die \emph{Jacobi-Identität}, 
\begin{align}
    \bm{a} \times (\bm{b}\times\bm{c}) + \bm{b} \times (\bm{c}\times\bm{a}) + \bm{c}\times (\bm{a}\times\bm{b}) = \bm{0}.
\end{align}
Weiterhin lauten die Vektorprodukte der Basisvektoren: 
\begin{align}
    \vu{e}_x \times \vu{e}_y = \vu{e}_z, \quad \vu{e}_y \times \vu{e}_z = \vu{e}_x, \quad \vu{e}_z \times \vu{e}_x = \vu{e}_y.
\end{align}


\subsection{Das Skalarprodukt}

Das Skalarprodukt ist eine Projektion zweier Vektoren aufeinander und stellt ein inneres Produkt dar (``Vektor mal Vektor gleich nicht-Vektor''), da es ein Skalar (hier aus $\mathbb{R}$) ergibt. Es ist 
\begin{align}
    \bm{r}_1 \cdot \bm{r}_2 = \mqty(x_1\\y_1\\z_1) \cdot \mqty(x_2\\y_2\\z_2) = x_1x_2 + y_1y_2+z_1z_2.
\end{align}
\begin{figure}[htp]
    \centering
    \begin{tikzpicture}
        \draw[thick, FSUblau, -{latex}] (0,0) -- (3.5,0)node[above right]{$\bm{r}_1$};
        \draw[thick, FSUblau, -{latex}] (0,0) -- (30:5)node[above]{$\bm{r}_2$}; 
        \draw[thick, {latex}-{latex}] ($(0,0)+(120:.2)$) --node[above,rotate=30]{$\frac{\bm{r}_1\cdot \bm{r}_2}{|\bm{r}_2|}$} +(30:cos{30}*3.5);
        \draw[dashed] ($(0,0)+(120:.4)+(30:cos{30}*3.5)$) -- (3.5,0);
        \draw[dotted] (3.5,0) -- (5,0);
        \draw[dashed] (30:5) -- +(0,-2.5);
        \draw[thick, {latex}-{latex}] (0,-.2) --node[below]{$\frac{\bm{r}_1 \cdot \bm{r}_2}{|\bm{r}_1|}$} (5*cos{30},-.2);
    \end{tikzpicture}
    \caption{Geometrische Bedeutung des Skalarproduktes. Das Skalarprodukt ist die Länge der Projektion des Vektors $\bm{r}_1$ auf $\bm{r}_2$ (oder umgekehrt) multipliziert mit der Länge des Vektors, auf den projiziert wird.}
\end{figure}
Wir können daraus schlussfolgern: 
\begin{itemize}
    \item Stehen zwei Vektoren senkrecht aufeinander, dann ist ihr Skalarprodukt Null, 
    \begin{align}
        \bm{a} \perp \bm{b} \quad \Longleftrightarrow \quad \bm{a}\cdot \bm{b} = 0. 
    \end{align}
    \item Der Nullvektor steht senkrecht auf allen Vektoren. 
    \item Der Betrag eines Vektors kann mit Hilfe des Skalarproduktes geschrieben werden, 
    \begin{align}
        |\bm{a}| = \sqrt{\bm{a}\cdot\bm{a}}.
    \end{align}
\end{itemize}
Das Skalarprodukt kann auch geschrieben werden als 
\begin{align}
    \bm{r}_1 \cdot \bm{r}_2 = |\bm{r}_1|\cdot|\bm{r}_2| \cdot \cos(\sphericalangle (\bm{r}_1,\bm{r}_2)). 
\end{align}

\paragraph{Algebraische Eigenschaften des Skalarproduktes}$~$

\begin{itemize}
    \item kommutativ, $\bm{a}\cdot \bm{b} = \bm{b}\cdot\bm{a}$; 
    \item nicht assoziativ\footnote{Assoziativität kann hier gar nicht vorliegen, da es sich um ein inneres Produkt handelt; es existiert kein Skalarprodukt aus drei Faktoren.}, $\bm{a}\cdot (\bm{b}\cdot\bm{c}) \neq (\bm{a}\cdot\bm{b})\cdot\bm{c}$;
    \item distributive, $\bm{a}\cdot (\bm{b}+\bm{c}) = \bm{a}\cdot\bm{b}+\bm{a}\cdot\bm{c}$.
\end{itemize}
Mit Hilfe von Skalar- und Vektorprodukt kann das Volumen des durch drei Vektoren aufgespannten Spates berechnet werden (\emph{Spatprodukt}): 
\begin{align}
    V = |(\bm{a}\times\bm{b}) \cdot\bm{c}|.
\end{align}

\subsection{Lineare Unabhängigkeit}
Eine Menge von Vektoren $\{\bm{a}_i \in \mathbb{R}^3\}$ heißt linear unabhängig, wenn sich keiner der Vektoren $\bm{a}_i$ als Linearkombination der übrigen Vektoren $\bm{a}_j, j \neq i,$ schreiben lässt. 

Eine Menge von Vektoren heißt linear abhängig, wenn sie nicht linear unabhängig ist. 

Anders gesagt: Lineare Abhängigkeit liegt genau dann vor, wenn der Nullvektor als Linearkombination 
\begin{align}
    \bm{0} = \sum_i \alpha_i \bm{a}_i
\end{align}
geschrieben werden kann, ohne, dass alle $\alpha_i$ Null sind.

Feststellungen: 
\begin{itemize}
    \item Im dreidimensionalen euklidischen Raum $\mathbb{R}^3$ können nicht mehr als drei Vektoren linear unabhängig sein. 
    \item Vektoren, die in einer Ebene liegen, sind linear abhängig, sofern es mehr als zwei sind. 
    \item Enthält eine Menge von Vektoren den Nullvektor, dann ist sie linear abhängig. 
    \item Ist das Skalarprodukt zweier Vektoren Null, dann sind diese beiden Vektoren linear unabhängig, sofern keiner der beiden der Nullvektor ist. 
    \item Die Basisvektoren $\vu{e}_x, \vu{e}_y, \vu{e}_z$ sind linear unabhängig.
\end{itemize}

\emph{Bsp.:} Die Vektoren $\bm{r}_1 = \mqty(1\\0\\1), \bm{r}_2 = \mqty(2\\1\\4), \bm{r}_3 = \mqty(0\\1/2\\1)$ sind linear abhängig, da $\bm{r}_1 - \frac{1}{2}\bm{r}_2 + \bm{r}_3 = \bm{0}$.

\subsection{Grundlagen der Matrix-Rechnung}

Eine Matrix $A$ ist eine Zusammenfassung von Elementen $a_{ij}$ in Form einer Tabelle (hier: zweidimensional). 

Eine $(m\times n)$-Matrix besitzt $m$ Zeilen und $n$ Spalten und das Element $a_{ij}$ befindet sich in der $i$-ten Zeile der $j$-ten Spalte ($i= 1,\hdots,m\qq{;} j=1,\hdots,n$). Man schreibt: 
\begin{align}
    A = (a_{ij}) = 
        \underbrace{\begin{rcases}\mqty(a_{11} & a_{12} & a_{13} & \hdots & a_{1n} \\
        a_{21} & a_{22} & a_{23} & \hdots & a_{2n} \\
        a_{31} & a_{32} & a_{33} & \hdots & a_{3n} \\
        \vdots & \vdots & \vdots & \ddots & \vdots \\
        a_{m1} & a_{m2} & a_{m3} & \hdots & a_{mn})
        \end{rcases}}_{n \text{ Spalten}} 
        \quad m \text{ Zeilen} \qq{,} a_{ij} \in \mathbb{R}.
\end{align}

\begin{itemize}
    \item Jeder Spaltenvektor kann als $(n\times 1)$-Matrix aufgefasst werden. 
    \item Eine $(n\times n)$-Matrix heißt \emph{quadratische Matrix}. 
    \item Hat eine quadratische Matrix nur Einträge auf der Hauptdiagonalen, $a_{ij} = 0$ für $i\neq j$, dann heißt sie \emph{Diagonalmatrix}. Man schreibt kurz: 
    \begin{align}
        A = \text{diag}(a_{11}, a_{22}, a_{33}, \hdots, a_{nn}).
    \end{align}
    \item Die Matrix $\mathds{1}_n = \text{diag}(1,1,\hdots,1)$ heißt \emph{Einheitsmatrix}.
\end{itemize}

Eine Matrix $A$ kann mit einem Skalar $k \in \mathbb{R}$ multipliziert werden 
\begin{align}
    k\cdot A = k\cdot \mqty(a_{11} & a_{12} & \hdots & a_{1n} \\
    a_{21} & a_{22} & \hdots & a_{2n} \\
    \vdots & \vdots & \ddots & \vdots \\
    a_{m1} & a_{m2} & \hdots & a_{mn}) = \mqty(k \cdot a_{11} & k \cdot a_{12} & \hdots & k \cdot a_{1n} \\
    k \cdot a_{21} & k \cdot a_{22} & \hdots & k \cdot a_{2n} \\
    \vdots & \vdots & \ddots & \vdots \\
    k \cdot a_{m1} & k \cdot a_{m2} & \hdots & k \cdot a_{mn}).
\end{align}
Zwei Matrizen $A,B$ können addiert/subtrahiert werden, 
\begin{align}
    A \pm B &= \mqty(a_{11} & a_{12} & \hdots & a_{1n} \\
    a_{21} & a_{22} & \hdots & a_{2n} \\
    \vdots & \vdots & \ddots & \vdots \\
    a_{m1} & a_{m2} & \hdots & a_{mn}) \pm \mqty(b_{11} & b_{12} & \hdots & b_{1n} \\
    b_{21} & b_{22} & \hdots & b_{2n} \\
    \vdots & \vdots & \ddots & \vdots \\
    b_{m1} & b_{m2} & \hdots & b_{mn}) \notag \\
    &= \mqty(a_{11} \pm b_{11} & a_{12} \pm b_{12}& \hdots & a_{1n} \pm b_{1n}\\
    a_{21} \pm b_{21}& a_{22} \pm b_{22}& \hdots & a_{2n} \pm b_{2n}\\
    \vdots & \vdots & \ddots & \vdots \\
    a_{m1} \pm b_{m1}& a_{m2} \pm b_{m2}& \hdots & a_{mn}\pm b_{mn}).
\end{align}

Die \emph{transponierte Matrix} ergibt sich durch Vertauschung von Zeilen und Spalten (bzw. Spiegelung an der Diagonalen), 
\begin{align}
    A^T = \mqty(a_{11} & a_{12} & \hdots & a_{1n} \\
    a_{21} & a_{22} & \hdots & a_{2n} \\
    \vdots & \vdots & \ddots & \vdots \\
    a_{m1} & a_{m2} & \hdots & a_{mn})^T = \mqty(a_{11} & a_{21} & \hdots & a_{m1} \\
    a_{12} & a_{22} & \hdots & a_{m2} \\
    \vdots & \vdots & \ddots & \vdots \\
    a_{1n} & a_{2n} & \hdots & a_{mn}).
\end{align}
\begin{itemize}
    \item Transponieren verwandelt eine ($m\times n$)-Matrix in eine  $(n\times m)$-Matrix.
    \item Transponieren verwandelt einen Spaltenvektor $\bm{x} = \mqty(x_1 \\ \vdots \\ x_n)$ in einen Zeilenvektor $\bm{x}^T = (x_1,x_2,\hdots, x_n)$. 
    \item Eine Matrix heißt \emph{symmetrisch}, wenn gilt: $A^T = A$.
    \item Eine Matrix heißt \emph{antisymmetrisch}, wenn gilt: $A^T = -A$.\\
    $\Rightarrow $ Jede Diagonalmatrix ist symmetrisch.
\end{itemize}
Als \emph{Spur} (engl. trace) einer quadratischen Matrix bezeichnet man die Summe der Diagonalelemente 
\begin{align}
    \trace(A) = \sum_{i=1}^n a_{ii}.
\end{align}
\emph{Beispiel für Matrixoperationen:}
\begin{align}
    A &= \mqty(\alpha & \beta & 0 \\ -\beta &\alpha & 0 \\ 0 & 0 & 1) \qq{,} B = \mqty(\alpha & -\beta & 0 \\ \beta &\alpha & 0 \\ 0 & 0 & 1) \notag \\
    \qq{Summe:} A+ B &= \mqty(\dmat[0]{2\alpha, 2\alpha, 2}) = 2 \mqty(\dmat[0]{\alpha,\alpha,1}) \\
    (A+B)^T &= \mqty(\dmat[0]{2\alpha, 2\alpha, 2}) = A + B \quad \Rightarrow \quad A+B \qq{symmetrisch.} \notag \\
    \trace(A+B) &= 4\alpha +2 = \trace{A} + \trace{B}. \notag \\
    \qq{Differenz:} A- B &= \mqty(0 & 2\beta & 0 \\ -2\beta & 0 & 0 \\ 0 & 0 &0) = 2\beta \mqty(0 & 1 & 0 \\ -1 & 0 & 0 \\ 0 & 0 &0) \\
    (A-B)^T &= \mqty(0 & -2\beta & 0 \\ 2\beta & 0 & 0 \\ 0 & 0 &0) = -(A - B) \quad \Rightarrow \quad A-B \qq{antisymmetrisch.} \notag \\
    \trace(A-B) &= 0 = \trace(A) - \trace(B).\notag
\end{align}

\subsection{Die Matrixmultiplikation}

Wir möchten ein assoziatives Produkt von Matrizen definieren. 

\emph{Definition:} Sei $A = (a_{ij})$ eine $(m\times n)$-Matrix und sei $B = (b_{ij})$ eine $(n\times a)$-Matrix. Dann ist das Produkt $A\cdot B = C$ eine $(m\times a)$-Matrix $C = (c_{ij})$ mit Einträgen 
\begin{align}
    c_{ij} = \sum_{k=1}^n a_{ik} b_{kj} \qq{;} i=1,\hdots,m; \quad j = 1,\hdots,a. 
\end{align}

Das heißt: Es ergibt sich das Element in $i$-ter zeile und $j$-ter Spalte der resultierenden Matrix, indem die Elemente der $i$-ten Zeile der ersten Matrix mit denen der $j$-ten Spalte der zweiten Matrix multipliziert und aufsummiert werden. 

Beispielsweise ergibt sich für $(3\times 3)$-Matrizen: 
\begin{align}
    \mqty(a_{11} & a_{12} & a_{13}\\ a_{21} & a_{22} & a_{23} \\ a_{31} & a_{32} & a_{33}) \cdot \mqty(b_{11} & b_{12} & b_{13}\\ b_{21} & b_{22} & b_{23} \\ b_{31} & b_{32} & b_{33}) = \mqty(
    a_{11} b_{11} + a_{12} b_{21} + a_{13} b_{31} & a_{11} b_{12} + a_{12} b_{22} + a_{13} b_{32} & \hdots\\
    a_{21} b_{11} + a_{22} b_{21} + a_{23} b_{31} & a_{21} b_{12} + a_{22} b_{22} + a_{23} b_{32} & \hdots\\
    a_{31} b_{11} + a_{32} b_{21} + a_{33} b_{31} & a_{31} b_{12} + a_{32} b_{22} + a_{33} b_{32} & \hdots).
\end{align}
Wichtig ist, dass die Spaltenzahl der ersten Matrix gleich der Zeilenzahl der zweiten Matrix ist. 
\begin{align}
    \qq{Beispiel:} \mqty(1 & 0 & 1 \\ 2 & 1 & -1 \\ 1 & 3 & 0) \cdot \mqty(3 & 7 & 1 \\ 2 & 0 & 5 \\ 1 & 0 & 1) &=\mqty(3+0+1 & 7+0+0 & 1+0+1 \\ 6+2-1 & 14+0+0 & 2+5-1 \\ 3+6+0 & 7+0+0 & 1+15+0) \notag \\
    &= \mqty(4 & 7 & 2 \\ 7&14&6\\9&7&16). 
\end{align}
Man kann das Skalarprodukt zweier Vektoren $\bm{r}_1, \bm{r}_2$ auffassen als das Produkt des Transponierten von $\bm{r}_1$ mit $\bm{r}_2$: 
\begin{align}
    (\bm{r}_1)^T \cdot (\bm{r}_2) = \mqty(x_1 & y_1 & z_1) \cdot \mqty(x_2 \\y_2 \\z_2) = x_1 x_2 + y_1y_2 + z_1z_2.
\end{align} 
Die Einheitsmatrix ist das Einselement der Matrixmultiplikation, 
\begin{align}
    A \cdot \mathds{1}_n = \mathds{1}_n \cdot A = A, \qquad A:(n\times n)\text{-Matrix.}
\end{align}

\paragraph{Algebraische Eigenschaften der Matrixmultiplikation}$~$

\begin{itemize}
    \item nicht kommutativ, $A \cdot B \neq B\cdot A$;
    \item assoziativ, $A\cdot(B\cdot C) = (A\cdot B)\cdot C = A\cdot B\cdot C$;
    \item distributiv, $A\cdot (B+C) = A\cdot B + A\cdot C.$
\end{itemize}

\subsection{Die Determinante}

Die Determinante ordnet jeder quadratischen Matrix eine (reelle) Zahl zu. 

\emph{Definition: } Die Determinante einer $(2\times 2)$-Matrix $A$ ist definiert als 
\begin{align}
    A = \mqty(a & b \\ c & d) \qquad \det(A) := a\cdot d - c \cdot b.
\end{align}
Die Determinante einer $(n\times n)$-Matrix kann mit Hilfe des \emph{Laplace'schen Entwicklungssatzes} bestimmt werden, wobei man die Berechnung auf Dterminanten von $(2\times 2)$-Matrizen zurückführt.

\paragraph{Beispiel einer $(3\times3)$-Matrix: }$~$

Um die Determinante einer $(3\times 3)$-Matrix zu bestimmen, entwickelt man nach einer (beliebigen) Zeile oder Spalte, wobei jeder Koeffizient $a_{ij}$ mit derjenigen Unterdeterminante multipliziert wird, die durch Streichung der $i$-ten Zeile und $j$-ten Spalte entsteht und ein alternierendes Vorzeichen nach dem Muster 
\begin{align}
    \mqty(+ & - & + \\ - & + & - \\ + & - & +) \qq{bzw.} (-1)^{i+j}
\end{align}
trägt. Sei nun also $A$ definiert als 
\begin{align}
    A = \mqty(a_{11} & a_{12} & a_{13}\\ a_{21} & a_{22} & a_{23} \\ a_{31} & a_{32} & a_{33}), \qq{alternative Schreibweise} \det(A) \equiv |A|,
\end{align}
dann ergibt sich die Determinante von $A$ zu 
\begin{align}
    \det(A) &= a_{11} \cdot \mqty|\textcolor{gray}{a_{11}} & \textcolor{gray}{a_{12}} & \textcolor{gray}{a_{13}}\\ \textcolor{gray}{a_{21}} & a_{22} & a_{23} \\ \textcolor{gray}{a_{31}} & a_{32} & a_{33}| - a_{12} \cdot \mqty|\textcolor{gray}{a_{11}} & \textcolor{gray}{a_{12}} & \textcolor{gray}{a_{13}}\\ a_{21} & \textcolor{gray}{a_{22}} & a_{23} \\ a_{31} & \textcolor{gray}{a_{32}} & a_{33}| + a_{13} \cdot \mqty|\textcolor{gray}{a_{11}} & \textcolor{gray}{a_{12}} & \textcolor{gray}{a_{13}}\\ a_{21} & a_{22} & \textcolor{gray}{a_{23}} \\ a_{31} & a_{32} & \textcolor{gray}{a_{33}}| \notag \\
    &= a_{11} \cdot \mqty|a_{22} & a_{23} \\ a_{32} & a_{33}| - a_{12} \cdot \mqty|a_{21} & a_{23} \\ a_{31} & a_{33}| + a_{13} \cdot \mqty|a_{21} & a_{22} \\ a_{31} & a_{32}| \notag \\
    &= a_{11} (a_{22}a_{33}-a_{32}a_{23}) - a_{12}(a_{21}a_{33}-a_{31}a_{23}) + a_{13} (a_{21}a_{32}-a_{31}a_{22}).
\end{align}
Hier wurde nach der ersten Zeile entwickelt. Wir hätten genauso nach bspw. der zweitne Spalte entwickeln können:
\begin{align}
    \det(A) &= -a_{12} \cdot \mqty|a_{21} & a_{23} \\ a_{31} & a_{33}| + a_{22} \cdot \mqty|a_{11} & a_{13} \\ a_{31} & a_{33}| - a_{32} \cdot \mqty|a_{11} & a_{13} \\ a_{21} & a_{23}| \notag \\
    &= -a_{12} (a_{21}a_{33}-a_{31}a_{23}) + a_{22}(a_{11}a_{33}-a_{31}a_{13}) - a_{32} (a_{11}a_{23}-a_{21}a_{13}).
\end{align}
Praktisch ist, immer nach derjenigen Zeile oder Spalte zu entwickeln, welche die meisten Nullen enthält. Dafür schauen wir uns ein Zahlenbeispiel an: 
\begin{align}
    \mqty|7&3&0\\1&2&4\\3&8&5| &= 0 - 4\cdot \mqty|7 & 3\\3&8| + 5\cdot\mqty|7&3\\1&2| \notag \\
    &= -4(56-9) + 5(14-3) = -188+55 = -133.
\end{align}

\paragraph{Eigenschaften von Determinanten}$~$

\begin{itemize}
    \item $\det(A\cdot B) = \det(A)\cdot \det(B)$
    \item $\det(k\cdot A) = k^n \cdot \det(A)$, wenn $k\in \mathbb{R}$ und $A: (n\times n)$-Matrix 
    \item $\det(A^T) = \det(A)$
    \item $\det(\mathds{1}) = 1.$
\end{itemize}

\paragraph{Beispiel: Spatprodukt} Das Volumen des durch die Vektoren 
\begin{align}
    \bm{a} = \mqty(a_x\\a_y\\a_z), \bm{b} = \mqty(b_x\\b_y\\b_z), \bm{c} = \mqty(c_x\\c_y\\c_z)
\end{align}
aufgespannten Spates kann als Determinante einer aus $\bm{a},\bm{b}$ und $\bm{c}$ gebildeten Matrix geschrieben werden, 
\begin{align}
    V &= \mqty|a_x & b_x&c_x\\a_y&b_y&c_y\\a_z&b_z&c_z|.\\
\qq{Denn: } \qty[\mqty(a_x\\a_y\\a_z)\times\mqty(b_x\\b_y\\b_z)] &= \mqty(a_yb_z-a_zb_y \\ a_zb_x-a_xb_z\\a_xb_y-a_yb_x)\cdot\mqty(c_x\\c_y\\c_z) \notag \\
&= c_x(a_yb_z-a_zb_y) - c_y(a_xb_z-a_zb_x + c_z(a_xb_y-a_yb_z)) \notag \\
\qq{und} \mqty|a_x & b_x&c_x\\a_y&b_y&c_y\\a_z&b_z&c_z| &= c_x \mqty|a_y &b_y\\a_z&b_z| - c_y\mqty|a_x&b_x\\a_z&b_z|+c_z \mqty|a_x&b_x\\a_y&b_y| \notag \\
&=c_x(a_yb_z-a_zb_y) - c_y(a_xb_z-a_zb_x + c_z(a_xb_y-a_yb_z)).
\end{align}
Wir sehen, dass beide Ergebnisse miteinander übereinstimmen.

\newpage
\subsection{Die inverse Matrix}

Die zu einer quadratischen Matrix $A$ gehörende inverse Matrix $A^{-1}$ ist durch die Bedingung 
\begin{align}
    A \cdot A^{-1} = A^{-1}\cdot A = \mathds{1}
\end{align}
definiert. Erinnerung: Im Falle reeller Zahlen ist jedem $x\in\mathbb{R}\backslash\{0\}$ ein multiplikatives inverses Element ($x^{-1} = \frac{1}{x}$) zugeordnet, sodass $x\cdot x^{-1} = 1$ gilt.

Beachte: Nicht jede Matrix besitzt ein Inverses (ist invertierbar)! Eine invertierbare Matrix heißt \emph{regulär}; eine nicht invertierbare Matrix heißt \emph{singulär}. 
\begin{satz}
    Eine quadratische Matrix $A$ ist genau dann invertierbar, wenn gilt: $\det(A) \neq 0.$
\end{satz}
\begin{itemize}
    \item Eine Matrix heißt \emph{selbstinvers}, wenn gilt $A^{-1} =A$. 
    \item Eine Matrix heißt \emph{orthogonal}, wenn gilt $A^{-1} = A^T$.
\end{itemize}
Die inverse Matrix $A^{-1}$ von $A$ kann, sofern existent, mit Hilfe des sogenannten \emph{Gauß-Jordan-Algorithmus} bestimmt werden. Dabei schreibt man die $(n\times n)$-Matrix $A$ zusammen mit der Einheitsmatrix $\mathds{1}_n$, 
\begin{align}
    (A | \mathds{1}_n),
\end{align}
und bringt diesen Ausdruck mit Hilfe von elementaren Zeilenumformungen in die Form $(\mathds{1}_n | A^{-1})$ aus der $A^{-1}$ abgelesen werden kann. Erlaubte Zeilenumformungen sind: 
\begin{itemize}
    \item Multiplikation einer Zeile mit einer Zahl $k \in \mathbb{R}\backslash\{0\}$, 
    \item Hinzuaddieren des $k$-Fachen einer beliebigen Zeile zu einer anderen, 
    \item Vertauschen zweier Zeilen.
\end{itemize}

\begin{align}
    \qq{Beispiel: } A = \mqty(1 & 0 \\ 2 & 3) \quad &\rightarrow \quad \qty(\begin{array}{c c|c c} 
       1 & 0 & 1 & 0 \\
       2 & 3 & 0 & 1 
    \end{array}) \quad \bigg| \quad (\text{II}) + (-2)\cdot (\text{I}) \notag \\
    &\rightarrow \quad \qty(\begin{array}{c c|c c} 
        1 & 0 & 1 & 0 \\
        0 & 3 & \minus2 & 1 
     \end{array}) \quad \bigg| \quad \frac{1}{3}\cdot(\text{II}) \notag \\
     &\rightarrow \quad \qty(\begin{array}{c c|c c} 
        1 & 0 & 1 & 0 \\
        0 & 1 & \minus \frac{2}{3} & \frac{1}{3} 
     \end{array})  \quad \Rightarrow A^{-1} = \mqty(1 & 0 \\-\frac{2}{3} & \frac{1}{3}).
\end{align}
Wir überprüfen das Ergebnis mit einer Probe: 
\begin{align}
    A \cdot A^{-1} = \mqty(1 & 0 \\ 2 & 3) \cdot \mqty(1 & 0 \\-\frac{2}{3} & \frac{1}{3}) = \mqty(1&0\\0&1) = \mathds{1}_2. \quad \checkmark
\end{align}

\subsection{Anwendungen von Matrizen}

\paragraph{Lösen von linearen Gleichungssystemen}$~$

Wir erinnern uns an das lineare Gleichungssystem mit 3 Unbekannten (siehe~\eqref{eqn:2_LGS_3a} bis~\eqref{eqn:2_LGS_3c})
\begin{subequations}
    \begin{align}
        a_1 x + b_1 y + c_1 z &= k_1, \\
        a_2 x + b_2 y + c_2 z &= k_2, \\
        a_3 x + b_3 y + c_3 z &= k_3.
    \end{align}
\end{subequations}
Dieses System kann auch als Matrixgleichung geschrieben werden: 
\begin{align}
    \underbrace{\mqty(a_1&b_1&c_1\\a_2&b_2&c_2\\a_3&b_3&c_3)}_{\mathclap{\text{Koeffizientenmatrix}}} \mqty(x\\y\\z) = \mqty(k_1\\k_2\\k_3).
\end{align}
Das System nach $x,y,z$ auflösen, heißt also, es auf eine Form zu bringen, in der die Koeffizientenmatrix diagonal ist.

\paragraph{Darstellung physikalischer Größen}$~$

Viele physikalische Größen sind selbst Matrix-wertig. Wir wollen im Folgenden einige Beispiele nennen:
\begin{itemize}
    \item Die Trägheitsmomente eines starren Körpers werden in einer Matrix zusammengefasst. 
    \item Bewegungen und vErformungen, die auf der Wechselwirkung elektromagnetischer Felder mit Ladungen beruhen, werden durch eine Matrix beschrieben, in der die Komponenten des elektrischen und magnetischen Feldes zusammengefasst sind (elektromagnetischer Spannungstensor).
    \item Das doppelbrechende Verhalten anisotroper Kristalle kann durch eine Matrix-wertige Brechzahl beschrieben werden. 
    \item Viele Observablen der Quantenmechanik (Energie, Drehimpulse) können als Matrizen dargestellt werden. 
    \item In der Allgemeinen Relativitätstheorie ist die Geometrie einer Raumzeit durch eine Matrix bestimmt.
    \item Die sogenannte Dichtematrix ist elementarer Gegenstand der statistischen Quantenmechanik.
\end{itemize}
\section{Grundzüge der Integralrechnung}

Wir können die mathematische Operation des Integrierens auffassen als die Umkehrung der Differentiation. Das heißt: Wir suchen zu einer gegebenen Funktion $f(x)$ eine sogenannte \emph{Stammfunktion} $F(x)$, sodass 
\begin{align}
    f(x) = \dv{F(x)}{x} \qq{und schreiben} F(x) = \int f(x) \dd{x}.
\end{align}
Wir nennen dies das unbestimmte Integral der Funktion $f(x)$. In symbolischer schreibweise lässt sich das Integral folgendermaßen konstruieren: 
\begin{align}
    f(x) &= \dv{F(x)}{x} \quad |\cdot \dd{x} \notag \\
    f(x) \dd{x} &= \dv{F(x)}{x}\dd{x} = \dd{F}(x) \quad \bigg|\; \int \notag \\
    \int f(x) \dd{x} &= \int \dd{F(x)} = F(x).    
\end{align}
Ein paar Beispiele wollen wir hier beispielhaft einmal notieren. Kennen wir die Ableitungen verschiedener Funktionen, so können wir das dazugehörige Integral leicht konstruieren: 
\begin{multicols}{2}
    \begin{itemize}
        \item $\displaystyle \int 1 \dd{x} = x + C,$
        \item $\displaystyle \int x^n \dd{x} = \frac{x^{n+1}}{(n+1)} + C,$
        \item $\displaystyle \int \frac{1}{\sqrt{x}} \dd{x} = 2\sqrt{x} + C,$
        \item $\displaystyle \int \frac{1}{x} \dd{x} = \ln|x| + C,$
        \item $\displaystyle \int \frac{1}{1+x^2} \dd{x} = \arctan(x) + C,$
        \item $\displaystyle \int \e^{cx} \dd{x} = \frac{\e^{cx}}{c} + C.$
    \end{itemize}
\end{multicols}
Hierbei ist $C = \text{const.}$ die sogennante \emph{Integrationskonstante}. Das unbestimmte Integral beschreibt nämlich die Menge aller Stammfunktionen, deren erste Ableitung $f(x)$ ergibt. 

Im Gegensatz zur Differentiation existiert im Allgemeinen kein Algorithmus zur Bestimmung eines Integrals. Außerdem besitzt nicht jede Funktion eine geschlossen analytisch darstellbare Stammfunktion.

\paragraph{Beispiele}$~$\\[-2.2cm]

\begin{align}
    \text{Stammfunktion von} \quad f(x) &= 3xy^2 + 2y - 4x +3 \\
    F(x) &= \int (3xy^2 + 2y - 4x + 3) \dd{x} \notag \\
         &= \frac{3}{2} x^2 + y^2 + 2xy - 2x^2 + 3x + C \notag \\
         &= \qty(\frac{3y^2}{2} - 2)x^2 + (2y-3)x + C \notag \\
    \text{Stammfunktion von} \quad f(y) &= 3xy^2 + 2y - 4x +3 \\
    F(x) &= \int (3xy^2 + 2y - 4x + 3) \dd{y} \notag \\
         &=xy^3 + y^2 - (4x-3)y + C.\notag \\[-1.5cm] \notag 
\end{align}

Wir sehen also, dass es wichtig, ist, die korrekte Variable zur Integration auszuwählen. 

Bei einem \emph{bestimmten Integral} wird die Stammfunktion an einer oberen und einer unteren Grenze ausgewertet. Das Ergbnis hängt nicht mehr von der Integrationsvariablen ab. Wir können damit den \emph{Hauptsatz der Differential- und Integralrechnung} formulieren: 
\begin{mymathbox}[ams align, title={Hauptsatz der Integral- und Differentialrechnung}, colframe={FSUblau}]
    \int\limits_{x_1}^{x_2} f(x) \dd{x} = F(x_1) - F(x_1).
\end{mymathbox}
Man schreibt auch $\displaystyle \int\limits_{x_1}^{x_2} f(x) \dd{x} = F(x)\eval_{x_1}^{x_2}.$

Als Integrationsgrenze kann ebenfalls $\pm \infty$ auftauchen, was im Sinne eines Grenzwertes zu verstehen ist, also 
\begin{align}
    \int\limits_{0}^{\infty} f(x) \dd{x} = \lim_{t\to \infty} \int\limits_{0}^{t} f(x) \dd{x}.
\end{align}
Man findet außerdem oft Schreibweisen wie 
\begin{align}
    \int\limits_{-\infty}^{\infty} f(x) \dd{x} = \int\limits_{\mathbb{R}} f(x) \dd{x}.
\end{align}
Es ist zu beachten, dass sowohl endliche als auch unendliche Integrale nicht immer konvergent sind, bspw. 
\begin{align}
    \begin{split}
        \int\limits_{0}^{\infty} \e^{x} \dd{x} &= \e^{x}\eval_0^\infty \to \infty, \qquad \int\limits_{0}^{10} \frac{1}{r^2} \dd{x} = -\frac{1}{r}\eval_0^{10} \to \infty, \\
        \int\limits_{0}^{\infty} \cos(x) \dd{x} &= \sin(x)\eval_0^\infty \to \quad ? \;.
    \end{split}
\end{align}

\subsection{Das Wegintegral}

Wir führen das Wegintegral ein am Beispiel der Frage ``\emph{Was ist die Arbeit?}''. Bewegt sich ein Teilchen (Massepunkt) unter dem Einfluss einer konstanten Kraft $F$ entlang eines Weges der Länge $s$, dann lautet die Antwort 
\begin{align}
    \text{Arbeit } = \text{ Kraft mal Weg,} \qq{bzw.} W = F\cdot s.
\end{align}
Betrachten wir das konkrete Beispiel eines Elektrons (Ladung $q = -e$) im Kondensator mit elektrischer Feldstärke $\bm{F}$, das sich unter Einfluss einer Coulomb-Kraft $\bm{F} = q \cdot \bm{E}$ bewegt. 
\begin{figure}[htp]
    \centering
    \begin{tikzpicture}
        \draw[thick] (-4,0) -- (-3,0);
        \draw[thick] (4,0) -- (3,0);
        \draw[thick] (-3,-1) -- (-3,1);
        \draw[thick] (3,-1) -- (3,1);
        \fill (-1,0) circle (2pt);
        \draw[very thick, -{latex}] (-1,0) --node[above]{$\bm{F}= q \cdot \bm{E}$} (1,0);
        \foreach \x in {0.8,0.6,...,-.8}{
            \draw[FSUblau, opacity=0.2, -{latex}] (3,\x) -- (-3,\x);
        }
        \draw[thick, {latex}-{latex}] ( -3,-1.2) --node[below]{s} (3,-1.2);
    \end{tikzpicture}
\end{figure}

Würden wir das elektrische Feld immer dann neu einstellen, wenn das Elektron ein bestimmtes Wegintervall $\Delta s$ zurückgelegt hat, so ergäbe sich die gesamte Arbeit als Summe der Arbeiten innerhalb der einzelnen Teilstrecken.
\begin{figure}[htp]
    \centering
    \begin{tikzpicture}
        \draw[thick] (-4,0) -- (-3,0);
        \draw[thick] (4,0) -- (3,0);
        \draw[thick] (-3,-1) -- (-3,1);
        \draw[thick] (3,-1) -- (3,1);
        % \fill (-1,0) circle (2pt);
        % \draw[very thick, -{latex}] (-1,0) --node[above]{$\bm{F}= -e \cdot \bm{E}$} (1,0);
        \foreach \x in {0.8,0.6,...,-.8}{
            \draw[FSUblau, opacity=0.2, -{latex}] (3,\x) -- (-3,\x);
        }
        \foreach \x in {-3,-2,-1,0,2}{
            \draw[thick, {latex}-{latex}] (\x,0.8) --node[above]{$\Delta s$} +(1,0);
        }
        \foreach \x in {-3,-2,-1,0,1}{
            \draw[dashed] (\x+1, 1) -- +(0,-2);
        }
        \node (A) at (-3+0.5,-0.5){$F_1$};
        \node (A) at (-2+0.5,-0.5){$F_2$};
        \node (A) at (-1+0.5,-0.5){$F_3$};
        \node (A) at (0+0.5,-0.5){$F_4$};
        \node (A) at (1+0.5,-0.5){$\hdots$};
        \node (A) at (2+0.5,-0.5){$F_n$};
        \draw[thick, {latex}-] ( -3,-0.8) --node[below]{s} (1.3,-0.8);
        \draw[thick, -{latex}] ( 1.7,-0.8) -- (3,-0.8);
        \node[anchor=west] (A) at (4.5,.3){$n$ Teilstücke};
        \node[anchor=west] (A) at (4.5,-.3){$s = n \cdot \Delta s$};
        \node (A) at (-6.5,0){};
    \end{tikzpicture}
\end{figure}
Für $n$ Teilstücke ergibt sich die Gesamt-Arbeit zu 
\begin{align}
    W &= W_1 + W_2 + \hdots + W_n \notag \\
      &= q E_1 \Delta s + q E_1 \Delta s + \hdots + e E_n \Delta s = q \sum_{i=1}^{s/\Delta s} E_i \Delta s.
\end{align}
Wollen wir nun die Gesamtarbeit bestimmen für den Fall, dass das $E$-Feld \emph{kontinuierlich} verändert wird, so habne wir den Weg in unendlich viele Intervalle einzuteilen, sodass das Feld in diesen unendlich kleinen (infinitesimalen) Intervallen konstant ist; das führt auf die Definition des Integrals: 
\begin{align}
    W = q \cdot \lim_{\Delta s \to 0} \qty(\sum_{i=1}^{s/\Delta s} E_i \Delta s) =: \int\limits_{0}^{s} E(s') \dd{s'}.
\end{align}
Dabei ist das elektrische Feld nun als eine (kontinuierliche) Funktion $E = E(s)$ aufzufassen. Bildlich gesprochen heißt das: Mit Hilfe des Integrals ``bewegen'' wir uns entlang eines Weges und ``sammeln'' an allen Punkten die Beiträge der Kraft zur Gesamtarbeit ein. 
\begin{quote}
    Arbiet ist das Integral über eine Kraft entlang eines Weges.
\end{quote}
Der Weg muss hierbei nicht notwendigerweise gerade sein. 
\begin{figure}[htp]
    \centering
    \begin{tikzpicture}
        \fill (0,0) circle (2pt)node[above]{$a$};
        \fill (4,0) circle (2pt)node[above]{$b$};
        \draw[thick] (0,0) to[out=0,in=150] (2,0.5) to [out=150-180, in =240]node[below left]{Kurve $C$} (4,0);
        \draw[thick, -{latex}] (5,0) -- (6,0);
        \node[anchor=west] at (6.5,0){$\displaystyle\int\limits_C \bm{F}(\bm{s}) \dd{\bm{s}}$};
        \begin{scope}[shift={(0,-2)}]
            \fill (2,0.6) circle (2pt)node[above]{$a=b$};
            \draw[thick] (2,0.6) to[out=180,in=150] (0,-.6) to [out=150-180, in =200](2,0) to [out=200-180, in=240]node[below left]{Kurve $C$}  (4,0) to [out=240-180, in=180] (2,0.6);
            \draw[thick, -{latex}] (5,0) -- (6,0);
            \node[anchor=west] at (6.5,0){$\displaystyle\oint\limits_C \bm{F}(\bm{s}) \dd{\bm{s}}$};
        \end{scope}
    \end{tikzpicture}
\end{figure}
Insbesondere können wir die Länge einer Kurve bestimmen als das Integral über die konstante Funktion $f(s) = 1$. Bildlich heißt das: Wir ``sammeln'' den Beitrag eines jeden infinitesimalen Streckenabschnitts zur Gesamtlänge ein.

\subsection{Flächen- und Volumenintegrale}

So wie wir Längen mit Hilfe des Wegintegrales berechnen können, lassen sich Flächen mit Hilfe von Flächenintegralen berechnen, 
\begin{align}
    \text{Doppelintegral: } \quad A = \iint\limits_{\text{Fläche}} \dd{x}\dd{y}. 
\end{align}

\paragraph{Beispiel: Das rechtwinklige Dreieck}$~$

\begin{wrapfigure}{r}{6cm}
    \centering
    \vspace{-5mm}
    \begin{tikzpicture}
        \draw[thick, -{latex}] (0,0) -- (4,0)node[right]{$x$};
        \draw[thick, -{latex}] (0,0) -- (0,2.5)node[above]{$y$};
        \draw [decorate, decoration = {calligraphic brace}, thick] (3.5,-.2) --node[below]{$a$}  (0,-.2);
        \draw [decorate, decoration = {calligraphic brace}, thick] (3.7,2) --node[right]{$b$}  (3.7,0);
        \draw[thick, {latex}-{latex}] (2.5,0) --node[right]{$x \frac{b}{a}$} (2.5, 2.5*2/3.5);
        \filldraw[draw=black, fill=FSUblau, fill opacity=0.4, thick] (0,0) -- (3.5,0) -- (3.5,2) -- cycle;
    \end{tikzpicture}
    \vspace{-2.4cm}
\end{wrapfigure}

Wir legen zunächst die Integrationsgrenzen fest: 
\begin{align}
    x \qq{von} &0 \qq{bis} a, \notag\\
    y \qq{von} &0 \qq{bis} \frac{b}{a}x \notag \\
    \Rightarrow A &= \int\limits_{0}^{a}\dd{x} \int\limits_{0}^{\frac{b}{a}x} \dd{y} = \int\limits_{0}^{a}\dd{x} y \eval_0^{\frac{b}{a}x} = \int\limits_{0}^{a} \frac{b}{a}x \dd{x} \notag \\
    &= \frac{b}{a} \int\limits_{0}^{a}x \dd{x} = \frac{b}{2a} x^2 \eval_0^a = \uuline{\frac{ab}{2}}.
\end{align}
In der Schule wird das Integral über eine Funktion $f(x)$ oft eingeführt als die Fläche, die diese Funktion mit der $x$-Achse einschließt:
\begin{align}
    A = \int\limits_{0}^{x_0} \dd{x} \int\limits_{0}^{f(x)} \dd{y} = \int\limits_{0}^{x_0} \dd{x} y \eval_0^{f(x)} = \int\limits_{0}^{x_0} f(x) \dd{x}.
\end{align}
Beachte, dass auch unendliche Flächenintegrale konvergieren können, wie bspw. 
\begin{align}
    A = \int\limits_{1}^{\infty} \frac{1}{r^2} \dd{r} = -\frac{1}{r}\eval_1^\infty = 1.
\end{align}
Es wird an dieser Stelle kaum verwunderlich sein, dass auch Volumen durch Integration bestimmt werden. Wir führen dafür das \emph{Volumenintegral} bzw. \emph{Dreifachintegral} ein 
\begin{align}
    V = \iiint\limits_{\text{Volumen}} \dd{x} \dd{y}\dd{z}.
\end{align}

\newpage
\paragraph{Beispiel: Prisma mit Parabelausschnitt}$~$

\begin{wrapfigure}{r}{6cm}
    \centering
    \vspace{-5mm}
    \begin{tikzpicture}
        \draw[thick, -{latex}] (0,0) -- (-2,-2)node[below]{$x$};
        \draw[thick, -{latex}] (0,0) -- (3,0)node[right]{$y$};
        \draw[thick, -{latex}] (0,0) -- (0,4.3)node[above]{$z$}; 
        \draw[ domain=0:2, smooth, variable=\x, thick] plot ({\x}, {-1*(\x)*(\x)+4});
        \draw[ domain=-1.4:0.6, smooth, variable=\x, thick] plot ({\x}, {-1*(\x+1.4)*(\x+1.4)+4-1.4});
        \draw[thick] (0,4) -- ++(-1.4,-1.4) -- ++(0,-4) -- ++(2,0) -- ++(1.4,1.4);
        \node (A) at (-1.35,-1.7){$x_0$};
        \node (A) at (2,-.4){$\sqrt{z_0}$};
        \node[anchor=west] (A) at (1,3){$z(y) = -y^2 +z_0$};
        \node (A) at (-.3,4){$z_0$};
    \end{tikzpicture}
    \vspace{-6cm}
\end{wrapfigure}

Das Volumen des Prismas ergibt sich zu 
\begin{align}
    V &= \int\limits_{0}^{x_0} \dd{x} \int\limits_{0}^{\sqrt{z_0}} \dd{y} \int\limits_{0}^{-y^2 + z_0} \dd{z} \notag \\
    &= \int\limits_{0}^{x_0} \dd{x} \int\limits_{0}^{\sqrt{z_0}} \dd{y} ( -y^2 + z_0) \notag \\
    &= \int\limits_{0}^{x_0} \dd{x} \eval(-\frac{y^3}{3}+z_0 y|_0^{\sqrt{z_0}} \notag \\
    &= x_0 \qty(-\frac{\sqrt{z_0}^3}{3} + z_0 \sqrt{z_0}) = \uuline{\frac{2}{3} x_0 \sqrt{z_0}^3}.
\end{align}
Hierbei ``laufen'' die Integrale über $x$ und $y$ die Grundfläche ab, während das $z$-Integral mit der (von $y$ abhängigen) Höhe multipliziert wird. 

\emph{Bemerkung: } Man schreibt auch $A = \iint \dd{x}\dd{y} = \int \dd{A}$ und $V = \iiint \dd{x}\dd{y}\dd{z} = \int \dd{V}$, sodass eine bildliche Vorstellung ähnlich dem Wegintegral möglich ist: Wir schreiten das Gebiet innerhalb der Integrationsgrenzen ab und ``sammeln'' infinitesimale Flächenstücke $\dd{A}$ bzw. Volumenstücke $\dd{V}$ ein.

Natürlich können auch Flächen- und Volumenintegrale über Funktionen betrachtet werden. Dafür betrachten wir beispielhaft die Masse eines Quaders der Kantenlängen $a,b,c$, dessen Dichte $\varrho$ ortsabhängig ist: 
\begin{align}
    M = \int\limits_{0}^{a}\dd{x} \int\limits_{0}^{b}\dd{y} \int\limits_{0}^{c}\dd{z} \varrho(x,y,z) = \int\limits_{V} \varrho(\bm{r}) \dd{V}.
\end{align}

\subsection{Eigenschaften und Rechenregeln}

\begin{itemize}
    \item Linearität: $\displaystyle \int\limits_{a}^{b} \qty(\alpha f(x)+\beta g(x)) \dd{x} = \alpha \int\limits_{a}^{b} f(x) \dd{x} + \beta \int\limits_{a}^{b} g(x) \dd{x}$ 
\end{itemize}

\end{document}
 