\section{Trigonometrische Funktionen} 

Dieser Abschnitt widmet sich der Definition von Sinus- und Kosinusfunktionen sowie deren Eigenschaften. 

\paragraph{Bogenmaß und Gradmaß}$~$

\begin{wrapfigure}{r}{7cm}
    \centering
    \vspace{-1cm}
        \begin{tikzpicture}[scale=2.5]
            \draw[thick, -{latex}] ( -1.1,0) -- (1.2,0)node[above]{$x$};
            \draw[thick, -{latex}] (0,-1.1) -- (0,1.2)node[right]{$y$};
            \draw (0,0) circle (1cm);
            \fill (1,0) circle (1pt)node[below left]{$1$};
            \fill (0,1) circle (1pt)node[below right]{$1$};
            \fill (-1,0) circle (1pt)node[above right]{$\minus1$};
            \fill (0,-1) circle (1pt)node[above left]{$\minus1$};
            \draw[thick, -{latex}] (0,0) --node[above left, rotate=40, pos=0.8]{$r=1$} (40:1);
            \draw[-{latex}] (0.4,0) arc (0:40:.4);
            \node (A) at (20:.3){$\alpha$};
            \draw[thick, PAForange] (1,0) arc (0:40:1); 
        \end{tikzpicture}
    \vspace{-3mm}
\end{wrapfigure}

Wir betrachten die (reelle) Zahlenebene, in der alle Abstände in Einheit 1 gemessen werden, also \emph{dimensionslos} sind. Der Umfang des Einheitskreises beträgt definitionsgemäß 
\begin{align}
    u = 2\pi r = 2\pi.
\end{align}
Wir können nun zwischen den beiden Maßsystemen \emph{Gradmaß} und \emph{Bogenmaß} unterscheiden. Für das Gradmaß ist der Vollkreis in 360 Abschnitte eingeteilt, während im Bogenmaß der Winkel in Bruchteilen des Kreisumfangs gemessen wird:
\begin{align}
    \begin{split}
        \text{Gradmaß: } [\alpha] &= \si{\degree} \quad(\text{deg}) \qq{,} \text{Vollkreis: } \SI{360}{\degree}\\
        \text{Borgenmaß: } [\alpha] &= 1 \quad(\text{rad}) \qq{,} \text{Vollkreis: } 2\pi.
    \end{split}
\end{align}
Wir können beide Maßsysteme mittels einer Verhältnisgleichung ineinander überführen: 
\begin{align}
    \frac{\alpha [\text{rad}]}{2\pi} = \frac{\alpha [\text{deg}]}{\SI{360}{\degree}}.
\end{align}
Einige Werte zur Umrechnung sind in folgender Tabelle aufgelistet:

\begin{table}[htp]
    \centering
    \caption{Umrechnungstabelle für Bogen- und Gradmaß}
    \begin{tabular}{c c c c c c c c c c c}
        \toprule 
        $\alpha$ [deg] & 30 & 45 & 60 & 90 & 120 & 150 & 180 & 270 & 360 \\
        \midrule
        $\alpha$ [rad] & $\frac{\pi}{6}$ & $\frac{\pi}{4}$ & $\frac{\pi}{3}$ & $\frac{\pi}{2}$ & $\frac{2\pi}{3}$ & $\frac{5\pi}{6}$ & $\pi$ & $\frac{3\pi}{2}$ & $2\pi$ 
    \end{tabular}
\end{table}

\subsection{Winkelfunktionen}

\begin{minipage}{0.5\textwidth}
    Alle Winkelfunktionen sind dimensionslos definiert. Die $x$-Koordinate eines Punktes des Einheitskreises ist der Kosinus des Winkels zwischen seinem Ortsvektor und der Abszisse, während die $y$-Koordinate der Sinus dieses Winkels ist. Weitere trigonometrische Funktionen sind wie folgt defniert: 
    \begin{itemize}
        \item Tangens:\hphantom{Ko} $\displaystyle \tan(\alpha) := \frac{\sin(\alpha)}{\cos(\alpha)}$
        \item Kotangens: \,$\displaystyle \cot(\alpha) := \frac{\cos(\alpha)}{\sin(\alpha)} = \frac{1}{\tan(\alpha)}$
        \item Sekans:\;\;\hphantom{Ko} $\displaystyle \sec(\alpha) := \frac{1}{\cos(\alpha)}$
        \item Kosekans: \;\;$\displaystyle \csc(\alpha) := \frac{1}{\sin(\alpha)}$
    \end{itemize}
\end{minipage}
\begin{minipage}{0.5\textwidth}
    \centering
    \begin{tikzpicture}[scale=5]
        \draw[thick, -{latex}] ( -.3,0) -- (1.2,0)node[above]{$x$};
        \draw[thick, -{latex}] (0,-.3) -- (0,1.2)node[right]{$y$};
        \draw (-10:1) arc (-10:100:1);
        \fill (1,0) circle (0.5pt)node[below left]{$1$};
        \fill (0,1) circle (0.5pt)node[below right]{$1$};
        % \fill (-1,0) circle (1pt)node[above right]{$\minus1$};
        % \fill (0,-1) circle (1pt)node[above left]{$\minus1$};
        \draw[thick, -{latex}] (0,0) --node[above left, rotate=40, pos=0.6]{$r=1$} (40:1);
        \draw[-{latex}] (0.4,0) arc (0:40:.4);
        \node (A) at (20:.3){$\alpha$};

        \draw[ultra thick, FSUblau] ($(0,sin{40})$) --node[above]{$\cos(\alpha)$} ($({cos{40}},sin{40})$);
        \draw[ultra thick, PAForange, pos=0.4] (40:1) --node[rotate=-90,yshift=0.3cm,, xshift=0.3cm]{$\sin(\alpha)$} ($({cos{40}},0)$);
        \draw[ultra thick, Gruen] (1,0) --node[rotate=-90, yshift=0.3cm]{$\tan(\alpha)$} ($(1,tan{40})$);
        \draw[ultra thick, red!70!black] (0,1) --node[above]{$\cot(\alpha)$} ($(cot{40},1)$);
        \draw[ultra thick, PAForange!60!white] (0,0) --node[above, rotate=40]{$\csc(\alpha)$} ($(cot{40},1)$);
        \draw[thick, FSUblau!60!white] (0,0) --node[below, rotate=40]{$\sec(\alpha)$} ($(1, tan{40})$);
        % \draw[thick, PAForange] (1,0) arc (0:40:1); 
    \end{tikzpicture}
\end{minipage}

Aus der Abbildung können wir mithilfe des Satzes des Pythagoras folgende Relation ablesen 
\begin{mymathbox}[ams align, title={Trigonometrischer Pythagoras}, colframe={FSUblau}]
    \sin^2(\alpha) + \cos^2(\alpha) = 1.
\end{mymathbox}

Durch weitere rechtwinklige Dreiecke können wir ebenfalls folgende Relationen ablesen: 
\begin{align}
    \begin{split}
        \sec^2(\alpha) &= \frac{1}{\cos^2(\alpha)} = 1 + \tan^2(\alpha) \\
        \csc^2(\alpha) &= \frac{1}{\sin^2(\alpha)} = 1 + \cot^2(\alpha).
    \end{split}
\end{align}
Beide Relationen lassen sich auch direkt mithilfe der Definitioen und des trigonometrischen Pythagoras herleiten.
Anhand der Konstruktion können wir auch folgende Eigenschaften der Funktionen ablesen, wie z.\,B. die Wertebereiche 
\begin{align}
    \begin{split}
        &\sin(\alpha) \in [-1,1] \qquad \tan(\alpha) \in (-\infty,\infty) \\
        &\cos(\alpha) \in [-1,1] \qquad \cot(\alpha) \in (-\infty,\infty).
    \end{split}
\end{align}
Die Vorzeichen der Funktionen in den einzelnen Quadranten lauten:
\begin{table}[htp]
    \centering
    \caption{Vorzeichen/Signum (sgn) der Funktionen}
    \begin{tabular}{c c c c c c c c c c c}
        \toprule 
        Quadrant & sgn($\sin\alpha$) & sgn($\cos\alpha$) & sgn($\tan\alpha$) & sgn($\cot\alpha$)  \\
        \midrule
        I & + & + & + & + \\
        II & + & - & - & - \\
        III & - & - & + & + \\
        IV & - & + & - & -
    \end{tabular}
    \vspace{-1cm}
\end{table}

Spezielle Werte der Funktionen sind in folgender Tabelle zusammengefasst: 
\begin{table}[htp]
    \centering
    \caption{Vorzeichen/Signum (sgn) der Funktionen}
    \begin{tabular}{c c c c c c c c c c c }
        \toprule 
         $\alpha$ [deg] & 0 & 30 & 45 & 60 & 90 & 120 & 150 & 180 & 270 & 360 \\
         $\alpha$ [rad] & 0 & $\dfrac{\pi}{6}$ & $\dfrac{\pi}{4}$ & $\dfrac{\pi}{3}$ & $\dfrac{\pi}{2}$ & $\dfrac{2\pi}{3}$ & $\dfrac{5\pi}{6}$ & $\pi$ & $\dfrac{3\pi}{2}$ & $2\pi$\\[.4em]
         \midrule 
         $\sin\alpha$   & 0 & $\dfrac{1}{2}$ & $\dfrac{\sqrt{2}}{2}$ & $\dfrac{\sqrt{3}}{2}$ & 1 & $\dfrac{\sqrt{3}}{2}$ & $\dfrac{1}{2}$ & 0 & -1 & 0 \\[.7em]
         $\cos\alpha$   & 1 & $\dfrac{\sqrt{3}}{2}$ & $\dfrac{\sqrt{2}}{2}$ & $\dfrac{1}{2}$ & 0 & -$\dfrac{1}{2}$ & -$\dfrac{\sqrt{3}}{2}$ & -1 & 0 & 1 \\[.7em]
         $\tan\alpha$   & 0 & $\dfrac{1}{\sqrt{3}}$ & $1$ & $\sqrt{3}$ & $\pm \infty$ & -$\sqrt{3}$ & -$\dfrac{1}{\sqrt{3}}$ & 0 & $\pm \infty$ & 0 \\[.7em]
         $\cot\alpha$   & $\pm\infty$ & $\sqrt{3}$ & $1$ & $\dfrac{1}{\sqrt{3}}$ & 0 & -$\dfrac{1}{\sqrt{3}}$ & -$\sqrt{3}$ & $\pm \infty$ & 0 & $\pm \infty$ \\

    \end{tabular}
    \vspace{-1cm}
\end{table}

\subsection{Graphische Darstellung der Winkelfunktionen}

Wir erlauben alle reellen Zahlen $x$ als Argumente in den Winkelfunktionen, d.\,h. der Definitionsbereich ist ganz $\mathbb{R}$ (für $\sin(x)$ und $\cos(x)$). Dabei sind die Funktionen periodisch 
\begin{align}
    \begin{split}
        \sin(x) &= \sin(x+2\pi n), \qquad \tan(x) = \tan(x+\pi n) \\
        \cos(x) &= \cos(x+2\pi n), \qquad \cot(x+\pi n)
    \end{split} \qq{mit} n \in \mathbb{Z}.
\end{align}

\begin{figure}[htp]
    \centering
    \begin{tikzpicture}
        \begin{axis}[disabledatascaling, axis lines=middle, xtick ={-pi,pi,2*pi,3*pi}, xticklabels={-$\pi$,$\pi$,$2\pi$, $3\pi$},ytick={-1,1}, xlabel={$x$}, ylabel={$y$}, height=5.5cm, width=0.9\textwidth, xmin =-pi-0.1, xmax=3*pi+0.2, samples =100, ymin=-1.3, ymax=1.3, legend pos = outer north east]
            \addplot[no marks, FSUblau, thick, domain=-pi:3*pi]{sin{deg(x)}};
            \addplot[no marks, PAForange, thick, domain=-pi:3*pi]{cos{deg(x)}};
            \legend{$\sin(x)$, $\cos(x)$};
        \end{axis}
    \end{tikzpicture}
    \begin{tikzpicture}
        \begin{axis}[disabledatascaling, axis lines=middle, xtick ={-pi,-pi/2,pi,2*pi,3*pi}, xticklabels={-$\pi$,,$\pi$,$2\pi$, $3\pi$},ytick={-1,1}, xlabel={$x$}, ylabel={$y$}, height=5.5cm, width=0.9\textwidth, xmin =-pi-0.1, xmax=3*pi+0.2, samples =100, ymin=-2.5, ymax=2.5, legend pos=outer north east]
            \addplot[no marks, FSUblau, thick, domain=-pi:-pi/2-0.1]{tan{deg(x)}};
            \addplot[no marks, PAForange, thick, domain=-pi:0-0.1]{cot{deg(x)}};
            \legend{$\tan(x)$, $\cot(x)$};
            \addplot[no marks, FSUblau, thick, domain=-pi/2+0.1:pi/2-0.1]{tan{deg(x)}};
            \addplot[no marks, FSUblau, thick, domain=-pi/2+0.1:pi/2-0.1]{tan{deg(x)}};
            \addplot[no marks, FSUblau, thick, domain=pi/2+0.1:3*pi/2-0.1]{tan{deg(x)}};
            \addplot[no marks, FSUblau, thick, domain=3*pi/2+0.1:5*pi/2-0.1]{tan{deg(x)}};
            \addplot[no marks, PAForange, thick, domain=0.1:pi-0.1]{cot{deg(x)}};
            \addplot[no marks, PAForange, thick, domain=pi+0.1:2*pi-0.1]{cot{deg(x)}};
            \addplot[no marks, PAForange, thick, domain=2*pi+0.1:3*pi-0.1]{cot{deg(x)}};
            \draw[thick, dashed] (2*pi,-2.5) -- (2*pi,2.5);
            \draw[thick, dashed] (5*pi/2,-2.5) -- (5*pi/2,2.5);
        \end{axis}
    \end{tikzpicture}
    \caption{Grafische Darstellung von Sinus bzw. Kosinus (oben) und Tangens bzw. Kotangens (unten).}
    \label{}
\end{figure}
Symmetrie-Eigenschaften der Funktionen lauten 
\begin{itemize}
    \item $\sin(-x) = -\sin(x), $ ungerade
    \item $\cos(-x) = +\cos(x), $ gerade
    \item $\tan(-x) = -\tan(x), $ ungerade
    \item $\cot(-x) = -\cot(x), $ ungerade
\end{itemize}

\paragraph{Umkehrfunktionen}$~$

Wir erhalten die Umkehrfunktionen durch Auflösen nach dem Argument $x$. Dabei definieren wir die folgendermaßen
\begin{itemize}
    \item Arkussinus: \hphantom{ko}\;\qquad$\arcsin(\sin x) = x$,
    \item Arkuskosinus: \qquad\;$\arccos(\cos x) = x$,
    \item Arkustangens: \quad\hphantom{ko}$\arctan(\tan x) = x$,
    \item Arkuskotangens: \quad$\text{acot}(\cot x) = x.$
\end{itemize}

Wir müssen beachten, dass der Definitionsbereich der periodischen Winkelfunktionen eingeschränkt werden muss, damit die Umkehrfunktionen eindeutig sind. Wir beschränken die Argumente (bzw. die Funktionswerte der Umkehrfunktionen) wie folgt: 
\begin{itemize}
    \item $\sin(x), \quad x\in\qty[-\frac{\pi}{2},\frac{\pi}{2}]$,
    \item $\cos(x), \quad x\in[0,\pi]$,
    \item $\tan(x), \quad x \in \qty(-\frac{\pi}{2},\frac{\pi}{2})$,
    \item $\text{acot}(x) \quad x \in (0,\pi).$
\end{itemize}

\begin{figure}[htp]
    \centering
    \begin{tikzpicture}
        \begin{axis}[disabledatascaling, axis lines=middle, xtick={-1,1}, ytick={-90,90, 180}, yticklabels={$-\frac{\pi}{2}$,$\frac{\pi}{2}$, $\pi$}, xlabel={$x$}, ylabel={$y$}, height=7cm, width=0.57\textwidth, xmin=-1.8, xmax=1.8, samples=100, ymax=220, ymin=-130]
            \addplot[no marks, FSUblau, thick, domain=-1:1]{asin(x)};
            \addplot[no marks, PAForange, thick, domain=-1:1]{acos(x)};
            \legend{$\arcsin(x)$, $\arccos(x)$};
            \draw[thin, dashed] (-1,180) -- (-.2,180);
            \draw[thin, dashed] (-1,-90) -- (-.4,-90);
        \end{axis}
    \end{tikzpicture}
    \hfill
    \begin{tikzpicture}
        \begin{axis}[disabledatascaling, axis lines=middle, ytick={-180,90,180}, yticklabels={$-\pi$,$\frac{\pi}{2}$,$\pi$}, xlabel={$x$}, ylabel={$y$}, height=7cm, width=0.57\textwidth, xmin=-5, xmax=5, samples=100, ymax=220, ymin=-130]
            \addplot[no marks, FSUblau, thick, domain=-5:5]{atan(x)};
            \addplot[no marks, PAForange, thick, domain=-5:5]{90 - atan(x)};
            \legend{$\arctan(x)$, $\text{acot}(x)$};
            \draw[thin,dashed] (-5,180) -- (-.7,180);
            \draw[thin,dashed] (-5,-90) -- (0,-90);\draw[thin,dashed] (0,90) -- (5,90);
        \end{axis}
    \end{tikzpicture}
    \caption{Grafische Darstellung der Umkehrfunktionen von Sinus bzw. Kosinus (links) und Tangens bzw. Kotangens (rechts).}
\end{figure}
Wir können aus der graphischen Darstellung die folgenden analytischen Eigenschaften ablesen: 
\begin{align}
    \arccos(x) &= \frac{\pi}{2} - \arcsin(x), \quad \text{atan}(x) = \frac{\pi}{2} - \arctan(x) \\
    \lim_{x\to \pm \infty} (\arctan(x)) &= \pm \frac{\pi}{2} \qq{und} \lim_{x\to\infty}(\text{acot}(x)) = 0, \quad \lim_{x\to-\infty}(\text{acot}(x)) = \pi.
\end{align}

\subsection{Definition durch Reihen}

Ebenso wie die Exponentialfunktion lassen sich auch die trigonometrischen Funktionen als Reihen darstellen bzw. definieren. Der Zusammenhang der Wineklfunktionen mit der e-Funktion wird im Vorkurs (mit Einführung der komplexen Zahlen) klar werden. Es gilt: 
\begin{align}
    \begin{split}
        \sin(x) &= \frac{x}{1!} - \frac{x^3}{3!} + \frac{x^5}{5!} - \frac{x^7}{7!} \pm \hdots = \sum_{n=0}^\infty (-1)^n \frac{(x)^{2n+1}}{(2n+1)!} \\
        \cos(x) &= 1 - \frac{x^2}{2!} + \frac{x^4}{4!} - \frac{x^6}{6!} \pm \hdots = \sum_{n=0}^\infty (-1)^n \frac{(x)^{2n}}{(2n)!}.
    \end{split}
\end{align}
Wir können weiterhin auch die Symmetrieeigenschaften von Sinus und Kosinus in den Reihenentwicklungen erkennen. Da die Sinusfunktion eine ungerade Funktion ist, tauchen hier nur ungerade Potenzen von $x$ auf. Weiterhin gilt 
\begin{align}
    \arctan(x) = x - \frac{x^3}{3} + \frac{x^5}{5} - \frac{x^7}{7} \pm \hdots = \sum_{n=0}^\infty (-1)^{n} \frac{(x)^{2n+1}}{2n+1}.
\end{align}
Die übrigen Reihen sehen etwas komplizierter aus und werden hier nicht aufgeführt.

\subsection{Additionstheoreme}

Für die Winkelfunktionen gelten die folgenden Additionstheoreme: 
\begin{mymathbox}[ams align, title={Additionstheoreme}, colframe={FSUblau}]
    \begin{split}
        \sin(x\pm y) &= \sin(x) \cos(y) \pm \cos(x) \sin(y),\\
    \cos(x\pm y) &= \cos(x) \cos(y) \mp \sin(x)\sin(y).
    \end{split}
\end{mymathbox}
Das zweite Theorem folgt aus dem Ersten (siehe Übung). Ebenfalls folgen Regeln für Tangens und Cotangens
\begin{align}
    \tan(x\pm y) = \frac{\tan(x) \pm \tan(y)}{1 \mp \tan(x) \tan(y)}.
\end{align}
Auf einen Beweis wird an dieser Stelle verzichtet.

Die Additionstheoreme können genutzt werden, um die Relationen für die Doppelwinkelfunktionen herzuleiten: 
\begin{align}
    \sin(2x) &= 2 \sin(x) \cos(x) \label{eqn:07_Doppelwinkel_sin}\\
    \cos(2x) &= 2 \cos^2(x) -1. \label{eqn:07_Doppelwinkel_cos}
\end{align}

Des weiteren lässt sich die Summe zweier Kosinusfunktionen auch als Produkt schreiben: 
\begin{align}
    \cos(x) + \cos(y) = 2 \cos(\frac{x+y}{2}) \cos(\frac{x-y}{2}).
\end{align}
% Diese Identität wird beispielsweise in der Mechanik genutzt, um die Schwebung zwischen zwei Tonsignalen der ähnlicher Frequenzen $f_1$ und $f_2$ zu veranschaulichen. Die (langsame) Frequenz des Schwebungssignals ist dann $\tilde{f} = \frac{|f_1 - f_2|}{2}$. 

\subsection{Ebene Trigonometrie}

\paragraph{Das rechtwinklige Dreieck}$~$

Sinus, Kosinus und Tangens können auch über Winkel und Längenverhältnisse im rechtwinkligen Dreieck definiert werden. 
\begin{figure}[htp]
    \centering
    \begin{tikzpicture}
        \coordinate (A) at (0,3){};
        \coordinate (B) at (-4,0){};
        \coordinate (C) at (0,0){};
        \draw[thick] (A) --node[above]{$c$} (B) --node[below]{$a$} (C) --node[right]{$b$} (A);
        \draw ($(A)+(0,-0.9)$) arc (-90:-90-53:.9);
        \draw ($(B)+(0.9,0)$) arc (0:37:.9);
        \draw ($(C) + (0,.7)$) arc (90:180:.7);
        \node (D) at ($(A) + (-110:.6)$){$\beta$};
        \node (D) at ($(B) + (20:.6)$){$\alpha$};
        \fill ($(C) + (135:.35)$) circle (1pt);
        \node[anchor=west] (text) at (2,3){$a:$ Ankathete (von $\alpha$)};
        \node[anchor=west] (text) at (2,2.2){$b:$ Gegenkathete (von $\alpha$)};
        \node[anchor=west] (text) at (2,1.4){$c:$ Hypotenuse};
        \node[anchor=west] (text) at (2,.4){Pythagoras: $a^2 + b^2 = c^2$};
        \node[anchor=west] (text) at (2,-.6){Innenwinkel: $\alpha + \beta = \frac{\pi}{2}$};
    \end{tikzpicture}
\end{figure}

Wir können nun die Winkelfunktionen definieren als 
\begin{align}
    \sin(\alpha) = \frac{b}{c}, \quad \cos(\alpha) = \frac{a}{c}, \quad \tan(\alpha) = \frac{b}{a}.
\end{align}
Setzen wir $c=1$ so erhalten wir wieder die Definition der Winkelfunktionen am Einheitskreis. Der Flächeninhalt des rechtwinkligen Dreiecks ergibt sich zu $A = \frac{1}{2}ab$.

\paragraph{Schiefwinkliges Dreieck}$~$

Für ein allgemeines, schiefwinkliges Dreieck können wir o.\,B.\,d.\,A. für $\alpha < \frac{\pi}{2}, \gamma < \frac{\pi}{2}$ zwei Fälle unterscheiden: 

\begin{figure}[htp]
    \centering
    \begin{tikzpicture}
        \coordinate (A) at (-4,0){};
        \coordinate (B) at (2,0){};
        \coordinate (C) at ($(A)+(45:5.5)$){};
        \draw[thick] (A) --node[below]{$c$} (B) --node[right]{$a$} (C) --node[above left]{$b$} (A);
        \draw ($(A)+(0.9,0)$) arc (0:45:.9);
        \draw ($(B)+(-0.9,0)$) arc (180:115:.9);
        \draw ($(C) + (225:.9)$) arc (225:300:.9);
        \node (D) at ($(B) + (150:.6)$){$\beta$};
        \node (D) at ($(A) + (22.5:.6)$){$\alpha$};
        \node (D) at ($(C) + (250+12.5:.6)$){$\gamma$};
        \draw[gray, dashed] (A) --node[above]{$h_a$} +(25:6);
        \draw[gray, dashed] (B) --node[above, pos=0.3]{$h_b$} +(180-42:5);
        \draw[gray, dashed] (C) --node[right,pos=0.7]{$h_c$} (0,0);

        \node[anchor=west] (text) at (-4,-1.5){Innenwinkel: $\alpha + \beta +\gamma = \pi$};
        \node[anchor=west] (text) at (-4,-2.3){Flächeninhalt: $A = \frac{1}{2}ab \sin\gamma$};
        \node[anchor=west] (text) at (-4,4){$\beta < \frac{\pi}{2}$};
        \draw[thick, {latex}-{latex}] (-4,-.5) --node[below]{$p$} +(5.5/1.41,0); 
        \draw[thick, {latex}-{latex}] (-4+5.5/1.41,-.5) --node[below]{$q$} (2,-.5); 

        \begin{scope}[shift={(6,0)}]
            \coordinate (A) at (-3,0){};
            \coordinate (B) at (2,0){};
            \coordinate (C) at ($(A)+(30:7.5)$){};
            \draw[thick] (A) --node[below]{$c$} (B) --node[right]{$a$} (C) --node[above left]{$b$} (A);
            \draw ($(A)+(1.2,0)$) arc (0:30:1.2);
            \draw ($(B)+(-0.9,0)$) arc (180:70:.9);
            \draw ($(C) + (180+30:.9)$) arc (180+30:180+30+40:.9);
            \node (D) at ($(B) + (125:.6)$){$\beta$};
            \node (D) at ($(A) + (15:.9)$){$\alpha$};
            \node (D) at ($(C) + (180+30+20:.6)$){$\gamma$}; 
            \draw[gray, dashed] (A) --node[above]{$h_a$} +(-22:5.5);
            \draw[gray, dashed] (B) --node[above]{$h_b$} +(120:3);
            \draw[gray, dashed] (C) --node[right]{$h_c$} +(0,-4);
            \draw[gray] (B) -- +(2,0);
            \draw[gray] (B) -- +(180+68:3.5);
            
        \node[anchor=west] (text) at (-3,4){$\beta > \frac{\pi}{2}$};
        \end{scope}
    \end{tikzpicture}
    \caption{Geometrie eines allgemeinen schiefwinkligen Dreiecks für die beiden Fälle $\beta < \frac{\pi}{2}$ und $\beta > \frac{\pi}{2}$. Die Höhen der einzelnen Seiten sind jeweils auch grau gestrichelt eingezeichnet.}
    \label{fig:07_Dreieck}
\end{figure}

In beiden Varianten lässt sich folgendes ablesen: 
\begin{align}
    \begin{rcases}
        h_a = c \sin(\beta) = b \sin(\gamma) \\
        h_b = c \sin(\alpha) = a \sin(\gamma) \\
        h_c = a \sin(\beta) = b \sin(\alpha)
    \end{rcases} \quad \Rightarrow \quad \frac{a}{\sin(\alpha)} = \frac{b}{\sin(\beta)} = \frac{c}{\sin(\gamma)}.
\end{align}
Dieses Ergebnis wird \emph{Sinussatz} genannt. Die Seiten eines Dreiecks verhalten sich also zueinander wie die Sinus der gegenüberliegenden Winkel.

\paragraph{Kosinussatz}$~$

Wir wollen uns nun noch einmal o.\,B.\,d.\,A. mit dem Fall $\beta < \frac{\pi}{2}$ beschäftigen und das Dreieck (siehe Abb.~\ref{fig:07_Dreieck} links) entlang der Höhe $h_c$ in zwei Teildreiecke aufteilen. Dann gilt nach Pythagoras 
\begin{align}
    \begin{rcases}
        a^2 = h_c^2 + q^2 \\
        b^2 = h_c^2 + p^2
    \end{rcases} \quad a^2 = (b^2-p^2) + q^2 \overset{q = c-p}{=} b^2 + c^2 -2cp.
\end{align}
Nutzen wir außerdem noch $p = b \cos(\alpha)$, so folgt daraus 
\begin{mymathbox}[ams align, title={Kosinussatz}, colframe={FSUblau}]
    \begin{split}
        a^2 &= b^2 + c^2 - 2bc \cos(\alpha)\\
        b^2 &= c^2 + a^2 - 2ac \cos(\beta)\\
        c^2 &= a^2 + b^2 - 2ab \cos(\gamma).
    \end{split}
\end{mymathbox}
Die anderen Gleichungen erhalten wir durch zyklische Vertauschung der Variablen $a,b,c$ und $\alpha,\beta,\gamma$.

Formulieren wir den Kosinussatz noch einmal in Worten: Das Quadrat einer Seite ist gleich der Summe der Quadrate der beiden anderen Seiten, verringert um das doppelte Produkt diesr beiden Seiten mit dem Kosinus des von ihnen eingeschlossenen Winkels.

Wir wollen abschließend noch anmerken, dass der Kosinussatz ebenfalls auch für den Fall $\beta > \frac{\pi}{2}$ gilt. Er stellt eine Verallgemeinerung des Satzes des Pythagoras für allgemeine Dreiecke dar.

\clearpage
\paragraph{Beispiel: Heron'sche Inhaltsformel (Heron von Alexandria)}$~$

Wir wollen in diesem Abschnitt eine Formel für den Flächeninhalt eines Dreiecks angeben, welches nur durch seine Seitenlängen bestimmt ist. Beginnen wir beim Kosinussatz und formen dies etwas um, so folgt 
\begin{align}
    \cos(\alpha) &= \frac{b^2 + c^2 - a^2}{2bc} \notag \\
    \Leftrightarrow 1+\cos(\alpha) &= \frac{b^2 + c^2 -a^2 +2bc}{2bc} = \frac{(b+c)^2 -a^2}{2bc} \overset{\eqref{eqn:1_binomische_Formeln}}{=} \frac{(b+c+a)(b+c-a)}{2bc} \notag \\
     2 \cos^2 \qty(\frac{\alpha}{2}) &= \frac{(b+c+a)(b+c-a)}{2bc}. 
\end{align}
Die letzte Gleichheit folgt dabei aus den Formeln für die Doppelwinkelfunktionen~\eqref{eqn:07_Doppelwinkel_cos}. Führen wir nun die Variable 
\begin{align}
    s = \frac{1}{2}(a+b+c)
\end{align}
als halben Umfang des Dreiecks ein, so erhalten wir mit 
\begin{align}
    s-a = \frac{b+c-a}{2}, \quad s-b = \frac{a+c-b}{2}, \quad s-c = \frac{a+b-c}{2}
\end{align}
durch zyklisches Durchtauschen folgendes Ergebnis: 
\begin{align}
    \Rightarrow 2\cos^2\qty(\frac{\alpha}{2}) = \frac{2s \cancel{2}(s-a)}{\cancel{2}bc} \quad \Rightarrow \quad \cos^2\qty(\frac{\alpha}{2}) &= \frac{s(s-a)}{bc} \notag \\
    \Rightarrow \quad \cos^2\qty(\frac{\beta}{2}) &= \frac{s(s-b)}{ac} \\
    \Rightarrow \quad \cos^2\qty(\frac{\gamma}{2}) &= \frac{s(s-c)}{ab}. \notag
\end{align}
Auf dem selben Weg folgen aus der Relation 
\begin{align}
    1- \cos(\alpha) &= \frac{a^2 -b^2 -c^2 +2bc}{2bc} \notag \\
    \Rightarrow \sin^2\qty(\frac{\alpha}{2}) &= \frac{(s-b)(s-c)}{bc}, \quad \sin^2\qty(\frac{\beta}{2}) = \frac{(s-a)(s-c)}{ac}, \quad \sin^2\qty(\frac{\gamma}{2}) = \frac{(s-a)(s-b)}{ab}.
\end{align}
Damit ergibt sich insgesamt für den Flächeninhalt des Dreiecks 
\begin{align}
    A = \frac{1}{2} ab \sin(\gamma) \overset{\eqref{eqn:07_Doppelwinkel_sin}}&{=} ab \sin(\frac{\gamma}{2})\cos(\frac{\gamma}{2}) = \cancel{ab} \sqrt{\frac{(s-a)(s-b)}{\cancel{ab}}} \sqrt{\frac{s(s-c)}{\cancel{ab}}} \notag \\
    &= \sqrt{s(s-a)(s-b)(s-c)}.
\end{align}
