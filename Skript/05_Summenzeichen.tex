\section{Das Summenzeichen}

Für viele Anwendungen erweist es sich als praktisch, sehr lange (oder auch unendliche) Summen kompakt aufzuschreiben. Betrachten wir eine Summe $S$ aus $n$ Summanden, 
\begin{align}
    S = s_1 + s_2 + s_3 + \hdots + s_n,
\end{align}
wobei jeder Summand mit einem \emph{Index} gekennzeichnet ist, 
\begin{align}
    s_i \qq{,} i=1,2,\hdots,n.
\end{align}
Die Indizes dienen zunächst nur dazu, die Summanden zu unterscheiden und sind im Allgemeinen willkürlich gesetzt - schließlich hängt der Wert der Summe nicht von der Summationsreihenfolge ab. Als Kurzschreibweise für Summen verwendet man den großen griechischen Buchstaben Sigma:
\begin{align}
    S = \sum_{i=1}^n s_i \qq{, mit Startwert $i=1$ und Endwert $i=n$.} 
\end{align}

\paragraph{Beispiel: Polynome}$~$

Hier sind die Summanden die jeweiligen Potenzen von $x$ mit ihren Vorfaktoren und es bietet sich an, die Potenzen als Indizes zu benutzen: 
\begin{align}
    P_n(x) &= a_0 + a_1 x^1 + a_2 x^2 + \hdots + a_{n-1} x^{n-1} + a_n x^n  = \sum_{i=0}^n a_i x^i.
\end{align}

\paragraph{Beispiel: Summe der ersten $n$ Zahlen}$~$

Hier bietet sich an, die Zahlen selbst als Indizes zu benutzen: 
\begin{align}
    \sum_{k=1}^n k = 1 + 2 + 3 + \hdots + n-1 +n.
\end{align}

\paragraph{Eigenschaften:}$~$

\begin{itemize}
    \item Die Benennung des Summationsindex ist irrelevant, 
    \begin{align}
        \sum_{i=1}^n s_i = \sum_{k=1}^n s_k.
    \end{align}
    \item Summen von Summen/Differenzen sind Summen/Differenzen von Summen, 
    \begin{align}
        \sum_{i=1}^n (a_i \pm b_i) = \sum_{i=1}^n a_i \pm \sum_{i=1}^n b_i.
    \end{align}
    \item Gleiche (vom Index unabhängige) Faktoren können ausgeklammert werden, 
    \begin{align}
        \sum_{i=1}^n (a s_i) &= a \sum_{i=1}^n s_i. \notag \\
        \Rightarrow \qq{speziell:} \sum_{i=1}^n a &= a \sum_{i=1}^n 1 = a\underbrace{(1+1+1+\hdots+1)}_{n\text{-mal}} = n \cdot a. \notag 
    \end{align}
    \item Summen können aufgeteilt werden,
    \begin{align}
        \sum_{i=1}^n s_i = \sum_{i=1}^m s_i + \sum_{i=m+1}^n s_i.
    \end{align}
    \item Startwerte können verändert werden, 
    \begin{align}
        \sum_{k=m}^n s_k &= \sum_{k=1}^n s_k - \sum_{k=1}^{m-1} s_k \\ 
        \Rightarrow \qq{speziell:} \sum_{k=m}^n 1 &= \sum_{k=1}^n 1 - \sum_{k=1}^{m-1} 1 = n -(m-1) = 1 + n -m. \notag 
    \end{align}
\end{itemize}

\paragraph{Bemerkung} Es gibt einige weitere Möglichkeiten, Summen zu schreiben, bspw. kann der Index Werte einer (abzählbaren) Menge $M$ annehmen, 
\begin{align}
    \sum_{i \in M} s_i \qq{,}
\end{align}
oder man lässt die Summationsgrenzen weg, wenn in irgendeiner Weise ``klar'' ist, wie summiert wird, 
\begin{align}
    \sum_i s_i. 
\end{align}
Mehrfache Summen können mit einem Summenzeichen geschrieben werden, sofern sie unabhängig voneinander laufen:
\begin{align}
    \sum_{i=1}^n \qty(\sum_{j=1}^n s_{ij}) = \sum_{i,j=1}^n s_{ij} = s_{11} + s_{12} + \hdots + s_{21} + s_{22} + \hdots + s_{nn}.
\end{align}
Dabei ist es egal, welche Summe zuerst ausgeführt wird. Letzteres gilt nicht in Fällen wie beispielsweise 
\begin{align}
    \sum_{i=1}^n \sum_{j=1}^i s_{ij},
\end{align}
da hier der Endwert der zweiten Summe vom Index der ersten Summe abhängt.

\clearpage
\paragraph{Beispiel: Differenzen benachbarter Quadratzahlen}$~$

Wenn wir untersuchen wollen, was die Summe aus den Differenzen benachbarter Quadratzahlen ist, so können wir das zunächst für die ersten paar Glieder aufschreiben: 
\begin{figure}[htp]
    \centering
    \begin{tikzpicture}
        \draw[thick, -{latex}] (0,0) -- (10,0)node[right]{$n$};
        \foreach \x in {0,1,4,9}{
            \draw (\x,.1) -- (\x,-.1)node[below]{\x};
        }
        \draw [decorate, decoration = {calligraphic brace}, thick] (1,-.7) --node[below]{1} (0,-.7);
        \draw [decorate, decoration = {calligraphic brace}, thick] (4,-.7) --node[below]{3} (1,-.7);
        \draw [decorate, decoration = {calligraphic brace}, thick] (9,-.7) --node[below]{5} (4,-.7);
    \end{tikzpicture}
\end{figure}
Dies legt die Vermutung nahe, dass es sich um die Summe aller ungeraden Zahlen handelt. Rechnen wir dies nach, ergibt sich 
\begin{align}
    \sum_{k=0}^n \qty[(k+1)^2 - k^2] &= \sum_{k=0}^n (k+1)^2 - \sum_{k=0}^n k^2 \notag \\   
    &= \sum_{k=0}^n (\cancel{k^2} +2k +1) - \cancel{\sum_{k=0}^n k^2} = \sum_{k=0}^n (2k+1).
\end{align}
Das ist in der Tat die Summe aller ungeraden Zahlen von 1 bis $2n+1$.

\paragraph{Indexverschiebung}$~$

Da die Indizierung einzelner Summanden willkürlich ist, ist es möglich, eine \emph{Indexverschiebung} vorzunehmen:
\begin{align}
    S &\tikzmarknode{eq1}{=} \sum_{i=1}^n s_i  \qq{, Substitution} i = j + a  \notag \\
                        & \hspace{3cm} \Rightarrow \qq{Startwert: aus} i=1 \qq{folgt} j = 1-a \notag \\
                        & \hspace{3cm} \Rightarrow \qq{Endwert: aus} i=n \qq{folgt} j = n-a \notag \\[-1em]
      &\tikzmarknode{eq2}{=} \sum_{j=1-a}^{n-a} s_{j+a}.
\end{align}
\tikz[overlay,remember picture]{
\draw[thick] ($(eq1)+(0,-.2)$)-- ($(eq2)+(0,.2)$);
}
Schauen wir uns dazu das vorherige Beispiel nochmal an:
\begin{align}
    \sum_{k=0}^n (k+1)^2 - \sum_{k=0}^n k^2 \overset{k = i+1}&{=} \sum_{k=0}^n (k+1)^2 - \sum_{i=-1}^{n-1} (i+1)^2 \overset{i\to k}{=} \underbrace{\sum_{k=0}^n (k+1)^2}_{(1)} - \underbrace{\sum_{k=-1}^{n-1} (k+1)^2}_{(2)} \notag\\
     &= \underbrace{(n+1)^2}_{\text{letztes Glied (1)}} + \cancel{\sum_{k=0}^{n-1}(k+1)^2} + \underbrace{(-1+1)^2}_{\text{erstes Glied (2)}} - \cancel{\sum_{k=0}^{n-1}(k+1)} \notag \\
     \Rightarrow \sum_{k=0}^n (2k+1) = (n+1)^2.
\end{align}
Die Summe der ersten $n$ ungeraden zahlen ist gleich der $(n+1)$-ten Quadratzahl.