\section{Der binomische Satz}

Der binomische Satz (auch: Binomialtheorem) ermöglicht die Entwicklung von Binomen $(a+b)^n$ in Potenzen von $a$ und $b$, also das ``Ausmultiplizieren''. 

\subsection{Binomialkoeffizienten}
Für zwei natürliche Zahlen $n,k \in \mathbb{N}_0$ ist der Binomialkoeffizient $\binom{n}{k}$, sprich: ``$n$ über $k$'', definiert als 
\begin{align}
    \binom{n}{k} := \frac{n(n-1)(n-2)\hdots (n-k+1)}{k!}.
\end{align}
Beachte, dass der Binomialkoeffizient nur von Null verschieden ist, wenn $k \le n$. Betrachten wir einige Beispiele: 
\begin{align}
    \begin{split}
        \binom{7}{3} &= \frac{7\cdot 6\cdot 5}{3!} = 35, \quad \binom{10}{7} = \frac{10\cdot 9 \cdot 8 \cdot 7 \cdot 6 \cdot 5 \cdot 4}{7!} = 120, \\
        \binom{3}{5} &= \frac{3 \cdot 2 \cdot 1 \cdot 0 \cdot (-1)}{5!} = 0.
    \end{split}
\end{align}
Sofern $k \le n$ gilt, kann man schreiben: 
\begin{align}
    \binom{n}{k} = \frac{n(n-1)\hdots (n-k+1)}{k!} \cdot \underbrace{\frac{(n-k)(n-k-1)\hdots 2 \cdot 1}{(n-k)!}}_{1} = \frac{n!}{k!(n-k)!}.
\end{align}
Eine interessante Eigenschaft des Binomialkoeffizienten ist, dass sich der Wert nicht ändern, wenn man $k$ durch $n-k$ ersetzt:
\begin{mymathbox}[ams align, title={Binomialkoeffizent}, colframe={FSUblau}]
\binom{n}{k} = \frac{n!}{k!(n-k)!} \qq{,} \binom{n}{k} = \binom{n}{n-k} \qq{für} k \le n.
\end{mymathbox}

Wir betrachten im Folgenden einige Spezialfälle: 
\begin{align}
    \binom{n}{0} &= \frac{n!}{n! 0!} = 1, \notag \\
    \binom{n}{1} &= \frac{n!}{1!(n-1)!} = \frac{n(n-1)!}{(n-1)!} =n, \\
    \binom{n}{n} &= \frac{n!}{n! 0!} = 1. \notag
\end{align}
Außerdem gilt die Rekursionsrelation 
\begin{align}
    \binom{n}{k} + \binom{n}{k+1} &= \binom{n+1}{k+1}, \label{eqn:11_Rekursion}\\
    \qq{denn:} \binom{n}{k} + \binom{n}{k+1} &= \frac{n(n-1)\hdots (n-k+1)}{k!} + \frac{n(n-1)\hdots (n-k)}{(k+1)!} \notag \\
    &= \frac{n(n-1)\hdots (n-k+1)}{(k+1)!} \qty[(\cancel{k}+1)+(n-\cancel{k})] \notag \\
    &= \frac{(n+1)n(n-1) \hdots (n-k+1)}{(k+1)!} = \binom{n+1}{k+1}.
\end{align}

Das Produkt $n(n-1)\hdots(n-k+1)$ wird auch als ``fallende Faktorielle'' oder ``absteigendes Pochhammer-Symbol'' bezeichnet und mit $[n]_k, (n)_k$ oder $n^{\underline{k}}$ abgekürzt.

Eine Verallgemeinerung der Binomialkoeffizienten für reelle $n$ (und natürliche $k$) ist mit oben gegebener Definition ohne Weiteres möglich. Beachte, dass $\binom{n}{k}$ dann auch für $k>n$ von Null verschiedene Werte annimmt, bspw. 
\begin{align}
    \binom{1/2}{4} = \frac{1/2\cdot (-1/2)\cdot (-3/2) \cdot (-5/2)}{4!} = -\frac{5}{128}. 
\end{align}
Die Binomialkoeffizienten können (sowohl in $n$ als auch in $k$) als Folge aufgefasst werden; sie ist weder arithmetisch noch geometrisch. 

\newpage
\subsection{Der binomische Satz}

Mithilfe des Binomialkoeffizienten des letzten Abschnitts können wir nun den binomsichen Satz formulieren: 
\begin{mymathbox}[ams align, title={binomischer Lehrsatz}, colframe={FSUblau}]
    (a+b)^n = \sum_{k=0}^n \binom{n}{k} a^{n-k} b^k \qq{für} a,b \in \mathbb{R}, n\in\mathbb{N}_0
\end{mymathbox}
Wir wollen im Folgenden den Satz durch Induktion beweisen. Dafür betrachten wir zunächst die Formel für $n=0,1,2$ 
\begin{enumerate}
    \item[(IA)] $\displaystyle n=0: \quad  \binom{0}{0} a^0 b^0 = 1\quad \checkmark \qquad \displaystyle n=1: \quad  \binom{1}{0} a^1 b^0 + \binom{1}{1} a^0 b^1= a+b \quad\checkmark$ \\
    $\displaystyle n=2: \quad  \binom{2}{0} a^2 b^0 + \binom{2}{1} a^1 b^1 + \binom{2}{2} a^0 b^2= a^2 + 2ab +b^2 \quad\checkmark$
    \item[(IV)] $\displaystyle n=l: \quad (a+b)^l = \sum_{k=0}^l \binom{l}{k} a^{l-k} b^k$
    \item[(IB)] $\displaystyle n=l: \quad (a+b)^{l+1} = \sum_{k=0}^{l+1} \binom{l+1}{k} a^{l+1-k} b^k$\\
    \begin{proof}$~$\\[-1.65cm]
        \begin{align}
            \;\;(a+b)^{l+1} &= (a+b)^l (a+b) \overset{\text{(IV)}}{=} \sum_{k=0}^l \binom{l}{k} a^{l-k} b^k (a+b) \notag \\
            &\tikzmarknode{eq1}{}= \sum_{k=0}^l \binom{l}{k} a^{l+1-k} b^k + \sum_{k=0}^l \binom{l}{k} a^{l-k} b^{k+1} \notag \\[3mm]
            &\qquad \text{Indextransformation } m=k+1 \Rightarrow k=m-1: \notag \\
            &\qquad \sum_{m=1}^{l+1} \binom{l}{m-1} a^{l+1-m} b^m \qq{,} \qq{Umbennennung} m \to k \notag \\[3mm]
            &\tikzmarknode{eq2}{}= \binom{l}{0} a^{l+1} + \sum_{k=1}^l \qty[\binom{l}{k} + \binom{l}{k-1}] a^{l+1-k} b^k + \binom{l}{l} b^{l+1} \notag \\
            &\tikzmarknode{eq3}{}= a^{l+1} + \sum_{k=1}^l \binom{l+1}{k} a^{l+1-k} b^k + b^{l+1} \notag \\[3mm]
            & \qquad \text{für } k=0: \quad \binom{l+1}{0} a^{l+1} = \binom{l}{0} a^{l+1} = a^{l+1}. \notag \\ 
            & \qquad \text{für } k=l+1: \quad \binom{l+1}{l+1} b^{l+1} = \binom{l}{l} b^{l+1} = b^{l+1}. \notag \\[3mm]
            &\tikzmarknode{eq4}{}= \sum_{k=0}^{l+1} \binom{l+1}{k} a^{l+1-k} b^k.  \\[-1.4cm] \notag
        \end{align} 
        \tikz[overlay, remember picture]{
        \draw[shorten >=2pt,shorten <=2pt, thick] ($(eq1)+(0.2,0)$) -- ($(eq2)+(0.2,0.2)$);
        \draw[shorten >=2pt,shorten <=2pt, thick] ($(eq3)+(0.2,0)$) -- ($(eq4)+(0.2,0.2)$);
        }
    \end{proof}
\end{enumerate}

\paragraph{Pascal'sches Dreieck}$~$

Eine einfache Möglichkeit, die Werte der Binomialkoeffizienten zu berechnen, stellt das Pascal'sche Dreieck dar (siehe Abbildung\ref{fig:11_Pascal1}). 
\begin{figure}[htp]
    \centering
    \begin{tikzpicture}
        \foreach{\y} in {0,...,5}{
            \node (A) at (-\y,-\y){$\displaystyle \binom{\y}{0}$};
        }
        \foreach{\y} in {1,...,5}{
            \node (A) at (2-\y,-\y){$\displaystyle \binom{\y}{1}$};
        }
        \foreach{\y} in {2,...,5}{
            \node (A) at (4-\y,-\y){$\displaystyle \binom{\y}{2}$};
        }
        \foreach{\y} in {3,...,5}{
            \node (A) at (6-\y,-\y){$\displaystyle \binom{\y}{3}$};
        }
        \foreach{\y} in {4,...,5}{
            \node (A) at (8-\y,-\y){$\displaystyle \binom{\y}{4}$};
        }
        \node (A) at (10-5,-5){$\displaystyle \binom{5}{5}$};
        \node (A) at (8,1){$\displaystyle n$};
        \foreach{\y} in {0,...,5}{
            \node (A) at (8,-\y){$\displaystyle \y$};
        }
        \foreach{\y} in {0,...,5}{
            \node (A) at (1+\y,-\y+1){$\displaystyle k = \y$};
        }
        \foreach{\x} in {1.5,...,5.5}{
            \draw[dashed] (\x,-\x+2) -- +(225:10-1.4*\x);  
        }
    \end{tikzpicture}
    \caption{Das Pascal'sche Dreieck der Binomialkoeffizienten.}
    \label{fig:11_Pascal1}
\end{figure}

Wir können nun die Binomialkoeffizienten im Pascal'schen Dreieck mithilfe der Rekursionsrelation~\eqref{eqn:11_Rekursion} berechnen. Dafür nehmen wir als Voraussetzung, dass $\binom{n}{1} = 1 = \binom{n}{n}$ gelten. In der Art und Weise wie wir das Pascal'sche Dreieck gezeichnet haben, ergibt sich dann der Wert eines Binomialkoeffizienten als die Summe der schräg darüberliegenden Binomialkoeffizienten (siehe rote Pfeile in Abb.~\ref{fig:11_Pascal2}).

\begin{figure}[htp]
    \centering
    \begin{tikzpicture}[scale=0.7]

        \foreach{\x} in {0,...,7}{
            \foreach{\y} in {\x,...,7}{
                \pgfmathsetmacro{\binomial}{int(round(factorial(\y)/(factorial(\x)*factorial(\y-\x))))}
            \node at (2*\x-\y,-\y) {$\binomial$};
            } 
            \node (A) at (10,-\x){$\displaystyle \x$};
            \node (A) at (1+\x,-\x+1){$\displaystyle k = \x$};
        }

        \node (A) at (10,1){$\displaystyle n$};
        % \foreach{\y} in {0,...,5}{
        %     \node (A) at (9,-\y){$\displaystyle \y$};
        % }
        % \foreach{\y} in {0,...,5}{
            
        % }
        \foreach{\x} in {1.5,...,7.5}{
            \draw[dashed] (\x,-\x+2) -- +(225:13-1.4*\x);  
        }
        \draw[thick, red, -{latex}] ($(0,-2)+(-.2,-.2)$) -- +(-.6,-.6);
        \draw[thick, red, -{latex}] ($(-2,-2)+(.2,-.2)$) -- +(.6,-.6);

    \end{tikzpicture}
    \caption{Das Pascal'sche Dreieck mit den berechneten Binomialkoeffizienten. Zur zeilenweisen Berechnung der Werte nutzen wir die Rekursionsrelation~\eqref{eqn:11_Rekursion}. Wir erkennen zudem im Pascal'schen Dreieck die Symmetrie unter $k \mapsto n-k$.}
    \label{fig:11_Pascal2}
\end{figure}

Wir können mithilfe des Pascal'schen Dreiecks Binome unmittelbar ausmultiplizieren, bspw. (siehe letzte Zeile von Abb.~\ref{fig:11_Pascal2}): 
\begin{align}
    (a+b)^7 = a^7 + 7 a^6 b + 21 a^5 b^2 + 35 a^4 b^3 + 35 a^3 b^4 + 21 a^2 b^5 + 7 ab^6 + b^7.
\end{align}

\paragraph{Weitere Bemerkungen}$~$

Der binomische Satz liefert uns insbesondere eine Reihendarstellung der Funktion $f_n(x) = (1+x)^n$: 
\begin{align}
    f_n(x) = \sum_{k=0}^n \binom{n}{k} x^k = \binom{n}{0} + \binom{n}{1} x + \binom{n}{2} x^2 + \hdots.
\end{align}
Durch Verallgemeinerung auf nicht-ganze $n$ können weitere nützliche Reihendarstellungen gefunden werden, bspw. 
\begin{align}
    \sqrt{1+x} = \sum_{k=0}^\infty \binom{1/2}{k} x^k \qq{(Newton'sches Binomialtheorem).}
\end{align}
Es existiert auch eine sogenanntens Multinomialtheorem für Ausdrücke der Form 
\begin{align}
    (x_1 + x_2 + \hdots + x_j)^n.
\end{align}