\section{Der binomische Satz}

Der binomische Satz (auch: Binomialtheorem) ermöglicht die Entwicklung von Binomen $(a+b)^n$ in Potenzen von $a$ und $b$, also das ``Ausmultiplizieren''. 

\subsection{Binomialkoeffizienten}
Für zwei natürliche Zahlen $n,k \in \mathbb{N}_0$ ist der Binomialkoeffizient $\binom{n}{k}$, sprich: ``$n$ über $k$'', definiert als 
\begin{align}
    \binom{n}{k} := \frac{n(n-1)(n-2)\hdots (n-k+1)}{k!}.
\end{align}
Beachte, dass der Binomialkoeffizient nur von Null verschieden ist, wenn $k \le n$. Betrachten wir einige Beispiele: 
\begin{align}
    \begin{split}
        \binom{7}{3} &= \frac{7\cdot 6\cdot 5}{3!} = 35, \quad \binom{10}{7} = \frac{10\cdot 9 \cdot 8 \cdot 7 \cdot 6 \cdot 5 \cdot 4}{7!} = 120, \\
        \binom{3}{5} &= \frac{3 \cdot 2 \cdot 1 \cdot 0 \cdot (-1)}{5!} = 0.
    \end{split}
\end{align}
Sofern $k \le n$ gilt, kann man schreiben: 
\begin{align}
    \binom{n}{k} = \frac{n(n-1)\hdots (n-k+1)}{k!} \cdot \underbrace{\frac{(n-k)(n-k-1)\hdots 2 \cdot 1}{(n-k)!}}_{1} = \frac{n!}{k!(n-k)!}.
\end{align}
Eine interessante Eigenschaft des Binomialkoeffizienten ist, dass sich der Wert nicht ändern, wenn man $k$ durch $n-k$ ersetzt:
\begin{mymathbox}[ams align, title={Binomialkoeffizent}, colframe={FSUblau}]
\binom{n}{k} = \frac{n!}{k!(n-k)!} \qq{,} \binom{n}{k} = \binom{n}{n-k} \qq{für} k \le n.
\end{mymathbox}

Wir betrachten im Folgenden einige Spezialfälle: 
\begin{align}
    \binom{n}{0} &= \frac{n!}{n! 0!} = 1, \notag \\
    \binom{n}{1} &= \frac{n!}{1!(n-1)!} = \frac{n(n-1)!}{(n-1)!} =n, \\
    \binom{n}{n} &= \frac{n!}{n! 0!} = 1. \notag
\end{align}
Außerdem gilt die Rekursionsrelation 
\begin{align}
    \binom{n}{k} + \binom{n}{k+1} &= \binom{n+1}{k+1}, \\
    \qq{denn:} \binom{n}{k} + \binom{n}{k+1} &= \frac{n(n-1)\hdots (n-k+1)}{k!} + \frac{n(n-1)\hdots (n-k)}{(k+1)!} \notag \\
    &= \frac{n(n-1)\hdots (n-k+1)}{(k+1)!} \qty[(\cancel{k}+1)+(n-\cancel{k})] \notag \\
    &= \frac{(n+1)n(n-1) \hdots (n-k+1)}{(k+1)!} = \binom{n+1}{k+1}.
\end{align}

Das Produkt $n(n-1)\hdots(n-k+1)$ wird auch als ``fallende Faktorielle'' oder ``absteigendes Pochhammer-Symbol'' bezeichnet und mit $[n]_k, (n)_k$ oder $n^{\underline{k}}$ abgekürzt.

Eine Verallgemeinerung der Binomialkoeffizienten für reelle $n$ (und natürliche $k$) ist mit oben gegebener Definition ohne Weiteres möglich. Beachte, dass $\binom{n}{k}$ dann auch für $k>n$ von Null verschiedene Werte annimmt, bspw. 
\begin{align}
    \binom{1/2}{4} = \frac{1/2\cdot (-1/2)\cdot (-3/2) \cdot (-5/2)}{4!} = -\frac{5}{128}. 
\end{align}
Die Binomialkoeffizienten können (sowohl in $n$ als auch in $k$) als Folge aufgefasst werden; sie ist weder arithmetisch noch geometrisch. 

\subsection{Der binomische Satz}

Mithilfe des Binomialkoeffizienten des letzten Abschnitts können wir nun den binomsichen Satz formulieren: 
\begin{mymathbox}[ams align, title={binomischer Lehrsatz}, colframe={FSUblau}]
    (a+b)^n = \sum_{k=0}^n \binom{n}{k} a^{n-k} b^k \qq{für} a,b \in \mathbb{R}, n\in\mathbb{N}_0
\end{mymathbox}
Wir wollen im Folgenden den Satz der Induktion beweisen. Dafür betrachten wir zunächst die Formel für $n=0,1,2$ 
\begin{enumerate}
    \item[(IA)] $\displaystyle n=0: \quad  \binom{0}{0} a^0 b^0 = 1\quad \checkmark$ \\
    $\displaystyle n=1: \quad  \binom{1}{0} a^1 b^0 + \binom{1}{1} a^0 b^1= a+b \quad\checkmark$ \\
    $\displaystyle n=1: \quad  \binom{2}{0} a^2 b^0 + \binom{2}{1} a^1 b^1 + \binom{2}{2} a^0 b^2= a^2 + 2ab +b^2 \quad\checkmark$
    \item[(IV)] $\displaystyle n=l: \quad (a+b)^l = \sum_{k=0}^l \binom{l}{k} a^{l-k} b^k$
    \item[(IB)] $\displaystyle n=l: \quad (a+b)^{l+1} = \sum_{k=0}^{l+1} \binom{l+1}{k} a^{l+1-k} b^k$\\
    \begin{proof}$~$\\[-1.65cm]
        \begin{align}
            \quad(a+b)^{l+1} &= (a+b)^l (a+b) \overset{\text{(IV)}}{=} \sum_{k=0}^l \binom{l}{k} a^{l-k} b^k (a+b) \notag \\
            &= \sum_{k=0}^l \binom{l}{k} a^{l+1-k} b^k + \sum_{k=0}^l \binom{l}{k} a^{l-k} b^{k+1} \notag \\
            &\qquad \text{Indextransformation } m=k+1 \Rightarrow k=m-1: \notag \\
            &\qquad \sum_{m=1}^{l+1} \binom{l}{m-1} a^{l+1-m} b^m \qq{,} \qq{Umbennennung} m \to k \notag \\
            &= \binom{l}{0} a^{l+1} + \sum_{k=1}^l \qty[\binom{l}{k} + \binom{l}{k-1}] a^{l+1-k} b^k + \binom{l}{l} b^{l+1} \notag \\
            &= a^{l+1} + \sum_{k=1}^l \binom{l+1}{k} a^{l+1-k} b^k + b^{l+1} \notag \\
            & \qquad \text{für } k=0: \quad \binom{l+1}{0} a^{l+1} = \binom{l}{0} a^{l+1} = a^{l+1}. \notag \\ 
            & \qquad \hp{für } k=l+1: \quad \binom{l+1}{l+1} b^{l+1} = \binom{l}{l} b^{l+1} = b^{l+1}. \notag \\
            &= \sum_{k=0}^{l+1} \binom{l+1}{k} a^{l+1-k} b^k.  
        \end{align}
    \end{proof}
\end{enumerate}