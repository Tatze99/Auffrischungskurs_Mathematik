\thispagestyle{plain}
\section{Grundlagen der Differentialrechnung (Ableiten)}

\begin{wrapfigure}{r}{8cm}
    \centering
    \vspace{-5mm}
        \begin{tikzpicture}
            \begin{axis}[disabledatascaling, axis lines=middle, xtick={1}, xticklabels={$x_0$}, ytick={0.83}, yticklabels={$f(x_0)$}, xlabel={$x$}, ylabel={$y$}, height=6cm, width=9cm, ymax= 5, ymin=-2, samples=100]
                \addplot[no marks, FSUblau, thick, domain=-2:3]{0.5*x^2 + 0.33*x^3 - x +1};
                \addplot[no marks, PAForange, thick, domain=-1:3]{x+0.83-1};
                \legend{Funktion $f(x)$, Tangente}
            \end{axis}
        \end{tikzpicture}
    \vspace{-5mm}
\end{wrapfigure}

Wir möchten in diesem Kapitel die Frage stellen, wie man den Anstieg einer beliebigen Funktion $f(x)$ an einem Punkt $x=x_0$ bestimmen kann. Dabei meinen wir den Anstieg der Geraden, die am Punkt $x_0$ als Tangente angelegt wird. Da das in jedem beliebigen Punkt $x=x_0$ möglich ist, ist der Anstieg selbst wieder eine Funktion von $x$ die wir Ableitung $f'(x)$ nennen wollen.

\paragraph{Konstruktion der Ableitung}$~$

\begin{figure}[htp]
    \centering
    \begin{tikzpicture}
        \begin{axis}[disabledatascaling, axis lines=middle, xtick={0.576, 1.951}, xticklabels={$x_0$,$x_0 + \varepsilon$}, ytick={0.652,3.402}, yticklabels={$f(x_0)$, $f(x_0 +\varepsilon)$}, xlabel={$x$}, ylabel={$y$}, height=6cm, width=0.8\textwidth, ymax= 5, ymin=-2, legend pos=outer north east, samples=100]
            \addplot[no marks, FSUblau, thick, domain=-2:3]{0.5*x^2 + 0.33*x^3 - x +1};
            \addplot[no marks, PAForange, thick, domain=-1:3]{2*x-0.5};
            \legend{Funktion $f(x)$, Sekante}
            \draw[dashed, gray] (0.576,0) -- (0.576,0.652) -- (0,0.652);
            \draw[dashed, gray] (1.951,0) -- (1.951,3.402) -- (0,3.402);
            \draw [decorate, decoration = {calligraphic brace}, thick] (0.2,3.402)--node[right]{\scriptsize$f(x_0+\varepsilon)-f(x_0)$}(0.2,0.652);
            \draw [decorate, decoration = {calligraphic brace}, thick] (1.951,-1)--node[below]{$\varepsilon$}(0.576,-1);
        \end{axis}
    \end{tikzpicture}
    \caption{Für die Konstruktion der Ableitung am Punkt $x_0$ legen wir zunächst eine Sekante des Graphen durch den Punkt $(x_0, f(x_0))$ und einen Punkt $(x_0 + \varepsilon, f(x_0+\varepsilon))$ und bilden anschließend den Grenzwert $\varepsilon \to 0$.} 
\end{figure}

Wir definieren die Ableitung als Grenzwert des Differenzenquotienten: 
\begin{align}
    f'(x) := \lim_{\varepsilon \to 0} \frac{f(x_0 + \varepsilon)-f(x_0)}{\varepsilon}.
\end{align}
Wir können die erhaltene Ableitungsfunktion erneut ableiten und erhalten damit die zweite Ableitung. Für Ableitungen $n$-ter Ordnung schreiben wir schließlich 
\begin{align}
    &\text{1. Ableitung:} \qquad f'(x) = \dv{f(x)}{x} \equiv \qty(\dv{f}{x})(x) \notag \\
    &\text{2. Ableitung:} \qquad f''(x) = \dv[2]{f(x)}{x} \notag \\
    &\text{n. Ableitung:} \qquad f^{(n)}(x) = \dv[n]{f(x)}{x}.
\end{align}
Ableitungen spezieller Funktionen sind zum Beispiel 
\begin{align}
    \begin{split}
        f(x) &= a =\text{const.} \quad \Rightarrow \quad f'(x) = \dv{x}(a) = 0, \\
        f(x) &= x \hphantom{= const.} \quad \Rightarrow \quad f'(x) = \dv{x}(x) = \lim_{\varepsilon \to 0} \frac{(x+\varepsilon) - x }{\varepsilon} = 1.
    \end{split}
\end{align}

\subsection{Allgemeine Eigenschaften}

Der Operator $\dv{x}$ weißt bestimmte Eigenschaften auf, die wir als Ableitungsregeln bezeichnen: 
\begin{itemize}
    \item Linearität: ($a,b \in \mathbb{R}$)
    \begin{align}
        \dv{x}(a f(x) + b g(x)) = a \dv{f}{x} + b \dv{g}{x}
    \end{align}
    \item Produktregel (Leibniz-Regel):
    \begin{align}
        \dv{x}(f(x)\cdot g(x)) = \dv{f}{x} \cdot g(x) + f(x) \cdot \dv{g}{x}.
    \end{align}
    \item Kettenregel: \vspace{-0.5cm}
    \begin{align}
        \dv{x} f(g(x)) = \underbrace{\qty(\dv{f}{g})(g(x))}_{\mathclap{\text{äußere Ableitung}}}\cdot \overbrace{\dv{g}{x}}^{\mathclap{\text{innere Ableitung}}}.
    \end{align}
    \item Quotientenregel: 
    \begin{align}
        \dv{x}\qty(\frac{f(x)}{g(x)}) = \frac{1}{g(x)^2} \qty(\dv{f}{x}\cdot g(x) - f(x) \cdot \dv{g}{x}).
    \end{align}
    \item Potenzregel: 
    \begin{align}
        \dv{x}\qty(x^n) = n \, x^{n-1} \quad \Rightarrow \quad \text{insbesondere: } \dv{x}\qty(\frac{1}{x^n}) = - \frac{n}{x^{n+1}.}
    \end{align}
\end{itemize}
Wenden wir einige dieser Regeln mal an einem praktischen Beispiel an: 
\begin{align}
    f(x) &=x \cdot g(x) + 7 h(y(x)) + \frac{x^2}{j(x)} \qq{mit} y(x) = ax + b \notag \\
    \dv{f}{x} &= \underbrace{\dv{x}{x} \cdot g(x) + x \cdot \dv{g}{x}}_{\text{Produktregel}} + \underbrace{7 \qty(\dv{h}{y})(y) \cdot \dv{y}{x}}_{\text{Kettenregel}} + \underbrace{\frac{2 x j(x) - x^2 \dv{j}{x}}{j(x)^2}}_{\text{Quotientenregel}}\notag \\
    &= g(x) +  x \cdot \dv{g}{x} + 7a \qty(\dv{h}{y})(y)  + \frac{2x j(x) - x^2 \dv{j}{x}}{j(x)^2}.
\end{align}
Wichtig ist es, bei der Kettenregel nach der richtigen Variable abzuleiten, d.\,h. wir müssen $h(y)$ nach dem Argument $y = ax+ b$ ableiten und nicht nach $x$. 


\subsection{Ableitungen spezieller Funktionen}

Wir wollen im Folgenden eine Liste von häufig verwendeten Funktionen und deren Ableitungen angeben. 

\begin{table}[htp]
    \centering
    \caption{Ableitungen spezieller Funktionen}
    \begin{tabular}[t]{l l}
        \toprule
        $f(x)$ & $f'(x)$ \\
        \midrule 
        $x^n$ & $n x^n$ \\
        $\sqrt{x}$ & $\dfrac{1}{2\sqrt{x}}$ \\
        $\ln(x)$ & $\dfrac{1}{x}$ \\
        $\sin(x)$ & $\cos(x)$ \\
        $\cos(x)$ & $-\sin(x)$ \\ 
        $\tan(x)$ & $1 + \tan^2(x)$ \\
        $\exp(x)$ & $\exp(x)$ \\
        $a^x$ & $\ln(a) a^x$ 
    \end{tabular}
    \hspace{1cm}
    \begin{tabular}[t]{l l}
        \toprule
        $f(x)$ & $f'(x)$ \\
        \midrule 
        $\arcsin(x)$ & $\dfrac{1}{\sqrt{1-x^2}}$ \\
        $\arccos(x)$ & $\dfrac{-1}{\sqrt{1-x^2}}$ \\
        $\arctan(x)$ & $\dfrac{1}{1+x^2}$ \\
        $\sinh(x)$ & $\cosh(x)$ \\
        $\cosh(x)$ & $\sinh(x)$ \\
    \end{tabular}
\end{table}

Schauen wir uns mal ein Beispiel an: 
\begin{align}
    f(x) &= \sin(\sqrt[n]{x^3}) \cdot \ln(\cos(x)-3\e^{x\cdot \ln(x)}) \notag \\
    \dv{f}{x} &= \dv{x}\qty[\sin(\sqrt[n]{x^3})]\cdot \ln(\cos(x)- 3\e^{x\ln(x)}) + \sin(\sqrt[n]{x^3}) \dv{x} \ln\underbrace{\big(\cos(x)-3\e^{x\cdot \ln(x)}\big)}_{\equiv A(x)} \notag \\
    &= \ln(A(x)) \cdot \cos(\sqrt[n]{x^3}) \cdot \dv{x}\underbrace{(\sqrt[n]{x^3})}_{x^{3/n}} + \sin(\sqrt[n]{x^3}) + \sin(\sqrt[n]{x^3}) \frac{1}{A(x)} \dv{A}{x} \notag \\
    &= \ln(A(x)) \cdot \cos(\sqrt[n]{x^3}) \cdot \frac{3}{n} \cdot \underbrace{x^{\frac{3}{n}-1}}_{\mathclap{x^{\frac{3-n}{n}} =\sqrt[n]{x^{3-n} = \sqrt[n]{x^3}\cdot x^{-1}}}} + \sin(\sqrt[n]{x^3}) \frac{1}{A(x)}\qty[-\sin(x)-3 \dv{x}\e^{x \ln(x)}] \notag\\
    &= \frac{3 \ln(A(x)) \sqrt[n]{x^3}}{nx}\cos(\sqrt[n]{x^3}) - \frac{\sin(\sqrt[n]{x^3})}{A(x)} \qty[\sin(x)+3 \e^{x \ln(x)}(\ln(x)+1)]
\end{align}

\newpage
\subsection{Kurvendiskussion} 
Verschwindet die Ableitung an einem Punkt, $f'(x) = 0$, dann bedeutet dies, dass dort ein lokales Extremum (bzw. ein Sattelpunkt) vorliegt. Bei einem Maximum ist die zweite Ableitung $f''(x_0) < 0$, während sie bei einem lokalen Minimum $f''(x_0) >0$ ist. Für den Fall $f''(x_0) = 0$ liegt ein Sattelpunkt vor, falls $f^{(3)}(x_0) \neq 0$ ist. Ansonsten muss durch das Bilden von höheren Ableitungen weiter entschieden werden\footnote{Ein Beispiel dafür ist $f(x) = x^4$. Im Nullpunkt liegt ein Minimum vor, jedoch verschwinden dort die ersten drei Ableitungen.}. 

\begin{figure}[htp]
    \centering
    \begin{tikzpicture}
        \begin{axis}[disabledatascaling, axis lines=middle, xtick={2.5}, xticklabels={$x_0$}, ytick, xlabel={$x$}, ylabel={$y$}, height=6cm, width=0.4\textwidth, xmin = 0, xmax=5, ymin=0, ymax=5]
            \addplot[no marks, FSUblau, thick, domain=0:5]{-0.5*(x-2.5)^2+4};
            \draw[thick] (1,4) -- (4,4);
            \draw[thick, dashed] (2.5,0) -- (2.5,4);
        \end{axis}
        \node (A) at (2.3,-1){$f''(x_0) <0$ Maximum};
    \end{tikzpicture}
    \hfill
    \begin{tikzpicture}
        \begin{axis}[disabledatascaling, axis lines=middle, xtick={2.5}, xticklabels={$x_0$}, ytick, xlabel={$x$}, ylabel={$y$}, height=6cm, width=0.4\textwidth, xmin = 0, xmax=5, ymin=0, ymax=5]
            \addplot[no marks, FSUblau, thick, domain=0:5]{0.5*(x-2.5)^2+1};
            \draw[thick] (1,1) -- (4,1);
            \draw[thick, dashed] (2.5,0) -- (2.5,1);
            
        \end{axis}
        \node (A) at (2.3,-1){$f''(x_0) < 0$ Minimum};
    \end{tikzpicture}
    \hfill
    \begin{tikzpicture}
        \begin{axis}[disabledatascaling, axis lines=middle, xtick={2.5}, xticklabels={$x_0$}, ytick, xlabel={$x$}, ylabel={$y$}, height=6cm, width=0.4\textwidth, xmin = 0, xmax=5, ymin=0, ymax=5]
            \addplot[no marks, FSUblau, thick, domain=0:5]{0.2*(x-2.5)^3+2.5};
            \draw[thick] (1,2.5) -- (4,2.5);
            \draw[thick, dashed] (2.5,0) -- (2.5,2.5);
            
        \end{axis}
        \node (A) at (2.3,-1){$f''(x_0) = 0$ Sattelpunkt};
    \end{tikzpicture}
    \caption{lokale Extremstellen einer Funktion abhängig vom Vorzeichen der zweiten Ableitung $f''(x_0)$. Für den Sattelpunkt muss außerdem noch gelten: $f^{(3)}(x_0) \neq 0$.}
\end{figure}

Das Vorzeichen der zweiten Ableitung gibt Auskunft über die Krümmung der Kurve:
\begin{figure}[htp]
    \centering
    \begin{tikzpicture}
        \begin{axis}[disabledatascaling, axis lines=middle, xtick={2,2.5,3}, xticklabels={$x_1$,$x_2$,$x_3$}, ytick, xlabel={$x$}, ylabel={$y$}, height=6cm, width=0.55\textwidth, xmin=1,xmax=4.5,ymin=0,ymax=10]
            \addplot[no marks, FSUblau, very thick, domain=0:3.7]{(x-1)^2+1} node[above]{$f(x)$};
            \fill (2,2) circle (2pt);
            \fill (3,5) circle (2pt);
            \fill (2.5,3.25) circle (2pt);
            \addplot[no marks, PAForange, domain=0:3.7]{2*x-2}node[right]{$f'(x_1)$};
            \addplot[no marks, PAForange, domain=0:3.7]{3*x-4.25}node[right]{$f'(x_2)$};
            \addplot[no marks, PAForange, domain=0:3.7]{4*x-7}node[right]{$f'(x_3)$};

            \draw[gray, dashed] (2,0) -- (2,2);
            \draw[gray, dashed] (2.5,0) -- (2.5,3.25);
            \draw[gray, dashed] (3,0) -- (3,5);
            \node (A) at (2,7){konvex};
        \end{axis}
    \end{tikzpicture}
    \hfill 
    \begin{tikzpicture}
        \begin{axis}[disabledatascaling, axis lines=middle, xtick={2,2.5,3}, xticklabels={$x_1$,$x_2$,$x_3$}, ytick, xlabel={$x$}, ylabel={$y$}, height=6cm, width=0.55\textwidth, xmin=1,xmax=4.5,ymin=0,ymax=10]
            \addplot[no marks, FSUblau, very thick, domain=0:3.7]{-(x-1)^2+8} node[left]{$f(x)$};
            \fill (2,8-1) circle (2pt);
            \fill (3,8-4) circle (2pt);
            \fill (2.5,8-2.25) circle (2pt);
            \addplot[no marks, PAForange, domain=0:3.7]{-2*x+2+9}node[right]{$f'(x_1)$};
            \addplot[no marks, PAForange, domain=0:3.7]{-3*x+4.25+9}node[right]{$f'(x_2)$};
            \addplot[no marks, PAForange, domain=0:3.7]{-4*x+7+9}node[right]{$f'(x_3)$};

            \draw[gray, dashed] (2,0) -- (2,8-1);
            \draw[gray, dashed] (2.5,0) -- (2.5,8-2.25);
            \draw[gray, dashed] (3,0) -- (3,8-4);
            \node (A) at (3.5,7){konkav};
        \end{axis}
    \end{tikzpicture}
    \caption{Links: Beispiel für eine konvexe Kurve. Der Anstieg $f'(x)$ nimmt zu, es gilt also $f''(x) > 0$ \\
    Rechts: Beispiel für eine konkave Kurve. Der Anstieg $f'(x)$ nimmt ab, es gilt also $f''(x) < 0$}
\end{figure}