\Titelbanner{9}{Arithmetische \& geometrische Reihen\\
                Der binomische Satz}

\textbf{Aufgabe 1: } \emph{Arithmetische Reihe}
\begin{align*}
&L_n=\pi r+\pi e(n-1),\ \ \ \ n=1,2,3,\dots \\
&\Rightarrow\hspace{0.3cm}S_n=n\pi r+\pi e\sum\limits_{k=1}^n(k-1)=\pi n\left(r+e\frac{n-1}{2}\right)
\end{align*}\\[0.5cm]
%
\textbf{Aufgabe 2: } \emph{Geometrische Folge}
\begin{align*}
& L_n=\left(\frac{11}{12}\right)^nL_0\stackrel{!}{=}\frac{1}{2}L_0 \hspace{0.8cm} \Rightarrow\hspace{0.8cm}n=\frac{\ln(1/2)}{\ln(11/12)}\approx 7,97
\end{align*}
Der Lichtstrahl muss 8 Platten durchdringen.\\[1.2cm]
%
\textbf{Aufgabe 3: } \emph{Geometrische Reihe}
\begin{align*}
I_\text{out}=\frac{T}{2-T}I_\text{in}\\[0.2cm]
\end{align*}
%
\textbf{Aufgabe 4: } \emph{Der binomische Satz}
\begin{enumerate}[label=(\alph*)]
\item $2a^6+30a^4+30a^2+2$
\item $2^n=\sum\limits_{k=0}^n\begin{pmatrix}n\\k\end{pmatrix}$\hspace{0.2cm}\text{ und }\hspace{0.2cm} $0=\sum\limits_{k=0}^n(-1)^k\begin{pmatrix}n\\k\end{pmatrix}$ für $n\ge 1$
\item $f(nx)+g(nx)=\sum\limits_{k=0}^n\binom{n}{k}f(x)^kg(x)^{n-k}$
\end{enumerate}
\vspace{1cm}
%
\textbf{Aufgabe 5: } \emph{Erzeugende Funktion}
\begin{enumerate}
\item $f(x)=\frac{x}{(1-x)(1-2x)}$
\item $a_n=2^n-1$\\[0.2cm]
\end{enumerate}
%
% \textbf{Aufgabe 6*: } \emph{Zustandssumme}
% \begin{align*}
% \Omega_N=\left(\frac{\e-\e^{-N}}{\e-1}\right)^k, && \Omega=\left(\frac{1}{1-1/\e}\right)^k
% \end{align*}
% %
% \textbf{Aufgabe 7*: } \emph{Fibonacci-Zahlen III}
% \begin{align*}
% &f(x)=\frac{x}{1-x-x^2},\ \ \ a_n=\frac{x_+^n-x_-^n}{\sqrt{5}}\,,\\[0.2cm]
% &\text{wobei } x_{\pm}=\frac{1\pm\sqrt{5}}{2}\,;\\
% &\text{vergleiche mit Aufgabe 2\,(f) von Thema 7 und Aufgabe 6 von Thema 8}
% \end{align*}