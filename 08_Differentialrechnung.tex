\section{Grundlagen der Differentialrechnung (Ableiten)}

\begin{wrapfigure}{r}{8cm}
    \centering
    \vspace{-5mm}
        \begin{tikzpicture}
            \begin{axis}[disabledatascaling, axis lines=middle, xtick={1}, xticklabels={$x_0$}, ytick={0.83}, yticklabels={$f(x_0)$}, xlabel={$x$}, ylabel={$y$}, height=6cm, width=9cm, ymax= 5, ymin=-2]
                \addplot[no marks, FSUblau, thick, domain=-2:3]{0.5*x^2 + 0.33*x^3 - x +1};
                \addplot[no marks, PAForange, thick, domain=-1:3]{x+0.83-1};
                \legend{Funktion $f(x)$, Tangente};
            \end{axis}
        \end{tikzpicture}
    \vspace{-5mm}
\end{wrapfigure}

Wir möchten in diesem Kapitel die Frage stellen, wie man den Anstieg einer beliebigen Funktion $f(x)$ an einem Punkt $x=x_0$ bestimmen kann. Dabei meinen wir den Anstieg der Geraden, die am Punkt $x_0$ als Tangente angelegt wird. 