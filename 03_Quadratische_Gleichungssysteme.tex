%% Potentielle Aufgabe: Goldener Schnitt

\thispagestyle{plain}
\section{Quadratische Gleichungssysteme}

In diesem Abschnitt geht es um quadratische Gleichungen, also Bestimmungsgleichungen, deren Variablen höchsten in zweiter Potenz auftreten, sowie um quadratische Funktionen und Systeme aus quadratischen Gleichungen.

\subsection{Die quadratische Gleichung}

Wir schauen uns zunächst die allgemeine Form einer quadratischen Gleichung an 
\begin{align}
    \tikzmarknode{eq1}{A} x^2 + \tikzmarknode{eq2}{B} x + \tikzmarknode{eq3}{C} =0.
\end{align}
\tikz[overlay,remember picture]{
\draw[shorten >=2pt,shorten <=2pt, thick, -{latex}] ($(eq1)+(-.4,-1)$)node[left]{quadratisches Glied} -- ($(eq1)+(0,-.2)$);
\draw[shorten >=2pt,shorten <=2pt, thick, -{latex}] ($(eq2)+(0,-1)$)node{lineares Glied} -- ($(eq2)+(0,-.2)$);
\draw[shorten >=2pt,shorten <=2pt, thick, -{latex}] ($(eq3)+(.4,-1)$)node[right]{Absolutglied} -- ($(eq3)+(0,-.2)$);
}

Wir nehmen im Folgenden o.\,B.\,d.\,A. an, es sei $A >0$. Zudem sind $A,B$ und $C$ reelle Konstanten. Wir betrachten zunächst zwei Spezialfälle
\begin{align}
    B &= 0 
    \begin{split}
        A x^2 + C &= 0 \\
        \Rightarrow \qq{für} C &\le 0: \qq{Lösungen} x_1 = \sqrt{-\frac{C}{A}}, x_2 = -\sqrt{-\frac{C}{A}} \\
        \Rightarrow \qq{für} C &>0: \qq{keine (reellen) Lösungen}
    \end{split}\\
    C &= 0 
    \begin{split}
        A x^2 + Bx &= 0 \\
         x(Ax + B) &= 0 \\
        \Rightarrow \qq{Lösungen} &x_1 = 0, x_2 = -\frac{B}{A}.\hspace{3.85cm}
    \end{split}
\end{align}

Zur Bestimmung einer allgemeinen Lösung verwenden wir die \emph{Methode der quadratischen Ergänzung}:
\begin{enumerate}
    \item Überführung in die \emph{Normalform} mit Umbenennung $p \equiv \frac{B}{A}, q \equiv \frac{C}{A}$
    \begin{align}
        x^2 + \frac{B}{A} x + \frac{C}{A} = 0\qq{,} x^2 + px + q = 0;
    \end{align}
    \item quadratische Ergänzung: \vspace{-1.1cm}
    \begin{align}
        x^2 + px &= -q \hphantom{+\qty(\frac{p}{2})^2} \bigg| +\qty(\frac{p}{2})^2 \\
        \Rightarrow x^2 + px +\qty(\frac{p}{2})^2 &= -q +\qty(\frac{p}{2})^2 \notag \\
        \qty(x+\frac{p}{2})^2 &= - q +\qty(\frac{p}{2})^2.
    \end{align}
    \item Auflösen nach $x$ 
    \begin{align}
        x + \frac{p}{2} = \pm \sqrt{\qty(\frac{p}{2})^2 - q}.
    \end{align}
\end{enumerate}
\begin{mymathbox}[ams align, title={$p$-$q$-Lösungsformel}, colframe={FSUblau}]
    x^2 + px + q = 0 \quad \Rightarrow \quad x_{1/2} = -\frac{p}{2} \pm \sqrt{\qty(\frac{p}{2})^2 - q}.
\end{mymathbox}
Man bezeichnet $D \equiv \qty(\frac{p}{2})^2 - q$ als \emph{Diskriminante}. Anhand ihres Vorzeichens können folgende Fälle auftreten:
\begin{itemize}
    \item $D > 0$, es existieren zwei reelle Lösungen 
    \item $D = 0$, beide Lösungen fallen zusammen $x_1 = x_2 = -\frac{p}{2}$ 
    \item $D < 0$, es existiert keine (reelle) Lösung.
\end{itemize}
Betrachten wir für den Fall $D \ge 0$ die Summe und das Produkt der beiden allgemeinen Lösungen,
\begin{align}
        x_1 + x_2 &= -\frac{p}{2} + \cancel{\sqrt{\qty(\frac{p}{2})^2 - q}} - \frac{p}{2} - \cancel{\sqrt{\qty(\frac{p}{2})^2 - q}} = -p \\ 
        x_1 \cdot x_2 &= \qty(-\frac{p}{2}+ \sqrt{\qty(\frac{p}{2})^2 - q})\qty(-\frac{p}{2}- \sqrt{\qty(\frac{p}{2})^2 - q}) \notag \\
        \overset{\eqref{eqn:1_binomische_Formeln}}&{=} \frac{p^2}{4} - \qty(\sqrt{\qty(\frac{p}{2})^2 - q})^2 = \cancel{\frac{p^2}{4}} - \cancel{\frac{p^2}{4}} + q = q, 
\end{align}
dann folgt damit der \emph{Vieta'sche Wurzelsatz}\footnote{Wurzeln sind eine Bezeichnung für die Nullstellen eines Polynoms.} für die Lösungen quadratischer Gleichungen, 
\begin{align}
    x_1 + x_2 = -p \qq{,} x_1 x_2 = q, \qq{Vieta'scher Wurzelsatz.}
\end{align}
Setzen wir dieses Resultat in die Normalform ein, so ergibt sich 
\begin{align}
        x^2 + px + q &= x^2 - (x_1 + x_2) x + x_1 x_2 \notag \\
        &= (x-x_1) (x-x_2), 
\end{align}
also die Zerlegung in Linearfaktoren.

\clearpage
\subsection{Quadratische Funktionen}

Für quadratische Funktionen lautet die allgemeine Form 
\begin{align}
    f(x) = Ax^2 + Bx + C \qq{,} A,B,C \in \mathbb{R}.
\end{align}
Wir nehmen wieder o.\,B.\,d.\,A. $A > 0$ an. Offenbar enspricht das Lösen einer quadratischen Gleichung der Suche nach den Nullstellen einer quadratischen Funktion (und umgekehrt). Betrachten wir zunächst den Fall $B = C = 0$: 

\begin{minipage}{0.45\textwidth}
    \begin{itemize}
        \item Definitionsbereich $x \in \mathbb{R}$ \\
        Wertebereich $y \ge 0$;
        \item $A = 1$: \emph{Normalparabel} 
        \item $A > 1$: gestauchte Normalparabel ;
        \item $A < 1$: gestreckte Normalparabel;
        \item doppelte Nullstele bei $x_{1/2} = 0.$
    \end{itemize}
\end{minipage}
\begin{minipage}{0.55\textwidth}
        \centering
        \begin{tikzpicture}
            \begin{axis}[disabledatascaling, axis lines=middle, xlabel={$x$}, ylabel={$y$},xtick={-3,-2,-1,1,2,3}, ytick={1,2,3,4,5,6,7,8,9}, height=6cm, width=\textwidth, xmin=-3.5, xmax=4, ymin=-1.5, ymax = 9.9]
                \addplot[no marks, FSUblau, thick, domain=-3.5:3]{x^2}node[right]{$A=1$};
                \addplot[no marks, FSUblau, dashed, thick, domain=-3.5:0]{1.4*x^2};
                \addplot[no marks, FSUblau, dashed, thick, domain=-3.5:0]{0.7*x^2};
                \draw[dashed] (1,0) -- (1,1) -- (0,1);
                \draw[dashed] (2,0) -- (2,4) -- (0,4);
                \draw[dashed] (3,0) -- (3,9) -- (0,9);
                \node(A) at (-1.5,8){$A>1$};
                \node(A) at (-2.5,2){$A<1$};
            \end{axis}
        \end{tikzpicture}
\end{minipage}

Durch Addition der Konstante $y_0$ verschiebt sich die Parabel entlang der $y$-Achse. Für $y_0 > 0$ bzw. $y_0 <0$ existieren keine bzw. zwei Nullstellen.

Durch die Substitution $x \mapsto x+x_0$ verschiebt sich die $y$-Achse um den Wert $x_0$, bzw. die Parabel um $\minus x_0$ entlang der $x$-Achse. Wir erhalten damit die quadratische Funktion
\begin{align}
    f(x) = A(x+x_0)^2 + y_0 \qq{Scheitelpunktform,}
\end{align}
wobei\quad$x_0 > 0$: Verschiebung der $y$-Achse (Parabel) nach rechts (links) \\
\hphantom{wobei}\quad$x_0 < 0$: Verschiebung der $y$-Achse (Parabel) nach links (rechts).

\begin{wrapfigure}{r}{6cm}
    \centering
    \vspace{-5mm}
        \begin{tikzpicture}
            \begin{axis}[disabledatascaling, axis lines=middle, xtick={1}, xticklabels={$-x_0$}, ytick={-2}, yticklabels={$y_0$}, xlabel={$x$}, ylabel={$y$}, height=6cm, width=6cm, ymin = -2.3]
                \addplot[no marks, FSUblau, thick, domain=-1:3] {(x-1)^2-2};
                \draw[thick, dashed] (0,-2) -- (1,-2) -- (1,0);
            \end{axis}
        \end{tikzpicture}
    \vspace{-5mm}
\end{wrapfigure}

Vergleichen wir mittels quadratischer Ergänzung die Scheitelpunktform mit der Normalform, so ergibt sich 
\begin{align}
    x_0 = \frac{B}{2A} \qq{,} y_0 = C -A x_0^2.
\end{align}
Das heißt, dass jede quadratische Funktion durch geeignete Verschiebung des Koordinatensystems auf eine Parabel der Form $f(x) = A x^2$ zurückgeführt werden kann.

\clearpage
\subsection{Quadratische Gleichungssystem mit zwei Unbekannten}

Die allgemeine Form eines quadratischen Gleichungssystems mit zwei Unbekannten Größen $x,y$ lautet 
\begin{align}
    a_i x^2 + b_i y^2 + c_i xy + d_i x + e_i y + f_i = 0.
\end{align}
Die allgemeine ist jedoch nur umständlich zu diskutieren und es existieren viele Lösungsmöglichkeiten. Daher beschränken wir uns hier auf zwei Beispiele. 

\paragraph{Beispiel 1}
\begin{subequations}
    \begin{align}
        x^2 +y^2 - 2x &= \frac{11}{2}, \label{eqn:3_QGS_Bsp1a}\\
        2xy - 2y &= \frac{5}{2}\label{eqn:3_QGS_Bsp1b}
    \end{align}
\end{subequations}
Wir addieren nun $\eqref{eqn:3_QGS_Bsp1a} \pm \eqref{eqn:3_QGS_Bsp1b}$ und erhalten 
\begin{align}
    \underbrace{x^2 + y^2 \pm 2xy}_{(x\pm y)^2} - 4(x\pm y) = 11 \pm 5.
\end{align}
Wir substituieren nun $u \equiv x+y$ und $v \equiv x-y$, damit haben wir zwei voneinander unabhängige quadratische Gleichungen 
\begin{align}
    \begin{split}
        u^2 - 2u -8 &= 0\\
        v^2 - 2v -3 &= 0
    \end{split} \qq{mit den Lösungen} \begin{split}
        u_{1/2} &= 1 \pm \sqrt{1+8} \quad \Rightarrow \quad u_1 = 4, u_2 = -2 \\
        v_{1/2} &= 1 \pm \sqrt{1+3} \quad \Rightarrow \quad v_1 = 3, v_2 = -1.
    \end{split}
\end{align}

Resubstituieren wir jetzt nun $x = \frac{1}{2}(u+v), y = \frac{1}{2}(u-v)$, dann existieren vier Lösungspaare für $(x,y)$, da vier Paare $(u_i, v_j)$ beildet werden können 
\begin{align}
    \begin{split}
        x_1 &= \frac{1}{2}(u_1+v_1) = \hphantom{\minus}\frac{7}{2} \qq{,} y_1 = \frac{1}{2}(u_1-v_1) = \frac{1}{2};\\
        x_2 &= \frac{1}{2}(u_1+v_2) = \hphantom{\minus}\frac{3}{2} \qq{,} y_2 = \frac{1}{2}(u_1-v_2) = \frac{5}{2};\\
        x_3 &= \frac{1}{2}(u_2+v_1) = \hphantom{\minus}\frac{1}{2} \qq{,} y_3 = \frac{1}{2}(u_2-v_1) = \minus\frac{5}{2};\\
        x_4 &= \frac{1}{2}(u_2+v_2) = \minus\frac{3}{2}             \qq{,} y_4 = \frac{1}{2}(u_2-v_2) = \minus\frac{1}{2};
    \end{split}
\end{align}
\begin{align}
    \Rightarrow \mathbb{L} = \bigg\{\qty(\frac{7}{2};\frac{1}{2}),\qty(\frac{3}{2};\frac{5}{2}),\qty(\frac{1}{2};\minus\frac{5}{2}),\qty(\minus\frac{3}{2};\minus\frac{1}{2})\bigg\}.
\end{align}
Im Zweifelsfall (``Sind wirklich alle Kombinationen erlaubt?'') sollte die Probe durchgeführt werden. 

\emph{Bemerkung:} Eine geometrische Interpretation ist auch hier möglich: Gleichung~\eqref{eqn:3_QGS_Bsp1a} beschreibt einen Kreis des Radius $\sqrt{\frac{13}{2}}$, dessen Mittelpunkt um 1 nach rechts verschoben ist, wohingegen \eqref{eqn:3_QGS_Bsp1b} als Funktion $y = \frac{5}{4}\frac{1}{x-1}$ geschrieben werden kann. Die Schnittpunktkoordinaten der Hyperbeln mit dem Kreis sind die Lösungspaare. 
\begin{figure}[htp]
    \centering
    \begin{tikzpicture}
        \begin{axis}[axis equal, disabledatascaling, axis lines=middle, xlabel={$x$}, ylabel={$y$}, width=0.7\textwidth, ymin=-3.5, ymax=3.5]
            \addplot[no marks, FSUblau, samples=100, thick, domain=1.1:4]{5/4*1/(x-1)};
            \addplot[no marks, FSUblau, samples=100, thick, domain=-2:0.9]{5/4*1/(x-1)};
            \draw[thick] (1,0) circle (2.55);
            \draw[dashed] (1,4) -- (1,-4);
            \fill (7/2,1/2) circle (2pt) node[above right]{$\qty(\frac{7}{2};\frac{1}{2})$};
            \fill (3/2,5/2) circle (2pt) node[above right]{$\qty(\frac{3}{2};\frac{5}{2})$};
            \fill (-3/2,-1/2) circle (2pt) node[below left]{$\qty(\minus\frac{3}{2};\minus\frac{1}{2})$};
            \fill (1/2,-5/2) circle (2pt) node[above right]{$\qty(\frac{1}{2};\minus\frac{5}{2})$};
        \end{axis}
    \end{tikzpicture}
    \caption{Grafische Lösung des quadratischen Gleichungssystems.}
\end{figure}

\paragraph{Beispiel 2}$~$
\begin{align}
    \begin{split}     
        x + y^2 &= 2, \\
        x y^2 &= -8 
    \end{split} 
    \qquad\Rightarrow\qq{Substitution} y^2 \equiv z 
    \begin{split}
        x+ z &= 2, \\
        xz &= -8
    \end{split}\label{eqn:3_QGS_Bsp2}
\end{align}
Der Satz von Vieta angewandt auf das Gleichungssystem~\eqref{eqn:3_QGS_Bsp2}, dass $x$ und $z$ Lösungen einer qudratischen Gleichung mit $p = -2$ und $q = -8$ sind, also der Gleichung 
\begin{align}
    u^2 + pu +q = u^2 -2u - 8 = 0 \quad \Rightarrow \quad u_{1/2} = 1 \pm \sqrt{9} = \begin{cases}
        4 \\ -2.
    \end{cases}
\end{align}
Wir könnten nun $u_1 = x, u_2 =z$ oder $u_1 = z, u_2 = x$ identifizieren, allerdings muss $z = y^2$ positiv sein. wir erhalten somit die Lösungen 
\begin{align}
    x= -2, y = \pm 2 \quad \Rightarrow \quad \mathbb{L} = \{(-2;2),(-2,-2)\}.
\end{align}