\Titelbanner{8}{Die Methode der vollständigen Induktion}

\textbf{Aufgabe 1: } \emph{Rekursive und explizite Zuordnungsvorschrift}
\begin{align}
    & a_1 = 1, a_2 = 3, a_3 = 7, a_4 = 15 \quad \Rightarrow \qq{Vermutung: } a_n=2^n-1 \hspace{0.5cm} \Rightarrow \hspace{0.5cm} a_{n+1}=2^{n+1}-1 \\
    &a_{k+1}=2a_k+1=2\qty(2^k-1)=2^{k+1}-1.
\end{align}\hfill $\Box$\\[0.7cm] 
%
\textbf{Aufgabe 2: } \emph{Vollständige Induktion I} $\quad S_n = 1^2 -2^2 + 3^2 -4^2 + \hdots + (-1)^{n-1}n^2 = (-1)^{n-1}\frac{n(n+1)}{2}$
\begin{enumerate}
    \item[(IA)] $n=1: \quad S_1 = 1. \quad\checkmark$ 
    \item[(IV)] $n=k: \quad S_k=(-1)^{k-1}\frac{k(k+1)}{2}$
    \item[(IB)] $n=k+1: \quad S_{k+1}=(-1)^{k}\frac{(k+1)(k+2)}{2}$
    \begin{proof}$~$\\[-1.5cm]
        \begin{align}
            \qquad S_{k+1} &= S_k + (-1)^k (k+1)^2 =(-1)^{k-1}\frac{k(k+1)}{2}+(-1)^k(k+1)^2\\
            &=(-1)^k(k+1)\qty[\frac{(-1)^{-1}k}{2}+(k+1)]=\frac{(-1)^{k}(k+1)(k+2)}{2}.
        \end{align}
    \end{proof}
\end{enumerate}
%
\textbf{Aufgabe 3: } \emph{Vollständige Induktion II} $\quad S_n=1^3+2^3+3^3+\dots+n^3 = \qty[\frac{n(n+1)}{2}]^2$

\begin{enumerate}
    \item[(IA)] $n=1: \quad S_1 = 1. \quad\checkmark$ 
    \item[(IV)] $n=k: \quad S_k=\qty[\frac{k(k+1)}{2}]^2$
    \item[(IB)] $n=k+1: \quad S_{k+1}=\qty[\frac{(k+1)(k+2)}{2}]^2$\\
    \begin{proof}$~$\\[-1.7cm]
        \begin{align}
            \qquad S_{k+1} &= S_k+(k+1)^3=\qty(\frac{k+1}{2})^2\qty[k^2+4(k+1)]=\qty[\frac{(k+1)(k+2)}{2}]^2.
        \end{align}
    \end{proof}
\end{enumerate}
%
\newpage
\ \\
\textbf{Aufgabe 4: } \emph{Vollständige Induktion III} $\quad S_n = (n-1)^3 + n^3 + (n+1)^3, \quad S_n = 9m \qq{mit} m \in \mathbb{N}$

\begin{enumerate}
    \item[(IA)] $n=1: \quad S_1 = 0 + 1^3 + 2^3 = 9. \quad\checkmark$ 
    \item[(IV)] $n=k: \quad S_k = (k-1)^3 + k^3 + (k+1)^3, \quad S_k = 9m \qq{mit} m \in \mathbb{N}$
    \item[(IB)] $n=k+1: \quad S_{k+1} = k^3 + (k+1)^3 + (k+2)^3, \quad S_k = 9m' \qq{mit} m' \in \mathbb{N}$\\
    \begin{proof}$~$\\[-1.4cm]
        \begin{align}
            \qquad S_{k+1}&=k^3+(k+1)^3+(k+2)^3=\underbrace{(k-1)^3+k^3+(k+1)^3}_{S_n}+(k+2)^3-(k-1)^3\\
            &=9m+(k+2)^3-(k-1)^3=9m+9(k^2+k+1)=9m' \text{ mit } m'=m+k^2+k+1\in\mathbb{N}
        \end{align}
    \end{proof}
\end{enumerate}
%
\textbf{Aufgabe 5: } \emph{Die Suche nach der richtigen Summenformel}\\
\begin{align}
\text{Vermutung: }\hspace{0.2cm} S_n=\frac{(1-x)(2-x)\dots(n-x)}{n!} \hspace{0.5cm}
\end{align}

\begin{enumerate}
    \setlength{\mathindent}{0cm}
    \item[(IA)] $n=1: \quad S_1 = 1-x. \quad\checkmark$ 
    \item[(IV)] $n=k: \quad S_k = \frac{(1-x)(2-x)\dots(k-x)}{k!}$
    \item[(IB)] $n=k+1: \quad S_{k+1} = \frac{(1-x)(2-x)\dots((k-x)(k+1-x)}{(k+1)!}$\\
    \begin{proof}
        \begin{align}
            S_{k+1}&=S_k+(-1)^{k+1}\frac{x(x-1)(x-2)\dots(x-k+1)(x-k)}{(k+1)!}\\
            &=\frac{(1-x)(2-x)\dots(k-x)}{k!}+(-1)^{k+1}\frac{x(x-1)(x-2)\dots(x-k+1)(x-k)}{(k+1)!}\\
            &=\frac{1}{(k+1)!}\qty[(1-x)(2-x)\dots(k-x)(k+1)+(-1)^{k+1}(-1)^kx(x-1)(x-2)\dots(x-k+1)(x-k)]\\
            &=\frac{(1-x)(2-x)\dots(k-x)(k-x+1)}{(k+1)!}\underbrace{\qty[\frac{k+1}{k-x+1}+(-1)^{2k+1}\frac{x}{k-x+1}]}_1\\
            &=\frac{(1-x)(2-x)\dots(k-x)(k-x+1)}{(k+1)!}
        \end{align}
    \end{proof}
\end{enumerate}

\newpage
\textbf{Aufgabe 6: } \emph{Fibonacci-Zahlen} $\quad a_n=\frac{1}{\sqrt{5}}\qty(x_+^n-x_-^n) \qq{mit}x_\pm = \frac{1\pm \sqrt{5}}{2}$\hfill (Zusatzaufgabe)

\begin{enumerate}
    \setlength{\mathindent}{0cm}
    \item[(IA)] $n=0: \quad a_0 = 0, a_1 = 1. \quad\checkmark$ 
    \item[(IV)] $n=k: \quad a_k = \frac{1}{\sqrt{5}}\qty(x_+^k-x_-^k)$
    \item[(IB)] $n=k+1: \quad a_{k+1} = \frac{1}{\sqrt{5}}\qty(x_+^{k+1}-x_-^{k+1})$
    \begin{proof}
        \begin{align}
            a_{n+1}&=a_n+a_{n-1}=\frac{1}{\sqrt{5}}\qty(x_+^n-x_-^n+x_+^{n-1}-x_-^{n-1})\\
            &=\frac{1}{\sqrt{5}}\qty[\qty(\frac{1+\sqrt{5}}{2})^n+\qty(\frac{1+\sqrt{5}}{2})^{n-1}-\qty(\frac{1-\sqrt{5}}{2})^n-\qty(\frac{1-\sqrt{5}}{2})^{n-1}]\\
            &=\frac{1}{\sqrt{5}}\Bigg[\underbrace{\qty(\frac{1+\sqrt{5}}{2}+1)}_{\qty(\frac{1+\sqrt{5}}{2})^2}\qty(\frac{1+\sqrt{5}}{2})^{n-1}-\underbrace{\qty(\frac{1-\sqrt{5}}{2}+1)}_{\qty(\frac{1-\sqrt{5}}{2})^2}\qty(\frac{1-\sqrt{5}}{2})^{n-1}\Bigg]
            =\frac{1}{\sqrt{5}}\qty(x_+^{n+1}-x_-^{n+1})
        \end{align}
    \end{proof}
\end{enumerate}
%
%
%\begin{minipage}{0.6\linewidth}
%\begin{align*}
%\text{Vermutung: }\hspace{0.2cm} S_n=(-1)^n \hspace{0.5cm} \Rightarrow \hspace{0.5cm} S_{n+1}=(-1)^{n+1}
%\end{align*}
%\end{minipage}\\
%\begin{align*}
%S_{n+1}&=\sum_{k=0}^{n+1}\sum_{l=0}^{n+1-k}(-1)^{k+l}\binom{n+1}{k}\binom{n+1-k}{l}\\
%&=\sum_{k=0}^{n}\sum_{l=0}^{n+1-k}(-1)^{k+l}\binom{n+1}{k}\binom{n+1-k}{l}+\underbrace{\sum_{l=0}^0(-1)^{n+1+l}\overbrace{\binom{n+1}{n+1}}^1\overbrace{\binom{0}{l}}^1}_{(-1)^{n+1}}\\
%&=\sum_{k=0}^{n}\sum_{l=0}^{n+1-k}(-1)^{k+l}\binom{n}{k}\binom{n-k}{l}+\sum_{k=0}^{n}\sum_{l=0}^{n+1-k}(-1)^{k+l}\binom{n}{k}\binom{n-k}{l-1}\\
%&\quad +\sum_{k=0}^{n}\sum_{l=0}^{n+1-k}(-1)^{k+l}\binom{n}{k-1}\binom{n-k}{l}+\sum_{k=0}^{n}\sum_{l=0}^{n+1-k}(-1)^{k+l}\binom{n}{k-1}\binom{n-k}{l-1}+(-1)^{n+1}\\
%\end{align*}
%\newpage
%\begin{align*}
%&=\underbrace{\sum_{k=0}^{n}\sum_{l=0}^{n-k}(-1)^{k+l}\binom{n}{k}\binom{n-k}{l}}_{(-1)^n}+\sum_{k=0}^{n}(-1)^{n+1}\binom{n}{k}\underbrace{\binom{n-k}{n+1-k}}_{0}\\
%&\quad -\sum_{k=0}^{n}\sum_{l=-1}^{n-k}(-1)^{k+l}\binom{n}{k}\underbrace{\binom{n-k}{l}}_\text{Null für $l=-1$}+\sum_{k=0}^{n}\sum_{l=0}^{n-k}(-1)^{k+l}\binom{n}{k-1}\binom{n-k}{l}\\
%&\quad +\sum_{k=0}^{n}(-1)^{n+1}\binom{n}{k-1}\underbrace{\binom{n-k}{n+1-k}}_0-\sum_{k=0}^{n}\sum_{l=-1}^{n-k}(-1)^{k+l}\binom{n}{k-1}\underbrace{\binom{n-k}{l}}_\text{Null für $l=-1$}+(-1)^{n+1}\\
%&=\cancel{(-1)^n}-\cancel{(-1)^n}+\bcancel{\sum_{k=0}^{n}\sum_{l=0}^{n-k}(-1)^{k+l}\binom{n}{k-1}\binom{n-k}{l}}-\bcancel{\sum_{k=0}^{n}\sum_{l=0}^{n-k}(-1)^{k+l}\binom{n}{k-1}\binom{n-k}{l}}+(-1)^{n+1}\\
%&=(-1)^{n+1}
%\end{align*}\hfill $\Box$