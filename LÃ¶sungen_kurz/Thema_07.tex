\Titelbanner{7}{Grundlagen der Differentialrechnung\\Kurvendiskussion}

\textbf{Aufgabe 1: } \emph{Ableitungen I}
\begin{enumerate}[label=(\alph*)]
\item $Q'(r)=3r\left(1+\ln\frac{r}{r_0}\right)$
\item $f'(x)=-12t\sin(3tx)\cos(3tx)$
\item $S'(\tau)=\tau\left(\e^{\tau}+\ln\tau\right)$
\item $y'(x)=x\cos(x)\e^{2x}$
\item $F'(x)=\frac{k(x-x_0)}{\sqrt{(x-x_0)^2+(y-y_0)^2}^{\ \!3}}$
\item $N'(z)=\frac{1}{\sqrt{1+\cos(z)}}$
\end{enumerate}
\vspace{0.7cm}
%
\textbf{Aufgabe 2: } \emph{Ableitungen II}
\begin{enumerate}[label=(\alph*)]
\item $f^{(n)}(x)=n!$
\item $f^{(n)}(x)=k^n\left(\e^{kx}+(-1)^n\e^{-kx}\right)$
\item $f^{(n)}(x)=0$
\item $f^{(n)}(x)=\left(\ln a\right)^n a^x$
\item $f^{(n)}(x)=\frac{n!}{(1-x)^{n+1}}$
\item* $f^{(n)}(x)=\frac{(-1)^{n+1} n!}{\sqrt{5}}\left(\frac{x_1^n}{(1+x_1x)^{n+1}}-\frac{x_2^n}{(1+x_2x)^{n+1}}\right)$,\ \ wobei $x_{1/2}$ die Nullstellen des Nenners bezeichnen.
\end{enumerate}
\vspace{0.7cm}
%
\newpage
\textbf{Aufgabe 3: } \emph{Kurvendiskussion I}
\begin{itemize}
\item Nullstelle: $x_0=-\frac{\ln 2}{\alpha}$\\[0.2cm]
Extremum: $x=0$, $U(x=0)=-D$\\[0.2cm]
Asymptotik: $\lim\limits_{x\rightarrow -\infty}U(x)=\infty$, $\lim\limits_{x\rightarrow \infty}U(x)=0$\\
% \item\adjustbox{valign=t}{
% \includegraphics[scale=0.6]{aufg3.pdf}}
\end{itemize}
\vspace{1cm}
%
\textbf{Aufgabe 4: } \emph{Kurvendiskussion II}
\begin{itemize}
\item Nullstelle: $r_0=2m$\\[0.2cm]
Extrema: $r_{1/2}=\frac{L^2}{2Em}\left(1\pm\sqrt{1-\frac{12Em^2}{L^2}}\right)$\\[0.2cm]
Asymptotik: $\lim\limits_{r\rightarrow\infty}U(r)=\frac{E}{2}$, $\lim\limits_{r\rightarrow 0}U(r)=-\infty$
\item 2 Extrema für $3E<L^2$, 1 Extremum für $3E=L^2$, kein (reelles) Extremum für $3E>L^2$
% \item\adjustbox{valign=t}{
% \includegraphics[scale=0.6]{aufg4.pdf}}
\end{itemize}
\vspace{1cm}
%
\textbf{Aufgabe 5*: } \emph{Gewöhnliche Differentialgleichung}
\begin{enumerate}[label=(\alph*)]
\item allgemeinste Lösung: $f(x)=c_1\e^{ax}+c_2\e^{-ax}-\frac{b}{a^2}x$ mit Konstanten $c_{1/2}$
\item allgemeinste Lösung: $f(x)=c\,x^x$ mit Konstante $c$ 
\end{enumerate}
%
%\newpage
%\textbf{Aufgabe 5: } \emph{Nichtlineare Differentialgleichung}\\[0.1cm]
%\begin{align*}
%f(t)\equiv \frac{\d u(x)}{\d x}=-\frac{4C}{a}\frac{1}{C^2\e^{x/a+at}+\e^{-(x/a+at)}} && \rightarrow && \frac{\d f(t)}{\d t}=\underline{\underline{4C\e^{x/a+at}\frac{C^2\e^{2(x/a+at)}-1}%%{\left(C^2\e^{2(x/a+at)}+1\right)^2}}}
%\end{align*}
%Abkürzung: $k\equiv C\e^{\frac{x}{a}+at}$
%\begin{align*}
%\sin(u)&=\sin(-4\arctan k)=-2\sin(2\arctan k)\cos(2\arctan k)\\
%&=-4\underbrace{\sin(\arctan k)\cos(\arctan k)}_{\frac{C \e^{x/a+at}}{C^2\e^{2(x/a+at)}+1}}\left[2\underbrace{\cos^2(\arctan k)}_{\frac{1}{C^2\e^{2(x/a+at)}+1}}-1\right]\\[2ex]
%&=\underline{\underline{4C\e^{x/a+at}\frac{C^2\e^{2(x/a+at)}-1}{\left(C^2\e^{2(x/a+at)}+1\right)^2}}}
%\end{align*}
%