\documentclass[parskip=half, fontsize=12pt]{scrartcl}

\usepackage[ngerman]{babel}
\usepackage{tabto} % alignment in enumerate
% \usepackage[english]{babel}         % Deutsches Sprachpaket

\usepackage{iftex}
\ifPDFTeX
   \usepackage[utf8]{inputenc}% Eingaben codieren
   \usepackage{fourier}
   % \usepackage[T1]{fontenc}   % Umlaute codieren, Silbentrennung
\else
   \usepackage{lmodern}
   % \usepackage[no-math]{fontspec}
   % \setmainfont{Utopia Std}
\fi

\usepackage{amsmath, amssymb}       % Mathe, \mathbb{R}
\usepackage{amsthm,amstext}         % Theoreme, \text im Mathe-Modus
\usepackage{mathtools}              % \Aboxed für Boxen in Align Umgebungen
\usepackage[arrowdel]{physics}      % Ableitungen \dv{B}{t} \pdv \dd{t}
\usepackage[left=2.5cm, right=2.5cm, top=3cm, bottom=3cm]{geometry}
\usepackage{graphicx}               % \includegraphics
\usepackage[extendedchars]{grffile} % extends file name processing of graphics
\usepackage[section]{placeins}      % \Floatbarrier
\usepackage{wrapfig}                % Bilder umfließen
\usepackage{enumerate}              % Aufzählungen
\usepackage{enumitem}               % \begin{enumerate}[label=\alph*]
\usepackage{footnote}               % Fußzeilen
\usepackage{booktabs}               % publication quality tables
\usepackage{tikz, pgfplots}         % TIKZ ist kein Zeichenprogramm
\usepackage[europeanvoltages,europeanresistors]{circuitikz}
\usepackage{bm}                     % bold symbols \bm{r}
\usepackage{dsfont}                 % identity matrix \mathds{1}
\usepackage{esint}                  % Doppelintegrale
\usepackage{mathrsfs}               % \mathscr{} statt \mathcal{}
\usepackage{placeins}               % FloatBarrier
\usepackage{subcaption}
\usepackage{multirow}
\usepackage{mdframed}               % Deckblatt
\usepackage{xcolor}                 % Deckblatt
\usepackage{fancyhdr}               % Kopfzeile
\usepackage{aligned-overset}        % Ausrichtungen mit stackrel oder overset
\usepackage{cancel}
\usepackage{float}
\usepackage{cite}
\usepackage[version=4]{mhchem}                 % Chemistry Package
\usepackage{multicol}
% \usepackage{pdfpages}             % insert whole pdf files

\definecolor{FSUblau}{cmyk}{1,0.7,0.1,0.5}
\definecolor{PAForange}{cmyk}{0.1,0.7,1,0}
\definecolor{Gruen}{cmyk}{1,0.1,0.7,0.5}

\usepackage[final,
    pdfauthor={Martin Beyer},
    pdffitwindow=false,     % resize document window to fit document size
    pdftoolbar=false,        % Adobe Toolbar
    bookmarks=true,         % Anzeigen der Kapitel
    bookmarksopen=true,
    bookmarksopenlevel=0,
    bookmarksnumbered=true,
    colorlinks=true,        % fuer Druckversion auf "false"
    linkcolor=FSUblau,         % Table of Contents, Footnotes
    urlcolor=FSUblau,          % fuer eingebunden URLs
    citecolor=FSUblau,         % Equations, References
    filecolor=FSUblau,
    pdfborder={0 0 0},      % keine Rahmen um Verlinkungen: {0 0 0}
    pagebackref=false
]{hyperref}

\pgfplotsset{compat=1.18}
\newcommand\mydots{\makebox[1em][c]{.\hfil.\hfil.}}
\newcommand{\minus}{\scalebox{0.75}[1.0]{$-$}}
\newcommand{\e}{\mathrm{e}}
\renewcommand{\i}{\mathrm{i}}

\usepackage[detect-all,
            locale=DE,
            exponent-product = \cdot,
            per-mode=fraction]{siunitx}
\usepackage[position=below,
            format=hang,
            figurename=Fig.,
            labelfont={bf},
            font=small]{caption}

\sisetup{range-phrase = {\mydots}}
% Commands
\usetikzlibrary{positioning,intersections,calc,external}
\usepgfplotslibrary{fillbetween, groupplots}
\pgfplotsset{
tick label style={font=\small},
label style={font=\small},
legend style={font=\footnotesize},
every axis post/.style={legend cell align={left}}}
\tikzstyle{every node}=[font=\small]


\setlength{\parindent}{0px}         % keine Absätze durch Leerzeilen im Code
\numberwithin{equation}{section}

% Remove page number from \thispagestyle{empty}
\makeatletter\let\ps@plain\ps@fancy\makeatother

% Deckblatt
\newcommand{\HRule}{\rule{\linewidth}{0.5mm}}
\newcommand{\Deckblatt}[5][\LaTeX-Satz und Design von Martin Beyer]{
  \begin{titlepage}
    \center
    \textsc{\LARGE Friedrich-Schiller-Universität Jena\\[1ex]
    \Large Physikalisch-Astronomische-Fakultät}
    \begin{figure}[h!]
       \centering
       \includegraphics[scale=0.75]{uni-Logo_neu.pdf}
    \end{figure}\\
    \vspace{2em}
    \textsc{\Large #2}\\[0.35cm]
    \HRule \\[0.4cm]
    { \Huge \bfseries #3}\\[0.15cm]
    \HRule \\[0.5cm]
    \textsc{\Large #4}\\[0.35cm]
    \vfill
    \begin{mdframed}[backgroundcolor=gray!20]
      \begin{center}
        #1
      \end{center}
    \end{mdframed}
  \end{titlepage}

  \pagestyle{fancy}
  \fancyhead[R]{\textbf{#5}}
  \fancyfoot[C]{\bfseries\thepage}
  \fancyhead[L]{\rightmark}

  \fancypagestyle{plain}{
    \fancyfoot[C]{\bfseries\thepage}
    \fancyhead[R]{}
    \fancyhead[L]{}
    \renewcommand{\headrulewidth}{0pt}
  }
}
\renewcommand{\sectionmark}[1]{\markright{#1}}
\renewcommand{\headrulewidth}{0.5pt}
\renewcommand{\footrulewidth}{0.5pt}

\newgeometry{left=2cm, right=2cm, top=2cm, bottom=2.5cm}
\setenumerate{labelindent=1em,labelsep=0.5cm,leftmargin=*}
\usetikzlibrary{tikzmark}
\everymath{\displaystyle}
\newcommand{\Semester}{2023/24}

\newcommand{\Titelbanner}[2]{
    \begin{center}
        \textsc{
        \LARGE Auffrischungskurs Mathematik\\[0.5cm]
        \large  -- ein Vorkurs für Studienanfänger --\\[0.5cm]}
        \footnotesize WS \Semester
    \end{center}
    \paragraph{Thema #1:} \hspace{0.2cm}
    \begin{minipage}[t]{0.8\linewidth}
        #2
    \end{minipage}
    \vspace{0.7cm}
}

%%%% 
% \includeonly{01_Grundrechenarten}
%%%%

\begin{document}
\graphicspath{{./Bilder/}{../}}
% \Deckblatt{Wintersemester \Semester}
%           {Mathematik - Ein Vorkurs für Studienanfänger}
%           {Martin Beyer}
%           {Mathematik}

\pagestyle{plain}
\renewcommand{\headrulewidth}{0pt}
\renewcommand{\footrulewidth}{0pt}
% \tableofcontents
% \newpage
  

\Titelbanner{1}{Grundrechenarten\\
                Brüche\\
                Potenzen\\
                Wurzeln}

\paragraph{Vorbereitung der Übung:} Wichtige Formeln an die Tafel schreiben!$~$

\begin{mymathbox}[ams align, title={Binomische Formeln}, colframe={FSUblau}]
      (a\pm b)^2 &= a^2 + b^2 \pm 2ab \notag \\
      (a+b)(a-b) &= a^2 - b^2.\notag 
\end{mymathbox}
\begin{mymathbox}[ams align, title={Potenzgesetze}, colframe={FSUblau}]
      a^m \cdot a^n = a^{m+n}, \quad a^n \cdot b^n &= (ab)^n, \quad (a^m)^n = (a^n)^m = a^{mn}\notag \\
      \frac{a^m}{a^n} = a^{m-n}, \quad &\frac{a^n}{b^n} = \qty(\frac{a}{b})^n.\notag 
\end{mymathbox}

\paragraph{Aufgabe 1: } \emph{Bruchrechnung} \hfill Ziel: (a) bis (f)\\[0.2cm]

\begin{enumerate}[label=(\alph*)]
    \item $\frac{\frac{b}{a}-\frac{a}{b}}{\frac{1}{a}+\frac{1}{b}} = \frac{b^2 - a^2}{a+b} = \uuline{b-a}$
    \item $\frac{\frac{1}{a-b}+\frac{1}{a+b}}{\frac{1}{a-b}-\frac{1}{a+b}} = \frac{a + \cancel{b} + a - \cancel{b}}{\bcancel{a} + b - \bcancel{a} + b} = \uuline{\frac{a}{b}}$
    \item $\frac{x^2-y^2}{xy}-\frac{x^2}{xy+x^2}+\frac{y^2}{x^2+xy} = \frac{1}{x} \qty(\frac{(x+y)(x-y)}{y} - \frac{x^2 - y^2}{x+y}) = \frac{x-y}{\cancel{x}} \underbrace{\qty(\frac{x+y}{y}-1)}_{\frac{\cancel{x}}{y}} = \uuline{\frac{x}{y}-1}$ \vspace{-5mm}
    \item $\frac{n+1}{2-\frac{1}{1-\frac{1}{n^2+1}}} = \frac{n+1}{2-\frac{n^2+1}{n^2}} = n^2 \frac{n+1}{n^2 -1} = \uuline{\frac{n^2}{n-1}}$
    \item $\frac{\frac{1}{y^2}+\frac{2}{xy}+\frac{1}{x^2}}{\frac{1}{y^2}-\frac{1}{x^2}} = \frac{x^2 +2xy + y^2 }{x^2 -y^2} = \frac{(x+y)^{\cancel{2}}}{\cancel{(x+y)}(x-y)} = \uuline{\frac{x+y}{x-y}}$
    \item $\frac{a^2-1}{a^2+a}-\cancel{a}\frac{a+1}{a^{\cancel{3}\textcolor{gray}{2}}-\cancel{a}}+\frac{1}{a}+\frac{(a+1)^2-(a-1)^2+4}{4(a^2-1)} $\\
    $= \frac{1}{a} \frac{(\cancel{a+1}) (a-\bcancel{1})}{\cancel{a+1}} - \frac{\cancel{a+1}}{(\cancel{a+1})(a-1)} + \bcancel{\frac{1}{a}} + \frac{\cancel{4a+4}}{\cancel{4(a+1)}(a-1)} = 1- \frac{1}{a-1}+\frac{1}{a-1} = \uuline{1}$
    \item $\frac{1+(a+x)^{-1}}{1-(a+x)^{-1}}\qty[\frac{\sqrt{2}}{ax}-\frac{1-(a^2+x^2)}{\sqrt{2}a^2x^2}] \text{ für } x=\frac{1}{a-1}$ \\
    $= \frac{a+x+1}{a+x-1}\qty[\frac{2ax -1 +a^2 +x^2}{\sqrt{2} a^2 x^2}] = \frac{a+x+1}{\cancel{a+x-1}} \overbrace{\frac{(a+x)^2 -1}{\sqrt{2}a^2 x^2}}^{\mathclap{(a+x+1)(\cancel{a+x-1})}} = \underbrace{\frac{(a+x+1)^2}{\sqrt{2}a^2 x^2} = \frac{1}{\sqrt{2}} \frac{a^2 (a-1)^2}{(a-1)^2}}_{\displaystyle \mathclap{x+a+1 = \frac{1}{a-1} + a+ 1 = \frac{a^2}{a-1}}} = \uuline{\frac{a^2}{\sqrt{2}}}$ 
\end{enumerate}

\paragraph{Aufgabe 2: } \emph{Potenzgesetze} \hfill Ziel: (a) bis (c)\\[0.2cm]

\begin{enumerate}[label=(\alph*)]
    \item $\qty(\frac{a^2-b^2}{x^2-y^2})^n\qty(\frac{x+y}{a-b})^n = \frac{(a+b)^n \cancel{(a-b)^n}}{\bcancel{(x+y)^n}(x-y)^n} \frac{\bcancel{(x+y)^n}}{\cancel{(a-b)^n}} = \uuline{\qty(\frac{a+b}{x-y})^n}$
    \item $\frac{b^xc^y(ab)^{2z+y}(cb)^{-x}}{(ac)^{y-x}\qty[\qty(abc^{-0,5})^z]^2} = a^{2z + y -(y-x+2z)} b^{x+2z+y-x-2z} c^{y-x-(y-x-z)} = \uuline{a^x b^y c^z}$
    \item $\frac{(a+b)^{3n-4}}{a^{n-1}b}\cdot\frac{a^{4n-3}(a+b)^{3-2n}}{b^{2n-5}}\cdot\frac{a^{4-3n}b^{3n-6}}{(a+b)^{n-2}} $\\
    $= a^{1-n+4n-3+4-3n} b^{-1-2n+5+3n-6} (a+b)^{3n-4+3-2n-n+2} $ \\
    $= \uuline{a^2 b^{n-2}(a+b)}$
    \item $\qty(a^{n+2}-a^n):(a^3+a^2) = \frac{a^n}{a^2}\frac{a^2-1}{a+1} = \uuline{(a-1) a^{n-2}}$
    \item $\qty(\frac{a^{-4}b^{-5}}{x^{-1}y^3})^2\cdot\qty(\frac{a^{-2}x}{b^3y^2})^3 = a^{-8-6} b^{-10-9} x^{2+3} y^{-6-6} = \uuline{\frac{x^5}{a^{14} b^{19} y^{12}}}$
\end{enumerate}
%
\newpage
\paragraph{Aufgabe 3: } \emph{Umformungen mit Wurzelausdrücken} \hfill Ziel: (a) bis (b)\\[0.2cm]

\begin{mymathbox}[ams align, title={Wurzelgesetze}, colframe={FSUblau}]
      &\sqrt[n]{a}\sqrt[n]{b} = \sqrt[n]{ab}, \quad \sqrt[n]{a^n b} =  a \sqrt[n]{b}, \quad \qty(\sqrt[n]{a})^m = \sqrt[n]{a^m}, \quad \sqrt[m]{\sqrt[n]{a}} = \sqrt[mn]{a} \notag\\
      &\sqrt[p]{a^m} \sqrt[q]{a^n} = \sqrt[pq]{a^{mq+np}}, \quad \frac{\sqrt[n]{a}}{\sqrt[n]{b}} = \sqrt[n]{\frac{a}{b}}, \quad \frac{\sqrt[p]{a^m}}{\sqrt[q]{a^n}} = \sqrt[pq]{a^{mq-np}}\notag
\end{mymathbox}
Beispiel für ``Rationalmachen des Nenners'':
\begin{align*}
    r + \sqrt{1+r^2} - \frac{1}{r+\sqrt{1+r^2}} = r + \cancel{1+r^2} - \frac{r-\sqrt{1+r^2}}{-1} = \uuline{2r}.
\end{align*}

\begin{enumerate}[label=(\alph*)]
    \item $\sqrt[6]{a^3}\dfrac{\frac{1}{\sqrt{a}}-\sqrt{b}}{1+\sqrt{ab}}+\dfrac{1}{\sqrt{2}}\dfrac{\sqrt{a}\sqrt{8b}}{1-ab} = \frac{1- \sqrt{ab}}{1+\sqrt{ab}} + \frac{2\sqrt{ab}}{1-ab} = \frac{(1-\sqrt{ab})^2 + 2\sqrt{ab}}{1-ab} =\uuline{\frac{1+ab}{1-ab}}$
    \item $\dfrac{\sqrt{a+bx}+\sqrt{a-bx}}{\sqrt{a+bx}-\sqrt{a-bx}} \quad \textnormal{für}\,\, x=\dfrac{2am}{b(1+m^2)}\,\,\textnormal{ mit }\,\, |m|<1$ \\
    $= \frac{\qty(\sqrt{a+bx}+\sqrt{a-bx})^2}{2bx} = \frac{\cancel{2}a + \cancel{2}\sqrt{a^2 - b^2 x^2}}{\cancel{2}bx}= \frac{a+\sqrt{a^2 - \frac{4a^2 m^2}{(1+m^2)^2}}}{2am}(1+m^2)$ \\
    $= \frac{1+m^2 + \sqrt{(1+m^2)^2 - 4m^2}}{2m} = \frac{1+m^2 +1 -m^2}{2m} = \uuline{\frac{1}{m}}$
    \item $\qty(\sqrt{ab}-\dfrac{ab}{a+\sqrt{ab}}):\dfrac{\textcolor{Gruen}{\sqrt[4]{ab}-\sqrt{b}}}{\textcolor{PAForange}{a-b}} = \frac{a \sqrt{b} + \cancel{b\sqrt{a}} - \cancel{\sqrt{a}b}}{\bcancel{\sqrt{a}+\sqrt{b}}} \frac{\textcolor{PAForange}{\bcancel{\qty(\sqrt{a}+\sqrt{b})} \qty(\sqrt[4]{a}+\sqrt[4]{b})\cancel{\qty(\sqrt[4]{a}-\sqrt[4]{b})}}}{\textcolor{Gruen}{\sqrt[4]{b}\cancel{\qty(\sqrt[4]{a}-\sqrt[4]{b})}}}$\\
    $ = \uuline{a\qty(\sqrt[4]{ab}+\sqrt{b})}$
\end{enumerate}

%
\newpage
\paragraph{Aufgabe 4: } \emph{Algebraische Umformungen} \hfill Ziel: (a) bis (c)\\[0.2cm]
Lösen Sie die folgenden Gleichungen jeweils nach $x$ auf.

\begin{enumerate}[label=(\alph*)]
    \item $~$\\[-1.45 cm]
    \begin{align*} 
        (a+nx)(b-nx)-(a-mx)(b+mx) &= x^2(m-n)(m+n)-1 \\ 
        \cancel{(m^2-n^2)x^2} + (bn -an + bm -am)x &= \cancel{x^2 (m^2 -n^2)} -1 \\
        (a-b)(m+n) x&=1 \qquad \Rightarrow \qquad \uuline{x = \frac{1}{(a-b)(m+n)}}
    \end{align*}
    \item $~$\\[-1.45 cm]
    \begin{align*}
        \frac{ax+b}{ab-b^2}-\frac{a-bx}{ab+b^2} &=\frac{2(ax+b)}{a^2-b^2} \quad |\cdot b(a^2 - b^2) \\
        (ax+b)(a+b) - (a-bx)(a-b) &= 2b(ax+b) \\
        (ax+b)\cancel{(a-b)} - (a-bx)\cancel{a-b} &= 0 \quad (a\neq b) \\
        (a+b) x &= a-b \qquad \Rightarrow \qquad \uuline{x= \frac{a-b}{a+b}}
    \end{align*}
    \item $~$\\[-1.45cm]
    \begin{align*}
        \frac{x-1}{n-1}+\underbrace{\frac{2n^2(1-x)}{n^4-1}}_{\mathclap{n^4-1 = (n^2+1)(n+1)(n-1)}}&=\frac{2x-1}{1-n^4}-\frac{1-x}{1+n} \quad |\cdot (n-1)(n+1)\\
        \qty(\cancel{n}+1 - \frac{2n^2}{n^2+1} + \frac{2}{n^2+1} -(\cancel{n}-1))x &= \cancel{n}+1 - \frac{2n^2}{n^2+1} + \frac{1}{n^2+1} -(\cancel{n}-1) \\
        4x &= 3 \qquad \Rightarrow \qquad \uuline{x = \frac{3}{4}}.
    \end{align*}
    \item $~$\\[-1.45cm] 
    \begin{align*}
        a\qty(\sqrt{x}-a)-b\qty(\sqrt{x}-b)+a+b &=\sqrt{x} \\
        \cancel{a-b-1} \sqrt{x} &= a^2-b^2 - (a+b) = (a+b)\cancel{(a-b-1)} \qquad \Rightarrow \qquad \uuline{x = (a+b)^2}
    \end{align*}
    \item $~$\\[-1.45cm] 
    \begin{align*}
        \frac{\frac{1}{x-\sqrt{1-4y^2}}+\frac{1}{x+\sqrt{1-4y^2}}}{\frac{1}{x-\sqrt{1-4y^2}}-\frac{1}{x+\sqrt{1-4y^2}}}&=\sqrt{1+\frac{y^2}{1+2y}}\sqrt{1+\frac{y^2}{1-2y}}, \qq{siehe 1b) mit $a=x, b= \sqrt{1-4y^2}$} \\
        \frac{x}{\cancel{\sqrt{1-4y^2}}} &= \frac{|y+1|}{\cancel{\sqrt{1+2y}}} \frac{|y-1|}{\cancel{\sqrt{1-2y}}} \qquad \Rightarrow \qquad \begin{cases}
            \uuline{x = y^2 -1} & \qq{für} |y| \ge 1 \\
            \uuline{x = 1- y^2} & \qq{für} |y| < 1.
        \end{cases}
    \end{align*}
\end{enumerate}
% \Titelbanner{2}{Lineare Gleichungssysteme}

\paragraph{Aufgabe 1: } \emph{Zwei lineare Gleichungen mit zwei Unbekannten}\\[0.2cm]
Lösen Sie die folgenden Gleichungen jeweils für $x$ und $y$.
\begin{enumerate}[label=(\alph*)]
    \item $33x+12y=25\,,$ \tab $11x-3y=6$
    \item $\dfrac{2x+3y}{3x-y}=\dfrac{17}{9}\,, $ \tab $ \dfrac{3x+4y}{6x-1}=2$
    \item $\frac{x+2}{y+3}=\frac{1}{3}\,, $ \tab $ \frac{y+3}{2y-5x}=\frac{3}{5}$
    \item $ax+by=2a\,, $ \tab $ \frac{x}{b}-\frac{y}{a}=\frac{2}{a}$
    \item $x+14y=\frac{1}{\sqrt{2}}-7\sqrt{2}\,, $ \tab $ 3\sqrt{2}x-\frac{y}{\sqrt{3}}=3+\frac{1}{\sqrt{6}}$
    \item $\frac{x}{a+b}+\frac{y}{a-b}=a+b\,, $ \tab $ \frac{x}{a}-\frac{y}{b}=2b$
    \item $39x-38y=1\,, $ \tab $ 91x-57y=4$
\end{enumerate}

\paragraph{Aufgabe 2: } \emph{Drei Gleichungen mit drei Unbekannten}\\[0.2cm]
Lösen Sie die folgenden Gleichungen jeweils für $x$, $y$ und $z$. %\\[1.5em]
%
\begin{center}
\begin{minipage}[t]{0.3\linewidth}
(a)\vspace{-2.7em}
\begin{align*}
    x-y+5z&=5\,,\\
    3x+7y-5z&=5\,,\\
    x+y-z&=1
\end{align*}

(c)\vspace{-2.7em}
\begin{align*}
    x+y&=b+a\,,\\
    x+z&=a+c\,,\\
    y+z&=c+b
\end{align*}
\end{minipage} \hspace{1.5cm}
\begin{minipage}[t]{0.4\linewidth}
%
(b)\vspace{-2.7em}
\begin{align*}
    3x-4y+3z&=4\,,\\
    -x+y-z&=-2\,,\\
    7x+4y-5z&=0
\end{align*}
(d)\vspace{-2.7em}
\begin{align*}
    6x-4y+8z&=0\,, \\
    -2x+y-z&=0\,,\\
    12x-7y+11z&=0
\end{align*}
\end{minipage} \\[0.2cm]
\end{center}
\vspace{0.7cm}

\newpage
% 
\textbf{Aufgabe 3: } \emph{Parametrisierung von Lösungsmengen}\\[0.2cm]
Geben Sie die Lösungsmenge der Gleichung $13x-7y=1$ an für
\begin{align*}
&\text{(a) } \hspace{0.2cm} x,y\in\mathbb{R}\,;\\[0.2cm]
&\text{(b) } \hspace{0.2cm} x,y\in\mathbb{N}\,.
\end{align*}

\paragraph{Aufgabe 4: } \emph{Gleichungssysteme}\\[0.2cm]
Lösen Sie die folgenden Gleichungssysteme jeweils für $x$ und $y$.\\[0.5cm]
\begin{minipage}[t]{0.45\linewidth}
(a)\vspace{-2.7em}
\begin{align*}
    x^2+y^2&=2(xy+2)\,,\\ x+y&=6
\end{align*}
(c)\vspace{-2,7em}
\begin{align*}
    \frac{x^2-y^2}{2x+3}+y^2&=(x+y)x-xy\,,\\
    y-2x&=3
\end{align*}
\end{minipage}\hfill
\begin{minipage}[t]{0.4\linewidth}
(b)\vspace{-2.7em}
\begin{align*}
    \frac{12}{\sqrt{x-1}}+\frac{5}{\sqrt{y+\frac{1}{4}}}&=5\,,\\ \frac{8}{\sqrt{x-1}}+\frac{10}{\sqrt{y+\frac{1}{4}}}&=6
\end{align*}
\end{minipage}
%
\paragraph{Aufgabe 5: } \emph{Ungleichungssysteme}
\begin{enumerate}[label=(\alph*), labelindent=1em,labelsep=0.5cm]
\item Es ist die Lösungsmenge des folgenden Systems an Ungleichungen zu skizzieren. Welche der Ungleichungen können weggelassen werden, ohne dass sich die Lösungsmenge ändert?
\begin{align*}
    2x-3y&\geq -6\,,\\
    x-2y&< 11\,,\\
    x&>-y-1\,,\\
    x&<5\,,\\
    x&\geq 0\,,\\
    y&\geq 0
\end{align*}
\item \textbf{*} Welches Gebiet im ersten Oktanden ($x\ge 0$, $y\ge 0$, $z\ge 0$) wird durch die folgenden Ungleichungen definiert?
\begin{align*}
    x+y&\ge z\,,\\
    x+z&\ge y\,,\\
    y+z&\ge x
\end{align*}
\end{enumerate} 
% \Titelbanner{3}{Quadratische Gleichungen und Gleichungssysteme}

\paragraph{Aufgabe 1: } \emph{Quadratische Gleichungen} \hfill Ziel: (a) und (b)\\[0.2cm]
Lösen Sie die folgenden Gleichungen jeweils für $x$ durch quadratische Ergänzung und kontrollieren Sie das Ergebnis mit der $pq$-Formel.
\begin{enumerate}[label=(\alph*)]
    \item $x^2-10x+9=0$
    \item $x^2+x-12=0$
    \item $x^2-\sqrt{8}x+1=0$
\end{enumerate}

\paragraph{Aufgabe 2: } \emph{Wurzeln quadratischer Gleichungen}
\begin{enumerate}[label=(\alph*)]
    \item Stellen Sie $\textstyle\frac{a}{b}-\frac{b}{a}$ als Produkt zweier Faktoren dar, deren Summe gleich $\textstyle\frac{a}{b}+\frac{b}{a}$ ist.
    \item Bestimmen Sie in der Gleichung $5x^2-kx+1=0$ den Koeffizienten $k$ so, dass die Differenz der Wurzeln 1 ergibt.
    \item Wählen Sie die Koeffizienten der quadratischen Gleichung $x^2+px+q=0$ so, dass die Wurzeln der Gleichung gleich $p$ und $q$ sind.
    \item Gegeben ist die quadratische Gleichung $ax^2+bx+c=0$. Gesucht ist diejenige neue quadratische Gleichung, deren Wurzeln gleich
    \begin{itemize}[labelindent=1em,labelsep=0.5cm]
        \item dem Doppelten der Wurzeln der gegebenen Gleichung,
        \item den reziproken Werten der Wurzeln der gegebenen Gleichung
    \end{itemize}
    sind.
\end{enumerate}

\paragraph{Aufgabe 3: } \emph{Gleichungssysteme}\\[0.2cm]
Lösen Sie die folgenden Gleichungssysteme jeweils für $x$ und $y$.\\[0.2cm]
\begin{minipage}[t]{0.25\linewidth}
(a)\vspace{-2.6em}
\begin{align*}
x+y^2&=7\,,\\ xy^2&=12
\end{align*}
\end{minipage}\hspace{0.05\linewidth}
\begin{minipage}[t]{0.3\linewidth}
(b)\vspace{-2.6em}
\begin{align*}
x+xy+y&=11\,,\\ x^2y+xy^2&=30
\end{align*}
\end{minipage}\hspace{0.05\linewidth}
\begin{minipage}[t]{0.3\linewidth}
    (c)\vspace{-2.8em}
\begin{align*}
x^2+y^2&=\frac{5}{2}xy\,,\\ x-y&=\frac{1}{4}xy
\end{align*}
\end{minipage}\\
\emph{Hinweis:} Substitution $z_1=xy$, $z_2=x+y$ in Aufgabe (b).

\paragraph{Aufgabe 4: } \emph{Wurzelgleichungen} \hfill Ziel: (a) bis (c)\\[0.2cm]
Lösen Sie die folgenden Gleichungen jeweils für $x$.
\begin{enumerate}[label=(\alph*)]
    \item $\sqrt{3x+1}-\sqrt{x-1}=2$
    \item $\sqrt{x+a}=a-\sqrt{x}$
    \item $\sqrt{a^2-x}+\sqrt{b^2-x}=a+b$
    \item $\sqrt{x+1+\sqrt{3x+4}}=3$
    \item $\sqrt{2x-1}+\sqrt{x-\num{1,5}}=\dfrac{6}{\sqrt{2x-1}}$
\end{enumerate}
%
\paragraph{Aufgabe 5: } \emph{Nullstellensuche} \hfill Ziel: (a) bis (c)\\[0.2cm]
Die folgenden Terme sind als Produkte von Linearfaktoren darzustellen.
\begin{enumerate}[label=(\alph*)]
    \item $x^2+2x-15$
    \item $4x^2+8x-5$
    \item $ax^3+bx^2+adx^2+bdx$
    \item $(a-x)^2+(x-b)^2-a^2-b^2$
    \item $a\sqrt{8}x^2-2kax-3ak+ax\sqrt{18}$
\end{enumerate}
% \Titelbanner{4}{Umgang mit Polynomen höheren Grades}

\paragraph{Aufgabe 1: } \emph{Nullstellensuche}
\begin{enumerate}[label=(\alph*), labelindent=1em,labelsep=0.5cm]
    \item Stellen Sie das Polynom dritten Grades auf, das Wurzeln $a$, $b$ und $c$ hat.
    \item Zerlegen Sie das Polynom $f_4(x)=x^3+2x^4+4x^2+2+x$ in Faktoren. Welche Aussage können Sie über dessen Nullstellen treffen?
    \item Bestimmen Sie alle Nullstellen des Polynoms $f_5(x)=x^5-3x^3+2x$.
    \item Bestimmen Sie die kleinste positive Nullstelle des Polynoms $\displaystyle f_4(x)=1-\frac{x^2}{2}+\frac{x^4}{24}$. Setzen Sie näherungsweise $\displaystyle \sqrt{3}\approx 3-\frac{\pi^2}{8}$.
\end{enumerate}

\paragraph{Aufgabe 2: } \emph{Polynomdivision}\\[0.2cm]
Berechnen Sie die folgenden Ausdrücke. Für welche Werte von $n$ bleibt die Polynomdivision in (d) ohne Rest?
\begin{multicols}{2}
    \begin{enumerate}[label=(\alph*), labelindent=1em,labelsep=0.5cm]
        \item $(21a^3-34a^2b+25b^3):(7a+5b)$
        \item $(9x^3+2y^3-7xy^2):(3x-2y)$
        \item $(25x^4+a^2x^2+25a^4):(5x^2+7ax+5a^2)$
        \item $(x^2+2x-15):(x+n)$
    \end{enumerate}
\end{multicols}

\paragraph{Aufgabe 3: } \emph{Kubische Gleichungen}
\begin{enumerate}[label=(\alph*)]
    \item Bestimmen Sie den Wert von $m$ in der Gleichung
    \begin{align*}
    6x^3-7x^2-16x+m=0\,,
    \end{align*}
    wenn eine Wurzel der Gleichung den Wert 2 hat. Berechnen Sie auch die beiden anderen Wurzeln.
    \item Die Zahlen 2 und 3 seien Wurzeln der Gleichung
    \begin{align*}
    2x^3+mx^2-13x+n=0\,.
    \end{align*}
    Bestimmen Sie die Zahlenwerte von $m$ und $n$, und geben Sie die dritte Wurzel an.
\end{enumerate}

\newpage
\paragraph{Aufgabe 4: } \emph{Nullstellenraten}\\[0.2cm]
Finden Sie jeweils mindestens eine Nullstelle der folgenden Ausdrücke und spalten Sie diese als Linearfaktor $(x-x_0)$ vom Polynom ab.
\begin{multicols}{2}
    \begin{enumerate}[label=(\alph*), labelindent=1em,labelsep=0.5cm]
        \item $x^3-5x^2+8x-4$
        \item $x^4-2x^3-13x^2+9x+9$
        \item $x^4-3x^2+3x+2$
        \item $x^5-x^4-3x^3+3x^2+x-1$
    \end{enumerate}
\end{multicols}
%
\paragraph{Aufgabe 5: } \emph{Partialbruchzerlegung}\\[0.2cm]
Schreiben Sie, so weit möglich, als Summe von Partialbrüchen.
\begin{multicols}{2}
    \begin{enumerate}[label=(\alph*)]
        \item $\frac{x-5}{x^2-2x-3}$
        \item $\frac{x^2+1}{x^2-1}$
        \item $\frac{2x^2-3x+1}{x^3-5x^2+8x-4}$
        \item $\frac{2x^4-4x^3-5x^2+(\sqrt{2}-7)x+\sqrt{2}+12}{x^2-2x-3}$
    \end{enumerate}
\end{multicols}
%
%
\paragraph{Aufgabe 6: } \emph{Polynome in Ungleichungen}\\[0.2cm]
Beweisen Sie die folgenden Ungleichungen.
\begin{enumerate}[label=(\alph*)]
    \item $(1+a+a^2)^2<3(1+a^2+a^4)\,,\qq{für} a\in\mathbb{R}\!\smallsetminus\!\{1\}$
    \item $x^4-x^2-6x+10>0\,,\qq{für} x\in\mathbb{R}$
\end{enumerate}
% \Titelbanner{5}{Exponentialfunktionen\\
                Logarithmen\\
                Natürliche Exponentialfunktion}

\paragraph{Aufgabe 1: } \emph{Logarithmische und Exponentialgleichungen}\\[0.2cm]
Lösen Sie jeweils für $x$ bzw. $y$. Beachten Sie mögliche Fallunterscheidungen bezüglich der auftretenden Konstanten.\\[-1.3em]
\begin{enumerate}[label=(\alph*)]
    \item $a\,2^x=\e^{bx}$
    \item $x=49^{1-\log_7(2\sqrt{x})}+5^{-\log_5(4x)}$
    \item $\e^{3ax}-2^{a+1}\e^{2ax}+4^a\e^{ax}=\log_b(1)$
    \item $\sqrt[x^2-1]{a^3}\cdot\sqrt[2x-2]{a}\cdot\sqrt[4]{a^{-1}}=1$
    \item $\log_ax+\log_ay=2\,, \quad\log_bx-\log_by=4$
    \item $3\log_{xa^2}x+\frac{1}{2}\log_\frac{x}{\sqrt{a}}x=2$
\end{enumerate}

\paragraph{Aufgabe 2: } \emph{Verdopplungszeit}\\[0.2cm]
Der Wissenszuwachs eines Physik-Studenten mit Anfangswissen $A$ sei beschrieben durch
\begin{align*}
    W(t)=A\e^{c t}\,, \hspace{1.5cm} c>0\,,
\end{align*}
sodass $W(t)$ die "`Menge"' an Wissen zur Zeit $t$ gibt.
\begin{enumerate}[label=(\alph*)]
    \item Nach welcher Zeit $\tau_2$ hat das Wissen eines Studenten auf das Doppelte zugenommen?
    \item Nach welcher Zeit $\tau_n$ hat das Wissen eines Studenten auf das $n$-Fache zugenommen?
    \item Drücken Sie $\tau_n$ als Vielfaches von $\tau_2$ aus. Welcher Zusammenhang besteht in den Fällen $n=3$ und $n=4$? Welche Aussage können Sie für $n=2^m$, $m\in\mathbb{N}$, treffen?
\end{enumerate}

\newpage
\paragraph{Aufgabe 3: } \emph{Hyperbelfunktionen}\hfill (Zusatzaufgabe)\\[0.2cm]
Es seien die Funktionen
\begin{align*}
    f(x)&:=\e^{x}+\e^{-x}\,,\\
    g(x)&:=\e^{x}-\e^{-x}
\end{align*}
definiert.
\begin{enumerate}[label=(\alph*)]
    \item Finden Sie jeweils einen Ausdruck für $f(2x)$ und $g(2x)$ in Abhängigkeit der Funktionen einfacher Argumente, $f(x)$ und $g(x)$.
    \item Finden Sie jeweils einen Ausdruck für $f(x+y)$ und $g(x+y)$ in Abhängigkeit von $f(x)$ und $f(y)$ sowie $g(x)$ und $g(y)$. Vergleichen Sie mit dem Ergebnis aus (a), indem Sie $x=y$ setzen.
    \item Schreiben Sie die Funktionen $f(x)$ und $g(x)$ in Reihendarstellung.
    \item Bestimmen Sie die Umkehrfunktion $g^{-1}$ von $g(x)$.
    % \item* Es sei eine dritte Funktion definiert als $h(x):=f(x)/g(x)$. Überzeugen Sie sich davon, dass diese Funktion mindestens einen \emph{Fixpunkt} besitzt, d.h. dass mindestens eine Zahl $u\in\mathbb{R}$ existiert, für die $h(u)=u$.
\end{enumerate}
% \Titelbanner{6}{Trigonometrische Funktionen\\Ebene Trigonometrie}

\paragraph{Aufgabe 1: } \emph{Additionstheoreme}\\[-1em]
\begin{enumerate}[label=(\alph*)]
\item Leiten Sie das Additionstheorem für Kosinusfunktionen aus dem für Sinusfunktionen her.
% \begin{align*}
%     \text{Es gilt:} \quad \sin(x\pm y) = \sin(x)\cos(y) \pm \cos(x) \sin(y)
% \end{align*}
\item Leiten Sie das Additionstheorem für Tangensfunktionen her,
\begin{align*}
    \tan(x\pm y)=\frac{\tan(x)\pm \tan(y)}{1\mp \tan(x)\tan(y)}\,.
\end{align*}
\item Zeigen Sie, dass für Doppelwinkelfunktionen gilt:
\begin{itemize}
    \item $\sin(2x)=2\sin(x)\cos(x)\,,$
    \item $\cos(2x)=2\cos^2\qty(x)-1\,.$
\end{itemize}
\end{enumerate}

\paragraph{Aufgabe 2: } \emph{Trigonometrische Umformungen I}\hfill Ziel: (a) bis (c)\\[0.2cm]
Zeigen Sie die Richtigkeit der folgenden Identitäten.
\begin{enumerate}[label=(\alph*)]
    \item $\dfrac{\cos(\alpha)+\sin(\alpha)}{\cos(\alpha)-\sin(\alpha)}=\tan\qty(\frac{\pi}{4}+\alpha)$
    \item $\dfrac{1+\sin (2\alpha)}{\cos (2\alpha)}=\dfrac{1+\tan(\alpha)}{1-\tan(\alpha)}=\tan\qty(\frac{\pi}{4}+\alpha)$
    \item $2\cos\qty(\frac{x+y}{2})\cos\qty(\frac{x-y}{2})=\cos(x)+\cos(y)$
    \item $\cot(\alpha)\cot(\beta)+\cot(\alpha)\cot(\gamma)+\cot(\beta)\cot(\gamma)=1 \text{ für } \alpha+\beta+\gamma=\pi$
\end{enumerate}

\paragraph{Aufgabe 3: } \emph{Trigonometrische Umformungen II}\hfill Ziel: (a) und (b)\\[0.2cm]
Formen Sie die folgenden Ausdrücke so um, dass sie sich einfach logarithmieren lassen. Das heißt, die Terme sollen möglichst in Produkte, Quotienten und Potenzen umgeformt werden.\\[-1.3em]

\begin{enumerate}[label=(\alph*)]
    \item $1+\cos(\alpha) + \cos\qty(\frac{\alpha}{2}), \qq{\emph{Hinweis:} Es ist} \cos\qty(\frac{\pi}{3})=\dfrac{1}{2}\,.$
    \item $\frac{2\sin(\beta) - \sin (2\beta)}{2\sin(\beta)+2\sin(2\beta)}$
    \item $\sin(\alpha)+\sin(\beta)+\sin(\gamma), \qq{für} \alpha+\beta+\gamma = \pi$
\end{enumerate}
%
\paragraph{Aufgabe 4: } \emph{Goniometrische Gleichungen und Gleichungssysteme}\hfill Ziel: (a) bis (c)\\[0.2cm]
Bestimmen Sie alle Lösungen der folgenden Gleichungen. Es reicht aus, die Lösungen in impliziter Form stehenzulassen!
\begin{enumerate}[label=(\alph*)]
    \item $\sin(x)+\cos(x)= 1$
    \item $\cos\qty(\frac{x+y}{2})\cos\qty(\frac{x-y}{2})=\frac{1}{2}\,, \hspace{0.5cm}\cos(x)\cos(y)=\frac{1}{4}$
    \item $\sin(3x)=\cos(2x)$
    \item $a\qty(3\cos^2(x)+\sin(x)\cos(x))-b\qty(3\sin^2(x)-\sin(x)\cos(x))=2a-b$
\end{enumerate}
\emph{Hinweis:} Versuchen Sie in (c) und (d) die Gleichung so umzuformen, dass nur noch eine Funktionsart auftritt. Dann reduziert sich das Problem zu einer Nullstellensuche eines Polynoms.

\paragraph{Aufgabe 5: } \emph{Dreiecksfläche}\\[0.2cm]
Berechnen Sie die Fläche eines Dreiecks, wenn die Seiten $a$ und $b$ sowie die Länge $w$ der Winkelhalbierenden des Winkels zwischen diesen Seiten gegeben sind.
% %
% \paragraph{Aufgabe 6: } \emph{Sehnen im Kreis}\\[0.2cm]
% Durch einen Punkt auf einem Kreis vom Radius $r$ seien zwei Sehnen der Längen $a$ und $b$ gelegt. Wenn man die Schnittpunkte der Sehnen mit der Peripherie untereinander geradlinig verbindet, erhält man ein Dreieck. Bestimmen Sie dessen Flächeninhalt $A$.

% \Titelbanner{7}{Grundlagen der Differentialrechnung\\Kurvendiskussion}

\paragraph{Aufgabe 1: } \emph{Ableitungen I}\\[0.2cm]
Berechnen Sie die Ableitungen der folgenden Funktionen.
\begin{enumerate}[label=(\alph*)]
    \item $Q(r)=\frac{3r^2}{2}\qty(\frac{1}{2}+\ln\frac{r}{r_0})$
    \item $f(x)=\cos^4(3tx)-\sin^4(3tx)$
    \item $S(\tau)=(\tau-1)\e{\tau}+\frac{\tau^2}{4}\qty(2\ln\tau-1)$
    \item $y(x)=\frac{\exp(2x)}{25}\qty[(5x-4)\sin x+(10x-3)\cos x]$
    \item $F(x)=-\frac{k}{\sqrt{(x-x_0)^2+(y-y_0)^2}}$
    \item $N(z)=\frac{2\cos\qty(\frac{z}{2})}{\sqrt{1+\cos(z)}}\qty[\ln\qty(\cos\frac{z}{4}+\sin\frac{z}{4})-\ln\qty(\cos\frac{z}{4}-\sin\frac{z}{4})]$
\end{enumerate}
%
\paragraph{Aufgabe 2: } \emph{Ableitungen II}\\[0.2cm]
Finden Sie die $n$-te Ableitung der folgenden Funktionen.
\begin{multicols}{2}
    \begin{enumerate}[label=(\alph*)]
        \item $f(x)=x^n$
        \item $f(x)=\e{kx}+\e{-kx}$
        \item $f(x)=x^{n-1}$
        \item $f(x)=a^x$ 
        \item $f(x)=\frac{1}{1-x}$
        \item[(f)*] $f(x)=\frac{x}{1-x-x^2}$ 
    \end{enumerate}
\end{multicols}
%
\paragraph{Aufgabe 3: } \emph{Kurvendiskussion I}\\[0.2cm]
Ein zweiatomiges Molekül lässt sich näherungsweise durch das sogenannte ``Morse-Potential'' beschreiben,
\begin{align*}
U(x)=D\qty(\e{-2\alpha x}-2\e{-\alpha x})\,, \hspace{1cm} D,\alpha=\operatorname{const}.
\end{align*}
\begin{itemize}
\item Bestimmen Sie Nullstellen und lokale Extrema der Funktion $U(x)$ sowie deren Verhalten für $x \to\pm\infty$.
\item Skizzieren Sie die Funktion $U(x)$ für $D=\alpha=1$ im Intervall $x\in[-1,5]$.
\end{itemize}
%
\paragraph{Aufgabe 4: } \emph{Kurvendiskussion II}\\[0.2cm]
Die Bewegung eines Teilchens mit Drehimpuls $L$ und Energie $E$ in der gekrümmten Raumzeit eines Schwarzen Loches der Masse $m$ wird beschrieben durch das Potential
\begin{align*}
U(r)=\frac{E}{2}-\frac{Em}{r}+\frac{L^2}{2r^2}-\frac{mL^2}{r^3}\,, \hspace{1cm} r>0\,.
\end{align*}
\begin{itemize}
\item Bestimmen Sie Nullstellen und lokale Extrema der Funktion $U(r)$ sowie deren Verhalten für $\to\infty$ und $\to 0$.
\item Setzen Sie $m=\frac{1}{2}$. Welche Bedingungen an $E$ und $L$ müssen erfüllt sein, damit $U(r)$ zwei, ein oder keine lokalen Extrema besitzt?
\item Skizzieren Sie die Funktion $U(r)$ für $E=1$ und $L=2$ (nicht maßstabsgerecht).
\end{itemize} 
%
\paragraph{Aufgabe 5*: } \emph{Gewöhnliche Differentialgleichungen}
\begin{enumerate}[label=(\alph*)]
\item Finden Sie eine Funktion $f(x)$, welche die folgende Gleichung erfüllt:
\begin{align*}
f''(x)=a^2f(x)+bx\,.
\end{align*}
\item Finden Sie eine Funktion $f(x)$, welche die folgende Gleichung erfüllt:
\begin{align*}
f'(x)=\qty(1+\ln(x))f(x)\,.
\end{align*}
\end{enumerate}
% \Titelbanner{8}{Die Methode der vollständigen Induktion}

\paragraph{Aufgabe 1: } \emph{Rekursive und explizite Zuordnungsvorschrift}\\[0.2cm]
Eine Reihe $(a_n)_{n\in\mathbb{N}_0}$ sei rekursiv definiert durch
\begin{align*}
a_{n+1}=2a_n+1\,, && a_0=0\,.
\end{align*}
Finden Sie einen expliziten Ausdruck für $a_n$ und beweisen Sie ihn per vollständiger Induktion.
%
%
\paragraph{Aufgabe 2: } \emph{Vollständige Induktion I}\\[0.6cm]
Beweisen Sie\\[-1.5cm]
\begin{align*}
S_n&=1^2-2^2+3^2-4^2+\hdots+(-1)^{n-1}n^2=(-1)^{n-1}\,\dfrac{n(n+1)}{2}\,.
\end{align*}
%
\paragraph{Aufgabe 3: } \emph{Vollständige Induktion II}\\[0.2cm]
Beweisen Sie, dass die Summe der Kuben der ersten $n$ natürlichen Zahlen gleich $\left[\frac{n(n+1)}{2}\right]^2$ ist. 
%
\paragraph{Aufgabe 4: } \emph{Vollständige Induktion III}\\[0.2cm]
Zeigen Sie, dass die Summe der Kuben dreier aufeinanderfolgender natürlicher Zahlen durch $9$ teilbar ist.
%
\paragraph{Aufgabe 5: } \emph{Die Suche nach der richtigen Summenformel}\\[0.2cm]
Stellen Sie eine Summenformel für das Polynom
\begin{align*}
S_n = 1-\dfrac{x}{1!}+\dfrac{x(x-1)}{2!}-\hdots+(-1)^n \dfrac{x(x-1)\dots(x-n+1)}{n!}
\end{align*}
auf und beweisen Sie deren Richtigkeit durch vollständige Induktion.
%
\paragraph{Aufgabe 6: } \emph{Fibonacci-Zahlen}\hfill (Zusatzaufgabe)\\[0.2cm]
Die Folge der Fibonacci-Zahlen ist definiert durch $a_{n+1}=a_n+a_{n-1}$ für $n\ge 1$ und die Startwerte $a_0=0$ und $a_1=1$. Beweisen Sie, dass der explizite Ausdruck für die $n$-te Fibonacci-Zahl gegeben ist durch
\begin{align*}
a_n=\frac{1}{\sqrt{5}}\left(x_+^n-x_-^n\right)\,, \qq{mit} x_{\pm}=\frac{1\pm\sqrt{5}}{2}\,.
\end{align*}
% \Titelbanner{9}{Arithmetische \& geometrische Reihen\\
                Der binomische Satz}

\paragraph{Aufgabe 1: } \emph{Arithmetische Reihe}\\[0.2cm]
Eine Spirale bestehe aus zwei Scharen konzentrischer Halbkreise um die Punkte $A$ und $B$. Es sei $r$ der Radius des innersten Halbkreises und die Strecke $\overline{AB}=e$.
\begin{enumerate}[label=(\alph*)]\setlength{\itemsep}{-0.5ex}
\item Wie lang ist der $n$-te Halbbogen?
\item Wie lang ist der Gesamtbogen der Spirale bis dahin?
\end{enumerate}
\begin{figure}[htp]
    \centering
    \begin{tikzpicture}[scale=0.5]
        \draw[thick] (-6,0) -- (6,0);
        \draw[thick] (2.3, 0) arc(0:-180:2.3); 
        \draw[thick] (4.3, 0) arc(0:-180:4.3);
        \draw (0,-.1)node[below]{$B$} -- (0,.1); 
        \draw[{latex}-{latex}] (0,-1) --node[below]{$e$} (1.3,-1);
        \begin{scope}[shift={(1,0)}]
            \node (A) at (0.65,0.3){$r$};
            \draw (0,-.1)node[below]{$A$} -- (0,.1);
            \draw[thick] (1.3, 0) arc(0:180:1.3); 
            \draw[thick] (3.3, 0) arc(0:180:3.3); 
        \end{scope}
    \end{tikzpicture}
    \vspace{-0.5cm}
\end{figure}
%
\paragraph{Aufgabe 2: } \emph{Geometrische Folge}\\[0.2cm]
Beim Durchgang durch eine Glasplatte verliert ein Lichtstrahl $\textstyle\frac{1}{12}$ seiner Lichtstärke. Wie viele Platten muss er durchdringen, wenn er nur noch die Hälfte der ursprünglichen Lichtstärke besitzen soll?\\[0.2cm]
\emph{Hinweis:} Es genügt die Angabe des Ergebnisses in impliziter Form.
%
\paragraph{Aufgabe 3: } \emph{Geometrische Reihe}\\[0.2cm]
Es sei eine Anordnung aus zwei Glasplatten gegeben, in die ein Laserstrahl der Intensität $\mathcal{I}_\text{in}$ eingeschossen werde. Jede Platte besitze einen Transmissionskoeffizienten $T$ ($0<T<1$), d.h. dass an jeder Platte von einem Lichtstrahl der Intensität $\mathcal{I}$ ein Anteil $T \mathcal{I}$ transmittiert und ein Anteil $(1-T)\mathcal{I}$ reflektiert wird.
Bestimmen Sie die Intensität $\mathcal{I}_\text{out}$, mit der das Licht auf der anderen Seite der Anordnung austritt.\\
\begin{center}
    \vspace{-1cm}
\begin{tikzpicture}[scale=0.8]
    \node [] at(0.2,0){$\mathcal{I}_\text{out}$};
    \draw [<-,line width=0.8pt] (0.7,0)--(2.7,0);
    \draw [line width=1pt] (3.0,-2)--(3.0,2);
    \draw [line width=1pt] (6.0,-2)--(6.0,2);
    \draw [<-,line width=0.8pt] (3.3,0.15)--(5.7,0.15);
    \draw [->,line width=0.8pt] (3.3,-0.15)--(5.7,-0.15);
    \draw [<-,line width=0.8pt] (6.3,0)--(8.3,0);
    \node [] at(8.8,0){$\mathcal{I}_\text{in}$};
    \node [below] at(3,-2){$T$};
    \node [below] at(6,-2){$T$};
\end{tikzpicture}
\vspace{-2cm}
\end{center}
%
\newpage
\paragraph{Aufgabe 4: } \emph{Der binomische Satz}
\begin{enumerate}[label=(\alph*)]
\item Berechnen Sie $(1+a)^6+(1-a)^6$.
\item Schreiben Sie die allgemeine Binomialformel für $(1+x)^n$ und $(1-x)^n$ auf und setzen Sie anschließend $x=1$. In letzterem Falle sei $n\ge 1$.
\item Es seien die Funktionen $f(x)=\left(\e^{x}+\e^{-x}\right)/2$ und $g(x)=\left(\e^{x}-\e^{-x}\right)/2$ definiert. Finden Sie einen Ausdruck, der $f(nx)+g(nx)$, mit $n\in\mathbb{N}$, auf (Potenzen von) $f(x)$ und $g(x)$ zurückführt.
\end{enumerate}
%
\paragraph{Aufgabe 5: } \emph{Erzeugende Funktion} \hfill (Zusatzaufgabe)\\[0.2cm]
Betrachten Sie die Rekursionsgleichung $a_{n+1}=2a_n+1$ mit Anfangswert $a_0=0$ aus Aufgabe 1, Thema 8. Es sei eine (unbekannte) Funktion definiert als
\begin{align*}
f(x)=\sum\limits_{n=0}^\infty a_n x^n\,.
\end{align*}
\begin{enumerate}[label=(\alph*)]
\item Multiplizieren Sie beide Seiten der Rekursionsgleichung mit $x^n$, summieren Sie über $n$ (von Null bis Unendlich) ab und versuchen Sie alle auftretenden Terme durch $f(x)$ auszudrücken. Lösen Sie für $f(x)$.
\item Nutzen Sie Ihr Wissen über geometrische Reihen, um $f(x)$ wieder in Reihendarstellung zu überführen und lesen Sie die Koeffizienten vor $x^n$ ab.
\end{enumerate}
%
% \paragraph{Aufgabe 6*: } \emph{Zustandssumme}\\[0.2cm]
% Vereinfachen Sie den Ausdruck
% \begin{align*}
% \Omega_N=\sum\limits_{n_1=0}^N\sum\limits_{n_2=0}^N\sum\limits_{n_3=0}^N\dots\sum\limits_{n_k=0}^N f(n_1)f(n_2)f(n_3)\dots f(n_k)\,, \qq{wobei}f(n)=\e^{-n}.
% \end{align*}
% Bestimmen Sie anschließend den Grenzwert $\Omega=\lim\limits_{N\rightarrow\infty}(\Omega_N)$.
% %
% \paragraph{Aufgabe 7*: } \emph{Fibonacci-Zahlen II}\\[0.2cm]
% Wenden Sie das Verfahren aus Aufgabe 5 auf die Folge $a_{n+1}=a_n+a_{n-1}$ der Fibonacci-Zahlen mit Startwerten $a_0=0$, $a_1=1$ an. Beachten Sie, dass hier $n\ge 1$ gelten muss.
%
% \Titelbanner{10}{Rechnen mit Vektoren und Matrizen}

\paragraph{Aufgabe 1: } \emph{Linear (un)abhängige Vektoren}
\begin{enumerate}[label=(\alph*)]
\item Sind die folgenden Vektoren linear unabhängig?
\begin{align*}
\mqty( 3\\1\\-5 ),\quad \mqty( 1\\2\\0 ),\quad \mqty(-2\\0\\1)
\end{align*}
\item Für welche Werte von $t$ befinden sich die folgenden Vektoren in einer Ebene?
\begin{align*}
\mqty( 1\\4\\0 ),\quad \mqty( 3\\t\\1 ),\quad \mqty( 4\\4\\2 )
\end{align*}
\end{enumerate}
%
\paragraph{Aufgabe 2: } \emph{Skalarprodukt und Vektorprodukt}
\begin{enumerate}[label=(\alph*)]
\item Berechnen Sie die folgenden Skalarprodukte.
\begin{align*}
\mqty(1\\5\\7)\cdot\mqty(-2\\8\\3), &&
\mqty(1\\0\\ \sqrt{2})\cdot\mqty(0\\-1\\ \sqrt{2}), &&
\mqty(a\\3\\-1)\cdot\mqty(5\\2\\5a+6)
\end{align*}
\item Berechnen Sie die folgenden Vektorprodukte.
\begin{align*}
\mqty(1\\0\\0 )\times\mqty(0\\1\\0),&&
\mqty(\cos(\phi)\\ \sin(\phi) \\0)\times\mqty(-\sin(\phi)\\ \cos(\phi)\\0),&&
\mqty(1\\3\\-2)\times\mqty(1\\1\\5)
\end{align*}
\item Nutzen Sie das Skalarprodukt, um den Kosinussatz im allgemeinen Dreieck herzuleiten.
\item Nutzen Sie das Vektorprodukt, um den Sinussatz im allgemeinen Dreieck herzuleiten.
\end{enumerate}
%
\paragraph{Aufgabe 3: } \emph{Matrizen}
\begin{enumerate}[label=(\alph*)]
\item Bilden Sie -- falls möglich -- die Inversen der folgenden Matrizen mit Hilfe des Gauß-Jordan-Algorithmus.
\begin{align*}
 A =\mqty(1 & 2\\ 3 & 4),\quad B=\mqty(0 & 2\\ 3 & 0),\quad  C=\mqty(-2 & 4\\ 3 & -6)
\end{align*}
\end{enumerate}
\newpage
\begin{enumerate}[label=(\alph*),resume]
\item Matrixmultiplikation ist im Allgemeinen nicht kommutativ. Berechnen Sie $ A \cdot B- B\cdot A$.
\item Welche Bedingungen müssen die Einträge $m_i$ einer Matrix
\begin{align*}
M = \mqty(m_1 & m_2\\ m_3 & m_4)
\end{align*}
erfüllen, damit $| M\cdot \vec{v}|=|\vec{v}|$ für einen beliebigen zweikomponentigen Vektor $\vec{v}$ gilt? 
\end{enumerate}
%
%\paragraph{Aufgabe 4: } \emph{Drehungen}\\[0.2cm]
%Gegeben seien eine Matrix $\hat{R}(\phi)$ und ein Vektor $\vec{\alpha}$ als
%\begin{align*}
%\hat{R}=\begin{pmatrix*}[c]\cos(\phi)	& -\sin(\phi)	& 0\\ \sin(\phi) & \cos(\phi) & 0\\ 0 & 0 & 1\end{pmatrix*}, \quad \vec{\alpha}=\mqty(x\\y\\z).
%\end{align*}
%Berechnen Sie $\hat{R}(\phi)\cdot\vec{\alpha}$ und $\det\hat{R}(\phi)$. Welche geometrische Bedeutung hat der Fall $\phi=\frac{\pi}{2}$, $x=1$, $y=z=0$?\\[1cm]
%
\paragraph{Aufgabe 4: } \emph{Laplace'scher Entwicklungssatz}\\[0.2cm]
Berechnen Sie die Determinante der folgenden Matrix. Für welche Werte von $a$ ist die Matrix nicht invertierbar?
\begin{align*}
A=\mqty(a &3 &1 &a\\ 0 &-2 &0 &1\\ 3 &2 &-4 &-1\\ -a& 0 &-2 &0)
\end{align*}
%
\paragraph{Aufgabe 5: } \emph{Vektorprodukt}\\[0.2cm]
Das Vektorprodukt $\vec{\omega}\times \vec{u}$ zweier Vektoren
$\quad \vec{\omega}=\mqty(\omega_1\\\omega_2\\\omega_3),\quad\vec{u}=\mqty(u_1\\u_2\\u_3)$

lässt sich auch als Matrixmultiplikation $\Omega \cdot \vec{u}$ mit einer ($3\times 3$)-Matrix $\Omega$ mit den Einträgen $\pm \omega_i$, $i\in\{1,2,3\}$ schreiben. Wie sieht  $\Omega$ aus? Welche Eigenschaften hat es (Determinante, Spur, Symmetrie)?
%
%
\paragraph{Aufgabe 6*: } \emph{Höhere Dimensionen}\\[0.2cm]
Im dreidimensionalen euklidischen Raum $\mathbb{R}^3$ kann jede beliebige Drehung in Drehungen um die $x$-, die $y$- und die $z$-Achse zerlegt werden, d.h. man benötigt genau drei linear unabhängige Drehmatrizen für eine solche Zerlegung. Wie viele linear unabhängige Drehmatrizen benötigt man im $d$-dimensionalen euklidischen Raum $\mathbb{R}^d$\,?
%
%
\paragraph{Aufgabe 7*: } \emph{Diskreter Laplace-Operator}\\[0.2cm]
Gegeben sei die ($n\times n$)-Matrix $\Delta_n$, welche die Einträge -2 auf der Haupt- und 1 auf den beiden Nebendiagonalen hat,
\begin{align*}
\Delta_n=\mqty(
-2 & 1 & 0 & 0 & 0 &\dots & 0\\
1 & -2 & 1 & 0& 0 &\dots & 0\\
0 & 1 & -2 & 1 & 0 &\dots & 0\\
\vdots &\vdots &\vdots &\vdots &\vdots & &\vdots\\
0 & 0 & 0 & 0 & 0 &\dots & -2
)\,.
\end{align*}
Bestimmen Sie die Determinante von $\Delta_n$\,, indem Sie wie folgt vorgehen:
\begin{enumerate}
\item Finden Sie eine Rekursionsrelation, die $\det(\Delta_n)$ auf $\det(\Delta_{n-1})$ und $\det(\Delta_{n-2})$ zurückführt.
\item Raten Sie einen allgemeinen Ausdruck für $\det(\Delta_{n})$ und beweisen Sie ihn mittels vollständiger Induktion. Alternativ kann der Versuch unternommen werden, das Verfahren aus Thema 9, Aufgabe 5 anzuwenden.
\end{enumerate}

\end{document}