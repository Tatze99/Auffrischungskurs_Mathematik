\Titelbanner{4}{Umgang mit Polynomen höheren Grades}

\paragraph{Aufgabe 1: } \emph{Nullstellensuche}
\begin{enumerate}[label=(\alph*), labelindent=1em,labelsep=0.5cm]
    \item Stellen Sie das Polynom dritten Grades auf, das Wurzeln $a$, $b$ und $c$ hat.
    \item Zerlegen Sie das Polynom $f_4(x)=x^3+2x^4+4x^2+2+x$ in Faktoren. Welche Aussage können Sie über dessen Nullstellen treffen?
    \item Bestimmen Sie alle Nullstellen des Polynoms $f_5(x)=x^5-3x^3+2x$.
    \item Bestimmen Sie die kleinste positive Nullstelle des Polynoms $\displaystyle f_4(x)=1-\frac{x^2}{2}+\frac{x^4}{24}$. Setzen Sie näherungsweise $\displaystyle \sqrt{3}\approx 3-\frac{\pi^2}{8}$.
\end{enumerate}

\paragraph{Aufgabe 2: } \emph{Polynomdivision}\\[0.2cm]
Berechnen Sie die folgenden Ausdrücke. Für welche Werte von $n$ bleibt die Polynomdivision in (d) ohne Rest?
\begin{multicols}{2}
    \begin{enumerate}[label=(\alph*), labelindent=1em,labelsep=0.5cm]
        \item $(21a^3-34a^2b+25b^3):(7a+5b)$
        \item $(9x^3+2y^3-7xy^2):(3x-2y)$
        \item $(25x^4+a^2x^2+25a^4):(5x^2+7ax+5a^2)$
        \item $(x^2+2x-15):(x+n)$
    \end{enumerate}
\end{multicols}

\paragraph{Aufgabe 3: } \emph{Kubische Gleichungen}
\begin{enumerate}[label=(\alph*)]
    \item Bestimmen Sie den Wert von $m$ in der Gleichung
    \begin{align*}
    6x^3-7x^2-16x+m=0\,,
    \end{align*}
    wenn eine Wurzel der Gleichung den Wert 2 hat. Berechnen Sie auch die beiden anderen Wurzeln.
    \item Die Zahlen 2 und 3 seien Wurzeln der Gleichung
    \begin{align*}
    2x^3+mx^2-13x+n=0\,.
    \end{align*}
    Bestimmen Sie die Zahlenwerte von $m$ und $n$, und geben Sie die dritte Wurzel an.
\end{enumerate}

\newpage
\paragraph{Aufgabe 4: } \emph{Nullstellenraten}\\[0.2cm]
Finden Sie jeweils mindestens eine Nullstelle der folgenden Ausdrücke und spalten Sie diese als Linearfaktor $(x-x_0)$ vom Polynom ab.
\begin{multicols}{2}
    \begin{enumerate}[label=(\alph*), labelindent=1em,labelsep=0.5cm]
        \item $x^3-5x^2+8x-4$
        \item $x^4-2x^3-13x^2+9x+9$
        \item $x^4-3x^2+3x+2$
        \item $x^5-x^4-3x^3+3x^2+x-1$
    \end{enumerate}
\end{multicols}
%
\paragraph{Aufgabe 5: } \emph{Partialbruchzerlegung}\\[0.2cm]
Schreiben Sie, so weit möglich, als Summe von Partialbrüchen.
\begin{multicols}{2}
    \begin{enumerate}[label=(\alph*)]
        \item $\frac{x-5}{x^2-2x-3}$
        \item $\frac{x^2+1}{x^2-1}$
        \item $\frac{2x^2-3x+1}{x^3-5x^2+8x-4}$
        \item $\frac{2x^4-4x^3-5x^2+(\sqrt{2}-7)x+\sqrt{2}+12}{x^2-2x-3}$
    \end{enumerate}
\end{multicols}
%
%
\paragraph{Aufgabe 6: } \emph{Polynome in Ungleichungen}\\[0.2cm]
Beweisen Sie die folgenden Ungleichungen.
\begin{enumerate}[label=(\alph*)]
    \item $(1+a+a^2)^2<3(1+a^2+a^4)\,,\qq{für} a\in\mathbb{R}\!\smallsetminus\!\{1\}$
    \item $x^4-x^2-6x+10>0\,,\qq{für} x\in\mathbb{R}$
\end{enumerate}