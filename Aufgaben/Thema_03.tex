\Titelbanner{3}{Quadratische Gleichungen und Gleichungssysteme}

\paragraph{Aufgabe 1: } \emph{Quadratische Gleichungen}\\[0.2cm]
Lösen Sie die folgenden Gleichungen jeweils für $x$ durch quadratische Ergänzung und kontrollieren Sie das Ergebnis mit der $pq$-Formel.
\begin{enumerate}[label=(\alph*)]
    \item $x^2-10x+9=0$
    \item $x^2+x-12=0$
    \item $x^2-\sqrt{8}x+1=0$
\end{enumerate}

\paragraph{Aufgabe 2: } \emph{Wurzeln quadratischer Gleichungen}
\begin{enumerate}[label=(\alph*)]
    \item Stellen Sie $\textstyle\frac{a}{b}-\frac{b}{a}$ als Produkt zweier Faktoren dar, deren Summe gleich $\textstyle\frac{a}{b}+\frac{b}{a}$ ist.
    \item Bestimmen Sie in der Gleichung $5x^2-kx+1=0$ den Koeffizienten $k$ so, dass die Differenz der Wurzeln 1 ergibt.
    \item Wählen Sie die Koeffizienten der quadratischen Gleichung $x^2+px+q=0$ so, dass die Wurzeln der Gleichung gleich $p$ und $q$ sind.
    \item Gegeben ist die quadratische Gleichung $ax^2+bx+c=0$. Gesucht ist diejenige neue quadratische Gleichung, deren Wurzeln gleich
    \begin{itemize}[labelindent=1em,labelsep=0.5cm]
        \item dem Doppelten der Wurzeln der gegebenen Gleichung,
        \item den reziproken Werten der Wurzeln der gegebenen Gleichung
    \end{itemize}
    sind.
\end{enumerate}

\paragraph{Aufgabe 3: } \emph{Gleichungssysteme}\\[0.2cm]
Lösen Sie die folgenden Gleichungssysteme jeweils für $x$ und $y$.\\[0.2cm]
\begin{minipage}[t]{0.25\linewidth}
(a)\vspace{-2.6em}
\begin{align*}
x+y^2&=7\,,\\ xy^2&=12
\end{align*}
\end{minipage}\hspace{0.05\linewidth}
\begin{minipage}[t]{0.3\linewidth}
(b)\vspace{-2.6em}
\begin{align*}
x+xy+y&=11\,,\\ x^2y+xy^2&=30
\end{align*}
\end{minipage}\hspace{0.05\linewidth}
\begin{minipage}[t]{0.3\linewidth}
    (c)\vspace{-2.8em}
\begin{align*}
x^2+y^2&=\frac{5}{2}xy\,,\\ x-y&=\frac{1}{4}xy
\end{align*}
\end{minipage}\\
\emph{Hinweis:} Substitution $z_1=xy$, $z_2=x+y$ in Aufgabe (b).

\paragraph{Aufgabe 4: } \emph{Wurzelgleichungen}\\[0.2cm]
Lösen Sie die folgenden Gleichungen jeweils für $x$.
\begin{enumerate}[label=(\alph*)]
    \item $\sqrt{3x+1}-\sqrt{x-1}=2$
    \item $\sqrt{x+a}=a-\sqrt{x}$
    \item $\sqrt{a^2-x}+\sqrt{b^2-x}=a+b$
    \item $\sqrt{x+1+\sqrt{3x+4}}=3$
    \item $\sqrt{2x-1}+\sqrt{x-\num{1,5}}=\dfrac{6}{\sqrt{2x-1}}$
\end{enumerate}
%
\paragraph{Aufgabe 5: } \emph{Nullstellensuche}\\[0.2cm]
Die folgenden Terme sind als Produkte von Linearfaktoren darzustellen.
\begin{enumerate}[label=(\alph*)]
    \item $x^2+2x-15$
    \item $4x^2+8x-5$
    \item $ax^3+bx^2+adx^2+bdx$
    \item $(a-x)^2+(x-b)^2-a^2-b^2$
    \item $a\sqrt{8}x^2-2kax-3ak+ax\sqrt{18}$
\end{enumerate}