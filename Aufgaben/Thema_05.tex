\Titelbanner{5}{Exponentialfunktionen\\
                Logarithmen\\
                Natürliche Exponentialfunktion}

\paragraph{Aufgabe 1: } \emph{Logarithmische und Exponentialgleichungen}\\[0.2cm]
Lösen Sie jeweils für $x$ bzw. $y$. Beachten Sie mögliche Fallunterscheidungen bezüglich der auftretenden Konstanten.\\[-1.3em]
\begin{enumerate}[label=(\alph*)]
    \item $a\,2^x=\e^{bx}$
    \item $x=49^{1-\log_7(2\sqrt{x})}+5^{-\log_5(4x)}$
    \item $\e^{3ax}-2^{a+1}\e^{2ax}+4^a\e^{ax}=\log_b(1)$
    \item $\sqrt[x^2-1]{a^3}\cdot\sqrt[2x-2]{a}\cdot\sqrt[4]{a^{-1}}=1$
    \item $\log_ax+\log_ay=2\,, \quad\log_bx-\log_by=4$
    \item $3\log_{xa^2}x+\frac{1}{2}\log_\frac{x}{\sqrt{a}}x=2$
\end{enumerate}

\paragraph{Aufgabe 2: } \emph{Verdopplungszeit}\\[0.2cm]
Der Wissenszuwachs eines Physik-Studenten mit Anfangswissen $A$ sei beschrieben durch
\begin{align*}
    W(t)=A\e^{c t}\,, \hspace{1.5cm} c>0\,,
\end{align*}
sodass $W(t)$ die "`Menge"' an Wissen zur Zeit $t$ gibt.
\begin{enumerate}[label=(\alph*)]
    \item Nach welcher Zeit $\tau_2$ hat das Wissen eines Studenten auf das Doppelte zugenommen?
    \item Nach welcher Zeit $\tau_n$ hat das Wissen eines Studenten auf das $n$-Fache zugenommen?
    \item Drücken Sie $\tau_n$ als Vielfaches von $\tau_2$ aus. Welcher Zusammenhang besteht in den Fällen $n=3$ und $n=4$? Welche Aussage können Sie für $n=2^m$, $m\in\mathbb{N}$, treffen?
\end{enumerate}

\paragraph{Aufgabe 3: } \emph{Hyperbelfunktionen}\\[0.2cm]
Es seien die Funktionen
\begin{align*}
    f(x)&:=\e^{x}+\e^{-x}\,,\\
    g(x)&:=\e^{x}-\e^{-x}
\end{align*}
definiert.
\begin{enumerate}[label=(\alph*)]
    \item Finden Sie jeweils einen Ausdruck für $f(2x)$ und $g(2x)$ in Abhängigkeit der Funktionen einfacher Argumente, $f(x)$ und $g(x)$.
    \item Finden Sie jeweils einen Ausdruck für $f(x+y)$ und $g(x+y)$ in Abhängigkeit von $f(x)$ und $f(y)$ sowie $g(x)$ und $g(y)$. Vergleichen Sie mit dem Ergebnis aus (a), indem Sie $x=y$ setzen.
    \item Schreiben Sie die Funktionen $f(x)$ und $g(x)$ in Reihendarstellung.
    \item Bestimmen Sie die Umkehrfunktion $g^{-1}$ von $g(x)$.
    % \item* Es sei eine dritte Funktion definiert als $h(x):=f(x)/g(x)$. Überzeugen Sie sich davon, dass diese Funktion mindestens einen \emph{Fixpunkt} besitzt, d.h. dass mindestens eine Zahl $u\in\mathbb{R}$ existiert, für die $h(u)=u$.
\end{enumerate}