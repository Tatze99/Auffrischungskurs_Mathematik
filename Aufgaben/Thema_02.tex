\Titelbanner{2}{Lineare Gleichungssysteme}

\paragraph{Aufgabe 1: } \emph{Zwei lineare Gleichungen mit zwei Unbekannten}\\[0.2cm]
Lösen Sie die folgenden Gleichungen jeweils für $x$ und $y$.
\begin{enumerate}[label=(\alph*)]
    \item $33x+12y=25\,,$ \tab $11x-3y=6$
    \item $\dfrac{2x+3y}{3x-y}=\dfrac{17}{9}\,, $ \tab $ \dfrac{3x+4y}{6x-1}=2$
    \item $\frac{x+2}{y+3}=\frac{1}{3}\,, $ \tab $ \frac{y+3}{2y-5x}=\frac{3}{5}$
    \item $ax+by=2a\,, $ \tab $ \frac{x}{b}-\frac{y}{a}=\frac{2}{a}$
    \item $x+14y=\frac{1}{\sqrt{2}}-7\sqrt{2}\,, $ \tab $ 3\sqrt{2}x-\frac{y}{\sqrt{3}}=3+\frac{1}{\sqrt{6}}$
    \item $\frac{x}{a+b}+\frac{y}{a-b}=a+b\,, $ \tab $ \frac{x}{a}-\frac{y}{b}=2b$
    \item $39x-38y=1\,, $ \tab $ 91x-57y=4$
\end{enumerate}

\paragraph{Aufgabe 2: } \emph{Drei Gleichungen mit drei Unbekannten}\\[0.2cm]
Lösen Sie die folgenden Gleichungen jeweils für $x$, $y$ und $z$. %\\[1.5em]
%
\begin{center}
\begin{minipage}[t]{0.3\linewidth}
(a)\vspace{-2.7em}
\begin{align*}
    x-y+5z&=5\,,\\
    3x+7y-5z&=5\,,\\
    x+y-z&=1
\end{align*}

(c)\vspace{-2.7em}
\begin{align*}
    x+y&=b+a\,,\\
    x+z&=a+c\,,\\
    y+z&=c+b
\end{align*}
\end{minipage} \hspace{1.5cm}
\begin{minipage}[t]{0.4\linewidth}
%
(b)\vspace{-2.7em}
\begin{align*}
    3x-4y+3z&=4\,,\\
    -x+y-z&=-2\,,\\
    7x+4y-5z&=0
\end{align*}
(d)\vspace{-2.7em}
\begin{align*}
    6x-4y+8z&=0\,, \\
    -2x+y-z&=0\,,\\
    12x-7y+11z&=0
\end{align*}
\end{minipage} \\[0.2cm]
\end{center}
\vspace{0.7cm}

\newpage
% 
\textbf{Aufgabe 3: } \emph{Parametrisierung von Lösungsmengen}\\[0.2cm]
Geben Sie die Lösungsmenge der Gleichung $13x-7y=1$ an für
\begin{align*}
&\text{(a) } \hspace{0.2cm} x,y\in\mathbb{R}\,;\\[0.2cm]
&\text{(b) } \hspace{0.2cm} x,y\in\mathbb{N}\,.
\end{align*}

\paragraph{Aufgabe 4: } \emph{Gleichungssysteme}\\[0.2cm]
Lösen Sie die folgenden Gleichungssysteme jeweils für $x$ und $y$.\\[0.5cm]
\begin{minipage}[t]{0.45\linewidth}
(a)\vspace{-2.7em}
\begin{align*}
    x^2+y^2&=2(xy+2)\,,\\ x+y&=6
\end{align*}
(c)\vspace{-2,7em}
\begin{align*}
    \frac{x^2-y^2}{2x+3}+y^2&=(x+y)x-xy\,,\\
    y-2x&=3
\end{align*}
\end{minipage}\hfill
\begin{minipage}[t]{0.4\linewidth}
(b)\vspace{-2.7em}
\begin{align*}
    \frac{12}{\sqrt{x-1}}+\frac{5}{\sqrt{y+\frac{1}{4}}}&=5\,,\\ \frac{8}{\sqrt{x-1}}+\frac{10}{\sqrt{y+\frac{1}{4}}}&=6
\end{align*}
\end{minipage}
%
\paragraph{Aufgabe 5: } \emph{Ungleichungssysteme}
\begin{enumerate}[label=(\alph*), labelindent=1em,labelsep=0.5cm]
\item Es ist die Lösungsmenge des folgenden Systems an Ungleichungen zu skizzieren. Welche der Ungleichungen können weggelassen werden, ohne dass sich die Lösungsmenge ändert?
\begin{align*}
    2x-3y&\geq -6\,,\\
    x-2y&< 11\,,\\
    x&>-y-1\,,\\
    x&<5\,,\\
    x&\geq 0\,,\\
    y&\geq 0
\end{align*}
\item \textbf{*} Welches Gebiet im ersten Oktanden ($x\ge 0$, $y\ge 0$, $z\ge 0$) wird durch die folgenden Ungleichungen definiert?
\begin{align*}
    x+y&\ge z\,,\\
    x+z&\ge y\,,\\
    y+z&\ge x
\end{align*}
\end{enumerate}