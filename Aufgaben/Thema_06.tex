\Titelbanner{6}{Trigonometrische Funktionen\\Ebene Trigonometrie}

\paragraph{Aufgabe 1: } \emph{Additionstheoreme}\\[-1em]
\begin{enumerate}[label=(\alph*)]
\item Leiten Sie das Additionstheorem für Kosinusfunktionen aus dem für Sinusfunktionen her.
% \begin{align*}
%     \text{Es gilt:} \quad \sin(x\pm y) = \sin(x)\cos(y) \pm \cos(x) \sin(y)
% \end{align*}
\item Leiten Sie das Additionstheorem für Tangensfunktionen her,
\begin{align*}
    \tan(x\pm y)=\frac{\tan(x)\pm \tan(y)}{1\mp \tan(x)\tan(y)}\,.
\end{align*}
\item Zeigen Sie, dass für Doppelwinkelfunktionen gilt:
\begin{itemize}
    \item $\sin(2x)=2\sin(x)\cos(x)\,,$
    \item $\cos(2x)=2\cos^2\qty(x)-1\,.$
\end{itemize}
\end{enumerate}

\paragraph{Aufgabe 2: } \emph{Trigonometrische Umformungen I}\hfill Ziel: (a) bis (c)\\[0.2cm]
Zeigen Sie die Richtigkeit der folgenden Identitäten.
\begin{enumerate}[label=(\alph*)]
    \item $\dfrac{\cos(\alpha)+\sin(\alpha)}{\cos(\alpha)-\sin(\alpha)}=\tan\qty(\frac{\pi}{4}+\alpha)$
    \item $\dfrac{1+\sin (2\alpha)}{\cos (2\alpha)}=\dfrac{1+\tan(\alpha)}{1-\tan(\alpha)}=\tan\qty(\frac{\pi}{4}+\alpha)$
    \item $2\cos\qty(\frac{x+y}{2})\cos\qty(\frac{x-y}{2})=\cos(x)+\cos(y)$
    \item $\cot(\alpha)\cot(\beta)+\cot(\alpha)\cot(\gamma)+\cot(\beta)\cot(\gamma)=1 \text{ für } \alpha+\beta+\gamma=\pi$
\end{enumerate}

\paragraph{Aufgabe 3: } \emph{Trigonometrische Umformungen II}\hfill Ziel: (a) und (b)\\[0.2cm]
Formen Sie die folgenden Ausdrücke so um, dass sie sich einfach logarithmieren lassen. Das heißt, die Terme sollen möglichst in Produkte, Quotienten und Potenzen umgeformt werden.\\[-1.3em]

\begin{enumerate}[label=(\alph*)]
    \item $1+\cos(\alpha) + \cos\qty(\frac{\alpha}{2}), \qq{\emph{Hinweis:} Es ist} \cos\qty(\frac{\pi}{3})=\dfrac{1}{2}\,.$
    \item $\frac{2\sin(\beta) - \sin (2\beta)}{2\sin(\beta)+2\sin(2\beta)}$
    \item $\sin(\alpha)+\sin(\beta)+\sin(\gamma), \qq{für} \alpha+\beta+\gamma = \pi$
\end{enumerate}
%
\paragraph{Aufgabe 4: } \emph{Goniometrische Gleichungen und Gleichungssysteme}\hfill Ziel: (a) bis (c)\\[0.2cm]
Bestimmen Sie alle Lösungen der folgenden Gleichungen. Es reicht aus, die Lösungen in impliziter Form stehenzulassen!
\begin{enumerate}[label=(\alph*)]
    \item $\sin(x)+\cos(x)= 1$
    \item $\cos\qty(\frac{x+y}{2})\cos\qty(\frac{x-y}{2})=\frac{1}{2}\,, \hspace{0.5cm}\cos(x)\cos(y)=\frac{1}{4}$
    \item $\sin(3x)=\cos(2x)$
    \item $a\qty(3\cos^2(x)+\sin(x)\cos(x))-b\qty(3\sin^2(x)-\sin(x)\cos(x))=2a-b$
\end{enumerate}
\emph{Hinweis:} Versuchen Sie in (c) und (d) die Gleichung so umzuformen, dass nur noch eine Funktionsart auftritt. Dann reduziert sich das Problem zu einer Nullstellensuche eines Polynoms.

\paragraph{Aufgabe 5: } \emph{Dreiecksfläche}\\[0.2cm]
Berechnen Sie die Fläche eines Dreiecks, wenn die Seiten $a$ und $b$ sowie die Länge $w$ der Winkelhalbierenden des Winkels zwischen diesen Seiten gegeben sind.
% %
% \paragraph{Aufgabe 6: } \emph{Sehnen im Kreis}\\[0.2cm]
% Durch einen Punkt auf einem Kreis vom Radius $r$ seien zwei Sehnen der Längen $a$ und $b$ gelegt. Wenn man die Schnittpunkte der Sehnen mit der Peripherie untereinander geradlinig verbindet, erhält man ein Dreieck. Bestimmen Sie dessen Flächeninhalt $A$.
