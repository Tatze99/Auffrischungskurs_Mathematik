\Titelbanner{7}{Grundlagen der Differentialrechnung\\Kurvendiskussion}

\paragraph{Aufgabe 1: } \emph{Ableitungen I} \hfill Ziel: (a) bis (e)\\[0.2cm]
Berechnen Sie die Ableitungen der folgenden Funktionen.
\begin{enumerate}[label=(\alph*)]
    \item $Q(r)=\frac{3r^2}{2}\qty(\frac{1}{2}+\ln\frac{r}{r_0})$
    \item $f(x)=\cos^4(3tx)-\sin^4(3tx)$
    \item $S(\tau)=(\tau-1)\e^{\tau}+\frac{\tau^2}{4}\qty(2\ln\tau-1)$
    \item $y(x)=\frac{\exp(2x)}{25}\qty[(5x-4)\sin x+(10x-3)\cos x]$
    \item $F(x)=-\frac{k}{\sqrt{(x-x_0)^2+(y-y_0)^2}}$
    \item $N(z)=\frac{2\cos\qty(\frac{z}{2})}{\sqrt{1+\cos(z)}}\qty[\ln\qty(\cos\frac{z}{4}+\sin\frac{z}{4})-\ln\qty(\cos\frac{z}{4}-\sin\frac{z}{4})]$
\end{enumerate}
%
\paragraph{Aufgabe 2: } \emph{Ableitungen II} \hfill Ziel: (a) bis (d)\\[0.2cm]
Finden Sie die $n$-te Ableitung der folgenden Funktionen.
\begin{multicols}{2}
    \begin{enumerate}[label=(\alph*)]
        \item $f(x)=x^n$
        \item $f(x)=\e^{kx}+\e^{-kx}$
        \item $f(x)=x^{n-1}$
        \item $f(x)=a^x$ 
        \item $f(x)=\frac{1}{1-x}$
        \item $f(x)=\frac{x}{1-x-x^2}$ 
    \end{enumerate} 
\end{multicols}
%
\paragraph{Aufgabe 3: } \emph{Kurvendiskussion I}\\[0.2cm]
Ein zweiatomiges Molekül lässt sich näherungsweise durch das sogenannte ``Morse-Potential'' beschreiben,
\begin{align*}
U(x)=D\qty(\e^{-2\alpha x}-2\e^{-\alpha x})\,, \hspace{1cm} D,\alpha=\operatorname{const}.
\end{align*}
\begin{itemize}
\item Bestimmen Sie Nullstellen und lokale Extrema der Funktion $U(x)$ sowie deren Verhalten für $x \to\pm\infty$.
\item Skizzieren Sie die Funktion $U(x)$ für $D=\alpha=1$ im Intervall $x\in[-1,5]$.
\end{itemize}
%
\paragraph{Aufgabe 4: } \emph{Kurvendiskussion II}\\[0.2cm]
Die Bewegung eines Teilchens mit Drehimpuls $L$ und Energie $E$ in der gekrümmten Raumzeit eines Schwarzen Loches der Masse $m$ wird beschrieben durch das Potential
\begin{align*}
U(r)=\frac{E}{2}-\frac{Em}{r}+\frac{L^2}{2r^2}-\frac{mL^2}{r^3}\,, \hspace{1cm} r>0\,.
\end{align*}
\begin{itemize}
\item Bestimmen Sie Nullstellen und lokale Extrema der Funktion $U(r)$ sowie deren Verhalten für $\to\infty$ und $\to 0$.
\item Setzen Sie $\textstyle m=\frac{1}{2}$. Welche Bedingungen an $E$ und $L$ müssen erfüllt sein, damit $U(r)$ zwei, ein oder keine lokalen Extrema besitzt?
\item Skizzieren Sie die Funktion $U(r)$ für $E=1$ und $L=2$ (nicht maßstabsgerecht).
\end{itemize} 
%
\paragraph{Aufgabe 5: } \emph{Gewöhnliche Differentialgleichungen}\hfill (Zusatzaufgabe)\\[0.2cm]
\begin{enumerate}[label=(\alph*)]
\item Finden Sie eine Funktion $f(x)$, welche die folgende Gleichung erfüllt:
\begin{align*}
f''(x)=a^2f(x)+bx\,.
\end{align*}
\item Finden Sie eine Funktion $f(x)$, welche die folgende Gleichung erfüllt:
\begin{align*}
f'(x)=\qty(1+\ln(x))f(x)\,.
\end{align*}
\end{enumerate}