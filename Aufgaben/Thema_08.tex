\Titelbanner{8}{Die Methode der vollständigen Induktion}

\paragraph{Aufgabe 1: } \emph{Rekursive und explizite Zuordnungsvorschrift}\\[0.2cm]
Eine Reihe $(a_n)_{n\in\mathbb{N}_0}$ sei rekursiv definiert durch
\begin{align*}
a_{n+1}=2a_n+1\,, && a_0=0\,.
\end{align*}
Finden Sie einen expliziten Ausdruck für $a_n$ und beweisen Sie ihn per vollständiger Induktion.
%
%
\paragraph{Aufgabe 2: } \emph{Vollständige Induktion I}\\[0.6cm]
Beweisen Sie\\[-1.5cm]
\begin{align*}
S_n&=1^2-2^2+3^2-4^2+\hdots+(-1)^{n-1}n^2=(-1)^{n-1}\,\dfrac{n(n+1)}{2}\,.
\end{align*}
%
\paragraph{Aufgabe 3: } \emph{Vollständige Induktion II}\\[0.2cm]
Beweisen Sie, dass die Summe der Kuben der ersten $n$ natürlichen Zahlen gleich $\left[\frac{n(n+1)}{2}\right]^2$ ist. 
%
\paragraph{Aufgabe 4: } \emph{Vollständige Induktion III}\\[0.2cm]
Zeigen Sie, dass die Summe der Kuben dreier aufeinanderfolgender natürlicher Zahlen durch $9$ teilbar ist.
%
\paragraph{Aufgabe 5: } \emph{Die Suche nach der richtigen Summenformel}\\[0.2cm]
Stellen Sie eine Summenformel für das Polynom
\begin{align*}
S_n = 1-\dfrac{x}{1!}+\dfrac{x(x-1)}{2!}-\hdots+(-1)^n \dfrac{x(x-1)\dots(x-n+1)}{n!}
\end{align*}
auf und beweisen Sie deren Richtigkeit durch vollständige Induktion.
%
\paragraph{Aufgabe 6: } \emph{Fibonacci-Zahlen}\hfill (Zusatzaufgabe)\\[0.2cm]
Die Folge der Fibonacci-Zahlen ist definiert durch $a_{n+1}=a_n+a_{n-1}$ für $n\ge 1$ und die Startwerte $a_0=0$ und $a_1=1$. Beweisen Sie, dass der explizite Ausdruck für die $n$-te Fibonacci-Zahl gegeben ist durch
\begin{align*}
a_n=\frac{1}{\sqrt{5}}\left(x_+^n-x_-^n\right)\,, \qq{mit} x_{\pm}=\frac{1\pm\sqrt{5}}{2}\,.
\end{align*}