\Titelbanner{10}{Rechnen mit Vektoren und Matrizen}

\paragraph{Aufgabe 1: } \emph{Linear (un)abhängige Vektoren}
\begin{enumerate}[label=(\alph*)]
\item Sind die folgenden Vektoren linear unabhängig?
\begin{align*}
\mqty( 3\\1\\-5 ),\quad \mqty( 1\\2\\0 ),\quad \mqty(-2\\0\\1)
\end{align*}
\item Für welche Werte von $t$ befinden sich die folgenden Vektoren in einer Ebene?
\begin{align*}
\mqty( 1\\4\\0 ),\quad \mqty( 3\\t\\1 ),\quad \mqty( 4\\4\\2 )
\end{align*}
\end{enumerate}
%
\paragraph{Aufgabe 2: } \emph{Skalarprodukt und Vektorprodukt}
\begin{enumerate}[label=(\alph*)]
\item Berechnen Sie die folgenden Skalarprodukte.
\begin{align*}
\mqty(1\\5\\7)\cdot\mqty(-2\\8\\3), &&
\mqty(1\\0\\ \sqrt{2})\cdot\mqty(0\\-1\\ \sqrt{2}), &&
\mqty(a\\3\\-1)\cdot\mqty(5\\2\\5a+6)
\end{align*}
\item Berechnen Sie die folgenden Vektorprodukte.
\begin{align*}
\mqty(1\\0\\0 )\times\mqty(0\\1\\0),&&
\mqty(\cos(\phi)\\ \sin(\phi) \\0)\times\mqty(-\sin(\phi)\\ \cos(\phi)\\0),&&
\mqty(1\\3\\-2)\times\mqty(1\\1\\5)
\end{align*}
\item Nutzen Sie das Skalarprodukt, um den Kosinussatz im allgemeinen Dreieck herzuleiten.
\item Nutzen Sie das Vektorprodukt, um den Sinussatz im allgemeinen Dreieck herzuleiten.
\end{enumerate}
%
\paragraph{Aufgabe 3: } \emph{Matrizen}
\begin{enumerate}[label=(\alph*)]
\item Bilden Sie -- falls möglich -- die Inversen der folgenden Matrizen mit Hilfe des Gauß-Jordan-Algorithmus.
\begin{align*}
 A =\mqty(1 & 2\\ 3 & 4),\quad B=\mqty(0 & 2\\ 3 & 0),\quad  C=\mqty(-2 & 4\\ 3 & -6)
\end{align*}
\end{enumerate}
\newpage
\begin{enumerate}[label=(\alph*),resume]
\item Matrixmultiplikation ist im Allgemeinen nicht kommutativ. Berechnen Sie $ A \cdot B- B\cdot A$.
\item Welche Bedingungen müssen die Einträge $m_i$ einer Matrix
\begin{align*}
M = \mqty(m_1 & m_2\\ m_3 & m_4)
\end{align*}
erfüllen, damit $| M\cdot \vec{v}|=|\vec{v}|$ für einen beliebigen zweikomponentigen Vektor $\vec{v}$ gilt? 
\end{enumerate}
%
%\paragraph{Aufgabe 4: } \emph{Drehungen}\\[0.2cm]
%Gegeben seien eine Matrix $\hat{R}(\phi)$ und ein Vektor $\vec{\alpha}$ als
%\begin{align*}
%\hat{R}=\begin{pmatrix*}[c]\cos(\phi)	& -\sin(\phi)	& 0\\ \sin(\phi) & \cos(\phi) & 0\\ 0 & 0 & 1\end{pmatrix*}, \quad \vec{\alpha}=\mqty(x\\y\\z).
%\end{align*}
%Berechnen Sie $\hat{R}(\phi)\cdot\vec{\alpha}$ und $\det\hat{R}(\phi)$. Welche geometrische Bedeutung hat der Fall $\phi=\frac{\pi}{2}$, $x=1$, $y=z=0$?\\[1cm]
%
\paragraph{Aufgabe 4: } \emph{Laplace'scher Entwicklungssatz}\\[0.2cm]
Berechnen Sie die Determinante der folgenden Matrix. Für welche Werte von $a$ ist die Matrix nicht invertierbar?
\begin{align*}
A=\mqty(a &3 &1 &a\\ 0 &-2 &0 &1\\ 3 &2 &-4 &-1\\ -a& 0 &-2 &0)
\end{align*}
%
\paragraph{Aufgabe 5: } \emph{Vektorprodukt}\\[0.2cm]
Das Vektorprodukt $\vec{\omega}\times \vec{u}$ zweier Vektoren
$\quad \vec{\omega}=\mqty(\omega_1\\\omega_2\\\omega_3),\quad\vec{u}=\mqty(u_1\\u_2\\u_3)$

lässt sich auch als Matrixmultiplikation $\Omega \cdot \vec{u}$ mit einer ($3\times 3$)-Matrix $\Omega$ mit den Einträgen $\pm \omega_i$, $i\in\{1,2,3\}$ schreiben. Wie sieht  $\Omega$ aus? Welche Eigenschaften hat es (Determinante, Spur, Symmetrie)?
%
%
\paragraph{Aufgabe 6*: } \emph{Höhere Dimensionen}\\[0.2cm]
Im dreidimensionalen euklidischen Raum $\mathbb{R}^3$ kann jede beliebige Drehung in Drehungen um die $x$-, die $y$- und die $z$-Achse zerlegt werden, d.h. man benötigt genau drei linear unabhängige Drehmatrizen für eine solche Zerlegung. Wie viele linear unabhängige Drehmatrizen benötigt man im $d$-dimensionalen euklidischen Raum $\mathbb{R}^d$\,?
%
%
\paragraph{Aufgabe 7*: } \emph{Diskreter Laplace-Operator}\\[0.2cm]
Gegeben sei die ($n\times n$)-Matrix $\Delta_n$, welche die Einträge -2 auf der Haupt- und 1 auf den beiden Nebendiagonalen hat,
\begin{align*}
\Delta_n=\mqty(
-2 & 1 & 0 & 0 & 0 &\dots & 0\\
1 & -2 & 1 & 0& 0 &\dots & 0\\
0 & 1 & -2 & 1 & 0 &\dots & 0\\
\vdots &\vdots &\vdots &\vdots &\vdots & &\vdots\\
0 & 0 & 0 & 0 & 0 &\dots & -2
)\,.
\end{align*}
Bestimmen Sie die Determinante von $\Delta_n$\,, indem Sie wie folgt vorgehen:
\begin{enumerate}
\item Finden Sie eine Rekursionsrelation, die $\det(\Delta_n)$ auf $\det(\Delta_{n-1})$ und $\det(\Delta_{n-2})$ zurückführt.
\item Raten Sie einen allgemeinen Ausdruck für $\det(\Delta_{n})$ und beweisen Sie ihn mittels vollständiger Induktion. Alternativ kann der Versuch unternommen werden, das Verfahren aus Thema 9, Aufgabe 5 anzuwenden.
\end{enumerate}