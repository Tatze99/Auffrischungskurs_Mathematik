% \usepackage[english]{babel}         % Deutsches Sprachpaket

\usepackage{iftex}
\ifPDFTeX
   \usepackage[utf8]{inputenc}% Eingaben codieren
   \usepackage{fourier}
   % \usepackage[T1]{fontenc}   % Umlaute codieren, Silbentrennung
\else
   \usepackage{lmodern}
   % \usepackage[no-math]{fontspec}
   % \setmainfont{Utopia Std}
\fi

\usepackage{amsmath, amssymb}       % Mathe, \mathbb{R}
\usepackage{amsthm,amstext}         % Theoreme, \text im Mathe-Modus
\usepackage{mathtools}              % \Aboxed für Boxen in Align Umgebungen
\usepackage[arrowdel]{physics}      % Ableitungen \dv{B}{t} \pdv \dd{t}
\usepackage[left=2.5cm, right=2.5cm, top=3cm, bottom=3cm]{geometry}
\usepackage{graphicx}               % \includegraphics
\usepackage[extendedchars]{grffile} % extends file name processing of graphics
\usepackage[section]{placeins}      % \Floatbarrier
\usepackage{wrapfig}                % Bilder umfließen
\usepackage{enumerate}              % Aufzählungen
\usepackage{enumitem}               % \begin{enumerate}[label=\alph*]
\usepackage{footnote}               % Fußzeilen
\usepackage{booktabs}               % publication quality tables
\usepackage{tikz, pgfplots}         % TIKZ ist kein Zeichenprogramm
\usepackage[europeanvoltages,europeanresistors]{circuitikz}
\usepackage{bm}                     % bold symbols \bm{r}
\usepackage{dsfont}                 % identity matrix \mathds{1}
\usepackage{esint}                  % Doppelintegrale
\usepackage{mathrsfs}               % \mathscr{} statt \mathcal{}
\usepackage{placeins}               % FloatBarrier
\usepackage{subcaption}
\usepackage{multirow}
\usepackage{mdframed}               % Deckblatt
\usepackage{xcolor}                 % Deckblatt
\usepackage{fancyhdr}               % Kopfzeile
\usepackage{aligned-overset}        % Ausrichtungen mit stackrel oder overset
\usepackage{cancel}
\usepackage{float}
\usepackage{cite}
\usepackage[version=4]{mhchem}                 % Chemistry Package
\usepackage{multicol}
% \usepackage{pdfpages}             % insert whole pdf files

\definecolor{FSUblau}{cmyk}{1,0.7,0.1,0.5}
\definecolor{PAForange}{cmyk}{0.1,0.7,1,0}
\definecolor{Gruen}{cmyk}{1,0.1,0.7,0.5}

\usepackage[final,
    pdfauthor={Martin Beyer},
    pdffitwindow=false,     % resize document window to fit document size
    pdftoolbar=false,        % Adobe Toolbar
    bookmarks=true,         % Anzeigen der Kapitel
    bookmarksopen=true,
    bookmarksopenlevel=0,
    bookmarksnumbered=true,
    colorlinks=true,        % fuer Druckversion auf "false"
    linkcolor=FSUblau,         % Table of Contents, Footnotes
    urlcolor=FSUblau,          % fuer eingebunden URLs
    citecolor=FSUblau,         % Equations, References
    filecolor=FSUblau,
    pdfborder={0 0 0},      % keine Rahmen um Verlinkungen: {0 0 0}
    pagebackref=false
]{hyperref}

\pgfplotsset{compat=1.18}
\newcommand\mydots{\makebox[1em][c]{.\hfil.\hfil.}}
\newcommand{\minus}{\scalebox{0.75}[1.0]{$-$}}
\newcommand{\e}{\mathrm{e}}
\renewcommand{\i}{\mathrm{i}}

\usepackage[detect-all,
            locale=DE,
            exponent-product = \cdot,
            per-mode=fraction]{siunitx}
\usepackage[position=below,
            format=hang,
            figurename=Fig.,
            labelfont={bf},
            font=small]{caption}

\sisetup{range-phrase = {\mydots}}
% Commands
\usetikzlibrary{positioning,intersections,calc,external}
\usepgfplotslibrary{fillbetween, groupplots}
\pgfplotsset{
tick label style={font=\small},
label style={font=\small},
legend style={font=\footnotesize},
every axis post/.style={legend cell align={left}}}
\tikzstyle{every node}=[font=\small]


\setlength{\parindent}{0px}         % keine Absätze durch Leerzeilen im Code
\numberwithin{equation}{section}

% Remove page number from \thispagestyle{empty}
\makeatletter\let\ps@plain\ps@fancy\makeatother

% Deckblatt
\newcommand{\HRule}{\rule{\linewidth}{0.5mm}}
\newcommand{\Deckblatt}[5][\LaTeX-Satz und Design von Martin Beyer]{
  \begin{titlepage}
    \center
    \textsc{\LARGE Friedrich-Schiller-Universität Jena\\[1ex]
    \Large Physikalisch-Astronomische-Fakultät}
    \begin{figure}[h!]
       \centering
       \includegraphics[scale=0.75]{uni-Logo_neu.pdf}
    \end{figure}\\
    \vspace{2em}
    \textsc{\Large #2}\\[0.35cm]
    \HRule \\[0.4cm]
    { \Huge \bfseries #3}\\[0.15cm]
    \HRule \\[0.5cm]
    \textsc{\Large #4}\\[0.35cm]
    \vfill
    \begin{mdframed}[backgroundcolor=gray!20]
      \begin{center}
        #1
      \end{center}
    \end{mdframed}
  \end{titlepage}

  \pagestyle{fancy}
  \fancyhead[R]{\textbf{#5}}
  \fancyfoot[C]{\bfseries\thepage}
  \fancyhead[L]{\rightmark}

  \fancypagestyle{plain}{
    \fancyfoot[C]{\bfseries\thepage}
    \fancyhead[R]{}
    \fancyhead[L]{}
    \renewcommand{\headrulewidth}{0pt}
  }
}
\renewcommand{\sectionmark}[1]{\markright{#1}}
\renewcommand{\headrulewidth}{0.5pt}
\renewcommand{\footrulewidth}{0.5pt}
