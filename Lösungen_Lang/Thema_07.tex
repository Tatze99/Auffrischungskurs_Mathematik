\Titelbanner{7}{Grundlagen der Differentialrechnung\\Kurvendiskussion}

\paragraph{Wiederholung Ableitungsregeln}$~$
\begin{itemize}
    \item Linearität: $\dv{x}(a f(x) + b g(x)) = a \dv{f}{x} + b \dv{g}{x}$
    \item Produktregel (Leibniz-Regel): $\dv{x}(f(x)\cdot g(x)) = \dv{f}{x} \cdot g(x) + f(x) \cdot \dv{g}{x}$
    \item Kettenregel: $\dv{x} f(g(x)) = \qty(\dv{f}{g})(g(x))\cdot \dv{g}{x}.$
    % \item Quotientenregel: $\dv{x}\qty(\frac{f(x)}{g(x)}) = \frac{1}{g(x)^2} \qty(\dv{f}{x}\cdot g(x) - f(x) \cdot \dv{g}{x}).$
    \item Potenzregel: $\dv{x}\qty(x^n) = n \, x^{n-1}.$
\end{itemize}
\emph{Ableitungen spezieller Funktionen: }
\begin{alignat}{3}
    \dv{x}\sin(x) &= +\cos(x) &\qquad\quad \dv{x}\exp(x) &= \exp(x)\\
    \dv{x}\cos(x) &= -\sin(x) & \dv{x}a^x &= \ln(a) a^x.
\end{alignat}

\emph{Beispiel:}\\[-1.5cm]
\begin{align}
    f(x) &= x^2 \ln\big(\underbrace{3x^2-5x+2}_{g(x)}\big) \\
    \quad\Rightarrow f'(x) &= 2x \ln(g(x)) + x^2 \frac{1}{g(x)} \dv{g(x)}{x}= 2x \ln(3x^2-5x+2) + \frac{x^2(6x-5)}{3x^2 - 5x+2}.
\end{align}

\paragraph{Aufgabe 1: } \emph{Ableitungen I} \hfill Ziel: (a) bis (e)\\[0.2cm]
\emph{Berechnen Sie die Ableitungen der folgenden Funktionen.}
\begin{enumerate}[label=(\alph*)]
    \setlength{\mathindent}{0cm}
    \item $Q(r)=\frac{3r^2}{2}\qty(\frac{1}{2}+\ln\frac{r}{r_0})$
    \begin{align}
        \dv{Q(r)}{r} &= 3r\qty(\frac{1}{2}+\ln(\frac{r}{r_0})) + \frac{3r^2}{2}\frac{r_0}{r}\frac{1}{r_0} = \uuline{3r\qty(1+\ln \frac{r}{r_0})}.
    \end{align}
    \item $f(x)=\cos^4(3tx)-\sin^4(3tx) = \underbrace{(\cos^2(3tx)+\sin^2(3tx))}_{1}(\cos^2(3tx)-\sin^2(3tx))$ 
    \begin{align}
        \dv{f(x)}{x} = -6t \sin(3tx)\cos(3tx) - 6t\sin(3tx)\cos(3tx) = \uuline{-12t \sin(3tx)\cos(3tx)}.
    \end{align}
    \item $S(\tau)=(\tau-1)\e^{\tau}+\frac{\tau^2}{4}\qty(2\ln\tau-1)$
    \begin{align}
        \dv{S(\tau)}{\tau} = \cancel{\e^{\tau}} + (\tau-\cancel{1})\e^\tau + \frac{\tau}{2}\qty(2\ln(\tau)-\cancel{1}) + \frac{\tau^{\cancel{2}}}{4} \frac{2}{\cancel{\tau}} = \uuline{\tau(\e^{\tau} + \ln(\tau))}.
    \end{align}
    \item $y(x)=\frac{\exp(2x)}{25}\qty[(5x-4)\sin x+(10x-3)\cos x]$
    \begin{align}
        \dv{y(x)}{x} &= \frac{2}{25} \e^{2x} \bigg[\hdots \bigg] + \frac{1}{25}\e^{2x} \qty[5 \sin(x) + (5x-4)\cos(x) + 10\cos(x)-(10x-3)\sin(x)] \\
        &= \frac{\e^{2x}}{25}\qty[\cancel{(10x-8)\sin(x)}+(20x-\bcancel{6})\cos(x) + (5x+\bcancel{6})\cos(x) - \cancel{(10x-8)\sin(x)}] \notag \\
        &= \uuline{x \cos(x) \e^{2x}}
    \end{align}
    \item $F(x)=-\frac{k}{\sqrt{(x-x_0)^2+(y-y_0)^2}}$
    \begin{align}
        \dv{F(x)}{x} &= \frac{k}{\cancel{2}} \frac{\cancel{2}(x-x_0)}{\sqrt{\hdots}^3} = \uuline{\frac{k(x-x_0)}{\sqrt{(x-x_0)^2+(y-y_0)^2}^3}}
    \end{align}
    \item$~$\\[-1.5cm]
    \begin{align}
        N(z) &=\underbrace{\frac{2\cos\qty(\frac{z}{2})}{\sqrt{1+\cos(z)}}}_{\mathclap{1+\cos(z) = 2 \cos^2\qty(\frac{z}{2})}}\qty[\ln\qty(\cos\frac{z}{4}+\sin\frac{z}{4})-\ln\qty(\cos\frac{z}{4}-\sin\frac{z}{4})] \qquad \sigma_{\pm} = \cos(\frac{z}{4})\pm \sin(\frac{z}{4}) \\
        &= \sqrt{2}(\ln(\sigma_+)-\ln(\sigma_-)) \\
        \Rightarrow \dv{N(z)}{z} &= \sqrt{2} \qty(\frac{(\sigma^+)'}{\sigma^+} - \frac{(\sigma^-)'}{\sigma^-}) = \sqrt{2} \frac{(\sigma^+)' \sigma^- - (\sigma^-)'\sigma^+}{\sigma^+ \sigma^-} \qq{mit} (\sigma^\pm)' = \pm \frac{1}{4}\sigma^\mp \\
        &= \frac{1}{2\sqrt{2}} \frac{(\sigma^+)^2 + (\sigma^-)^2}{\sigma^+ \sigma^-} = \frac{1}{\cancel{2}\sqrt{2}} \frac{\cancel{2}}{\cos^2\qty(\frac{z}{4})-\sin^2\qty(\frac{z}{4})} \notag \\
        &= \frac{1}{\sqrt{2}} \frac{1}{2\cos^2\qty(\frac{z}{4}-1)} = \frac{1}{\sqrt{2}} \frac{1}{\cos(\frac{z}{2})} = \uuline{\frac{1}{\sqrt{1+\cos(z)}}}.  
    \end{align}
\end{enumerate}
%
\paragraph{Aufgabe 2: } \emph{Ableitungen II}\hfill Ziel: (a) bis (d)\\[0.2cm]
\emph{Finden Sie die $n$-te Ableitung der folgenden Funktionen.}
    \begin{enumerate}[label=(\alph*)]
        \item $f(x)=x^n$
        \item $f(x)=\e^{kx}+\e^{-kx}$
        \item $f(x)=x^{n-1}$
        \item $f(x)=a^x$ 
        \item $f(x)=\frac{1}{1-x}$
        \item[(f)*] $f(x)=\frac{x}{1-x-x^2}$ 
    \end{enumerate}
%
\paragraph{Aufgabe 3: } \emph{Kurvendiskussion I}\\[0.2cm]
\emph{Ein zweiatomiges Molekül lässt sich näherungsweise durch das sogenannte ``Morse-Potential'' beschreiben,}
\begin{align*}
U(x)=D\qty(\e^{-2\alpha x}-2\e^{-\alpha x})\,, \hspace{1cm} D,\alpha=\operatorname{const}.
\end{align*}
\begin{itemize}
\item Bestimmen Sie Nullstellen und lokale Extrema der Funktion $U(x)$ sowie deren Verhalten für $x \to\pm\infty$.
\item Skizzieren Sie die Funktion $U(x)$ für $D=\alpha=1$ im Intervall $x\in[-1,5]$.
\end{itemize}
%
\paragraph{Aufgabe 4: } \emph{Kurvendiskussion II}\\[0.2cm]
\emph{Die Bewegung eines Teilchens mit Drehimpuls $L$ und Energie $E$ in der gekrümmten Raumzeit eines Schwarzen Loches der Masse $m$ wird beschrieben durch das Potential}
\begin{align*}
U(r)=\frac{E}{2}-\frac{Em}{r}+\frac{L^2}{2r^2}-\frac{mL^2}{r^3}\,, \hspace{1cm} r>0\,.
\end{align*}
\begin{itemize}
\item \emph{Bestimmen Sie Nullstellen und lokale Extrema der Funktion $U(r)$ sowie deren Verhalten für $\to\infty$ und $\to 0$.}
\item \emph{Setzen Sie $\textstyle m=\frac{1}{2}$. Welche Bedingungen an $E$ und $L$ müssen erfüllt sein, damit $U(r)$ zwei, ein oder keine lokalen Extrema besitzt?}
\item \emph{Skizzieren Sie die Funktion $U(r)$ für $E=1$ und $L=2$ (nicht maßstabsgerecht).}
\end{itemize} 
%
\paragraph{Aufgabe 5: } \emph{Gewöhnliche Differentialgleichungen}\hfill (Zusatzaufgabe)\\[0.2cm]
\begin{enumerate}[label=(\alph*)]
\item \emph{Finden Sie eine Funktion $f(x)$, welche die folgende Gleichung erfüllt:}
\begin{align*}
f''(x)=a^2f(x)+bx\,.
\end{align*}
\item \emph{Finden Sie eine Funktion $f(x)$, welche die folgende Gleichung erfüllt:}
\begin{align*}
f'(x)=\qty(1+\ln(x))f(x)\,.
\end{align*}
\end{enumerate}