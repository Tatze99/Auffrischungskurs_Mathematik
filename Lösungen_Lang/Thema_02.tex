\Titelbanner{2}{Lineare Gleichungssysteme}

\paragraph{Tafelbeispiel:} $\displaystyle \frac{10(x+y) + 3}{x-2y+4} = 1, \quad \frac{36x-3y}{7(x-y)+3} = 3$
\begin{enumerate}
    \item Sortieren nach $x$ und $y$ 
    \begin{align*}
        9x +12 y &= 1 \quad (1) \\
        15x+18y &= 9 \quad (2)
    \end{align*}
    \item Prüfe auf lineare Unabhängigkeit, d.\,h. Berechne $a_1 b_2 - a_2 b_1$ (Determinante, vgl. Vorlesung): 
    \begin{align*}       
        9\cdot 18 - 15\cdot 12 = 162-180 \neq 0 \quad \Rightarrow \qq{genau ein Lösungspaar ($x,y$)}
    \end{align*}      
    \item Lösen, bspw. durch geschickte Linearkombination (Elimination von $y$)
    \begin{align*}
        2\cdot (2) - 3\cdot (1): \quad \hphantom{1}3x &= \hphantom{-}15 \quad \Rightarrow x=5 \\
        \qq{in} (1): \quad 12y &= -44 \quad \Rightarrow y = -\frac{11}{3} \\
        \Rightarrow \qq{Lösungsmenge} \mathbb{L} &= \uuline{\Bigg\{\qty(5;-\frac{11}{3})\Bigg\}}
    \end{align*}
\end{enumerate}

\paragraph{Aufgabe 1: } \emph{Zwei lineare Gleichungen mit zwei Unbekannten} \hfill Ziel (a) bis (d)\\[0.2cm]
\begin{enumerate}[label=(\alph*)]
    \item $~$\\[-1.45 cm]\begin{align*}
        &33x+12y=25 \quad (1) \\ 
        &11x-\hphantom{1}3y=\hphantom{2}6 \quad (2) \\
        &(1) - 3\cdot (2): \quad 21y = 7 \quad\Rightarrow \quad y = \frac{1}{3}, \qq{in (2):} x = \frac{7}{11} \qquad \Rightarrow \qquad \uuline{\mathbb{L} = \Bigg\{\qty(\frac{7}{11};\frac{1}{3})\Bigg\}}
    \end{align*}
    \item $~$\\[-1.45cm] 
    \begin{align*}
        &\begin{rcases}
            \frac{2x + 3y}{3x-y} = \frac{17}{9} \\
            \dfrac{3x+4y}{6x-1}=2
        \end{rcases} \quad \Longrightarrow \quad 
        \begin{array}{l}
            3x-4y = 0 \quad (1) \\
            9x -4y = 2 \quad (2)
        \end{array} \\
        &\hphantom{3\cdot}\;\,(1) -(2): \quad 6x = 2 \quad\Rightarrow\quad x = \frac{1}{3}\\
        &3\cdot(1) -(2): \quad 8y = 2 \quad\Rightarrow\quad y = \frac{1}{4} \qquad \Rightarrow \qquad \uuline{\mathbb{L} = \Bigg\{\qty(\frac{1}{3};\frac{1}{4})\Bigg\}}
    \end{align*}
    \item $~$\\[-1.45cm] 
    \begin{align*}
        &\begin{rcases}
            \frac{x+2}{y+3}=\frac{1}{3} \\
            \frac{y+3}{2y-5x}=\frac{3}{5}
        \end{rcases} \quad \Longrightarrow \quad 
        \begin{array}{l}
            \hphantom{1}3x-y = \hphantom{1}-3 \quad (1) \\
            15x -y = -15 \quad (2)
        \end{array} \\
        &(1) - (2): \quad -12x = 12 \quad \Rightarrow \quad  x = -1\\
        &\text{in } (1): \quad y = 0 \qquad \Rightarrow \qquad \uuline{\mathbb{L} = \big\{\qty(-1;0)\big\}}
    \end{align*}
    \item $ax+by=2a\,, $ \tab $ \frac{x}{b}-\frac{y}{a}=\frac{2}{a}$
    \item $x+14y=\frac{1}{\sqrt{2}}-7\sqrt{2}\,, $ \tab $ 3\sqrt{2}x-\frac{y}{\sqrt{3}}=3+\frac{1}{\sqrt{6}}$
    \item $\frac{x}{a+b}+\frac{y}{a-b}=a+b\,, $ \tab $ \frac{x}{a}-\frac{y}{b}=2b$
    \item $39x-38y=1\,, $ \tab $ 91x-57y=4$
\end{enumerate}

\paragraph{Aufgabe 2: } \emph{Drei Gleichungen mit drei Unbekannten}\\[0.2cm]
%
\begin{center}
\begin{minipage}[t]{0.3\linewidth}
(a)\vspace{-2.7em}
\begin{align*}
    x-y+5z&=5\,,\\
    3x+7y-5z&=5\,,\\
    x+y-z&=1
\end{align*}

(c)\vspace{-2.7em}
\begin{align*}
    x+y&=b+a\,,\\
    x+z&=a+c\,,\\
    y+z&=c+b
\end{align*}
\end{minipage} \hspace{1.5cm}
\begin{minipage}[t]{0.4\linewidth}
%
(b)\vspace{-2.7em}
\begin{align*}
    3x-4y+3z&=4\,,\\
    -x+y-z&=-2\,,\\
    7x+4y-5z&=0
\end{align*}
(d)\vspace{-2.7em}
\begin{align*}
    6x-4y+8z&=0\,, \\
    -2x+y-z&=0\,,\\
    12x-7y+11z&=0
\end{align*}
\end{minipage} \\[0.2cm]
\end{center}
\vspace{0.7cm}

\newpage
% 
\textbf{Aufgabe 3: } \emph{Parametrisierung von Lösungsmengen}\\[0.2cm]
Geben Sie die Lösungsmenge der Gleichung $13x-7y=1$ an für
\begin{align*}
&\text{(a) } \hspace{0.2cm} x,y\in\mathbb{R}\,;\\[0.2cm]
&\text{(b) } \hspace{0.2cm} x,y\in\mathbb{N}\,.
\end{align*}

\paragraph{Aufgabe 4: } \emph{Gleichungssysteme}\\[0.2cm]
Lösen Sie die folgenden Gleichungssysteme jeweils für $x$ und $y$.\\[0.5cm]
\begin{minipage}[t]{0.45\linewidth}
(a)\vspace{-2.7em}
\begin{align*}
    x^2+y^2&=2(xy+2)\,,\\ x+y&=6
\end{align*}
(c)\vspace{-2,7em}
\begin{align*}
    \frac{x^2-y^2}{2x+3}+y^2&=(x+y)x-xy\,,\\
    y-2x&=3
\end{align*}
\end{minipage}\hfill
\begin{minipage}[t]{0.4\linewidth}
(b)\vspace{-2.7em}
\begin{align*}
    \frac{12}{\sqrt{x-1}}+\frac{5}{\sqrt{y+\frac{1}{4}}}&=5\,,\\ \frac{8}{\sqrt{x-1}}+\frac{10}{\sqrt{y+\frac{1}{4}}}&=6
\end{align*}
\end{minipage}
%
\paragraph{Aufgabe 5: } \emph{Ungleichungssysteme}
\begin{enumerate}[label=(\alph*), labelindent=1em,labelsep=0.5cm]
\item Es ist die Lösungsmenge des folgenden Systems an Ungleichungen zu skizzieren. Welche der Ungleichungen können weggelassen werden, ohne dass sich die Lösungsmenge ändert?
\begin{align*}
    2x-3y&\geq -6\,,\\
    x-2y&< 11\,,\\
    x&>-y-1\,,\\
    x&<5\,,\\
    x&\geq 0\,,\\
    y&\geq 0
\end{align*}
\item \textbf{*} Welches Gebiet im ersten Oktanden ($x\ge 0$, $y\ge 0$, $z\ge 0$) wird durch die folgenden Ungleichungen definiert?
\begin{align*}
    x+y&\ge z\,,\\
    x+z&\ge y\,,\\
    y+z&\ge x
\end{align*}
\end{enumerate}