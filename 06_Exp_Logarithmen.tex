\section{Exponentialfunktionen und Logarithmen}

Wir wollen uns nun mit Funktionen beschäftigen, die exponentielles Wachstum beschreiben. \emph{Exponentialfunktionen} sind Funktionen, deren Variable im Exponenten steht $f(x) = a^x$. Hierbei muss die Basis $a > 0$ sein. Unabhängig von $a$ gilt dann $a^0 = 1$, d.\,h. alle Exponentialfunktionen schneiden die $y$-Achse im Punkt $(0,1)$. 

Wir können, abhängig von $a$, drei Fälle unterscheiden 
\begin{itemize}
    \item $a > 1: \quad \lim_{x\to -\infty} a^x = 0$, asymptotische Annäherung an $x$-Achse von rechts
    \item $a < 1: \quad \lim_{x\to \infty} a^x = 0$, asymptotische Annäherung an $x$-Achse von links 
    \item $a = 1: \quad y=1$ für alle $x$
\end{itemize}

\begin{figure}[htp]
    \centering
    \begin{tikzpicture}
        \begin{axis}[disabledatascaling, axis lines=middle, xlabel={$x$}, ylabel={$y$}, height=6cm, width=0.9\textwidth, ymax=2.99, legend pos = outer north east, xmin=-1.1, xmax=1.1, ytick={2}]
            \addplot[no marks, FSUblau, thick, domain=-1:1]{2^x};
            \addplot[no marks, FSUblau, thick, dashed, domain=-1:1]{5^x};
            \addplot[no marks, PAForange, thick, domain=-1:1]{(1/2)^x};
            \addplot[no marks, PAForange, thick, dashed, domain=-1:1]{(1/5)^x};
            \addplot[no marks, Gruen, thick, domain=-1:1]{1};
            \node (A) at (-0.04,0.8){1};
            \legend{$a = 2 >1$, $a = 5 > 1$,$a = 1/2 <1$, $a=1/5 < 1$, $a=1$};
        \end{axis}
    \end{tikzpicture}
    \caption{Darstellung von Exponentialfunktionen für verschiedene Werte von $a$.}
\end{figure} 
Die Funktionen sind spiegelbildlich zur $x$-Achse, denn 
\begin{align}
    \begin{rcases}
        f_1(x) = a^x \\
        f_2(x) = \qty(\frac{1}{a})^x = a^{-x} = f_1(-x) 
    \end{rcases} \qq{wenn} a >1, \qq{dann} \frac{1}{a} < 1.
\end{align}
Das heißt, zu jeder \textcolor{FSUblau}{blauen} Funktion mit Basis $a$, findet man eine spiegelsymmetrische \textcolor{PAForange}{orangene} Funktion mit Basis $\frac{1}{a}$.

\subsection{Logarithmen}

Wir wollen uns nun die Frage stellen, welchen Wert $n$ ein Exponent zu einer gegebenen Basis $b$ haben muss, damit der Potenzwert $a$ herauskommt. Also es gelte $a = b^n$ für ein bekanntes $a$ und $b$, was ist dann $n$?

Die Antwort auf diese Frage liefert die Logarithmusfunktion:
\begin{align}
    \tikzmarknode{eq1}{}n = \log_b\tikzmarknode{eq2}{}(\tikzmarknode{eq3}{}a).
\end{align}
\tikz[overlay,remember picture]{
\draw[shorten >=2pt,shorten <=2pt, thick, -{latex}] ($(eq2)+(0.8,-0.7)$)node[right]{Radikand} -- ($(eq2)+(-0.2,-0.1)$);
\draw[shorten >=2pt,shorten <=2pt, thick, -{latex}] ($(eq1)+(-0.7,0.1)$)node[left]{Exponent} -- ($(eq1)+(0,0.1)$);
\draw[shorten >=2pt,shorten <=2pt, thick, -{latex}] ($(eq3)+(1.3,0.15)$)node[right]{Numerus (Potenzwert)} -- ($(eq3)+(0.5,0.15)$);
}