\Titelbanner{7}{Grundlagen der Differentialrechnung\\Kurvendiskussion}

\paragraph{Wiederholung Ableitungsregeln}$~$
\begin{itemize}
    \item Linearität: $\dv{x}(a f(x) + b g(x)) = a \dv{f}{x} + b \dv{g}{x}$
    \item Produktregel (Leibniz-Regel): $\dv{x}(f(x)\cdot g(x)) = \dv{f}{x} \cdot g(x) + f(x) \cdot \dv{g}{x}$
    \item Kettenregel: $\dv{x} f(g(x)) = \qty(\dv{f}{g})(g(x))\cdot \dv{g}{x}.$
    % \item Quotientenregel: $\dv{x}\qty(\frac{f(x)}{g(x)}) = \frac{1}{g(x)^2} \qty(\dv{f}{x}\cdot g(x) - f(x) \cdot \dv{g}{x}).$
    \item Potenzregel: $\dv{x}\qty(x^n) = n \, x^{n-1}.$
\end{itemize}
\emph{Ableitungen spezieller Funktionen: }
\begin{alignat}{3}
    \dv{x}\sin(x) &= +\cos(x) &\qquad\quad \dv{x}\exp(x) &= \exp(x)\\
    \dv{x}\cos(x) &= -\sin(x) & \dv{x}a^x &= \ln(a) a^x.
\end{alignat}

\emph{Beispiel:}\\[-1.5cm]
\begin{align}
    f(x) &= x^2 \ln\big(\underbrace{3x^2-5x+2}_{g(x)}\big) \\
    \quad\Rightarrow f'(x) &= 2x \ln(g(x)) + x^2 \frac{1}{g(x)} \dv{g(x)}{x}= 2x \ln(3x^2-5x+2) + \frac{x^2(6x-5)}{3x^2 - 5x+2}.
\end{align}

\paragraph{Aufgabe 1: } \emph{Ableitungen I} \hfill Ziel: (a) bis (e)\\[0.2cm]
\emph{Berechnen Sie die Ableitungen der folgenden Funktionen.}
\begin{enumerate}[label=(\alph*)]
    \setlength{\mathindent}{0cm}
    \item $Q(r)=\frac{3r^2}{2}\qty(\frac{1}{2}+\ln\frac{r}{r_0})$
    \begin{align}
        \dv{Q(r)}{r} &= 3r\qty(\frac{1}{2}+\ln(\frac{r}{r_0})) + \frac{3r^2}{2}\frac{r_0}{r}\frac{1}{r_0} = \uuline{3r\qty(1+\ln \frac{r}{r_0})}.
    \end{align}
    \item $f(x)=\cos^4(3tx)-\sin^4(3tx) = \underbrace{(\cos^2(3tx)+\sin^2(3tx))}_{1}(\cos^2(3tx)-\sin^2(3tx))$ 
    \begin{align}
        \dv{f(x)}{x} = -6t \sin(3tx)\cos(3tx) - 6t\sin(3tx)\cos(3tx) = \uuline{-12t \sin(3tx)\cos(3tx)}.
    \end{align}
    \item $S(\tau)=(\tau-1)\e^{\tau}+\frac{\tau^2}{4}\qty(2\ln\tau-1)$
    \begin{align}
        \dv{S(\tau)}{\tau} = \cancel{\e^{\tau}} + (\tau-\cancel{1})\e^\tau + \frac{\tau}{2}\qty(2\ln(\tau)-\cancel{1}) + \frac{\tau^{\cancel{2}}}{4} \frac{2}{\cancel{\tau}} = \uuline{\tau(\e^{\tau} + \ln(\tau))}.
    \end{align}
    \item $y(x)=\frac{\exp(2x)}{25}\qty[(5x-4)\sin x+(10x-3)\cos x]$
    \begin{align}
        \dv{y(x)}{x} &= \frac{2}{25} \e^{2x} \bigg[\hdots \bigg] + \frac{1}{25}\e^{2x} \qty[5 \sin(x) + (5x-4)\cos(x) + 10\cos(x)-(10x-3)\sin(x)] \\
        &= \frac{\e^{2x}}{25}\qty[\cancel{(10x-8)\sin(x)}+(20x-\bcancel{6})\cos(x) + (5x+\bcancel{6})\cos(x) - \cancel{(10x-8)\sin(x)}] \notag \\
        &= \uuline{x \cos(x) \e^{2x}}
    \end{align}
    \item $F(x)=-\frac{k}{\sqrt{(x-x_0)^2+(y-y_0)^2}}$
    \begin{align}
        \dv{F(x)}{x} &= \frac{k}{\cancel{2}} \frac{\cancel{2}(x-x_0)}{\sqrt{\hdots}^3} = \uuline{\frac{k(x-x_0)}{\sqrt{(x-x_0)^2+(y-y_0)^2}^3}}
    \end{align}
    \item$~$\\[-1.5cm]
    \begin{align}
        N(z) &=\underbrace{\frac{2\cos\qty(\frac{z}{2})}{\sqrt{1+\cos(z)}}}_{\mathclap{1+\cos(z) = 2 \cos^2\qty(\frac{z}{2})}}\qty[\ln\qty(\cos\frac{z}{4}+\sin\frac{z}{4})-\ln\qty(\cos\frac{z}{4}-\sin\frac{z}{4})] \qquad \sigma_{\pm} = \cos(\frac{z}{4})\pm \sin(\frac{z}{4}) \\
        &= \sqrt{2}(\ln(\sigma_+)-\ln(\sigma_-)) \\
        \Rightarrow \dv{N(z)}{z} &= \sqrt{2} \qty(\frac{(\sigma^+)'}{\sigma^+} - \frac{(\sigma^-)'}{\sigma^-}) = \sqrt{2} \frac{(\sigma^+)' \sigma^- - (\sigma^-)'\sigma^+}{\sigma^+ \sigma^-} \qq{mit} (\sigma^\pm)' = \pm \frac{1}{4}\sigma^\mp \\
        &= \frac{1}{2\sqrt{2}} \frac{(\sigma^+)^2 + (\sigma^-)^2}{\sigma^+ \sigma^-} = \frac{1}{\cancel{2}\sqrt{2}} \frac{\cancel{2}}{\cos^2\qty(\frac{z}{4})-\sin^2\qty(\frac{z}{4})} \notag \\
        &= \frac{1}{\sqrt{2}} \frac{1}{2\cos^2\qty(\frac{z}{4}-1)} = \frac{1}{\sqrt{2}} \frac{1}{\cos(\frac{z}{2})} = \uuline{\frac{1}{\sqrt{1+\cos(z)}}}.  
    \end{align}
\end{enumerate}
%
\paragraph{Aufgabe 2: } \emph{Ableitungen II}\hfill Ziel: (a) bis (d)\\[0.2cm]
\emph{Finden Sie die $n$-te Ableitung der folgenden Funktionen.}
    \begin{enumerate}[label=(\alph*)]
        \setlength{\mathindent}{0cm}
        \item $f(x)=x^n$ 
        \begin{align}
            f'(x) &= n x^{n-1}, \quad f''(x) = n(n-1) x^{n-2}, \quad \hdots \quad \Rightarrow \quad \dv[n]{f(x)}{x} = n!
        \end{align}
        \item $f(x)=\e^{kx}+\e^{-kx}$
        \begin{align}
            f'(x) = k\qty(\e^{kx}-\e^{-kx}), \quad f''(x) = k^2 \qty(\e^{kx}+\e^{-kx})\quad \hdots \quad \Rightarrow \quad \dv[n]{f(x)}{x} = k^n \qty(\e^{kx} + (-1)^n \e^{kx})
        \end{align}
        \item $f(x)=x^{n-1}$
        \begin{align}
            \dv[n-1]{f(x)}{x} \overset{(a)}{=} (n-1)! \quad \Rightarrow \quad \dv[n]{f(x)}{x} = 0.
        \end{align}
        \item $f(x)=a^x$ 
        \begin{align}
            f'(x) = (\ln a) a^x, \quad f''(x) = (\ln a)^2 a^x, \quad \hdots \quad \Rightarrow \quad \dv[n]{f(x)}{x} = (\ln a)^n a^x
        \end{align}
        \item $f(x)=\frac{1}{1-x}$
        \begin{align}
            f'(x) = \frac{1}{(1-x)^2}, \quad f''(x) = \frac{2}{(1-x)^3}, \quad \hdots \quad \Rightarrow \quad \dv[n]{f(x)}{x} = \frac{n!}{(1-x)^{n+1}}
        \end{align}
        \item $f(x)=\frac{x}{1-x-x^2}\qq{,} $ Nullstellen des Nenners $\frac{-1\pm\sqrt{5}}{2}$  
        \begin{align}
            &\text{Partialbruchzerlegung:}\quad \frac{-x}{1-x-x^2} \overset{!}{=} \frac{\alpha}{x-x_1} + \frac{\beta}{x-x_2} = \frac{\alpha (x-x_2)+\beta (x-x_1)}{(x-x_1)(x-x_2)} \\
            &\text{Koeffizientenvergleich:}\quad 1 = \alpha + \beta \qq{und} 0 = \alpha x_1 + \beta x_2 \\
            &\Rightarrow x_1 = \alpha (x_1 - x_2) \quad \Rightarrow \quad \alpha = \frac{x_1}{x_1-x_2} \quad \Rightarrow\quad \beta = 1-\alpha = -\frac{x_2}{x_1-x_2} \\
            &\Rightarrow f(x) = \frac{-1}{x_1-x_2}\qty(\frac{x_1}{x-x_1}- \frac{x_2}{x-x_2})
        \end{align}
        \begin{align}
            \overset{(e)}{\Rightarrow} \dv[n]{f(x)}{x} &= \frac{(-1)^n n!}{x_1-x_2} \qty(\frac{x_1}{(x-x_1)^{n+1}} - \frac{x_2}{(x-x_2)^{n+1}}) \\
            &= \frac{(-1)^n n!}{x_1-x_2} \qty(\frac{x_2^{n+1}}{x_2^{n+1}} \frac{x_1}{(x-x_1)^{n+1}} - \frac{x_1^{n+1}}{x_1^{n+1}} \frac{x_2}{(x-x_2)^{n+1}}), \quad x_1-x_2 =  \sqrt{5}\\
            &= \frac{(-1)^n n!}{\sqrt{5}} \qty( \frac{x_1 x_2^{n+1}}{(x_2x-x_1x_2)^{n+1}} -  \frac{x_2 x_1^{n+1}}{(xx_1-x_1x_2)^{n+1}}), \quad x_1 x_2 = -1 \\
            &= \frac{(-1)^n n!}{\sqrt{5}} \qty( \frac{x_1^n}{(x_2x+1)^{n+1}} -  \frac{x_2^n}{(xx_1+1)^{n+1}}).
        \end{align}
        Es ergibt $\frac{1}{n!} \dv[n]{f(x)}{x}\eval_{x=0} = \frac{(-1)^n}{\sqrt{5}}(x_1^n -x_2^n)$ die $n$-te Fibonacci-Zahl.
    \end{enumerate}
%
\newpage
\paragraph{Aufgabe 3: } \emph{Kurvendiskussion I}\\[0.2cm]
\emph{Ein zweiatomiges Molekül lässt sich näherungsweise durch das sogenannte ``Morse-Potential'' beschreiben,}
\begin{align*}
U(x)=D\qty(\e^{-2\alpha x}-2\e^{-\alpha x})\,, \hspace{1cm} D,\alpha=\operatorname{const}.
\end{align*}
\begin{itemize}
\item \emph{Bestimmen Sie Nullstellen und lokale Extrema der Funktion $U(x)$ sowie deren Verhalten für $x \to\pm\infty$.} 
\begin{align}
    \text{Nullstellen: } \e^{-2\alpha x_0} \overset{!}{=} 2 \e^{-\alpha x_0} \quad \Rightarrow \quad -2\alpha x_0 &= \ln(2) - \alpha x_0 \\
    \Rightarrow x_0 &= \uuline{-\frac{\ln(2)}{\alpha}} 
\end{align}
\begin{align}
    \text{Extrema: } \dv{U(x)}{x} &= D \qty(-\cancel{2\alpha} \e^{-2\alpha x} + \cancel{2\alpha} \e^{-\alpha x}) \overset{!}{=} 0 \\
    \Rightarrow \quad -2\alpha x &= -\alpha x \quad \Rightarrow \quad \uuline{x=0, \quad U(x=0) = -D}. 
\end{align}
\begin{align}
    \text{Grenzwerte: } \lim_{x\to \infty} U(x) = 0 \qq{,} \lim_{x\to -\infty} U(x) = \infty
\end{align}
\item \emph{Skizzieren Sie die Funktion $U(x)$ für $D=\alpha=1$ im Intervall $x\in[-1,5]$.}
\end{itemize}
\begin{figure}[htp]
    \centering
    \begin{tikzpicture}
        \begin{axis}[disabledatascaling, axis lines=middle, xtick={-0.693}, xticklabels={$\frac{-\ln2}{\alpha}$}, ytick={-1}, xlabel={$x$}, ylabel={$U(x)/D$}, height=6cm, width=0.9\textwidth, samples=100, ymin=-1.2, xticklabel style = {fill=white, fill opacity=0.6, text opacity=1}, yticklabel style = {fill=white, fill opacity=0.6, text opacity=1}, set layers = axis on top]
            \addplot[no marks, FSUblau, thick, domain=-1:5]{exp(-2*x)-2*exp(-x)};
        \end{axis}
    \end{tikzpicture}
\end{figure}
%
\newpage
\paragraph{Aufgabe 4: } \emph{Kurvendiskussion II}\\[0.2cm]
\emph{Die Bewegung eines Teilchens mit Drehimpuls $L$ und Energie $E$ in der gekrümmten Raumzeit eines Schwarzen Loches der Masse $m$ wird beschrieben durch das Potential}
\begin{align*}
U(r)=\frac{E}{2}-\frac{Em}{r}+\frac{L^2}{2r^2}-\frac{mL^2}{r^3}\,, \hspace{1cm} r>0\,.
\end{align*}
\begin{itemize}
\item \emph{Bestimmen Sie Nullstellen und lokale Extrema der Funktion $U(r)$ sowie deren Verhalten für $\to\infty$ und $\to 0$.}
\begin{align}
    \text{Nullstellen: } 0 \overset{!}&{=} \frac{E}{2} - E m u + \frac{L^2}{2} u^2 - m L^2 u^3 \qq{,} u = \frac{1}{r} \\
    0 &= 1 - 2m u + \frac{L^2}{E}u^2 - \frac{2m L^2}{E} u^3 \\
    &= 1- 2m u + \frac{L^2}{E}(1-2mu) u^2 = (1-2mu)\qty(1+\frac{L^2}{E}u^2) \\
    &\Rightarrow u_0 = \frac{1}{2m} \qq{,} \uuline{r_0 = 2m}.
\end{align}
\begin{align}
    \text{Extrema: } \dv{U(r)}{r} &=  \frac{E m}{r^2} - \frac{L^2}{r^3} + \frac{3mL^2}{r^4}\overset{!}{=} 0 \quad |\cdot r^4 \\
    \Rightarrow \quad 0 &= r^2 - \frac{L^2}{E m} r + \frac{3L^2}{E}  \\
    \Rightarrow \quad r_{1/2} &= \frac{L^2}{2E m} \pm \sqrt{\frac{L^4}{4E^2 m^2}- \frac{3L^2}{E}} = \uuline{\frac{L^2}{2Em} \qty(1\pm \sqrt{1-\frac{12Em^2}{L^2}})}
\end{align}
\begin{align}
    \text{Grenzwerte: } \lim_{r\to \infty} U(r) = \frac{E}{2} \qq{,} \lim_{r\to 0} U(r) = -\infty.
\end{align}
\item \emph{Setzen Sie $\textstyle m=\frac{1}{2}$. Welche Bedingungen an $E$ und $L$ müssen erfüllt sein, damit $U(r)$ zwei, ein oder keine lokalen Extrema besitzt?}
Für $m = 1/2$ nimmt die Diskriminante den Wert $1-\frac{3E}{L^2}$ an. Damit können wir zwischen drei Lösungen unterscheiden: 
\begin{align}
    & 3E < L^2: \quad \qq{zwei Extrema.}\\
    & 3E = L^2: \quad \qq{ein Extremum.}\\
    & 3E > L^2: \quad \qq{keine Extrema.}
\end{align}
\item \emph{Skizzieren Sie die Funktion $U(r)$ für $E=1$ und $L=2$ (nicht maßstabsgerecht).}
Für diesen Fall finden wir zwei Extrema. Es ergibt sich 
\begin{align}
    r_{1/2} = 4 \qty(1\pm \sqrt{\frac{1}{4}}) = \begin{cases}
        6&, \quad U(r_1) = \frac{25}{54}= \frac{1}{2}-\frac{1}{27} \\ 2&, \quad U(r_2) = \frac{1}{2}
    \end{cases}
\end{align}
\end{itemize} 
\begin{figure}[htp]
    \centering
    \begin{tikzpicture}
        \begin{axis}[disabledatascaling, axis lines=middle, xtick={1,2,3,4,5,6,7,8,9,10, 12,14,16,18,20}, ytick={0.5,25/54}, yticklabels={$\frac{1}{2}$, $\frac{25}{54}$} , xlabel={$x$}, ylabel={$U(x)/D$}, height=6cm, width=0.9\textwidth, samples=200, xticklabel style = {fill=white, fill opacity=0.6, text opacity=1}, yticklabel style = {fill=white, fill opacity=0.6, text opacity=1}, set layers = axis on top, ymin=0.43, ymax=0.51, xmin=0, xmax=21]
            \addplot[no marks, FSUblau, thick, domain=0.9:20]{1/2 - 1/(2*x) + 2/x^2 - 2/x^3};
            \draw[dashed] (0,25/54) -- (6,25/54) -- (6,0.3);
            \fill (6,25/54) circle (2pt);
        \end{axis}
    \end{tikzpicture}
\end{figure}

%
\paragraph{Aufgabe 5: } \emph{Gewöhnliche Differentialgleichungen}\hfill (Zusatzaufgabe)\\[0.2cm]
\begin{enumerate}[label=(\alph*)]
\item \emph{Finden Sie eine Funktion $f(x)$, welche die folgende Gleichung erfüllt:}
\begin{align*}
f''(x)=a^2f(x)+bx\,.
\end{align*}
Wir suchen eine Funktion, die sich selbst (teilweise) reproduziert 
\begin{align}
    &\text{Versuch 1: } f(x) = \e^{\lambda x} \quad \Rightarrow \quad f''(x) = \lambda^2 \e^{\lambda x} = \lambda^2 f(x) \\
    &\text{Versuch 2: } f(x) = \e^{ax} + mx  \\
    &\quad \Rightarrow f''(x) = a^2 \e^{ax} \overset{!}{=} a^2 f(x) + bx = a^2 \e^{ax} + a^2 mx + bx \\
    &\text{offenbar: } m = -\frac{b}{a^2} \quad \Rightarrow \quad \uuline{f(x) = \e^{ax} - \frac{b}{a^2} x}\\
    &\qty(\text{allgemeine Lösung: } f(x) = c_1 \e^{ax} + c_2 \e^{-ax} - \frac{b}{a^2} x)
\end{align}
\item \emph{Finden Sie eine Funktion $f(x)$, welche die folgende Gleichung erfüllt:}
\begin{align*}
f'(x)=\qty(1+\ln(x))f(x)\,.
\end{align*}
Überlegung: Ableitung von $\e^{g(x)}$ ergibt $g'(x) \e^{g(x)}$
\begin{itemize}
    \item[$\Rightarrow$] Welche Funktion ergibt abgeleitet $1+ \ln(x)$? Keine Funktion ergibt abgeleitet $\ln(x)$. 
    \item[$\Rightarrow$] Es kann $\ln(x)$ durch die Produktregel entstehen. \\
    $\dv{x} (x \ln(x)) = \ln(x) + x \frac{1}{x} = \ln(x)+1$ 
\end{itemize}
\begin{align}
    \Rightarrow \uuline{f(x) = \e^{x \ln(x)} = x^x}.
\end{align}
\end{enumerate}