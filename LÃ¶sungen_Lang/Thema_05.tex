\Titelbanner{5}{Exponentialfunktionen\\
                Logarithmen\\
                Natürliche Exponentialfunktion}

\begin{mymathbox}[ams align, title={Logarithmengesetze, Basiswechsel}, colframe={FSUblau}]
    \begin{split}
        \log_b(uv) = \log_b(u) + \log_b(v), \quad \log_b\qty(\frac{u}{v}) = \log_b(u)-\log_b(v) \\
        \log_b(u^m) = m \log_b(u), \quad \log_a(x) = \frac{1}{\log_b(a)} \log_b(x).       
    \end{split}
\end{mymathbox}

\paragraph{Aufgabe 1: } \emph{Logarithmische und Exponentialgleichungen}\\[0.2cm]
\emph{Lösen Sie jeweils für $x$ bzw. $y$. Beachten Sie mögliche Fallunterscheidungen bezüglich der auftretenden Konstanten.}

\emph{Tafelbeispiele:}
\begin{enumerate}
    \item$~$\\[-1.7cm]
    \begin{align}
        3 \ln[\frac{\sqrt[3]{\e^x}}{\exp(3\log_2(\sqrt{2}))}] + 2x &= 0 \\
        3 \ln(\frac{\e^{x/3}}{\e^{3/2}}) +2x &= 0\\
        3\qty(\frac{x}{3}-\frac{3}{2}) + 2x &= 0 \quad \Rightarrow \quad \uuline{x = \frac{3}{2}}.
    \end{align}
    \item$~$\\[-1.4cm]
    \begin{align}
        2 \log_{\sqrt[5]{9}}(3x) - 5\ln(x) &= \log_3(\e^{5}) \\
        \text{Basiswechsel: } \log_a(\xi) &= \frac{\ln(\xi)}{\ln(a)} \\
        2\cdot \frac{\ln(3x)}{\ln(3^{2/5})} - 5\ln(x) &= \frac{5}{\ln(3)} \\
        \cancel{5} \frac{\ln(3)+\ln(x)}{\ln(3)} - \cancel{5} \ln(x) &= \frac{\cancel{5}}{\ln(3)} \\
        \Rightarrow \ln(x) = \frac{1-\ln(3)}{1-\ln(3)} &= 1 \quad \Rightarrow \quad \uuline{x=\e}.
    \end{align}
\end{enumerate}
\emph{Lösung:}
\begin{enumerate}[label=(\alph*)]
    \setlength{\mathindent}{0cm}
    \item $a\,2^x=\e^{bx} \quad \Rightarrow \quad \ln(a)+x\ln(2) = bx \quad \Rightarrow \quad \uuline{x = \frac{\ln(a)}{b-\ln(2)}},\quad$ offenbar $a>0$
    \item $x=49^{1-\log_7(2\sqrt{x})}+5^{-\log_5(4x)} = \frac{49}{(7\cdot 7)^{\log_7(2\sqrt{x})}} + \frac{1}{4x} = \frac{25}{2x} \quad \Rightarrow \quad \uuline{x = \frac{5}{\sqrt{2}}},\quad$ (nur positive Lsg.)
    \item$~$\\[-1.4cm]
    \begin{align}
        \e^{3ax}-2^{a+1}\e^{2ax}+4^a\e^{ax} &=\log_b(1) \quad |\cdot \e^{-ax} \\
        \e^{2ax} - 2\cdot 2^a \e^{ax} + (2^a)^2 &= 0 \\
        (\e^{ax}+2^a)^2 &= 0  \quad \Rightarrow \quad \e^x = 2 \quad \Rightarrow \quad \uuline{x = \ln(2)}.
    \end{align}
    \item$~$\\[-1.4cm]
    \begin{align}
        \sqrt[x^2-1]{a^3}\cdot\sqrt[2x-2]{a}\cdot\sqrt[4]{a^{-1}} &=1 \quad \qq{offenbar $a>0$} \\
        a^{\frac{3}{x^2-1} + \frac{1}{2x-2} - \frac{1}{4}} &= 1. \\
        \text{Fall } a=1: \qquad & \uuline{x \in \mathbb{R}\backslash \{\pm 1\}} \\
        \text{Fall } a\neq 1: \qquad & 0=\frac{3}{x^2-1} + \frac{1}{2x-2} - \frac{1}{4} = \frac{1}{4} \frac{15+ 2x-x^2}{x^2-1} \\
        &0 = x^2 -2x-15 \quad \Rightarrow \quad \uuline{x_1 =5, x_2 = -3}
    \end{align}
    \item$~$\\[-1.4cm]
    \begin{align}
        &\log_a(x) + \log_a(y) = 2 \quad\Rightarrow\quad \log_a(xy) = 2 \quad\Rightarrow\quad xy = a^2 \\
        &\log_b(x) - \log_b(y) =4 \quad\Rightarrow\quad \log_b\qty(\frac{x}{y}) = 4 \quad \Rightarrow \quad \frac{x}{y} = b^4\\
        &\Rightarrow x^2 = a^2 b^4 \quad \Rightarrow \quad x = \pm ab^2, y = \pm \frac{a}{b^2}\\
        &\text{Die Argumente der Logarithmen müssen positiv sein:}\\
        & \qquad \uuline{x = \begin{cases}
            a b^2, & a > 0 \\ -ab^2, & a< 0
        \end{cases}} \qq{und} \uuline{y = \begin{cases}
            \frac{a}{b^2}, & a> 0 \\ -\frac{a}{b^2}, & a < 0
        \end{cases}}, \qq{($a=0$ nicht zulässig)}
    \end{align}
    \item $3\log_{xa^2}x+\frac{1}{2}\log_\frac{x}{\sqrt{a}}x=2 \overset{\text{Basiswechsel}}{\Longrightarrow} \frac{3\ln(x)}{\ln(xa^2)} + \frac{\ln(x)}{2\ln(\frac{x}{\sqrt{a}})} = 2$\\
    Substitution: $z \equiv \ln(x), b\equiv \ln(a)$
    \begin{align}
        \Rightarrow \quad \frac{3z}{z+2b} + \frac{z}{2z-b} &= 2 \quad |\cdot (z+2b)(2z-b) \\
        7z^2 - z &= 4z^2+6bz -4b^2  \\
        \Rightarrow 0&= z^2 -\frac{7b}{3}z + \frac{4b^2}{3} \quad \Rightarrow \quad z_{1/2} = \frac{7b}{6}\pm \sqrt{\frac{49b^2}{36}-\frac{4b^2}{3}} = \begin{cases}
            \frac{4b}{3} \\ b
        \end{cases} \\
        \Rightarrow \quad\uuline{x_1 = a^{4/3}, x_2 = a}&
    \end{align}
\end{enumerate}

\paragraph{Aufgabe 2: } \emph{Verdopplungszeit}\\[0.2cm]
\emph{Der Wissenszuwachs eines Physik-Studenten mit Anfangswissen $A$ sei beschrieben durch
\begin{align*}
    W(t)=A\e^{c t}\,, \hspace{1.5cm} c>0\,,
\end{align*}
sodass $W(t)$ die "`Menge"' an Wissen zur Zeit $t$ gibt.}
\begin{enumerate}[label=(\alph*)]
    \item \emph{Nach welcher Zeit $\tau_2$ hat das Wissen eines Studenten auf das Doppelte zugenommen?}\\
    \begin{align}
        W(t_0 + \tau_2) \overset{!}&{=} 2 W(t_0) \quad \Rightarrow \quad \cancel{A} \e^{c(\bcancel{t_0} + \tau_2)} = 2 \cancel{A} \bcancel{e^{c t_0}} \\
        \Rightarrow \quad \tau_2 &= \uuline{\frac{\ln(2)}{c}}.
    \end{align}
    \item \emph{Nach welcher Zeit $\tau_n$ hat das Wissen eines Studenten auf das $n$-Fache zugenommen?}
    \begin{align}
        W(t_0 + \tau_n) \overset{!}&{=} n W(t_0) \quad \Rightarrow \quad \e^{c \tau_n} = n  \\
        \Rightarrow \quad \tau_n &= \uuline{\frac{\ln(n)}{c}}.
    \end{align}
    \item \emph{Drücken Sie $\tau_n$ als Vielfaches von $\tau_2$ aus. Welcher Zusammenhang besteht in den Fällen $n=3$ und $n=4$? Welche Aussage können Sie für $n=2^m$, $m\in\mathbb{N}$, treffen?}
    \begin{align}
        &\tau_n \overset{!}{=} x \tau_2 \quad \Rightarrow \quad \frac{\ln(n)}{\cancel{c}} = x \frac{\ln(2)}{\cancel{c}} \\
        &\Rightarrow \quad x = \frac{\ln(n)}{\ln(2)} = \log_2(n)  \quad \Rightarrow \quad \uuline{\tau_n = \log_2(n) \tau_2} \\
        & n=3: \quad \tau_3 = \log_2(3) \tau_2 \approx 1.6 \tau_2 \\
        & n=4: \quad \tau_4 = \log_2(4) \tau_2 = 2 \tau_2 \\
        & n= 2^m  \quad \tau_{2^m} = \log_2(2^m) \tau_2 = m \tau_2.
    \end{align}
\end{enumerate}

\newpage
\paragraph{Aufgabe 3: } \emph{Hyperbelfunktionen}\hfill (Zusatzaufgabe)\\[0.2cm]
\emph{Es seien die Funktionen
\begin{align*}
    f(x)&:=\frac{1}{2}(\e^{x}+\e^{-x})\,,\\
    g(x)&:=\frac{1}{2}(\e^{x}-\e^{-x})
\end{align*}
definiert.}
\begin{enumerate}[label=(\alph*)]
    \item \emph{Finden Sie jeweils einen Ausdruck für $f(2x)$ und $g(2x)$ in Abhängigkeit der Funktionen einfacher Argumente, $f(x)$ und $g(x)$.}
    \begin{align}
        f(2x) &= \frac{\e^{2x}+ \e^{-2x}}{2} = \frac{\qty(\e^x)^2 + \qty(\e^{-x})^2}{2} = \frac{(\e^x + \e^{-x})^2 - 2 \cancel{\e^x} \cancel{\e^{-x}}}{2} \\
        &= 2\qty(\frac{\e^x + \e^{-x}}{2})^2 - 1 = \uuline{2 f(x)^2 - 1}.\\
        g(2x) &= \frac{\e^{2x}-\e^{-2x}}{2} = \frac{(\e^x +\e^{-x})(\e^x - \e^{-x})}{2} = 2\qty(\frac{\e^{x}+\e^{-x}}{2})\qty(\frac{\e^{x}-\e^{-x}}{2}) = \uuline{2 f(x) g(x)}.
    \end{align}
    \item \emph{Finden Sie jeweils einen Ausdruck für $f(x+y)$ und $g(x+y)$ in Abhängigkeit von $f(x)$ und $f(y)$ sowie $g(x)$ und $g(y)$. Vergleichen Sie mit dem Ergebnis aus (a), indem Sie $x=y$ setzen.}\\
    Wir stellen zunächst fest: $\e^{x} = f(x) + g(x)$ und $\e^{-x} = f(x) - g(x)$ 
    \begin{align}
        f(x+y) &= \frac{1}{2}(\e^x \e^y + \e^{-x} \e^{-y}) = \frac{1}{2}\qty[(f(x)+g(x))\cdot(f(y)+g(x))+(f(x)-g(x))\cdot(f(y)-g(y))] \\
        &= \uuline{f(x)f(y) + g(x)g(y)} \\
        g(x+y) &= \frac{1}{2}(\e^x \e^y -\e^{-x} \e^{-y}) = \frac{1}{2}\qty[(f(x)-g(x))\cdot(f(y)+g(x))+(f(x)-g(x))\cdot(f(y)-g(y))] \\
        &= \uuline{f(x)g(y)+g(y)f(x)}
    \end{align}
    Ein Vergleich mit (a) zeigt 
    \begin{align}
        &f(2x) = f(x+x) = f(x)^2 + g(x)^2 \overset{!}{=} f(x)^2 - 1 \notag \\
        &\Rightarrow \qq{Es gilt} f(x)^2 - g(x)^2 = 1\\
        &g(2x) = g(x+x) = f(x)g(x) + g(x)f(x) = 2 f(x) g(x).
    \end{align}
    Bemerkung: DAs sind die Additionstheoreme für Hyperbelfunktionen sowie die hyperbolische Variante des trigonometrischen Pythagoras, $\cosh(x)^2 - \sinh(x)^2 = 1.$
    \item \emph{Schreiben Sie die Funktionen $f(x)$ und $g(x)$ in Reihendarstellung.}
    \begin{align}
        &\text{Vorlesung:} \quad \e^x = \sum_{n=0}^\infty \frac{x^n}{n!} = 1 + x + \frac{x^2}{2} + \frac{x^3}{6} + \hdots  \\
        &f(x) = \frac{\e^{x}+ \e^{-x}}{2} = \frac{1}{2}\sum_{n=0}^\infty \frac{x^n + (-x)^n }{n!} = \frac{1}{2} \sum_{n=0, \text{gerade}}^\infty \frac{2 x^n}{n!} = \uuline{\sum_{n=0}^\infty \frac{x^{2n}}{(2n)!}} \\
        &g(x) = \frac{\e^{x}- \e^{-x}}{2} = \frac{1}{2}\sum_{n=0}^\infty \frac{x^n - (-x)^n }{n!} = \frac{1}{2} \sum_{n=0, \text{ungerade}}^\infty \frac{2 x^n}{n!} = \uuline{\sum_{n=0}^\infty \frac{x^{2n+1}}{(2n+1)!}}.
    \end{align}
    \item \emph{Bestimmen Sie die Umkehrfunktion $g^{-1}$ von $g(x)$.}\\
    Wir schreiben $y = \frac{1}{2}(\e^x - \e^{-x})$ und versuchen nach $x$ aufzulösen. Dafür wenden wir zunächst die Substitution $z \equiv \e^x$ an: 
    \begin{align}
        y = \frac{1}{2}\qty(z - \frac{1}{z}) \quad \Rightarrow &\quad 0 = z^2 - 2y z -1  \\
        &\quad z_{1/2} = y \pm \sqrt{y^2 +1} \\
        \Rightarrow & \quad x = \ln(z) = \ln(y \pm \sqrt{y^2+1}).
    \end{align}
    Da $\sqrt{y^2 +1} > y$, muss das ``+'' gewählt werden, um ein positives Argument im Logarithmus zu erhalten. Es folgt 
    \begin{align}
        \uuline{g^{-1}(x) = \ln(x+\sqrt{x^2+1})}.
    \end{align}
    Bemerkung: Wir erhalten hiermit die Umkehrfunktion des $g(x) = \sinh(x)$, den $\text{arsinh}(x)$.
\end{enumerate}