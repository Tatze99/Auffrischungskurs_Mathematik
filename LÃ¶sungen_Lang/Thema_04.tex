\Titelbanner{4}{Umgang mit Polynomen höheren Grades\\
                Das Summenzeichen}

\paragraph{Aufgabe 1: } \emph{Nullstellensuche} \hfill Ziel: (a) bis (c)\\[-4mm]
\begin{enumerate}[label=(\alph*), labelindent=1em,labelsep=0.5cm]
    \item \emph{Stellen Sie das Polynom dritten Grades auf, das Wurzeln $a$, $b$ und $c$ hat.}
    \begin{align}
        f_3(x) = (x-a)(x-b)(x-c) = \uuline{x^3-(a+b+c) x^2 + (ab+ac+bc) x - abc}
    \end{align}
    \emph{Bemerkung}: Hier kann man versuchen, sich klarzumachen, wie der verallgemeinerte Satz von Vieta aussieht.
    \item \emph{Zerlegen Sie das Polynom $f_4(x)=x^3+2x^4+4x^2+2+x$ in Faktoren. Welche Aussage können Sie über dessen Nullstellen treffen?}
    \begin{align}
        f_4(x) = x^3 + \underbrace{2x^4 +4x^2 +2}_{\mathclap{2x^4+2x^2+2x^2 +2 = 2x^2(x^2+1) + 2(x^2+1)}} +x = (x^2+1)(2x^2+2) + \underbrace{x^3+x}_{x(x^2+1)} = \uuline{(x^2+1)(2x^2+x+2)}
    \end{align}
    Da weder $x^2 +1 = 0$ noch $\textstyle x^2 +\frac{1}{x}x+1=0$ eine (reelle) Lösung hat, besitzt das Polynom keine (reellen) Nullstellen.
    \item \emph{Bestimmen Sie alle Nullstellen des Polynoms $f_5(x)=x^5-3x^3+2x$.}
    Offensichtlicherweise gilt $x_1 = 0$ und wir können das Polynom mit Division durch $x$ vereinfachen. Mit der Substitution $z \equiv x^2$ erhalten wir 
    \begin{align}
        0 = z^2 -3z +2 \quad \Rightarrow \quad z_{1/2} = \frac{3}{2}\pm \sqrt{\frac{9}{4}-2} = \begin{cases}
            2 \\ 1
        \end{cases} \quad \Rightarrow \quad \uuline{x_{2/3} = \pm \sqrt{2}}, \quad \uuline{x_{4/5} = \pm 1}.
    \end{align}
    \item \emph{Bestimmen Sie die kleinste positive Nullstelle des Polynoms $\textstyle f_4(x)=1-\frac{x^2}{2}+\frac{x^4}{24}$. Setzen Sie näherungsweise $\textstyle \sqrt{3}\approx 3-\frac{\pi^2}{8}$.}
    Wir führen wieder die Substitution $z \equiv x^2$ aus und erhalten 
    \begin{align}
        0 = z^2 - 12 z + 24 \quad \Rightarrow \quad z_{1/2}= 6\pm2\sqrt{3} \qq{bzw.} x_{1/2/3/4} = \pm \sqrt{6\pm 2\sqrt{3}}.
    \end{align}
    Wir suchen nun das kleinste positive $x$, also 1.VZ: ``$+$'' und 2.VZ: ``$-$'' 
    \begin{align}
            x_0 = \sqrt{6-2\sqrt{3}} \overset{\sqrt{3}\approx 3-\frac{\pi^2}{8}}{\approx} \sqrt{6-\qty(3-\frac{\pi^2}{8})} = \uuline{\frac{\pi}{2}}.
    \end{align}
    \emph{Bemerkung}: Die Funktion $f_4(x)$ ist das dritte Taylorpolynom der Kosinusfunktion. Es gilt $\cos(x) = f_4(x) +\order{x^6}$.
\end{enumerate}

\paragraph{Aufgabe 2: } \emph{Polynomdivision}\\[0.2cm]
Berechnen Sie die folgenden Ausdrücke. Zusatz: Für welche Werte von $n$ bleibt die Polynomdivision in (d) ohne Rest?
\begin{enumerate}[label=(\alph*), labelindent=1em,labelsep=0.5cm]
    \item$~$\\[-1.3cm]
    \begin{align}
        \begin{array}{r@{} r@{} r@{} r@{} r}
            (21a^3 &{}+34a^2b\hp{)}&{}&{}+25b^3) &\;:(7a+5b) = \uuline{3a^2-7ab+5b^2} \\
          -(21a^3 &{}+15a^2b) \\ 
          \cmidrule{1-3}
                & -49a^2 b\hp{)} &{}&{}+25b^3\hp{)}\\
                &-(-49a^2 b\hp{)}&{}-35ab^2&{}\hp{+25b^3}) \\
          \cmidrule{2-4}
                & &{}35ab^2&{}+25b^3\hp{)}\\
                & &-(35ab^2&{}+25b^3)\\
          \cmidrule{3-4} 
                & & & 0
        \end{array}
    \end{align}
    \item$~$\\[-1.3cm]
    \begin{align}
        \begin{array}{r@{} r@{} r@{} r@{} r}
            (9x^3 &{}&{}-7xy^2\hp{)}&{}+2y^3) &\;:(3x-2y) = \uuline{3x^2+2xy-y^2} \\
          -(9x^3 &{}-6x^2y&{}) \\ 
          \cmidrule{1-3}
                & 6x^2y &{} -7xy^2 &{}+2y^3\hp{)}\\
                &-(6x^2y&{} -4xy^2&{}\hp{+2y^3}) \\
          \cmidrule{2-4}
                & &{}-3xy^2&{}+2y^3\hp{)}\\
                & -&(-3xy^2&{}+2y^3)\\
          \cmidrule{3-4} 
                & & & 0
        \end{array}
    \end{align}
    \item$~$\\[-1.3cm]
    \begin{align}
        \begin{array}{r@{} r@{} r@{} r@{} r@{} r}
            (25x^4 &{}&{}-a^2x^2\hp{)}&{}&{}+25a^4) &\;:(5x^2 
            + 7ax +5a^2) = \uuline{5x^2 -7ax +5a^2} \\
          -(-25x^4 &{}+35ax^3&{}+25a^2x^2) \\ 
          \cmidrule{1-3}
                & -35ax^3 &{} -24a^2x^2 &{}&{}+25a^4\hp{)}\\
                &-(-35ax^3&{} -49a^2x^2&{}-35a^3 x&{}\hp{+25a^4}) \\
          \cmidrule{2-5}
                & &{}25a^2 x^2&{}+35a^3x&{}+{25a^4}\hp{)}\\
                & &-(25a^2 x^2&{}+35a^3x&{}+{25a^4})\\
          \cmidrule{3-5} 
                & & & 0
        \end{array}
    \end{align}
    \item$~$\\[-1.3cm]
    \begin{align}
        \begin{array}{r@{} r@{} r@{} r}
            (x^2 &{} +2x \hp{)}&{}-15) &\;:(x+n) = \uuline{x+2-n + \frac{n(n-2)-15}{x+n}} \\[-3mm]
          -(x^2 &{}+nx) \\ 
          \cmidrule{1-2}
                & (2-n)x \hp{)}&{} -15\hp{)}\\
                &-((2-n)x\hp{)}&{} +n(2-n)) \\
          \cmidrule{2-3}
                & -15\hp{)}&{}+n(n-2) & \Rightarrow 0 \overset{!}{=} n^2 - 2n -15 \quad \Rightarrow \quad n_{1/2} = 1 \pm \sqrt{16}
        \end{array}
    \end{align}
    Wir erhalten damit die beiden Werte $n_{1/2}$ die den Wurzeln des quadratischen Gleichungssystems entsprechen 
    \begin{align}
        x^2 + 2x -15 = (x+5)(x-3).
    \end{align}
\end{enumerate}

\newpage
\paragraph{Aufgabe 3: } \emph{Kubische Gleichungen} \hfill Ziel: (a)\\[-4mm]
\begin{enumerate}[label=(\alph*)]
    \item \emph{Bestimmen Sie den Wert von $m$ in der Gleichung}
    \begin{align}
    6x^3-7x^2-16x+m=0\,, \quad \overset{x=2}{\Rightarrow} \quad -12 + m \overset{!}{=} 0 \quad \Rightarrow \quad \uuline{m=12}.
    \end{align}
    \emph{wenn eine Wurzel der Gleichung den Wert 2 hat. Berechnen Sie auch die beiden anderen Wurzeln.} Wir bestimmen die restlichen Wurzeln per Polynomdivision und $pq$-Formel: 
    \begin{align}
        &\begin{array}{r@{} r@{} r@{} r@{} r}
            (6x^3 &{}-7x^2\hp{)}&{}-16x\hp{)}&{}+12) &\;:(x-2) = \uline{6x^2+5x-6} \\
          -(6x^3 &{}-12x^2&{}) \\ 
          \cmidrule{1-3}
                & 5x^2 &{} -16x &{}+12\hp{)}\\
                &-(5x^2&{} -10x&{}\hp{+12}) \\
          \cmidrule{2-4}
                & &{}-6x&{}+12\hp{)}\\
                & -&(-6x&{}+12)\\
          \cmidrule{3-4} 
                & & & 0
        \end{array}\\
        & \Rightarrow \quad 0 \overset{!}{=} x^2 +\frac{5}{6}x -1 \quad \Rightarrow x_{1/2} = -\frac{5}{12} \pm \sqrt{\frac{25}{144}+1} \quad \Rightarrow \quad \uuline{x_1 = \frac{2}{3}, x_2 = -\frac{3}{2}}.
    \end{align}
    \item \emph{Die Zahlen 2 und 3 seien Wurzeln der Gleichung}
    \begin{align}
    2x^3+mx^2-13x+n=0\,.
    \end{align}
    \emph{Bestimmen Sie die Zahlenwerte von $m$ und $n$, und geben Sie die dritte Wurzel an.}
    Zu Bestimmung der Lösung setzen wir die beiden Wurzeln in die Gleichung ein: 
    \begin{align}
        &x_1 = 2: \quad 4m + n = \hp{-}10 \quad (1)\\
        &x_2 = 3: \quad 9m + n = -15 \quad (2)\\
        & \Rightarrow \quad (1)-(2): \quad -5m = 25 \quad \Rightarrow \quad \uuline{m= -5}, \quad \uuline{n=30}.
    \end{align}
    Die dritte Wurzel ermitteln wir durch Polynomdivision mit $(x-2)(x-3) = x^2 - 5x+6$
    \begin{align}
        \begin{array}{r@{} r@{} r@{} r@{} r}
            (2x^3 &{}-5x^2\hp{)}&{}-13x\hp{)}&{}+30) &\;:(x^2-5x+6) = 2x+5 \quad \Rightarrow  \quad \uuline{x_3 = -\frac{5}{2}}\\
          -(2x^3 &{}-10x^2&{}+12x) \\ 
          \cmidrule{1-3}
                & 5x^2 &{} -25x &{}+30\hp{)}\\
                &-(5x^2&{} -25x&{}+30)\\
          \cmidrule{2-4} 
                & & & 0
        \end{array}\\
    \end{align}
\end{enumerate}

\newpage
\paragraph{Aufgabe 4: } \emph{Nullstellenraten} \hfill (Zusatzaufgabe)\\[0.2cm]
\emph{Finden Sie jeweils mindestens eine Nullstelle der folgenden Ausdrücke und spalten Sie diese als Linearfaktor $(x-x_0)$ vom Polynom ab.}
    \begin{enumerate}[label=(\alph*), labelindent=1em,labelsep=0.5cm]
        \item $x^3-5x^2+8x-4, \quad \uuline{x_0=1}$ oder $\uuline{x_0=2}$ \\
        $= (x-1)(x^2-4x+4) = (x-1)(x-2)^2$ \quad (stückweises konstruieren ohne Polynomdivision)
        \item $x^4-2x^3-13x^2+9x+9, \quad \uuline{x_0=3}$ \hfill(einzige reelle Nullstelle)\\
        $=(x+3)(x^3-5x^2+2x+3)$
        \item $x^4-3x^2+3x+2, \quad \uuline{x_0=-2}$ \hfill(einzige reelle Nullstelle) \\
        $=(x+2)(x^3-2x^2+x+1)$
        \item $x^5-x^4-3x^3+3x^2+x-1, \quad \uuline{x_0 = 1}$\\[-3mm]
        $=(x-1)(x^4-3x^2+1)$ \hfill die anderen Nullstellen sind: $x_{2/3/4/5} = \frac{\pm 1 \pm \sqrt{5}}{2}$\\
        $= (x-1)(x^2-x-1)(x^2+x-1)$
    \end{enumerate}
%
\paragraph{Aufgabe 5: } \emph{Partialbruchzerlegung} \hfill Ziel: (a) bis (c)\\[0.2cm]
Schreiben Sie, so weit möglich, als Summe von Partialbrüchen.
\begin{enumerate}[label=(\alph*)]
    \item $\frac{x-5}{x^2-2x-3} \overset{!}{=} \frac{\alpha}{x+1} + \frac{\beta}{x-3} = \frac{(x-3)\alpha + (x+1)\beta}{x^2-2x -3} = \frac{(\alpha+\beta)x +\beta - 3\alpha}{x^2-2x-3}$\\[2mm]
    Koeffizientenvergleich im Zähler: $\quad\begin{rcases}
        x^1: \quad\alpha+\beta = 1 \\ x^0: \quad3\alpha - \beta = 5
    \end{rcases}$ Addition: $\uline{\alpha = \frac{3}{2}}, \quad \uline{\beta = -\frac{1}{2}}$ \\
    $\Rightarrow \uuline{\frac{x-5}{x^2-2x-3} = \frac{3}{2(x+1)} - \frac{1}{2(x-3)}}$
    \item $\frac{x^2+1}{x^2-1} \overset{!}{=} \frac{\alpha}{x+1} + \frac{\beta}{x-1} + \gamma = \frac{(x-1)\alpha + (x+1)\beta + (x^2-1)\gamma}{x^2-1}$\\[2mm]
    Koeffizientenvergleich im Zähler: $\quad\begin{rcases}
        x^2: \quad\uline{\gamma = 1} \\ x^1: \quad\alpha + \beta = 0 \\ x^0: \quad1 = \beta - \alpha - \gamma
    \end{rcases} \quad \uline{\beta = 1}, \uline{\alpha = -1}$\\
    $\Rightarrow \uuline{\frac{x^2+1}{x^2-1} = 1 - \frac{1}{x+1}+\frac{1}{x-1}}$\\
    Alternativ kann auch eine Poylnomdivision mit Rest durchgeführt werden mit anschließender Partialbruchzerlegung des Restglieds:
    $\textstyle\frac{x^2}{x^2-1} = 1 + \frac{2}{x^2-1}.$
    \item $\frac{2x^2-3x+1}{x^3-5x^2+8x-4} \overset{!}{=} \frac{\alpha}{x-1} + \frac{\beta}{x-2} + \frac{\gamma}{(x-2)^2} = \frac{(x^2-4x+4)\alpha + (x^2-3x+2)\beta + (x-1)\gamma}{(x-1)(x-2)^2}$\\[2mm]
    Koeffizientenvergleich: $\quad\begin{rcases}
        x^2: \quad\alpha + \beta = 2 \\
        x^1: \quad-4\alpha - 3\beta + \gamma = -3 \quad (2) \\
        x^0: \quad 4\alpha +2\beta - \gamma = \hp{-}1 \quad (3)
    \end{rcases} \quad (2) + (3): \quad \beta = 2 \Rightarrow \uline{\alpha = 0}, \uline{\gamma = 3}$\\
    $\Rightarrow \uuline{\frac{2x^2 - 3x + 1}{x^3 - 5x^2 +8x-4} = \frac{2}{x-2} + \frac{3}{(x-2)^2}}$\\
    Die Faktorisierung des Nenners haben wir in Aufgabe 4a bereits gesehen. Wir müssen die Vielfachheit der Nullstelle in unserem Ansatz berücksichtigen. Durch $\alpha = 0$ sehen wir, dass die rechte Seite bei $x=1$ keine Pollstelle hat. Die Partialbruchzerlegung hat uns also den Limes $x\to 1$ der linken Seite verschafft.
    \item $\frac{2x^4-4x^3-5x^2+(\sqrt{2}-7)x+\sqrt{2}+12}{x^2-2x-3}$\\
    Da der Zähler von höherem Grade ist als das Nennerpolynom, wird zunächst eine Polynomdivision durchgeführt: 
    \begin{align}
        \begin{array}{r@{} r@{} r@{} r@{} r@{} r}
            (2x^4 &{} -4x^3 &{}-5x^2\hp{)}&{}+(\sqrt{2}-7)x&{}+\sqrt{2}+12) &\;:(x^2 - 2x-3) = \uline{2x^2+1+ \frac{(\sqrt{2}-5)x + \sqrt{2}+15}{x^2-2x-3}} \\[-3mm]
          -(-2x^4 &{}-4x^3&{}-6x^2) \\ 
          \cmidrule{1-3}
                &{} &{} x^2 &{} +(\sqrt{2}-7)x &{} +\sqrt{2} + 12\hp{)}\\
                &{} &{}-(x^2&{} -2x&{}-3) \\
          \cmidrule{3-5}
                &{} &{} &{} (\sqrt{2}-5)x &{}+\sqrt{2} + 15
        \end{array}
    \end{align}
    Die Nullstellen des Nenners lauten: $x_1 = 3, x_2 = -1$. Damit folgt 
    \begin{align}
        &\frac{(\sqrt{2}-5)x + \sqrt{2}+15}{x^2-2x-3} \overset{!}{=} \frac{\alpha}{x-3} + \frac{\beta}{x+1} = \frac{(x+1)\alpha + (x-3)\beta}{x^2-2x-3}\\
        &\text{Koeffizientenvergleich:} \quad\begin{rcases}
            x^1: \alpha + \beta = \sqrt{2} -5 \\
            x^0: \alpha - 3\beta = \sqrt{2} + 15
        \end{rcases} \quad \text{Differenz:} \quad \uline{\beta = -5}, \quad\Rightarrow \uline{\alpha = \sqrt{2}}\\
        &\Rightarrow \uuline{\frac{2x^4-4x^3-5x^2+(\sqrt{2}-7)x+\sqrt{2}+12}{x^2-2x-3} = 2x^2 + 1 + \frac{\sqrt{2}}{x-3} - \frac{5}{x+1}}
    \end{align}
\end{enumerate}
%

% \paragraph{Aufgabe 6: } \emph{Polynome in Ungleichungen}\\[0.2cm]
% Beweisen Sie die folgenden Ungleichungen.
% \begin{enumerate}[label=(\alph*)]
%     \item $(1+a+a^2)^2<3(1+a^2+a^4)\,,\qq{für} a\in\mathbb{R}\!\smallsetminus\!\{1\}$
%     \item $x^4-x^2-6x+10>0\,,\qq{für} x\in\mathbb{R}$
% \end{enumerate}

\paragraph{Aufgabe 6: } \emph{Summen}\hfill Ziel: (a) bis (b)\\[0.2cm]
Vereinfachen bzw. berechnen Sie die folgenden Summen. 

\begin{enumerate}[label=(\alph*), labelindent=1em,labelsep=0.5cm]
    \item $\underbrace{\sum_{k=1}^n \frac{1}{2k}}_{\displaystyle =\frac{1}{2}+\frac{1}{4} + \frac{1}{2n}} + \underbrace{\sum_{l=0}^{n-1} \frac{1}{2l+1}}_{\displaystyle 1+\frac{1}{3}+\frac{1}{5}+\hdots+\frac{1}{2n-1}} = 1 + \frac{1}{2} + \frac{1}{3} + \frac{1}{4} + \hdots \frac{1}{2n} = \uuline{\sum_{k=0}^{2n} \frac{1}{k}}$
    \item $\sum_{k=1}^n \frac{1}{k(k+1)}$
    \item $\sum_{k=0}^n x^k$ 
\end{enumerate}

\emph{Hinweis zu (b)}: Der Term in der Summe kann mithilfe von Partialbruchzerlegung vereinfacht und die entstehende Summe auseinandergezogen werden. 
\begin{align}
    \frac{1}{k(k+1)} = \frac{A}{k} + \frac{B}{k+1} \quad \Rightarrow \quad 1 = A(k+1) + Bk = (A+B) k + A.
\end{align}
Der Koeffizientenvergleich liefert $A=1, B=-1$ und wir können weiter vereinfachen 
\begin{align}
    \sum_{k=1}^n \frac{1}{k(k+1)} &= \sum_{k=1}^n \frac{1}{k} - \sum_{k=1}^n \frac{1}{k+1} \qq{Indexverschiebung} k+1 = l \\
    & =\sum_{k=1}^n \frac{1}{k} - \sum_{l=2}^{n+1} \frac{1}{l} = \qty(1 + \cancel{\sum_{k=2}^n}) - \qty(\cancel{\sum_{l=2}^n} + \frac{1}{n+1}) = \uuline{\frac{n}{n+1}}.
\end{align}

\emph{Hinweis zu (c)}: Hierfür kann ein expliziter Ausdruck gefunden werden, wenn man die Formel mit $(1-x)$ multipliziert und analog vorgeht wie in (b) 
\begin{align}
        (1-x) \sum_{k=0}^n x^k &= \sum_{k=0}^n x^k - x \sum_{k=0}^n x^k = \sum_{k=0}^n x^k - \sum_{k=0}^n x^{k+1} \qq{Indexverschiebung} l = k+1 \\
        &= \sum_{k=0}^n x^k - \sum_{l=1}^{n+1} x^l = 1 + \cancel{\sum_{k=1}^n x^k} - \qty(\cancel{\sum_{l=1}^n x^l} + x^{n+1}) = 1 - x^{n+1} \\
        \Rightarrow \quad \sum_{k=0}^n x^k &= \uuline{\frac{1 - x^{n+1}}{1-x}}.
\end{align}