\Titelbanner{6}{Trigonometrische Funktionen\\Ebene Trigonometrie}

\paragraph{Wiederholung Trigonometrische Funktionen}$~$

\begin{mymathbox}[ams align, title={Additionstheoreme}, colframe={FSUblau}]
    \sin(x\pm y) &= \sin(x)\cos(y) \pm \cos(x)\sin(y) \\
    \cos(x\pm y) &= \cos(x)\cos(y) \mp \sin(x)\sin(y).
\end{mymathbox}
Weitere wichtige Relationen, die nützlich sein können: 
\begin{alignat}{3}
    \cos(x) &= \sin(x+\frac{\pi}{2}) \quad & \sin(-x) &= - \sin(x)\\
    \sin(x) &= \cos(x-\frac{\pi}{2}) \quad & \cos(-x) &= \cos(x) \\
    \sin(\frac{\pi}{2}) &= 1 \quad \cos(\frac{\pi}{2}) = 0 & \sin(0) &= 0 \quad \cos(0) &= 1\\
    \sin(\frac{\pi}{4}) &= \cos(\frac{\pi}{4}) = \frac{\sqrt{2}}{2} &\qquad  \sin^2(x) + \cos^2(x) &= 1, \quad \tan^2(x) +1 = \frac{1}{\cos^2(x)}.
\end{alignat}

\paragraph{Aufgabe 1: } \emph{Additionstheoreme}\\[-1em]
\begin{enumerate}[label=(\alph*)]
\item Leiten Sie das Additionstheorem für Kosinusfunktionen aus dem für Sinusfunktionen her.
\begin{align*}
    \cos(x\pm y)&=\sin\qty(x\pm y+\frac{\pi}{2})=\sin(x\pm z^{\pm}), && z^{\pm}\equiv y\pm\frac{\pi}{2}\\
    &=\sin(x)\cos(z^\pm)\pm\cos(x)\sin(z^\pm)\\
    &=\sin(x)\underbrace{\cos\qty(y\pm\frac{\pi}{2})}_{\mp\sin(y)}\pm\cos(x)\underbrace{\sin\qty(y\pm\frac{\pi}{2})}_{\pm\cos(y)}\\
    &=\cos(x)\cos(y)\mp\sin(x)\sin(y)
\end{align*} 
\item Leiten Sie das Additionstheorem für Tangensfunktionen her,
\begin{align*}
    \tan(x\pm y)&=\frac{\sin(x\pm y)}{\cos(x\pm y)}=\frac{\sin(x)\cos(y)\pm\cos(x)\sin(y)}{\cos(x)\cos(y)\mp\sin(x)\sin(y)}=\frac{\bcancel{\cos(x)\cos(y)}\qty(\frac{\sin(x)}{\cos(x)}\pm\frac{\sin(y)}{\cos(y)})}{\bcancel{\cos(x)\cos(y)}\qty(1\mp\frac{\sin(x)\sin(y)}{\cos(x)\cos(y)})}\\
    &=\frac{\tan(x)\pm\tan(y)}{1\mp\tan(x)\tan(y)}
    \end{align*}
\item \emph{Zeigen Sie, dass für Doppelwinkelfunktionen gilt:}
\begin{itemize}
    \item $\sin(2x)=\sin(x+x)=\sin(x)\cos(x)+\cos(x)\sin(x)=2\sin(x)\cos(x)$
    \item $\cos(2x)=\cos(x+x)=\cos^2(x)-\underbrace{\sin^2(x)}_{1-\cos^2(x)}=2\cos^2(x)-1$
\end{itemize}
\end{enumerate}

\paragraph{Aufgabe 2: } \emph{Trigonometrische Umformungen I}\hfill Ziel: (a) bis (c)\\[0.2cm]
\emph{Zeigen Sie die Richtigkeit der folgenden Identitäten.}

\emph{Tafelbeispiel:}
\begin{enumerate}
    \item Identität: $\frac{1}{\cos^2(\alpha)} = 1 + \tan^2(\alpha)$ \\
    $\Rightarrow \frac{1}{\cos^2(\alpha)} = \frac{\sin^2(\alpha) + \sin^2(\alpha)}{\cos^2(\alpha)} = \tan^2(\alpha) +1.$\hfill$\Box$
    \item $\sin(2x) = \frac{2\tan(x)}{1 + \tan^2(x)}$ \\
    $\Rightarrow \sin(2x) = 2 \sin(x)\cos(x) = 2 \tan(x) \cos^2(x) = \frac{2 \tan(x)}{1+\tan^2(x)}$\hfill$\Box$
\end{enumerate}
\emph{Lösung:}
\begin{enumerate}[label=(\alph*)]
    \setlength{\mathindent}{0cm}
    \item $\tan\qty(\frac{\pi}{4}+\alpha)=\frac{\sin\qty(\frac{\pi}{4}+\alpha)}{\cos\qty(\frac{\pi}{4}+\alpha)}=\frac{\cos(\frac{\pi}{4})\sin\alpha+\sin(\frac{\pi}{4})\cos\alpha}{\cos(\frac{\pi}{4})\cos\alpha-\sin(\frac{\pi}{4})\sin\alpha}=\frac{\cancel{\sqrt{2}}}{\cancel{\sqrt{2}}}\frac{\cos\alpha+\sin\alpha}{\cos\alpha-\sin\alpha}$\hfill$\Box$
    \item$~\vphantom{\frac{\frac{1}{1}}{\frac{1}{1}}}$\\[-1.5cm]
    \begin{align}
        \frac{1+\sin 2\alpha}{\cos 2\alpha}=\frac{1+2\sin\alpha\cos\alpha}{\cos^2\alpha-\sin^2\alpha}=\frac{(\cos\alpha+\sin\alpha)^{\cancel{2}}}{\cancel{(\cos\alpha+\sin\alpha)}(\cos\alpha-\sin\alpha)} \overset{(a)}&{=} \tan(\frac{\pi}{4}+\alpha) \\
        &= \frac{\cancel{\cos(\alpha)}}{\cancel{\cos(\alpha)}} \frac{1+ \tan(\alpha)}{1-\tan(\alpha)}
    \end{align}
    \item$~$\\[-1.3cm]
    \begin{align}
        2\cos\qty(\frac{x+y}{2})\cos\qty(\frac{x-y}{2})&=2\qty(\cos\frac{x}{2}\cos\frac{y}{2}-\sin\frac{x}{2}\sin\frac{y}{2})\qty(\cos\frac{x}{2}\cos\frac{y}{2}+\sin\frac{x}{2}\sin\frac{y}{2})\\
        &=2\bigg(\cos^2\frac{x}{2}\cos^2\frac{y}{2}-\underbrace{\sin^2\frac{x}{2}\sin^2\frac{y}{2}}_{\qty(1-\cos^2 \frac{x}{2})\qty(1-\cos^2 \frac{y}{2})}\bigg)=2\qty(\cos^2\frac{x}{2}+\cos^2\frac{y}{2}-1)\\
        &=2\cos^2 \frac{x}{2} -1 + 2\cos^2 \frac{y}{2}-1 \overset{(1c)}{=} \cos(x)+\cos(y)
    \end{align} 
    \item $\cot\alpha\cot\beta+\cot\alpha\cot\gamma+\cot\beta\cot\gamma=\frac{\overbrace{\cos\alpha\cos\beta\sin\gamma+\cos\alpha\cos\gamma\sin\beta}^{\cos\alpha\sin(\beta+\gamma)}+\cos\beta\cos\gamma\sin\alpha}{\sin\alpha\sin\beta\sin\gamma}\\[0.2cm]
    =\frac{\cos\alpha\sin(\beta+\gamma)+\sin\alpha[\cos(\beta+\gamma)+\sin\beta\sin\gamma]}{\sin\alpha\sin\beta\sin\gamma}=\frac{\sin(\alpha+\beta+\gamma)}{\sin\alpha\sin\beta\sin\gamma}+1=1$\hfill$\Box$
\end{enumerate}

\paragraph{Aufgabe 3: } \emph{Trigonometrische Umformungen II}\hfill Ziel (a) und (b)\\[0.2cm]
\emph{Formen Sie die folgenden Ausdrücke so um, dass sie sich einfach logarithmieren lassen. Das heißt, die Terme sollen möglichst in Produkte, Quotienten und Potenzen umgeformt werden.}\\[-1.3em]

\emph{Tafelbeispiel:}\\
Es ist mit ``einfach logarithmierbar'' gemeint, dass möglichst in Produkte, Quotienten und Potenzen umgeformt werden soll.
\begin{align}
    \cot(\alpha) + \cot(\beta) &= \frac{\cos(\alpha)}{\sin(\alpha)} + \frac{\cos(\beta)}{\sin(\beta)} = \frac{\cos(\alpha)\sin(\beta)+\sin(\alpha)\cos(\alpha)}{\sin(\alpha)\sin(\beta)} \\
    &= \frac{\sin(\alpha+\beta)}{\sin(\alpha)\sin(\beta)} \quad \longrightarrow \qq{Ziel erreicht.}
\end{align}
\emph{Lösung:}
\begin{enumerate}[label=(\alph*)]
    \setlength{\mathindent}{0cm}
    \item$~$\\[-1.3cm]
    \begin{align}
        1+\cos(\alpha) + \cos\qty(\frac{\alpha}{2})&, \qq{\emph{Hinweis:} Es ist} \cos\qty(\frac{\pi}{3})=\frac{1}{2}\, \\
        1+\cos(\alpha) + \cos\qty(\frac{\alpha}{2})&= 2 \cos^2\qty(\frac{\alpha}{2}) + \cos(\frac{\alpha}{2}) = 2 \cos(\frac{\alpha}{2})\qty(\cos(\frac{\alpha}{2}+\cos(\frac{\pi}{3}))) \\
        \overset{(2c)}&{=} \uuline{4 \cos(\frac{\alpha}{2})\cos(\frac{\alpha}{4}+\frac{\pi}{6})\cos(\frac{\alpha}{4}-\frac{\pi}{6})}
    \end{align}
    \item$~$\\[-1.3cm]
    \begin{align}
        \frac{2\sin(\beta) - \sin (2\beta)}{2\sin(\beta)+2\sin(2\beta)} &= \frac{2\cancel{\sin(\beta)}-2\cancel{\sin(\beta)}\cos(\beta)}{2\cancel{\sin(\beta)}+4\cancel{\sin(\beta)}\cos(\beta)} = \frac{1-\cos(\beta)}{1+2\cos(\beta)} \\
        &= \frac{2-2\cos^2\qty(\frac{\beta}{2})}{2\qty(\cos(\frac{\pi}{3}+\cos(\beta)))} \overset{(2c)}{=} \uuline{\frac{\sin^2\qty(\frac{\beta}{2})}{2 \cos(\frac{\beta}{2}+\frac{\pi}{6})\cos(\frac{\beta}{2}-\frac{\pi}{6})}}.
    \end{align}
    \item$~$\\[-1.3cm]
    \begin{align}
        &\hp{=}\sin(\alpha)+\sin(\beta)+\sin(\gamma), \qq{für} \alpha+\beta+\gamma = \pi \\
        &= \sin(\alpha) + \sin(\beta) + \sin(\alpha+\beta) \\
        &= \sin(\alpha) + \sin(\beta) + \sin(\alpha)\cos(\beta) + \sin(\beta)\cos(\alpha) \\
        &= \sin(\alpha) (1+\cos(\beta)) + \sin(\beta)(1+\cos(\alpha)) \\
        &= 2 \sin(\alpha)\cos^2\qty(\frac{\beta}{2}) + 2 \sin(\beta) \cos^2\qty(\frac{\alpha}{2}) \\
        &= 4 \sin(\frac{\alpha}{2})\cos(\frac{\alpha}{2}) \cos^2(\frac{\beta}{2}) + 4 \sin(\frac{\beta}{2})\cos(\frac{\beta}{2}) \cos^2\qty(\frac{\alpha}{2}) \\
        &= 4 \cos(\frac{\alpha}{2})\cos(\frac{\beta}{2})\qty(\sin(\frac{\alpha}{2})\cos(\frac{\beta}{2})+ \cos(\frac{\alpha}{2})\sin(\frac{\beta}{2}))\notag \\
        &= \uuline{4 \cos(\frac{\alpha}{2})\cos(\frac{\beta}{2})\sin(\frac{\alpha+\beta}{2})}
    \end{align}
\end{enumerate}
%
\newpage
\paragraph{Aufgabe 4: } \emph{Goniometrische Gleichungen und Gleichungssysteme}\hfill Ziel: (a) bis (c)\\[0.2cm]
\emph{Bestimmen Sie alle Lösungen der folgenden Gleichungen und machen Sie jeweils die Probe.}

\emph{Tafelbeispiel:}
\begin{align}
    &\begin{split}
        \cos(x) \sin(y) &= 1 \\
        \cos(x) - \sin(y) &= \frac{1}{2}
    \end{split} \qquad \overset{z \equiv -y}{\Longrightarrow} \qquad \begin{rcases}
        \cos(x) \sin(z) = -\frac{1}{2} \\
        \cos(x)+\sin(z) = \frac{1}{2}
    \end{rcases} \qq{Vieta!}
\end{align}
\begin{align}
    &\cos(x) \qq{und} \sin(z) \qq{sind Wurzeln des Polynoms} \\
    & \xi^2 - \frac{1}{2}\xi - \frac{1}{2} = 0 \quad \Longrightarrow \quad \xi_{1/2} = \frac{1}{4}\pm \sqrt{\frac{1}{16}+\frac{1}{2}} = \begin{cases}
        1 \\ -1/2
    \end{cases} \\
    & \text{Fall 1: } \cos(x)  =1, \sin(y) = 1/2 \\
    & \qquad \Rightarrow \uuline{x= 2\pi k}, \quad \uuline{y = \frac{\pi}{6} + 2\pi k} \qq{oder} \uuline{y = \frac{5\pi}{6} + 2\pi k}, \quad k \in \mathbb{Z} \\
    &\text{Fall 2: } \cos(x) = 1/2, \sin(y) = -1 \\
    & \qquad \Rightarrow \uuline{x = \frac{2\pi}{3} + 2\pi k} \qq{oder} \uuline{x =-\frac{2\pi}{3} + 2\pi k} \\
    & \qquad \Rightarrow \uuline{y = \frac{3\pi}{2} + 2\pi k} \qq{oder} \uuline{y = -\frac{\pi}{2}+2\pi k}, \quad k \in \mathbb{Z}.
\end{align}
Natürlich kann auch die implizite Form stehen gelassen werden!

\emph{Lösung:}
\begin{enumerate}[label=(\alph*)]
    \setlength{\mathindent}{0cm}
    \item $\sin(x)+\cos(x)= 1 \quad |()^2$
    \begin{align}
        \sin^2(x)+\cos^2(x)& + 2\sin(x)\cos(x) = 1 \quad \Rightarrow \quad \sin(x)\cos(x) = 0 \\
        \text{Fall 1: } &\sin(x) = 0 \quad \Rightarrow \quad x = \pi k, k\in\mathbb{Z} \\
         &\text{allerdings muss } \cos(x) = 1 \quad \Rightarrow \qq{nur für $k$ gerade} \\
         & \Rightarrow \uuline{x_1 = 2\pi k}, k \in \mathbb{Z} \\
         \text{Fall 2: } &\cos(x) = 0 \quad \Rightarrow \quad x = \frac{\pi}{2} + \pi k, k\in\mathbb{Z} \\
         &\text{allerdings muss } \sin(x) = 1 \quad \Rightarrow \qq{nur für} \frac{\pi}{2} + 2\pi k \\
         & \Rightarrow \uuline{x_2 = \frac{\pi}{2} + 2\pi k},  k \in \mathbb{Z} 
    \end{align}
    \item$~$\\[-1.5cm]
    \begin{align}
        &\underbrace{\cos\qty(\frac{x+y}{2})\cos\qty(\frac{x-y}{2})}_{2 \cos(x)\cos(y)}=\frac{1}{2}\,, \hspace{0.5cm}\cos(x)\cos(y)=\frac{1}{4} \\
        &\text{Vieta: } \xi^2 - \xi + \frac{1}{4} = 0 \quad \Rightarrow \quad \xi_{1/2} = \frac{1}{2} \pm \sqrt{\frac{1}{4}-\frac{1}{4}} = \frac{1}{2} \\
        &\Rightarrow \uuline{\cos(x) = \frac{1}{2} = \cos(y)} \qq{bzw.} x = \pm \frac{\pi}{3} + 2\pi k, y = \pm \frac{\pi}{3} + 2\pi l, \quad k,l \in \mathbb{Z}
    \end{align}
    \item $\sin(3x)=\cos(2x)$
    \begin{align}
        \sin(3x) &= \sin(2x+x) = \sin(2x)\cos(x) + \cos(2x)\sin(x) \\
        &= 2 \sin(x)\cos^2(x) + (2\cos^2(x)-1) \sin(x) \\
        &= 4 \sin(x) \cos^2(x) - \sin(x) \\
        \cos(2x) &= 2 \cos^2(x) -1 \\
        0&= 2\cos^2(x)(1-2\sin(x)) + \sin(x)-1 \\
        &= 2(1-\sin^2(x))(1-2\sin(x)) + \sin(x) - 1.
    \end{align}
    Wir substituieren nun $u \equiv \sin(x)$:
    \begin{align}
        0 = 4u^3 - 2u^2 -3u +1 \quad \Rightarrow \qq{draufschauen} u = 1
    \end{align}
    Die restlichen Lösungen erhalten wir durch Polynomdivision: 
    \begin{align}
        &\begin{array}{r@{} r@{} r@{} r@{} r}
            (4u^3 &{}-2u^2\hp{)}&{}-3u&{}+1) &\;:(u-1) = 4u^2+2u-1 \\
          -(4u^3 &{}+4u^2) \\ 
          \cmidrule{1-3}
                & 2u^2 \hp{)} &{}-3u\hp{)}&{}\\
                &-(2u^2 \hp{)}&{}-2u)&{} \\
          \cmidrule{2-4}
                & &{}-u&{}+1\hp{)}\\
                & &-(-u&{}+1)\\
          \cmidrule{3-4} 
                & & & 0
        \end{array}\\
        & \Longrightarrow 0 = u^2 + \frac{1}{2}u-\frac{1}{4} \quad \Rightarrow \quad u_{1/2} = - \frac{1}{4}\pm \sqrt{\frac{1}{16}+\frac{1}{4}} = \frac{-1 \pm \sqrt{5}}{4} \\
        & \Longrightarrow \uuline{\sin(x_1) = 1, \quad \sin(x_{2/3}) = \frac{-1 \pm \sqrt{5}}{4}} \\
        & \text{aufgelöst nach $x$: } \mathbb{L} = \qty{\frac{\pi}{2}+2\pi k, \frac{\pi}{10}+2\pi k, \frac{9\pi}{10}+2\pi k, -\frac{3\pi}{10}+2\pi k, -\frac{7\pi}{10} + 2\pi k |k \in \mathbb{Z} }
    \end{align}
    \item$~$\\[-1.5cm] 
    \begin{align}
        a\qty(3\cos^2(x)+\sin(x)\cos(x))-b\qty(3\sin^2(x)-\sin(x)\cos(x))&=2a-b \\
        3a \cos^2(x) - 3b \sin^2(x) + (a+b)\sin(x)\cos(x) &= 2a-b \quad |\; :\cos^2(x)\\
        3a - 3b \tan^2(x) + (a+b) \tan(x) &= \frac{2a-b}{\cos^2(x)} = (2a-b) (1+\tan^2(x)).  
    \end{align}
    Wir substituieren jetzt $z \equiv \tan(x)$ 
    \begin{align}
        0&= 2(a+b)z^2 - (a+b)z -(a+b) \\
        0&= z^2 - \frac{1}{2}z - \frac{1}{2} \quad \Rightarrow z_{1/2} = \frac{1}{4} \pm \sqrt{\frac{1}{16}+\frac{1}{2}} = \begin{cases}
            1 \\ -1/2
        \end{cases} \\
        \Rightarrow \quad& \uuline{\tan(x_1) = 1, \tan(x_2) = -\frac{1}{2}} 
    \end{align}
\end{enumerate}

\newpage
\paragraph{Aufgabe 5: } \emph{Dreiecksfläche}\\[0.2cm]
\emph{Berechnen Sie die Fläche eines Dreiecks, wenn die Seiten $a$ und $b$ sowie die Länge $w$ der Winkelhalbierenden des Winkels zwischen diesen Seiten gegeben sind.}

\begin{wrapfigure}{r}{6cm}
    \centering
    \vspace{-5mm}
    \begin{tikzpicture}
        \coordinate (A) at (-4,0);
        \coordinate (B) at (1.5,0);
        \coordinate (C) at (0,3);
        \draw[thick] (A) -- (B) --node[right]{$a$} (C) --node[above]{$b$} cycle; 
        \draw[thick] (C) --node[left]{$w$} (-0.8,0);
        \draw[thick] ($(C)+(216:1.3)$) arc (215:295:1.3);
        \node (A) at ($(C)+(215+20:1)$){$\gamma/2$};
        \node (A) at ($(C)+(295-20:1)$){$\gamma/2$};
        \node (A) at (-1.7,0.7){\textbf{I}};
        \node (A) at (0.2,0.7){\textbf{II}};
        % \draw[thick] ($(C)+(216:1.6)$) arc (215:255:1.6);
    \end{tikzpicture}
    \vspace{-5mm}
\end{wrapfigure}

\begin{align}
    &\text{Fläche in I:} \quad A_1 = \frac{bw}{2} \sin(\frac{\gamma}{2})\\
    &\text{Fläche in II:} \quad A_2 = \frac{aw}{2} \sin(\frac{\gamma}{2})\\
    &\text{Gesamtfläche:} \quad A = \frac{ab}{2} \sin(\gamma) \\
    &\text{andererseits: } \quad A= A_1 + A_2  \quad \Rightarrow \quad \frac{ab}{2}\cancel{\sin\gamma} = \frac{(a+b)w}{2} \sin(\frac{\gamma}{2}) = \frac{(a+b)w}{4} \frac{\cancel{\sin\gamma}}{\cos(\frac{\gamma}{2})} \\
    &\Rightarrow \cos(\frac{\gamma}{2}) = \frac{(a+b)w}{2ab} \\
    & \Rightarrow A = \frac{(a+b)w}{2} \sin(\frac{\gamma}{2}) =\frac{(a+b)w}{2}\sqrt{1-\cos^2\qty(\frac{\gamma}{2})} = \uuline{\frac{(a+b)w}{2}\sqrt{1-\frac{(a+b)^2 w^2}{4a^2b^2}}}.
\end{align}