\section{Umgang mit beliebigen Potenzen}

Bisher habne wir uns auf Gleichungen/Funktionen beschränkt, deren Variablen höchstens in erster oder zweiter Potenz aufgetreten sind. Nun sollen Methoden zum Umgang mit beliebigen (ganzzahligen) Potenzen behandelt werden. 

\subsection{Polynome und Polynomdivision}

Ein \emph{Polynom} $n$\emph{-ten Grades} ist eine Funktion der Form 
\begin{align}
    f_n(x) = a_n x^n + a_{n-1} x^{n-1} + \hdots + a_2 x^2 + a_1 x + a_0
\end{align}
mit den reellen Konstanten $a_i,\; i = 0,1,\hdots,n$. Die Nullstellen der Funktion $f_n(x)$ werden auch \emph{Wurzeln} des Polynoms genannt; ein Polynom $n$-ten Grades besitzt höchstens $n$ reelle Wurzeln. 

Besitzt ein Polynom $f_n(x)$ genau $n$ reelle Wurzeln, dann kann es als Produkt von Linearfaktoren geschrieben werden (vergleiche den Fall $n=2$ mit dem Satz von Vieta), 
\begin{align}
    f_n(x) = a_n (x-x_1)(x-x_2)\hdots (x-x_{n-1})(x-x_n),
\end{align}
mit Wurzeln $x_i, \; i=1,2,\hdots,n$. Demnach kann eine Wurzel gemäß $f_n(x) = (x-x_n)f_{n-1}(x)$ aus einem Polynom abgespalten werden, wobie $f_{n-1}(x)$, bei bekanntem $x_n$, mit Hilfe der \emph{Methode der Polynomdivision} zu bestimmten ist,
\begin{align}
    f_{n-1}(x) = f_n(x) : (x-x_n).
\end{align}
Möchte man einen unbekannten Linearfaktor abspalten, so ist ein $x_i$ zu erraten.

Wir wollen jetzt das Verfahren der Polynomdivision am Beispiel folgender Funktion diskutieren: 
\begin{align}
    f_3(x) = x^3 - 5x^2 + 8x -4, \qquad x_3 = 2.
\end{align}
Der Algorithmus für die schriftliche Polynomdivision besteht aus drei Schritten:
\begin{enumerate}
    \item Division: Man dividiere das Glied der höchsten Potenz des Zählerpolynoms durch das Glied der höchsten Potenz des Nennerpolynoms und schreibt das Ergebnis neben dem Gleichheitszeichen auf. 
    \item Multiplikation: Man multipliziert das Ergebnis von Schritt 1 mit dem Nennerpolynom und schreibt das Ergebnis unter das Zählerpolynom. 
    \item Subtraktion: Man subtrahiert das Ergebnis von Schritt 2 vom Zählerpolynom und beginnt wieder von vorne.
\end{enumerate}

\begin{align}
    \qq{Ergebnis: }\begin{array}{r@{} r@{} r@{} r r}
        x^3 &{}+5x^2\hphantom{)}&{}+8x\hphantom{)}&-4\hphantom{)} &:(x-2) = x^2 -3x +2 \\
      -(x^2 &{}-2x^2) &\\
      \cmidrule{1-2}
            & -3x^2\hphantom{)} &{}+8x\hphantom{)}&-4\hphantom{)}\\
            &-(-3x^2\hphantom{)} &{}+6x\hphantom{)}&\hphantom{-4}) \\
      \cmidrule{2-4}
            & &{}2x&-4\hphantom{)}\\
            & &-(2x&-4)\\
      \cmidrule{3-4} 
            & & & 0
    \end{array}
\end{align}

Die Nullstellen des Restpolynoms $x^2 -3x +2$ erhalten wir mittels $p$-$q$-Fromel: 
\begin{align}
    x_{1/2} = \frac{3}{2} \pm \sqrt{\frac{9}{4}-2} = \frac{3}{2} \pm \frac{1}{2} \quad \Rightarrow \quad x_1 = 2, x_2 =1.
\end{align}
Damit lautet die Linearfaktorzelegung des Polynoms  
\begin{align}
    f_3(x) = (x-1) (x-2)^2 \textcolor{gray}{\underbrace{= (x-1)(x^2-4x+4) = x^3-5x^2 + 8x-4}_{\text{Probe}}}
\end{align}